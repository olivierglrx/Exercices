\subsection{Géométrie représentation paramétrique de droite.}


\begin{exercice}
On considère les vecteurs de l'espace $\vec{u} =(-2,1,2)$ et $\vec{v} =(1,4,-1)$.
\begin{enumerate}
\item Calculer $\|\vec{u} \|$ et $\| (-2) \vec{v}\|$.
\item Montrer que les vecteurs $\vec{u}$ et $\vec{v}$ sont orthogonaux. 
\item Donner la représentation paramétrique de la droite $D$ passant par $A= (2,-3,1)$ et dirigée par $\vec{v}$. 
\item Déterminer si le point $B=(3,1,0)$ appartient à $D$.  
\end{enumerate} 
\end{exercice}

\begin{correction}
\begin{enumerate}
\item $$\|\vec{u} \| = \sqrt{ (-2)^2+1^2 +2^2 }= \sqrt{ 9}=3$$
et 
$$ \vec{v}= \sqrt{ 1^2+4^2 +(-1)^2 }= \sqrt{ 18}=3\sqrt{2}$$
\item Calculons le produit scalaire entre $\vec{u}$ et $\vec{v}$:  $$\langle \vec{u}, \vec{v}\rangle = -2 \times 1 + 1 \times 4 +2\times -1  = -2+4-2=0$$
D'après le cours, les deux vecteurs sont donc orthogonaux. 
\item $M=(x,y,z)\in D$ si et seulement si $\vec{AM}$ est colinéaire à $\vec{v}$ si et seulement si il existe $\lambda \in \R $ tel que $\vec{AM} = \lambda \vec{v}$. On obtient donc 
$$\left\{ \begin{array}{ccl}
x-2&=&\lambda \times 1\\
y+3&=&\lambda \times 4\\
z-1&=&\lambda \times (-1)
\end{array}\right.  \quad \equivaut \quad \left\{ \begin{array}{ccl}
x&=&2+\lambda \\
y&=&-3+4\lambda \\
z&=&1-\lambda 
\end{array}\right.  $$

\item Il faut vérifier si il existe $\lambda \in \R$ tel que 
$$\left\{ \begin{array}{ccl}
3&=&2+\lambda \\
1&=&-3+4\lambda \\
0&=&1-\lambda 
\end{array}\right.  $$

La première équation donne $\lambda=1$, les autres équations sont compatibles : $1 = -3+4\times 1 $ et $ 0 = 1- 1\times 1$.  Ainsi $$B\in D$$
\end{enumerate} 
\end{correction}