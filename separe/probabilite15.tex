\subsection{Dérangements}

\begin{exercice} 
Une urne contient $n$ boules num\'erot\'ees de 1 \`a $n$. On les extrait successivement et sans remise et apr\`es chaque tirage, on observe le num\'ero de la boule tir\'ee. On dit qu'il y a rencontre au $i$-i\`eme tirage si la boule tir\'ee porte le num\'ero $i$. D\'eterminer la probabilit\'e de l'\'ev\'enement $E$: \og Il n'y a aucune rencontre\fg.

\begin{rems}
Le probl\`{e}me des rencontres peut prendre des formes diverses:
\begin{itemize}
\item[$\bullet$] Un facteur poss\`{e}de $n$ lettres adress\'ees \`{a} $n$ personnes diff\'erentes. Il les distribue au hasard. Quelle est la probabilit\'e pour qu'aucune n'arrive \`{a} destination ?
\item[$\bullet$] \`{A} l'op\'era, $n$ spectateurs d\'epose au vestiaire leur chapeau num\'erot\'e selon leur ordre d'arriv\'e et un ticket leur est alors donn\'e. Mais le responsable a m\'elang\'e tous les tickets et tous les chapeaux sont rendus au hasard. Quelle est la probabilit\'e pour qu'aucun spectateur ne retrouve son chapeau.
\item[$\bullet$] $n$ couples se pr\'esentent \`{a} un concours de danse. Chaque danseur choisit sa partenaire au hasard. Quelle est la probabilit\'e pour que personne ne danse avec son conjoint ?
\end{itemize}
\end{rems}

\end{exercice}



\begin{correction}   \;
\begin{enumerate}
\item Pour tout $i\in\intent{ 1,n}$, on note $N_i$ l'\'ev\'enement \og tirer une boule noire au tirage i\fg \, et on note $G$ l'\'ev\'enement \og \^{e}tre gagnant\fg. Ainsi on a: $G=N_1\cap N_2\cap \dots\cap N_n$. Comme il y a remise, on r\'ep\`ete bien la m\^{e}me exp\'erience $n$ fois dans les m\^{e}mes conditions. Ainsi les \'ev\'enements $(N_1,N_2,\dots, N_n)$ sont mutuellement ind\'ependants et on a: $P(G)=P(N_1)P(N_2)\dots P(N_n)=\left( \ddp\frac{n-1}{n} \right)^n=\left(1- \ddp\frac{1}{n} \right)^n$. On pouvait aussi calculer cette probabilit\'e sans utiliser la mutuelle ind\'ependance mais avec du d\'enombrement car on est dans un cadre d'ordre et de r\'ep\'etition.
\item Pour cela, on montre que la fonction $f: x\mapsto \left( 1-\ddp\frac{1}{x} \right)^x=e^{x\ln{\left( 1-\frac{1}{x}  \right)}}$ est croissante sur $\lbrack 2,+\infty\lbrack$. Il s'agit ici d'une \'etude classique de fonction. La fonction $f$ est d\'erivable sur $\lbrack 2,+\infty\lbrack$ comme quotient, somme et compos\'ee de fonctions et pour tout $x\geq 2$: $f^{\prime}(x)=f(x)\left\lbrack  \ln{\left( 1-\ddp\frac{1}{x}  \right)} +\ddp\frac{1}{x-1}\right\rbrack$. On pose pour tout $x\geq 2$: $g(x)= \ln{\left( 1-\ddp\frac{1}{x}  \right)} +\ddp\frac{1}{x-1}$. Cette fonction est elle aussi d\'erivable sur $\lbrack 2,+\infty\lbrack$ et pour tout $x\geq 2$: $g^{\prime}(x)=\ddp\frac{-1}{x(x-1)^2}$. Ainsi comme on est sur $\lbrack 2,+\infty\lbrack$, $g^{\prime}$ est n\'egative et ainsi la fonction $g$ est strictement d\'ecroissante sur $\lbrack 2,+\infty\lbrack$. De plus $\lim\limits_{x\to +\infty} g(x)=0$ par propri\'et\'e sur les somme, quotient et compos\'ee de limites. Ainsi la fontion $g$ reste positive sur $\lbrack 2,+\infty\lbrack$. Donc $f^{\prime}$ est positive sur $\lbrack 2,+\infty\lbrack$ comme produit de deux nombres positifs et car $f$ est bien positive car c'est une exponentielle. Ainsi la fonction $f$ est bien croissante sur $\lbrack 2,+\infty\lbrack$. Et donc en particulier on a la croissance de la fonction $n\mapsto \left(1- \ddp\frac{1}{n} \right)^n$ sur $\N^{\star}\setminus\lbrace 1\rbrace$. De plus si $n=1$, il n'y a pas de boule noire et ainsi $P(G)=0$. Ceci prouve la croissance sur $\N^{\star}$ car pour tout $n\geq 2$: $f(n)\geq 0\Leftrightarrow f(n)\geq f(1)$. Donc \fbox{la fonction $n\mapsto \left(1- \ddp\frac{1}{n} \right)^n$ est bien croissante sur $\N^{\star}$}.
\item 
\begin{itemize}
\item[$\bullet$] Comme la fonction $n\mapsto \left(1- \ddp\frac{1}{n} \right)^n$ est croissante sur $\N^{\star}$, plus le nombre de boules totales $n$ augmente, plus le nombre $\left(1- \ddp\frac{1}{n} \right)^n$ augmente aussi, \`{a} savoir plus la probabilit\'e de gagner augmente. Le joueur a donc int\'er\^{e}t \`{a} ce que le nombre de boules totales soient le plus grand possible. 
\item[$\bullet$] En utilisant les \'equivalents usuels, on a: $\ln{\left(1- \ddp\frac{1}{n} \right)}\underset{=\infty}{\thicksim} -\ddp\frac{1}{n}$. On obtient donc que: $n\ln{\left(1- \ddp\frac{1}{n} \right)}\underset{=\infty}{\thicksim} -1$. Ainsi, on a: $\lim\limits_{n\to +\infty} n\ln{\left(1- \ddp\frac{1}{n} \right)}=-1$ puis par propri\'et\'e sur la composition de limite, on a: $\lim\limits_{n\to +\infty} e^{n\ln{\left(1- \frac{1}{n} \right)}}=e^{-1}$. Or comme $p_n=e^{n\ln{\left(1- \frac{1}{n} \right)}}$ pour tout $n\in\N^{\star}$, on obtient que: \fbox{$\lim\limits_{n\to +\infty} p_n=e^{-1}$}.  
\end{itemize}
\end{enumerate}
\end{correction}