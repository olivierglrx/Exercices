\subsection{Puissance matrice $M^n =\alpha_n M +\beta_n I_3$ }


\begin{exercice}
Soit $M \in\cM_3(\R) $, définie par 
$ M =\left( 
\begin{array}{ccc}
2 & -2 & 1\\
2 & -3 & 2\\
-1 & 2 & 0
\end{array}
\right)$
\begin{enumerate}
\item Montrer que $(M-\Id_3) (M+3\Id_3)=0_3$.
\item Montrer que pour tout $n\in \N$ il existe des réels $\alpha_n $ et $\beta_n$ tel que $M^n =\alpha_n M +\beta_n I_3$
\item Détermine pour tout $n\in \N$ les réels $\alpha_n$ et $\beta_n$ et en déduire l'expressionde $M^n$ en fonction de $n$.  (On pourra montrer que $\suite{\alpha}$ vérifie une relation de récurrence linéaire d'ordre 2 à coefficients constants) 
\end{enumerate}


\end{exercice}

\begin{correction}
\begin{enumerate}
\item $$(M - Id) =\left( 
\begin{array}{ccc}
1 & -2 & 1\\
2 & -4 & 2\\
-1 & 2 & -1 
\end{array}
\right) \quadet 
(M +3Id) = \left( 
\begin{array}{ccc}
5 & -2 & 1\\
2 & 0 & 2\\
-1 & 2 & 3 
\end{array}
\right) 
$$ 
On calcule coefficient par coefficient le produit : 
$$(M - Id) (M +3Id)  =\left( 
\begin{array}{ccc}
5-4-1 &  -2+0+2 & 1-4+3\\
10 -8-2 & -4+0+4 & 2-8+6\\
-5+4+1& 2+0-2 & -1+4-3 
\end{array}
\right) =\left( 
\begin{array}{ccc}
0&0&0\\
0&0&0\\
0&0&0
\end{array}
\right)  $$

\item Soit $P$ la propriété définie pour tout $n\in \N$ par 
$P(n) : $" il existe des réels $\alpha_n $ et $\beta_n$ tel que $M^n =\alpha_n M +\beta_n I_3$"  

 La question précédente montre que $M^2=-2M +3Id$, la proposition est donc  vraie au rang $n=2$ (elle est évidente au rand $n=0$ et $n=1$) 
 
Supposons la propriété vraie à un rang $n$ fixé et  montrons son hérédité. 
\begin{align*}
M^{n+1} &= M \times M^n
\end{align*}
Par hypothèse de récurrence, il existe des réels $\alpha_n $ et $\beta_n$ tel que 
\begin{align*}
M^{n+1} &= M \times (\alpha_n M +\beta_n \Id_3)\\
				&= \alpha_n M^2 +\beta_n M
\end{align*}
 On remplace $M^2$ par son expression obtenue précédemment : 
 \begin{align*}
M^{n+1} &= \alpha_n (-2M +3\Id_3) +\beta_n M\\
				&= (-2\alpha_n +\beta_n) M +3 \alpha_n \Id_3\\
				&=\alpha_{n+1} M +\beta_{n+1} \Id_3
\end{align*}
La propriété est héréditaire. Elle est donc vraie pour tout $n\in \N$. 


\item On a vu que $\alpha_n$ et $\beta_n$ vérifiaient les relations de récurrences suivantes : 
$$\left\{ \begin{array}{rcl}
\alpha_{n+1} &=& -2\alpha_n +\beta_n\\
\beta_{n+1} &=& 3\alpha_n
\end{array}\right.$$

On obtient donc 
\begin{align*}
\alpha_{n+2} &= -2\alpha_{n+1} + \beta_{n+1}  \\
					&= -2\alpha_{n+1} +3\alpha_n
\end{align*}
C'est une suite récurrente linéaire d'ordre 2. Son polynome caractéristique est $P(X)=X^2 +2X -3=(X-1)(X+3)$ Les deux racines sont donc $1 $ et $-3$. Il existe donc $u,v\in \R^2$ tel que pour tout $n\in \N$:
$$\alpha_n = u (1)^n +v (-3)^n$$
Comme $\alpha_0 =0 $ et $\alpha_1 =1 $, les calculs donnent : 
$u = \frac{1}{4}$, $v=-\frac{1}{4}$. 
$$\left\{ 
\begin{array}{cc}
\alpha_n &= \frac{1}{4}  - \frac{1}{4} (-3)^n \\
\beta_n &= \frac{3}{4}  - \frac{3}{4} (-3)^n \\
\end{array}\right.
$$





\end{enumerate}
\end{correction}