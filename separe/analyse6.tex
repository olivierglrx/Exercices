\subsection{Résolution de $\sqrt{e^x-2} \geq e^{x}-4$}
\begin{exercice}
Donner l'ensemble de définition de 
$$f(x) = \sqrt{e^x-2}$$

Résoudre $$f(x)\geq e^{x}-4$$
\end{exercice}

\begin{correction}
$f$ est bien définie pour tout $x$ tel que $e^x-2\geq 0$ c'est à dire pour $e^x\geq2$ soit $x\geq \ln(2)$

\conclusion{ $D_f =[\ln(2),+\infty[$}

On fait le changement de variable $e^x=X$, l'équaiton $f(x) \geq e^x-4$ équivaut alors à 
$$\sqrt{X-2} \geq X-4 \quad (E')$$
$(E')$ est bien définie sur $[2,+\infty[$

On étudie alors le signe de $X-4$

\begin{itemize}
\item Si $X-4\geq 0$, ie $X\in [4,+\infty[$. 

\begin{align*}
E' \equivaut X-2 \geq X^2 - 8X +16\\
	\equivaut X^2 -9X +18\leq 0
\end{align*}
Le discriminant de $X^2 -9X +18$ vaut $\Delta =9^2 - 4*18 = 81- 72 =9=3^2$
On a donc deux racines réelles : 
$$X_1= \frac{9 +3}{2} = 6 \quadet X_2 = \frac{9-3}{2}= 3$$

Donc $(E') \equivaut  (X-6)(X-3)\leq 0$ d'où les solutions sur $ [4,+\infty[$ : 
$$\cS_1= [4,6]$$


\item Si $X-4<0$, ie $X\in ]-\infty,4[$. 

Alors comme $\sqrt{X-2} \geq 0$  et $X-4<0$, tous les réels de l'ensemble de définition sont solutions 
$$\cS_2 = [2,4]$$


Ainsi les solutions de $(E')$ sont 
$$\bS' = [2,6]$$



\end{itemize}

On repasse à la variable $x$ on a $e^x =X$ donc $x =\ln(X)$

\conclusion{Les solutions de l'équation $f(x) \geq e^{x}-4$ sont 
$\cS = [\ln(2), \ln(6)]$}

\end{correction}