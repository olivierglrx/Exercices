\subsection{Equation intégrale $f(x) =\int_0^{ax} f(t)dt$ (Pb)}

\begin{exercice}
Soit $a\in ]-1,1[. $ On suppose l'existence d'une application $f$, continue sur $\R$, telle que :
$$\forall x\in \R, \quad f(x) =\int_0^{ax} f(t)dt.$$
\begin{enumerate}
\item Calcul des dérivées successives de $f$. 
\begin{enumerate}
\item Justifier l'existence d'une primitive $F$ de $f$ sur $\R$ et écrire alors, pour tout nombre réel $x,$
$f(x)$ en fonction de $x, a$ et $F$. 
%En déduire une expression de $f(x)$  en fonction de $x, a$ et $F$. 
\item Justifier la dérivabilité de $f$ sur $\R$ et exprimer, pour tout nombre réel $x$, $f'(x)$  en fonction de $x, a$ et $f$. 
\item Démontrer que $f$ est de classe $\cC^\infty $ sur $\R$ et que pour tout nombre entier naturel $n,$ on a 
$$\forall x\in \R \quad  f^{(n)} (x) =a^{n(n+1)/2} f(a^nx).$$
\item En déduire, pour tout nombre entier naturel $n$ la valeur de $f^{(n)} (0)$. 
\end{enumerate}
\item Démontrer que, pour tout nombre réel $x$ et tout nombre entier $n$, on a :
$$f(x) = \int_0^x \frac{(x-t)^n}{n!} f^{(n+1)} (t) dt.$$
\footnotesize{ \textit{On pourra faire une récurrence et utiliser une intégration par parties}}
\normalsize{}
\item Soit $A$ un nombre réel strictement positif. 
\begin{enumerate}
\item Justifier l'existence d'un nombre réel positif ou nul $M$ tel que : 
$$\forall x\in [-A,A], \quad |f(x) | \leq M$$
et en déduire que pour tout nombre entier naturel $n$, on a :
$$\forall x\in [-A,A], \quad |f^{(n)}(x) | \leq M$$.
\item Soit $x$ un nombre réel apartenant à $[-A,A].$ Démontrer que, pour tout nombre entier naturel $n$, on  a 
$$|f(x)| \leq M\frac{A^{n+1}}{(n+1)!}.$$
\item En déduire que $f(x) = 0$ pour tout $x\in [-A,A]$
\item Que peut-on en déduire sur la fonction $f$ ? 
\end{enumerate} 
\end{enumerate}
\end{exercice}



\begin{correction}
\begin{enumerate}
\item 
\begin{enumerate}
\item $f$ est continue sur $\R$ donc admet une primitive, notée $F$. On a 
par définition de l'intégrale $f(x) = F(ax) - F(0)$. 
\item Une primitive est par définiton une fonction de classe $\cC^1$ donc $F$ est de classe $\cC^1$ et finalemtn $f$ est de classe $\cC^1$. On a 
$$f'(x) = a F'(ax) = af(ax).$$ 
\item On pose $P(n)$ : " $f$ est de classe $C^n$ et $\forall x\in \R \quad  f^{(n)} (x) =a^{n(n+1)/2} f(a^nx)$ ".
\begin{itemize}
\item $P(0)$ est vraie par hypothèse. 
\item Supposons qu'il existe $n\in \N$ tel que $P(n)$ soit vraie. On a alors $f$ de classe $\cC^n$, et $\forall x\in \R \quad  f^{(n)} (x) =a^{n(n+1)/2} f(a^nx)$. 
Or comme $f$ est de classe $\cC^1$ d'après la question précédente, on a alors que $ f^{(n)}$ est de classe $\cC^1$ c'est à dire $f$ de classe $\cC^{n+1}$. Enfin 
$\forall x\in \R$,   \begin{align*}
 f^{(n+1)} (x) &= a^{n(n+1)/2}  f'(a^nx)\\ 
 						&= a^{n(n+1)/2+n} a f(a a^n x) \quad \text{d'après la question précédente}\\
 						&= a^{n(n+1)/2+n+1} f(a^{n+1} x) \\
 						&= a^{(n+1)(n+2)/2} f(a^{n+1} x) \\
\end{align*} 
\item On a montré par récurrence que pour tout $n\in \N$, $f$ est de classe $\cC^n$. Elle est donc de classe $\cC^\infty$ et $\forall x\in \R \quad  f^{(n)} (x) =a^{n(n+1)/2} f(a^nx)$.
\end{itemize}
\item  On a donc $f^{(n)} (0) = a^{n(n+1)/2 } f(0) $. Or $f(0) = \int_0^{0} f(t)df =0$
Donc pour tout $n\in \N$ $$f^{(n)} (0) =0$$
\end{enumerate}
\item On montre le résultat par récurrence. On pose pour tout nombre réel $x$ et tout nombre entier $n$, la proposition 
$P(n) : "f(x) = \int_0^x \frac{(x-t)^n}{n!} f^{(n+1)} (t) dt."$
\begin{itemize}
\item Réécrivons $P(0)$. On a $P(0) : "f(x) = \int_0^x \frac{(x-t)^0}{0!} f^{(0+1)} (t) dt. " $, c'est à dire : $f(x) = \int_0^x  f'(t) dt.$ Ce qui est vrai par définition de l'intégrale. 
\item Supposons qu'il existe $n\in \N$  tel que $P(n)$ soit vraie. On  a alors pour tout nombre réel $x$, $f(x) = \int_0^x \frac{(x-t)^n}{n!} f^{(n+1)} (t) dt$. 
Comme suggérer par l'énoncé on fait une IPP. On pose 
\begin{minipage}{0.4 \textwidth}
$u(t) = f^{(n+1)}(t)$\\
$v(t) = -\frac{(x-t)^{n+1}}{(n+1)!}$
\end{minipage}
\begin{minipage}{0.4 \textwidth}
$u'(t) = f^{(n+2)}(t)$\\
$v'(t) = \frac{(x-t)^{n}}{n!}$
\end{minipage}
On a donc 
\begin{align*}
f(x)&=\left[ \frac{(x-t)^{n+1}}{(n+1)!}  f^{(n+1)}(t)\right]_0^x - \int_0^x - \frac{(x-t)^{n+1}}{(n+1)!}f^{(n+2)}(t)dt\\
\end{align*}
Le crochet vaut $\frac{(x-x)^{n+1}}{(n+1)!}  f^{(n+1)}(x)- \frac{(x-0)^{n+1}}{(n+1)!}  f^{(n+1)}(0)$ les deux termes valent 0 (le second à l'aide de la question précédente). On obtient bien 
 \begin{align*}
f(x)&=  \int_0^x  \frac{(x-t)^{n+1}}{(n+1)!}f^{(n+2)}(t)dt
\end{align*}
\item Par récurrence la propriété est vraie pour tout $n\in \N$. 
\end{itemize}
\item \begin{enumerate}
\item Soit $A>0$. Comme $f$ est continue et $[-A,A]$ est un segment, le théorème de continuité sur un segment assure que $f$ est bornée et atteint ses bornes. Donc il existe $M>0$ tel que pour tout $x\in [-A,A]$, $|f(x)|\leq M$.

D'après 1c) on sait que pour tout $x\in \R$, $f^{(n)} (x) = a^{n(n+1)/2} f(a^n x)$ En particulier $|f^{(n)} (x)| =|a^{n(n+1)/2}| | f(a^n x)|$ 
Or comme $|a|<1$ , $|a^{n(n+1)/2}|  \leq 1$ et pour tout $x\in [-A,A]$, on a $a^n x \in  [-A,A]$ et ainsi $ | f(a^n x)| \leq M$. Au final pour tout  $x\in [-A,A]$ : 
$$|f^{(n)} (x)|\leq M.$$


\item  D'après la question 2 on a : 
$\ddp f(x) = \int_0^x \frac{(x-t)^n}{n!} f^{(n+1)} (t) dt$, donc $|f(x)| \leq\ddp   \int_0^x \left|  \frac{(x-t)^n}{n!} f^{(n+1)} (t) \right| dt$ c'est l'inégatilité triangulaire sur les intégrales. On majore maintenant $\left|  f^{(n+1)} (t) \right| $ à l'aide de la question précédente, on obtient pour tout $x\in [-A,A]$ :
$$f(x) \leq  \ddp M  \int_0^x \left|  \frac{(x-t)^n}{n!}  \right| dt.$$
Donc $f(x) \leq \ddp M \left[ \frac{|(x-t)|^{n+1}}{(n+1)!}\right]_0^x \leq M  \frac{|x|^{n+1}}{(n+1)!}$ Or comme $x\in [-A,A]$ on a bien : 
$$|f(x)|\leq M\frac{A^{n+1}}{(n+1)!}$$
\item Par croissance comparée, en passant à la limite on a $$\lim_{n\tv \infty} \frac{A^{n+1}}{(n+1)!} = 0$$
Ainsi le théorème des gendarmes assure que pour tout $x\in [-A,A]$ on a 
$$\lim_{n\tv \infty} f(x) = 0.$$ Evidemment $f(x) $ ne dépend pas de $n$ donc par unicité de la limite $f(x) = 0$

Ceci étant vrai pour tout $x \in [-A,A]$ et comme $A$ est arbitraire, ceci est vrai pour tout $x \in \R$. 

$$f\equiv 0$$

\end{enumerate}

\end{enumerate}

\end{correction}