\subsection{EV - exemple et c-ex}

\begin{exercice}
On admet que l'ensemble des fonctions réelles $\cF(\R)$ est un espace vectoriel. 
Dire si les ensembles suivants sont des sous-espaces vectoriels de $\cF(\R)$: 
(Si oui, le prouver, si non expliquer pourquoi )
\begin{itemize}
\item L'ensemble des fonctions qui valent $0$ en $0$ : $E_1 = \{ f \in  \cF(\R)\, |\, f(0)=0 \}$ 
\item L'ensemble des fonctions qui valent $1$ en $0$ : $E_2 = \{ f \in  \cF(\R)\, |\, f(0)=1 \}$ 
\item L'ensemble des fonctions qui vallent $0$ en $1$ : $E_3 = \{ f \in  \cF(\R)\, |\, f(1)=0 \}$
\item L'ensemble des fonctions $E_4 = \{ f \in  \cF(\R)\, |\, (f(0))^2 +2f(0)=0 \}$
\end{itemize}


\end{exercice}
\begin{correction}
\begin{itemize}
\item $E_1$  et $E_3$ sont des  sev. La preuve est la même. Soit $f, g\in E_1 $ (ou $E_3$) et $\lambda\in \R$. On a 
$$(f+\lambda g) (0)  = f(0)+\lambda g(0) = 0$$
Donc $f+\lambda g \in E_1$. 
\item $E_2$ et $E_4$ ne sont pas des sev.  $E_2$ ne contient pas la fonction nulle. 
La fonction constante égale à $-2$, ($f(x) =-2$) appartient à  $E_3$  mais $2f(x) =-4$ n'appartient pas à $E_3$. $E_3$ n'est donc pas stable par multiplication par un sclaire. Ce n'est pas un sev.  

\end{itemize}
\end{correction}