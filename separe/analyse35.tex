\subsection{Equation trigonométrique / changement de variable}


\begin{exercice}
Résoudre l'inéquation d'inconnue $x$: 

$$\frac{\frac{1}{2}}{x-\frac{1}{2}}\leq x+\frac{1}{2}$$

Résoudre sur $[0,2\pi[$ :  
$$\frac{\frac{1}{2}}{\sin(x)-\frac{1}{2}}\leq \sin(x)+\frac{1}{2}$$

Représenter les solutions sur le cercle trigonométrique. 
\end{exercice}

\begin{correction}
Pour tout $x\in \R\setminus\{ \frac{1}{2}\}$ l'inéquation est équivalente à 
$$\frac{\frac{1}{2} }{x-\frac{1}{2}} - \frac{(x+\frac{1}{2})(x-\frac{1}{2})}{x-\frac{1}{2}}\leq 0$$
D'où
$$ \frac{-x^2 +\frac{3}{4}}{x-\frac{1}{2}}\leq 0$$
$$\frac{(x-\frac{\sqrt{3}}{2})(x+\frac{\sqrt{3}}{2})}{x-\frac{1}{2}}\geq 0$$

Les solutions sont 
$$\cS  = [-\frac{\sqrt{3}}{2}, \frac{1}{2}[\cup [\frac{\sqrt{3}}{2} , +\infty[.$$

En posant $X= \sin(x)$, $x$ est solution de la seconde inéquation si et seulement si 
$$\sin(x) \in \cS$$
On résoud donc 
$$\sin(x) \in   [-\frac{\sqrt{3}}{2}, \frac{1}{2}[\cup [\frac{\sqrt{3}}{2} , +\infty[$$
Comme $\sin(x)\leq 1$ ceci équivaut à 
$$\sin(x) \in  [-\frac{\sqrt{3}}{2}, \frac{1}{2}[\cup [\frac{\sqrt{3}}{2} , 1]$$

Sur $[0, 2\pi[ $ on a donc 
$$x \in [0, \frac{\pi}{6}[\cup [\frac{\pi}{3}, \frac{2\pi}{3}]\cup ]\frac{5\pi}{6}, 
\frac{4\pi}{3}]\cup [\frac{5\pi}{3}, 2\pi[$$

\end{correction}