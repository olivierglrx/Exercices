\subsection{Matrice, famille libre, commutant}

\begin{exercice}
Soit $A$ la matrice suivante : 
$\left(\begin{array}{ccc}
0&0&0\\
1&0&0\\
0&1&0
\end{array}
\right).$
\begin{enumerate}
\item Déterminer les réels $\lambda\in \R$ pour lesquels la matrice $A-\lambda \Id$ n'est pas  inversible. On appelle ces réels les \emph{valeurs propres} de $A$. 
\item
\begin{enumerate}
\item  Calculer $A^2$ et $A^3$. 
\item Quelle est la dimension de $\cM_3(\R)$ ? 
\item Montrer que $(Id_3, A, A^2)$ est une famille libre de l'espace vectoriel $\cM_3(\R)$.
\item Est-ce une base ? 
\end{enumerate}

\item On considère $\cS$ l'ensemble des matrices  $M\in \cM_3(\R)$ telles que $AM=MA$. 
\begin{enumerate}
\item Montrer que $\cS$ est un sous-espace vectoriel de $\cM_3(\R)$. 
\item Soit $\alpha, \beta, \gamma $  trois réels et $M = \alpha \Id_3 +\beta A +\gamma A^2$. Vérifier que $M\in \cS$
\item Réciproquement, on considère $a, b, c, d, e, f, g, h, $ et $i$ des réels tel que $M =   \left(\begin{array}{ccc}
a&b&c\\
d&e&f\\
g&h&i
\end{array}
\right) \in \cS$. Déterminer, en fonction des coefficients de $M$, trois réels $\alpha, \beta, \gamma$ tels que $ M =\alpha \Id_3 +\beta A +\gamma A^2$
\item En déduire, une base de $\cS$. 
\end{enumerate} 
\item On considère $S'$ l'ensemble des matrices $M$ de $\cM_3(\R)$ telles que $M^3=0$  et $M^2 \neq 0$. 
\begin{enumerate}
\item Est ce que $\cS'$ est un sous-espace vectoriel de $\cM_3(\R)$ ?
\item Soit $P\in \cM_3(\R)$  une matrice inversible et $M=P A P^{-1} $. Vérifier que $M\in \cS'$.\\
Dans la suite, tout vecteur de $\R^3$ sera assimilé à une matrice colonne de $\cM_{3,1}(\R)$ de sorte que, pour tout vecteur $X\in \R^3$, le produit matriciel $MX$ soit correctement défini. 
\item Soit $M =\left( 
\begin{array}{ccc}
-1&1& 1\\
1&1&-1\\
0&2&0
\end{array}
\right).$.
\begin{enumerate}
\item  Vérifier que $M \in \cS'$. 
%On considère $f$ l'endomorphisme de $\R^3$ dont $M$ est la matrice dans la base canonique, c'est-à-dire $f : \R^3 \tv \R^3 $, $X\mapsto MX$.
\item Prouver qu'il existe un vecteur $X\in \R^3$ tel que $M^2 X$ soit non nul.
\item Montrer que la famlille $B=(X,MX,M^2X) $ est une base de $\R^3$.
%\item Déterminer la matrice de $f$ dans la base $\cB$. 
\end{enumerate}


\end{enumerate}
\end{enumerate}
\end{exercice}
\vspace{1cm}



\begin{correction}
\begin{enumerate}
\item $A-\lambda \Id_3 =\left(\begin{array}{ccc}
-\lambda&0&0\\
1&-\lambda&0\\
0&1&-\lambda
\end{array}
\right).$ Cette matrice est déjà échelonnée, elle n'est pas  inversible si et seulement si $\lambda =0$.
\item $A^2=\left(\begin{array}{ccc}
0&0&0\\
0&0&0\\
1&0&0
\end{array}
\right)$  et  $A^3 =0$.
\item 
\begin{enumerate}
\item $\cS$ contient la matrice nulle, $\cS$ est donc non vide. 
De plus si $M,N \in \cS$ et $\lambda \in \R$, on a 
\begin{align*}
(M +\lambda N) A &= M A + \lambda NA \\
							&= AM +\lambda AN\\
							&= A (M +\lambda N) 
\end{align*}
Donc $\cS$ est stable par combinaison linéaire, c'est donc un sev de $\cM_3(\R)$. 
\item Soit $(\alpha, \beta, \gamma )\in \R^3$  et $M = \alpha \Id_3 +\beta A +\gamma A^2$. On  a
\begin{align*}
AM &= A (  \alpha \Id_3 +\beta A +\gamma A^2 )\\
	&= \alpha A +\beta A^2 +\gamma A^3 \\
	&= (  \alpha \Id_3 +\beta A +\gamma A^2 ) A \\
	&=MA
\end{align*}
Ainsi $M\in \cS$. 

\item Soit $M =   \left(\begin{array}{ccc}
a&b&c\\
d&e&f\\
g&h&i
\end{array}
\right) \in \cS$ On a 
$$AM = \left(\begin{array}{ccc}
0&0&0\\
a&b&c\\
d&e&f
\end{array} \right)  \quad \text{ et } \quad MA = \left(\begin{array}{ccc}
b&e&0\\
e&f&0\\
h&i&0
\end{array}  \right) $$ 

Si $M\in \cS$ on obtient les équations suivantes : 
$\left\{ 
\begin{array}{c}
b=0\\
e=0\\
0=0\\
a=e\\
b=f \\
c=0\\
d=h\\
e=i\\
f=0
\end{array}
\right.$
Ce qui se simplifie en
$\left\{ 
\begin{array}{l}
b=c=f=0\\
a=e=i\\
d=h
\end{array}
\right.$
Au final si $M\in \cS$, $M$ est de la forme 

$$\left(\begin{array}{ccc}
a&0&0\\
d&a&0\\
g&d&a
\end{array}
\right)  = a \Id_3 +d A +gA^2$$
\item On en déduit que $\cS =\{ M =  \alpha \Id_3 +\beta A +\gamma A^2\, |\, (\alpha, \beta, \gamma) \in\R^3\} = \Vect( \Id_3, A, A^2)$. On a vu à la question 2b) que $( \Id_3, A, A^2)$ etait une famille libre de $\cM_3(\R)$, comme elle est génératrice de $\cS$ par définition d'un espace vectoriel engendré c'est donc une base de $\cS$. 


 
 
\end{enumerate}
\item 
\begin{enumerate}
\item La matrice nulle n'appartient pas à $\cS'$ car $0^2=0$. 
\item Soit $P\in \cM_3(\R)$  une matrice inversible et $M=P A P^{-1} $, on  d'une part
$M^2 = PA^2P^{-1} $ qui est non nul car $A^2$ est non null et $P$ et $P^{-1}$ sont inversibles  et d'autre part  $M^3 =PA^3 P^{-1} = P 0 P^{-1} =0$. Donc $M \in \cS'$
\item \begin{enumerate}
\item $M^2 =\left(\begin{array}{ccc}
2&2&-2\\
0&0&0\\
2&2&-2
\end{array}
\right)  $  et $M^3 = \left(\begin{array}{ccc}
0&0&0\\
0&0&0\\
0&0&0
\end{array}
\right)  $. Ainsi $M\in \cS'$. 
\item Si tous les vecteurs de $\R^3$ vérifiaient $M^2X=0$, alors $M^2 = 0$ (il suffit de vérifier avec les vecteurs de la base canonique). Ainsi il existe un vecteur tel $M^2X\neq 0$
\item Soit $\lambda_0,\lambda_1,\lambda_2\in \R^3$ tel que  
$$\lambda_0 X +\lambda_1 M X +\lambda_2 M^2 X= 0$$
On a en composant par $M:$
$$\lambda_0 M X +\lambda_1 M^2 X +\lambda_3 M^3 X=0$$
Or $M^3 =0 $ donc 
$$\lambda_0 M X +\lambda_1 M^2 X=0$$
En réitérant le processus on obtient 
$$\lambda_0 M^2 X =0  $$
Or comme $M^2X \neq 0$ par hypothèse, on a $\lambda_0=0$ 
L'équation précédente donne alors $\lambda_1=0$ et finalement 
$$\lambda_0=\lambda_1 =\lambda_2 =0$$
La famille est libre. Comme elle contient 3 vecteurs dans un ev de dimension $3$ c'est une base. 


\end{enumerate}
\end{enumerate}

\end{enumerate}
\end{correction}