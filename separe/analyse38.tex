\subsection{Calcul de $\sum_{k=0}^n k^4$ }

\begin{exercice}
\begin{enumerate}
\item Rappeler  la valeur  de $ R_3=\ddp \sum_{k=0}^n k^3$ en fonction de $n\in \N$
\item Soit $k\in \N$, développer $(k+1)^5 -k^5$. 
\item A l'aide de la  somme téléscopique  $\sum_{k=0}^n(k+1)^5 -k^5$ donner la valeur de 
$R_4= \ddp \sum_{k=0}^n k^4$  en fonction de $n\in \N$.  (On pourra garder une formule développée, malgré ce que j'ai pu dire en classe... ) 
\item Soit $x\in \N$, on note $R_x(n) =  \ddp \sum_{k=0}^n k^x$
Ecrire une fonction Python qui prend en paramètre $n\in \N $ et $x\in \N $  et rend la valeur de $R_x(n)$
\item Soit $x\in \N$, on note $R_x(n) =  \ddp \sum_{k=0}^n k^x$. 
Ecrire une fonction Python qui prend en paramètre $n\in \N $ et $x\in \N $, qui affiche un message d'erreur si $x$ n'est pas un entier positif et rend la valeur de $R_x(n)$ sinon. 

\item Montrer que les suites $ a_n=\ddp \sum_{k=1}^n \frac{1}{k^2}$  et $b_n= a_n + \frac{1}{n}$ sont adjacentes. 

\item  Ecrire une fonction   Python qui prend en paramètre $e>0$ et qui rend le premier rang $n_0\in \N$ tel que $\left|a_{n_0}-b_{n_0}\right| \leq e$
et la valeur de $a_{n_0}$
\end{enumerate}
\end{exercice}

\begin{correction}
\begin{enumerate}
\item $R_3 =\left(\frac{n(n+1)}{2}\right)^2$
\item $(k+1)^5-k^5 = 5k^4 +10k^3+10k^2 +5k +1 $
\item On a d'une part 
$$\sum_{k=0}^n(k+1)^5 -k^5 = (n+1)^5$$
et d'autre part 
\begin{align*}
\sum_{k=0}^n(k+1)^5 -k^5  &=\sum_{k=0}^n 5k^4 +10k^3+10k^2 +5k +1 \\
											&= 5 R_4  +10 R_3 +10 \frac{n(n+1)(2n+1)}{6} +5 \frac{n(n+1)}{2}+ (n+1)
\end{align*}

Donc 
$$R_4 =\frac{1}{5} \left( (n+1)^5- 10\left(\frac{n(n+1)}{2}\right)^2 -10  \frac{n(n+1)(2n+1)}{6}  - 5 \frac{n(n+1)}{2}-(n+1) \right)$$
On trouve à la fin des calculs 
$$R_4 = \frac{n}{30} (6n^4+15n^3 +10n^2-1)$$

\item 
\begin{lstlisting}
def R(x,n):
  if x<0 or type(x)!=int: #on teste que x est bien positif et entier
    print('erreur')
  else:
    R=0
    for k in range(n+1): # on fait une boucle for pour obtenir la somme
      R=R+k**x #x est l'exposant, k l'indice de somme 
  return(R) #on indente le return au niveau du if...else et pas au 
  				#niveau de la boucle for. 
   	

\end{lstlisting}

\item Regardons la monotonie de $\suite{a}$ et $\suite{b}$. On a pour tout $n\in \N$
$$a_{n+1} -a_{n}= \frac{1}{(n+1)^2}>0$$ 
donc $\suite{a}$ est  croissante. 
\begin{align*}
b_{n+1} -b_{n} &= a_{n+1} +\frac{1}{n+1}-a_{n}+\frac{1}{n}\\
						&= \frac{1}{(n+1)^2} +\frac{1}{n+1}+\frac{1}{n}\\
						&= \frac{n +n(n+1)-(n+1)^2}{n(n+1)^2}\\
						&= \frac{-1}{n(n+1)^2} <0
\end{align*}
Donc la suite $\suite{b}$ est décroissante. 

Enfin  $a_n-b_n= \frac{-1}{n}$, et donc $\lim_{n\tv+\infty} a_n-b_n =0$. 

Ainsi les suites $\suite{a}$ et $\suite{b}$ sont adjacentes. 

\item 
\begin{lstlisting}
from math import abs
def limite(e):   # fonction qui prend en argument le parametre e
   n=0   #on initialise la valeur de n 
   a=1   #on initialise la valeur de a_0
   b=2   #on initialise la valeur de b_0
  
   while abs(a-b) >e:   #tant que la difference est plus grande que e
      n+=1    #on actualise la valeur de n
      a=a+1/(n**2)     #on actualise la valeur de a_n
      b=a+1/(n)     #on actualise la valeur de b_n
   return(n,a ) 
    
\end{lstlisting}
\end{enumerate}
\end{correction}