\subsection{Diagonalisation (Pb)}
\begin{exercice}
Soit $M$ la matrice : 
$$M=\left( \begin{array}{ccc}
2 &1& 0\\
0 &1 & 0  \\
 -1&0&1
\end{array}\right) $$

\begin{enumerate}
\item  Résoudre le système $MX=\lambda X$ d'inconnue $X =\left(
\begin{array}{c}
x\\
y\\
z
\end{array}
 \right)$ où $\lambda$ est un paramètre réel. 
\item Calculer $(M- \Id)^2$. Donner son rang.

 \item Soit $e_1= \left(
\begin{array}{c}
1\\
0\\
-1
\end{array}
 \right)$,  $e_2= \left(
\begin{array}{c}
0\\
0\\
1
\end{array}
 \right)$, et  $e_3= \left(
\begin{array}{c}
1\\
-1\\
0
\end{array}
 \right)$.
Exprimer  $Me_1, Me_2$ en fonction de $e_1, e_2$.
\item Montrer qu'il existe $(\alpha, \beta)\in \R^2$ tel que $M e_3 = \alpha e_2 +\beta e_3$.
 
\item Soit $P= \left(
\begin{array}{ccc}
1&0&1\\
0&0&-1\\
-1&1&0
\end{array}
 \right)$ 
 
 Montrer que $P$ est inversible et calculer son inverse. 
 \item Soit $T=P^{-1}MP$. Calculer $T$. 
 \item Montrer par récurrence que pour tout $n\in \N^*$: 
 $$T^n = P^{-1}M^n P$$
 \item Montrer qu'il existe une matrice diagonale $D$ et une matrice $N$ telles que 
 $$T =D+N \quadet ND=DN$$
\item Montrer que $N^2=0$
\item Montrer   que $T^n = D^n +nND^{n-1}$.
\item En déduire la valeur de $M^n$.
\end{enumerate}
\end{exercice}
\begin{correction}
\begin{enumerate}
\item
$$MX=\lambda X \equivaut   \left( \begin{array}{c}
2x +y  \\
 y \\
 -x +z
\end{array}\right) = \left(
\begin{array}{c}
\lambda x\\
\lambda y\\
\lambda z
\end{array} \right)$$ 

$$\left\{ \begin{array}{ccccc}
2x &+y& & =&\lambda x \\
 &y & & =& \lambda y \\
 -x& &+z&=&\lambda z
\end{array}\right. 
\equivaut \left\{ \begin{array}{ccccc}
(2-\lambda)x &+y& & =&0 \\
 &(1-\lambda)y & & =& 0 \\
 -x& &+(1-\lambda)z&=&0
\end{array}\right. 
$$ 
 En échangeant les lignes et les colonnes on peut voir que le système est déjà échelonné.
$L_3\leftarrow L_1, L_2 \leftarrow _3, L_1\leftarrow L_2$
$$MX=\lambda X 
\equivaut  \left\{ \begin{array}{ccccc}
 -x& &+(1-\lambda)z&=&0\\
(2-\lambda)x &+y& & =&0 \\
 &(1-\lambda)y & & =& 0 
\end{array}\right.$$
$ C_3\leftarrow C_1, C_2 \leftarrow C_3, C_1\leftarrow C_2$
$$
\equivaut \left\{ \begin{array}{ccccc}
 (1-\lambda)z&-x& &=&0\\
 &(2-\lambda)x&+y & =&0 \\
 &  & (1-\lambda)y& =& 0 
\end{array}\right.$$

Si $\lambda \notin \{ 1,2\} $ alors le système est de rang 3, il est donc de Cramer et l'unique solution est 
\conclusion{ $\cS= \{ (0,0,0)\}$}

Si $\lambda =1$, le système est équivalent à 
$$\left\{ \begin{array}{cccc}
 -x& &=&0\\
 (2-1)x&+y & =&0 \\
   & 0& =& 0 
\end{array}\right. \equivaut \left\{ \begin{array}{cc}
 x& =0\\
 y&  =0
\end{array}\right.$$
Le système est de rang 2. L'ensemble des solutions est 
\conclusion{ $\cS= \{ (0,0,z) \, |\, z\in \R\}$}

Si $\lambda =2$, le système est équivalent à 
$$\left\{ \begin{array}{ccccc}
(1-2)z&-x& &=&0\\
 & 0 &+y & =&0 \\
 &  & (1-2)y& =& 0 
\end{array}\right. \equivaut \left\{ \begin{array}{cccc}
-z&-x& &=0\\
 &  &y  &=0 \\
 &  & y &= 0 
\end{array}\right.\equivaut \left\{ \begin{array}{cl}
x &=-z\\
  y  &=0 
\end{array}\right.$$
Le système est de rang 2. L'ensemble des solutions est 
\conclusion{ $\cS= \{ (-z,0,z) \, |\, z\in \R\}$}

\item $M-\Id= \left( \begin{array}{ccc}
1 &1& 0\\
0 &0& 0  \\
 -1&0&0
\end{array}\right) $

Donc \conclusion{$(M-\Id)^2= \left( \begin{array}{ccc}
1 &1& 0\\
0 &0& 0  \\
 -1&-1&0
\end{array}\right) $}

Le système associé est 

$\left\{  \begin{array}{ccr}
x &+y&  =0\\
 & & 0  =0\\
 -x&-y&=0
\end{array}\right. \equivaut \left\{  \begin{array}{cr}
x +y&  =0\\
\end{array}\right. $
Il est de rang 1. Donc
\conclusion{$(M-\Id)^2$ est de rang $1$}
\item 
Le calcul montre que $Me_1 =2e_1$ et  $Me_2=e_2$
\item Le calcul montre que 
$Me_3 =  \left( \begin{array}{c}
1\\
-1\\
-1\\
\end{array}\right) = \left( \begin{array}{c}
1\\
-1\\
0\\
\end{array}\right) - \left( \begin{array}{c}
0\\
\\
1\\
\end{array}\right)  e_3-e_2 $


Ainsi on peut prendre 
\conclusion{$\alpha =-1$ et $\beta =1$}

\item  On considère la matrice augmentée : 
$\left(\begin{array}{ccc|ccc}  
1&0&1 & 1&0&0 \\
0&0&-1& 0&1&0 \\
-1&1&0& 0&0&1 
\end{array}\right)$

$L_3\leftarrow L_3+L_1$  donnent
$$\left(\begin{array}{ccc|ccc}  
1&0&1 & 1&0&0 \\
0&0&-1& 0&1&0 \\
0&1&1& 1&0&1 
\end{array}\right)$$
$L_1\leftarrow L_1+L_2$ et $L_3\leftarrow L_3+L_2$
donne 
$$\left(\begin{array}{ccc|ccc}  
1&0&0 & 1&1&0 \\
0&0&-1& 0&1&0 \\
0&1&0& 1&1&1 
\end{array}\right)$$

$L_2\leftarrow -L_2$
donne 
$$\left(\begin{array}{ccc|ccc}  
1&0&0 & 1&1&0 \\
0&0&1& 0&-1&0 \\
0&1&0& 1&0&1 
\end{array}\right)$$
Enfin 
$L_2\leftrightarrow L_3$
donne 
$$\left(\begin{array}{ccc|ccc}  
1&0&0 & 1&1&0 \\
0&1&0& 1&1&1 \\
0&0&1& 0&-1&0 
\end{array}\right)$$

\conclusion{ $P$ est inversible d'inverse $\left(\begin{array}{ccc}  
1&1&0 \\
 1&1&1 \\
0&-1&0 
\end{array}\right)$}

\item Le calcul donne  
\conclusion{ $T=\left(\begin{array}{ccc}  
2&0&0 \\
0 &1&-1 \\
0&0&1 
\end{array}\right)$}
(sur une copie, le produit intermédiaire $MP$ serait apprécié)

\item 


(CF ex 6-3 du DM de Noël) \\

On pose $P(n) : "T^n =P^{-1} M^n P"$
\begin{itemize}
\item[Initialisation] 
$T^1 =T$ et $P^{-1} M^1 P= P^{-1} M P=T$ d'après la définition de $T$.
Donc $P(1) $ est vrai. 

\item[Hérédité] On suppose qu'il existe $n\in \N$ tel que $P(n)$ soit vraie. 
On a alors 
\begin{align*}
 (T)^{n+1}&=  T^n  T
\end{align*}
et donc par Hypothése de récurrence : 
\begin{align*}
 T^{n+1}&=  (P^{-1}M^n P )  (P^{-1}M P )\\
 							&=  (P^{-1}M^n P  P^{-1}M P )\\
 							&=  (P^{-1}M^n \Id M P )\\
 							&=  (P^{-1}M^n M P )\\
 							&=  (P^{-1}M^{n+1} P )
\end{align*}
\item[Conclusion] $P(n)$ est vraie pour tout $n$. 




\end{itemize}
\item On a 
$T= \left(\begin{array}{ccc}  
2&0&0 \\
0 &1&-1 \\
0&0&1 
\end{array}\right)  = \left(\begin{array}{ccc}  
2&0&0 \\
 0&1&0 \\
0&0&1 
\end{array}\right) + \left(\begin{array}{ccc}  
0&0&0 \\
0 &0&-1 \\
0&0&0 
\end{array}\right)  $  

On pose $D= \left(\begin{array}{ccc}  
2&0&0 \\
 0&1&0 \\
0&0&1 
\end{array}\right) $ et $N= \left(\begin{array}{ccc}  
0&0&0 \\
 0&0&-1 \\
0&0&0 
\end{array}\right)  $ 
On a bien $T =D+N$ et  le calcul donne 
$DN = \left(\begin{array}{ccc}  
0&0&0 \\
0 &0&-1 \\
0&0&0 
\end{array}\right) =DN $ 


\item C'est un calcul. La question \og normale\fg\,  devrait être \og Calculer $N^2$ \fg \,  , mais ne permet pas de faire la question suivante si on n'a pas trouvé la forme de $N$. 

\item Solution 1 : On peut appliquer le binome de Newton à $T= D+N$   car $D$ et $N$ commutent. On a alors 

$$T^n =\sum_{k=0}^n \binom{n}{k} N^k D^{n-k}$$
Comme pour tout $k\geq 2$, $N^2=0$ il reste dans cette somme seulement les termes $k=0$ et $k=1$. On obtient donc 
\begin{align*}
T^n  &= \binom{n}{0} N^0 D^{n-0}+ \binom{n}{1} N^1 D^{n-1}\\
		&=D^n + nND^{n-1}
\end{align*}






Solution 2: 

On pose $P(n) : \og  T^n =D^n +n D^{n-1} N \fg$

\begin{itemize}
\item \underline{Initialisation }
$T^1 =T$ et $D^1+1D^0 N = D^1 +\Id N=D+N =T$ d'après la définition de $D,N$.
Donc $P(1) $ est vrai. 

\item \underline{Hérédité} On suppose qu'il existe $n\in \N$ tel que $P(n)$ soit vraie. 
On a alors 
\begin{align*}
 (T)^{n+1}&=  T^n  T
\end{align*}
et donc par Hypothése de récurrence : 
\begin{align*}
 T^{n+1}&= (D^n +nD^{n-1} N)(D+N)\\
 							&=  D^n D +n D^{n-1} N D + D^n N + nD^{n-1}N^2\\
\end{align*}
Comme $ND=DN$ on a $D^{n-1} N D= D^{n-1} DN  =D^{n} N$.  on a par ailleurs $N^2=0$ donc 

\begin{align*}
 T^{n+1}&=D^{n+1} +D^n N +nD^n N\\
 			&=D^{n+1} + (n+1) D^{(n+1)-1} N 
\end{align*}
Ainsi la propriété est héréditaire. 

\item \underline{Conclusion} $P(n)$ est vraie pour tout $n$. 
\end{itemize}

\item On a d'après la question 7 
$$M^n = P T^n P^{-1}$$
et d'après la question précédente : 
$$T^n = D^{n} + n D^{n-1} N  =\left(\begin{array}{ccc}  
2^n&0&0 \\
0 &1&-n \\
0&0&1 
\end{array}\right)$$
Le calcul donne 

$T^n P^{-1} = \left(\begin{array}{ccc}  
2^n&2^n&0 \\
1 &1+n&1 \\
0&-1&0 
\end{array}\right)$
et 
\conclusion{
 $M^n=\left(\begin{array}{ccc}  
2^n&2^n-1&0 \\
0&1&0 \\
-2^n+1&-2^n+1+n&1 
\end{array}\right)$

}




\end{enumerate}
\end{correction}