\subsection{Racines de $z^n+z+1$ (Pb)}

\begin{exercice}
On considère l'équation suivante , d'inconnue $z\in \bC$ : 
\begin{equation}\tag{$E$}
z^3 +z+1=0
\end{equation} 



%\paragraph{Partie II : Cas n=3}
\begin{enumerate}
\item On note $f : \R\mapsto \R, $ la fonciton définie par $f(t) = t^3+t+1.$
A l'aide de l'étude de $f$, justifier que l'équation $(E)$ possède une unique solution réelle, que l'on notera $r$. Montrer que $r \in ]-1, \frac{-1}{2}[$.
\item On note $z_1$ et $z_2$ les deux autres solutions complexes de $(E)$ qu'on ne cherche pas à calculer. On sait alors que le polynôme $P(X) = X^3+X+1$ se factorise de la manière suivante : 
$$P(X)  = (X-r)(X-z_1) (X-z_2).$$
En déduire que $z_1+z_2=-r$ et $z_1z_2=\frac{-1}{r}$.
\item Justifier l'encadrement  : $\frac{1}{2}<|z_1+z_2 |<1.$\\
De même montrer que  $1< |z_1z_2|< 2.$
\item Rappeler l'inégalité triangulaire et donner une minoration de $|x-y|$ pour tout $x,y\in \bC$. 

\item En déduire que $$|z_1+z_2| >|z_1| -\frac{2}{|z_1|}$$

\item Grâce à un raisonnement par l'absurde montrer que $|z_1|<2$.



\item Conclure que toutes les solutions de $(E)$ sont de modules strictement inférieures à $2$. 

%\paragraph{Partie 3 : Cas général}
%\begin{enumerate}
%\item Soit $n$ un entier $n\geq 2$. Etudier les variations, le signe et les limites de la fonction $\phi :\R \mapsto \R$ définie par 
%$$\phi(t) = t^n -t-1.$$
%
%\item Montrer que pour $z\in \bC$, on a l'implication, pour tout entier $n\geq 2$ :
%$$\left( z^n +z+1=0\right) \Longrightarrow \left( |z|<2\right).$$
%
%\item Est ce que la réciproque est vraie ? (a justifier évidemment... ) 
%\end{enumerate}


\end{enumerate}

\end{exercice}


\begin{correction}
\begin{enumerate}
\item  Comme $f(-1)=-1<0$ et $f(\frac{-1}{2}) = \frac{3}{8}>0$, le théoréme des valeurs intermédiaires assure qu'il existe une solution a 
$f(t)=0$ dans l'intervalle $]-1, \frac{-1}{2}[$. De plus 
$f'(t)=2t^2+1$ donc $f'>0$ pour tout $t\in \R$, donc $f$ est strictement croissante et cette racine est unique. 

\item En développant on obtient 
$$P(X) =X^3  +(-r-z_1-z_2) X^2+\alpha X -z_1z_2r$$
On n'est pas obligé de calculer $\alpha$. 
Par identification on obtient : 
$$-r-z_1-z_2=0\quad \text{et} \quad z_1z_2r=-1$$
$$z_1+z_2=-r\quad \text{et} \quad z_1z_2=\frac{-1}{r}$$
($r\neq 0$)

\item On a $\frac{1}{2} < -r < 1$ et $|z_1+z_2 | =|-r|=-r$. D'où 
$$\frac{1}{2} < |z_1+z_2 | < 1.$$

On a $1 < \frac{-1}{r} < 2$ et $|z_1z_2 | =\left|\frac{-1}{r}\right|=  \frac{-1}{r} $. D'où 
$$1 < |z_1z_2 | < 2.$$

\item L'inégalité triangulaire 'inversée' donne 
$$|x-y| \geq |x|-|y|.$$

\item On a donc 
$$|z_1+z_2| \geq |z_1|-|z_2|$$
Or $|z_1z_2| <2$, donc $|z_2| <\frac{2}{|z_1|}$
D'où $-|z_2| >-\frac{2}{|z_1|}$. On obtient donc l'inégalité voulue. 

\item Supposons par l'absurde que $|z_1|\geq 2$. On a alors d'après la questions précédente 
$$|z_1-z_2| > 2 -1 =1$$
Ceci est en contradiction avec le résultat de la question $3$. Donc 
$$|z_1|\leq 2.$$
\item Le raisonement de la question 5 et 6 s'applique de façon similaire à $z_2$. Comme $|r|\leq 1$, toutes les racines de $P$ sont bien de module strictement inférieur à $2$. 

\end{enumerate}
\end{correction}