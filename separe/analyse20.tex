\subsection{ Suite définites implicitement $x^3+nx-1$ (Pb) }


\begin{exercice}
\begin{enumerate}
\item Montrer que pour tout $n\in \N^*$ l'équation $x^3+nx=1$ admet une unique solution dans $\R^+$. On la note $x_n$. 
\item Montrer que $x_{n+1}^3+n x_{n+1}-1<0$.
\item En déduite que la suite $\suite{x} $ est décroissante. 
\item Justifier que la suite est minorée par $0$ et majorée par $1$. 
\item En déduire que $\suite{x}$ converge. 
\item A l'aide d'un raisonement par l'absurde justifier que cette limite vaut $0$. 
\end{enumerate}
\end{exercice}

\begin{correction}
\begin{enumerate}
\item  Soit $n\in \N$. Pour tout $x\in \R$, on note $f_n(x) = x^3+nx-1$. C'est un polynome de degré 3, il est dérivable sur $\R$ et on a 
$$f'(x) = 3x^2 +n $$
Comme $n\geq 0$, la dérivée est strictement positive sur $\R$ et ainsi la fonction $f_n$ est strictement croissante. 

On a par ailleurs $f_n(0) = -1$ et $f_n(1) =n\geq  1$. Comme $f_n$ est continue sur $[0,1]$ et strictement croissante on peut appliquer le théorème de la bijection pour la valeur $0\in [f_n(0), f_n(1)]=[-1, 1]$. Ce théorème assure qu'il existe un unique réel $x_n\in [0,1]$ tel que $f_n(x_n) = 0$. 


\item On calcule $f_n(x_{n+1}) = x_{n+1}^3 +n x_{n+1}-1$, on va montrer que $f_n(x_{n+1})<0$. Or par définition de $x_{n+1}$ on a $f_{n+1} (x_{n+1})=0$ ce qui donne:
$$ x_{n+1}^3 +(n+1) x_{n+1}-1=0$$
Donc $x_{n+1}^3 +n x_{n+1}-1=-x_{n+1}$

Finalement en remplaçant dans la première égalité on obtient : 
$$f_n(x_{n+1}) =-x_{n+1}$$
Comme pour tout $n\in \N$, $x_n\geq 0$ d'après la première question,on a bien 
$$f_n(x_{n+1}) <0$$

\item Comme pour tout $n\in \N$,  $f_n $ est strictement croissante, 
et $f_n(x_{n+1}) \leq f_n(x_n)$ on a 
$$x_{n+1} \leq x_n$$
Ainsi, $\suite{x}$ est décroissante. 

\item Le raisonement effectué à la question 1 montre que la suite $\suite{x}$ est minorée par 0 et majorée par 1. 

\item La suite $\suite{x}$ est décroissante et minorée. Le théorème de la limite monotone assure que $\suite{x}$ converge. Notons $\ell\in \R$ cette limite. 

\item Comme $x_n\geq 0$ pour tout $n\in \N$, on a $\lim x_n \geq 0. $. Supposons par l'absurde que $\ell>0$. 

On a alors d'une par $f_n(x_n) =0$ donc $\lim x_n^3 +nx_n-1=0$. Par ailleurs,  $\lim x_n^3 -1=\ell^3 -1$ et $\lim nx_n =+\infty$. Donc 
$\lim x_n^3+nx_n -1=+\infty$. 
Comme $0\neq +\infty$ et que la limite est unique, c'est une contradiction. Ainsi $\ell=0$. 




\end{enumerate}


\end{correction}