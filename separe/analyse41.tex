\subsection{EDL - concentration de glucose}
\begin{exercice}
En l'abscence d'apport énergétique la concentration en glucose dissout dans le sang dans le temps mesurée en heure  $t\mapsto c(t)$   (en $g\cdot L^{-1}$) vérifie l'équation différentielle $$y'+0.01y= -0.02$$
La concentration en glucose après un repas est égale à $c_0=1,2gL^{-1}$.

Donner les solutions de l'équation différentielle $y'+0.01y= -0.02$.

Donner l'expression de la concentration en glucose $c(t)$ en utilisant la condition initiale $c(0) =1.2$

Au bout de combien de temps après un repas la concentration en glucose dans le sang sera inférieure à $0,8gL^{-1}$ ? 

Exprimer le résulat avec un calcul litéral, puis en donner une valeur approchée (on pourra utiliser que $\ln(7/8)  \approx -0.13$)

\end{exercice}

\begin{correction}
\begin{enumerate}
\item Les solutions de l'équation différentielle homogène sont $\{ f : t\mapsto \lambda e^{-0.01t} |\lambda\in \R\}$
Une solution particulière de l'équaiton est $c_0(t) = -2$ 

Les solutions de l'équation différentielle sont donc $S=\{ f : t\mapsto \lambda e^{-0.01t} - 2\, | \lambda\in \R\}$

Après le repas, la concentration est de $1.2$ on pose donc $c(0)=1.2$ et $c(t) =  \lambda e^{-0.01t} - 2$. Ce qui donne $\lambda -2 = 1.2 $ d'où $\lambda=3.2$. 


On cherche à résoudre $3.2e^{-0.01t} -2 \leq 0.8$. On obtient 
$e^{-0.01t} \leq \frac{2.8}{3.2}$.

On a donc $t\geq \frac{-1}{0.01}\ln(\frac{7}{8})$
La valeur approchée de $\frac{-1}{0.01}\ln(\frac{7}{8})\sim 13$. 

La concentration en glucose sera inférieure à $0.8gL^{-1}$ au bout de $13h$ environ. 



\end{enumerate}
\end{correction}