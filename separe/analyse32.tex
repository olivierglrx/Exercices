\subsection{Fibonacci}

\begin{exercice}
%\paragraph{Exercice 1 : Suite de Fibonacci}
Soit $\suite{F}$ la suite définie par $F_0 =0, \, F_1=1 $ \text{ et pour tout $n \geq 0 $ \,} 
$$ F_{n+2} = F_{n+1} +F_n.$$

\begin{enumerate}
\item Montrer que pour tout $n\in \N$ on a : $\ddp \sum_{k=0}^n F_{2k+1} =F_{2n+2}$
et $\ddp \sum_{k=0}^n F_{2k} =F_{2n+1}-1$.
\item Montrer que pout tout $n\in \N$ on a $\ddp \sum_{k=0}^n F_{k}^2 =F_nF_{n+1}$.
\item \begin{enumerate}
\item On note $\varphi = \frac{1+\sqrt{5}}{2}$ et $\psi=\frac{1-\sqrt{5}}{2}$. Montrer que 
$\varphi^2 =\varphi+1$ et $\psi^2 =\psi+1$.
\item Montrer que l'expression explicite de $F_n$ st donnée par $F_n =\frac{1}{\sqrt{5}}(\varphi^n-\psi^n)$.
\item En déduire que $\ddp \lim_{n\tv \infty} \frac{F_{n+1}}{F_n}=\varphi.$
\end{enumerate}
\end{enumerate}

\end{exercice}


\begin{correction}
\begin{enumerate}
\item Nous allons montrer ces propriétés par récurrence sur l'entier $n\in \N$. 
Soit $\cP(n)$ la prorpriété définie pour tout $n\in\N$ par:
$$\cP(n) := \text{ \og }\ddp  \sum_{k=0}^n F_{2k+1} =F_{2n+2} \text{ 
et } \ddp \sum_{k=0}^n F_{2k} =F_{2n+1}-1 \text{ \fg }.$$
Montrons  $\cP(0)$. Vérifions la première égalité : 
$$\sum_{k=0}^0 F_{2k+1} =F_{0+1}=F_1=1$$
et 
$$F_2 =F_1+F_0 =1$$
Donc la première égalité est vraie au rang $0$. 

Vérifions la sedonde égalité : 
$$\sum_{k=0}^0 F_{2k} =F_{0}=0$$
et 
$$F_{2*0+1}-1 =F_1-1=0$$
Donc la seconde égalité est vraie au rang $0$. 
Ainsi $\cP(0)$ est vraie. 


 \textbf{H\'er\'edit\'e:}\\
Soit $n\geq 0$ fix\'e. On suppose la propri\'et\'e vraie \`a l'ordre $n$. Montrons qu'alors $\mathcal{P}(n+1)$ est vraie.\\

Considérons la première égalité de $\cP(n+1)$. Son membre de gauche vaut : 
\begin{equation*}
 \sum_{k=0}^{n+1} F_{2k+1}=  \sum_{k=0}^n F_{2k+1} +F_{2n+3}
\end{equation*}
Par hypothèse de récurrence on a $ \ddp \sum_{k=0}^n F_{2k+1} = F_{2n+2}$, donc 
\begin{align*}
 \sum_{k=0}^{n+1} F_{2k+1} & =F_{2n+2}+F_{2n+3}.\\
								& = F_{2n+4}.\quad \text{ d'après la définition de $\suite{F}$}\\
								& = F_{2(n+1)+2}.\\
\end{align*}
La première égalité est donc héréditaire. 


Considérons la sedonde égalité de $\cP(n+1)$. Son membre de gauche vaut : 
\begin{equation*}
 \sum_{k=0}^{n+1} F_{2k}=  \sum_{k=0}^n F_{2k} +F_{2n+2}
\end{equation*}
Par hypothèse de récurrence on a $ \ddp \sum_{k=0}^n F_{2k} = F_{2n+1}-1$, donc 
\begin{align*}
 \sum_{k=0}^{n+1} F_{2k} & =F_{2n+1}-1+F_{2n+2}.\\
								& = F_{2n+3}-1.\quad \text{ d'après la définition de $\suite{F}$}\\
								& = F_{2(n+1)+1}-1.\\
\end{align*}
La seconde égalité est donc héréditaire. Finalement la propriété $\cP(n+1)$ est vraie. 


\textbf{Conclusion:}\\
Il r\'esulte du principe de r\'ecurrence que pour tout $ n\geq 0$:
\begin{center}
\fbox{$\ddp \sum_{k=0}^n F_{2k+1} =F_{2n+2} \text{ 
et } \ddp \sum_{k=0}^n F_{2k} =F_{2n+1}-1 $}
\end{center}


\item On  va montrer par récurrence que $\mathcal{P}(n) :\ddp \sum_{k=0}^n F_{k}^2 =F_{n}F_{n+1}$. 



\textbf{Initialisation:}  Pour $n=0$, on a $\sum_{k=0}^0 F_{k}^2 = F_0^2=0$ et $F_0F_1=0$. 
La propriété est donc vraie au rang $0$. 
 
 \textbf{H\'er\'edit\'e:}\\
Soit $n\geq 0$ fix\'e. On suppose la propri\'et\'e vraie \`a l'ordre $n$. 

On a $\ddp \sum_{k=0}^{n+1} F_{k}^2 =\sum_{k=0}^{n} F_{k}^2 +F_{n+1}^2$
Par hypothèse de récurrence on a $\sum_{k=0}^{n} F_{k}^2 = F_n F_{n+1}$ donc : 
\begin{align*}
 \sum_{k=0}^{n+1} F_{k}^2 &=  F_n F_{n+1} +F_{n+1}^2\\
				&= F_{n+1} (F_n +F_{n+1}) \\
				&=   F_{n+1}F_{n+2}  \quad \text{ par définition de $\suite{F}$ }													
\end{align*}

La propriété $\cP$ est donc vraie au rang $n+1$.

\textbf{Conclusion:}\\
Il r\'esulte du principe de r\'ecurrence que pour tout $ n\geq 0$:
\begin{center}
\fbox{$\mathcal{P}(n): \ddp \sum_{k=0}^n F_{k}^2 =F_{n}F_{n+1}$}
\end{center}

\item Le polynôme du second degrès $X^2-X-1$ a pour discriminant $\Delta =1+4=5$ les racines sont donc 
$\varphi = \frac{1+\sqrt{5}}{2}$ et $\psi=\frac{1-\sqrt{5}}{2}$. 
En particulier, ces nombres vérifient : $\varphi^2 -\varphi -1 =0$ et $\psi^2 -\psi-1=0$, c'est-à-dire 

\begin{center}
\fbox{$\varphi^2 =\varphi+1$ et $\psi^2 =\psi+1$.}
\end{center}





\item  Notons  :$u_n =\frac{1}{\sqrt{5}}(\varphi^n-\psi^n)$ 
On a 
$$u_0= \frac{1}{\sqrt{5}}(\varphi^0-\psi^0)=0$$
$$u_1= \frac{1}{\sqrt{5}}(\varphi^1-\psi^1)=1$$
et pour tout $n\in\N$ on a 
\begin{align*}
u_{n+2} &= \frac{1}{\sqrt{5}}(\varphi^{n+2}-\psi^{n+2}) \\
			&= \frac{1}{\sqrt{5}}(\varphi^n (\varphi^2)-\psi^n (\psi^2) ) \\
			&= \frac{1}{\sqrt{5}}(\varphi^n (\varphi +1)-\psi^n (\psi +1)  ) \quad \text{ D'après la question précédente} \\			
			&= \frac{1}{\sqrt{5}}(\varphi^{n+1} +\varphi^n-\psi^{n+1} -\psi^n   ) \\
			&= \frac{1}{\sqrt{5}}(\varphi^{n+1} -\psi^{n+1}) +  \frac{1}{\sqrt{5}} \varphi^n-\psi^n   ) \\
			&=u_{n+1}+u_n
\end{align*}
Donc $u_n$ satisfait aussi la relation de récrurrence. 
Ainsi  pour tout $n\in \N$, $u_n=F_n= \frac{1}{\sqrt{5}}(\varphi^n-\psi^n)$. 


\item D'après la question précédente on a pour tout $n\in \N$: 
$$\frac{F_{n+1}}{F_n} =  \frac{\varphi^{n+1}-\psi^{n+1}}{\varphi^n-\psi^n}$$
Donc,
\begin{align*}
\frac{F_{n+1}}{F_n} &=\varphi \frac{\varphi^{n}\left(1-\frac{\psi^{n+1}}{\varphi^{n+1}}\right)}{\varphi^n\left(1-\frac{\psi^n}{\varphi^n}\right)}\\
&=\varphi \frac{1-\left(\frac{\psi}{\varphi}\right)^{n+1}}{1-\left(\frac{\psi}{\varphi}\right)^n}
\end{align*}


Remarquons que $|\varphi| >|\psi|$ en particulier $|\frac{\psi}{\varphi}|<1$ et donc 
$$\lim_{n\tv \infty} \left(\frac{\psi}{\varphi}\right)^{n+1} =0.$$
Finalemetn 
\begin{center}
\fbox{$ \lim_{n\tv \infty} \frac{F_{n+1}}{F_n} =\varphi$.} 
\end{center}

\end{enumerate}

\end{correction}