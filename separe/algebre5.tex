\subsection{Calcul du rang de $M_a$}

\begin{exercice}
Soit $a\in \R$ et  $M_a \in\cM_3(\R) $, définie par 
$$ M_a =\left( 
\begin{array}{ccc}
1 & 0 & a\\
0 & -a & 1\\
a & 1 & 0
\end{array}
\right)$$
\begin{enumerate}
\item  Calculer le rang de $M_a$ en fonction de $a$. 
\item Déterminer les valeurs de $a$ pour lesquelles $M_a$ est inversible et calculer l'inverse dans ces cas. 
\end{enumerate}


\end{exercice}
\begin{correction}
Appliquant l'algorithme du pivot de gauss à la matrice $M_a$. On est amené à aussi chercher l'inverse de $M_a$ donc on se contentera de faire que des opérations sur les lignes et on gardera trace des opérations sur la matrice $\Id$ afin d'obtenir l'inverse. 
$$
\begin{array}{ccc}
 M_a & : &\left( 
\begin{array}{ccc|ccc}
1 & 0 & a &1 & 0 & 0\\
0 & -a & 1&  0& 1 & 0\\
a & 1 & 0&0 & 0 & 1
\end{array}
\right)\vsec \\
 &\stackrel{L_3\leftarrow L_3-aL_1}{\sim}&
\left( 
\begin{array}{ccc|ccc}
1 & 0 & a &1 & 0 & 0\\
0 & -a & 1&  0& 1 & 0\\
0 & 1 & -a^2&-a & 0 & 1
\end{array}
\right)\vsec \\
& \stackrel{L_2\longleftrightarrow L_3}{\sim}&
\left( 
\begin{array}{ccc|ccc}
1 & 0 & a &1 & 0 & 0\\
0 & 1 & -a^2&  -a& 0 & 1\\
0 & -a & 1&0 & 1 & 0
\end{array}
\right) \vsec \\
& \stackrel{L_3\leftarrow L_3+aL_2}{\sim}&
\left( 
\begin{array}{ccc|ccc}
1 & 0 & a &1 & 0 & 0\\
0 & 1 & -a^2&  -a& 0 & 1\\
0 & 0 & 1-a^3&-a^2 & 1 & +a
\end{array}
\right)\\
\end{array}
$$
On voit dès à présent que la matrice est de rang $3$ si et seulementsi $1-a^3\neq 0$ c'est-à-dire si $a^\neq 1$. Si $a=1$ elle est de rang  $2$. 

On finit avec le calcul de l'inverse dans le cas où $1-a^3 \neq 1$: 

$$
\begin{array}{ccc}
 M_a 
& \sim &
\left( 
\begin{array}{ccc|ccc}
1 & 0 & a &1 & 0 & 0\\
0 & 1 & -a^2&  -a& 0 & 1\\
0 & 0 & 1-a^3&-a^2 & 1 & a
\end{array}
\right) \vsec \\
& \stackrel{L_3\leftarrow \frac{L_3}{1-a^3} }{\sim} &
\left( 
\begin{array}{ccc|ccc}
1 & 0 & a &1 & 0 & 0\\
0 & 1 & -a^2&  -a& 0 & 1\\
0 & 0 & 1 &\frac{-a^2}{1-a^3} & \frac{1}{1-a^3}  & \frac{a}{1-a^3} 
\end{array}
\right) \vsec \\

& \stackrel{L_2\leftarrow L_2+a^2L_3 }{\stackrel{L_1\leftarrow L_1-aL_3 }{\sim} } &
\left( 
\begin{array}{ccc|ccc}
1 & 0 & 0 &  \frac{1}{1-a^3} &  \frac{-a}{1-a^3} &  \frac{-a^2}{1-a^3}\\
0 & 1 & 0&   \frac{-a}{1-a^3}&  \frac{a^2}{1-a^3} &  \frac{1}{1-a^3}\\
0 & 0 & 1 &\frac{-a^2}{1-a^3} & \frac{1}{1-a^3}  & \frac{a}{1-a^3} 
\end{array}
\right) \vsec \\


\end{array}
$$

Pour tout $a$ différent de $1$, l'invese de $M_a$ est donc donné par  
$$M_a^{-1} = \left( 
\begin{array}{ccc}
\frac{1}{1-a^3} &  \frac{-a}{1-a^3} &  \frac{-a^2}{1-a^3}\\
  \frac{-a}{1-a^3}&  \frac{a^2}{1-a^3} &  \frac{1}{1-a^3}\\
\frac{-a^2}{1-a^3} & \frac{1}{1-a^3}  & \frac{a}{1-a^3} 
\end{array}
\right) =  \frac{1}{1-a^3}\left( 
\begin{array}{ccc}
1 &  -a &  -a^2\\
  -a&  a^2&  1\\
-a^2& 1& a
\end{array}
\right) $$



\end{correction}