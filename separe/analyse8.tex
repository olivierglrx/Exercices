\subsection{Equation différentielle, changement de variable}
\begin{exercice}
Le but de cet exercice est de déterminer l'ensemble $\cS$ des fonctions
$f :]0,+\infty[\tv \R$ telles que : 
$$ \text{$f$ est dérivable sur $]0,+\infty[$ et } \forall t>0, \, f'(t) = f(1/t)$$

On fixe une fonction $f\in \cS$ et on définit la fonction $g$ par 
$$g(x) =f(e^{x})$$
\begin{enumerate}
\item Justifier que $f$ est deux fois dérivable sur $]0,+\infty[$ et exprimer sa dérivée seconde en fonction de $f$. 
\item Justifier que $g$ est deux fois dérivable sur $\R$ et montrer que $g$ est solution de l'équation diffrentielle suivante : 
$$y''-y'+y=0\quad(E)$$
\item Résoudre $(E)$. 
\item En déduire que $f$ est de la forme $$f(t) = A \sqrt{t}  \cos\left(\frac{\sqrt{3}}{2}\ln(t)\right) +B\sqrt{t}  \sin\left(\frac{\sqrt{3}}{2}\ln(t)\right)$$ où $(A,B)$ sont deux constantes réelles.

On appelle $f_1(t) =  \sqrt{t}  \cos\left(\frac{\sqrt{3}}{2}\ln(t)\right)$ et 
$f_2(t)= \sqrt{t}  \sin\left(\frac{\sqrt{3}}{2}\ln(t)\right)$
\item Calculer les dérivées premières de $f_1$ et $f_2$
\item En considérant les cas $t=1$ et $t=e^{\pi/\sqrt{3}}$, montrer que $A$ et $B$ sont solutions de
$$(S) \left\{ \begin{array}{ccc}
A-B\sqrt{3}&=&0\\
A\sqrt{3}-3B&=&0
\end{array}\right.$$
\item Résoudre $(S)$. 
\item Conclure. 
\end{enumerate} 
\end{exercice}

\begin{correction}
\begin{enumerate}
\item Remarquons qu'étant donné que $f$ est dérivable et $t\mapsto \frac{1}{t}$ est aussi dérivable, la fonction $f'$ est dérivable par composée de fonctions dérivables. Ainsi $f$ est dérivable deux fois sur $]0,+\infty[$ et on a \conclusion{$f''(t) = \frac{-1}{t^2}f'(1/t)  = \frac{-1}{t^2}f(t)$}

\item La fonction $x\mapsto e^x$ est dérivable deux fois sur $\R$ et $exp(\R) = ]0,+\infty[$, de nouveau par composition, $g$ est dérivable deux fois sur $\R$. 

Calculons les dérivées successives de $g$ en fonction de celles de $f$ :
$$g'(x) = e^x f'(e^x) \quadet g''(x) = e^xf'(e^x) +e^{2x} f''(e^x)$$
On  a donc 
\begin{align*}
g''(x)-g'(x)+g(x)  &= e^xf'(e^x) +e^{2x} f''(e^x)-e^x f'(e^x)+f(e^{x})\\
						&= e^{2x} f''(e^x)+f(e^{x})
\end{align*}
On utilise alors la relation vérifiée par $f$ : $f'(x) = f(1/x)$, on a par dérivation $f''(x) = \frac{-1}{x^2}f'(1/x)  = \frac{-1}{x^2}f(x)$, d'où 
$$f''(e^x) = \frac{-1}{e^{2x}}f(e^{x})=-e^{-2x} f(e^x)$$
Donc pour tout $x\in \R$ :
\begin{align*}
g''(x)-g'(x)+g(x)  &=  -e^{2x}e^{-2x} f(e^x) +f(e^x)\\
							&=0
\end{align*}
\conclusion{La fonction $g$ est donc solution de l'équation différentielle $g''-g'+g=0$.}

\item Résolvons $(E)$ avec la méthode vue en cours. Le polynôme caractéristique est 
$X^2-X+1$ qui admet comme discriminant $\Delta = 1-4 =-3<0$ et donc deux racines complexes : $r_1=\frac{1-i\sqrt{3}}{2}$ et $r_2=\frac{1+i\sqrt{3}}{2}$. 
Les solutions de $(E)$ sont donc de la forme 
\conclusion{
$\cS = \left\{ x\mapsto  e^{x/2} \left(A\cos\left(\frac{\sqrt{3}}{2}x\right)+B\sin\left(\frac{\sqrt{3}}{2}x\right)\right) \, | \, A, B\in \R \right\}$}

\item On vient de voir que $f(e^{x})$ est de la forme $e^{x/2} (A\cos\left(\frac{\sqrt{3}}{2}x\right)+B\sin\left(\frac{\sqrt{3}}{2}x\right)) $, donc 
$f(t) $ est de la forme 
$$f(t) = A \sqrt{t}  \cos\left(\frac{\sqrt{3}}{2}\ln(t)\right) +B\sqrt{t}  \sin\left(\frac{\sqrt{3}}{2}\ln(t)\right)$$
avec $A,B$ deux constantes réelles. Ceci est bien la forme demandée par l'énoncé, avec 
$$f_1(t) = \sqrt{t}  \cos\left(\frac{\sqrt{3}}{2}\ln(t)\right)$$ et 
$$f_2(t) = \sqrt{t}  \sin\left(\frac{\sqrt{3}}{2}\ln(t)\right)$$

\item Calculons les dérivées des fonctions $f_1$ et $f_2$. 
On a 
\begin{align*}
f_1'(t) &= \frac{1}{2\sqrt{t}}  \cos\left(\frac{\sqrt{3}}{2}\ln(t)\right)  - \sqrt{t} \frac{\sqrt{3}}{2t} \sin\left(\frac{\sqrt{3}}{2}\ln(t)\right)\\
&= \frac{1}{2\sqrt{t}}  \cos\left(\frac{\sqrt{3}}{2}\ln(t)\right)  - \frac{\sqrt{3}}{2\sqrt{t} } \sin\left(\frac{\sqrt{3}}{2}\ln(t)\right)
\end{align*}

De même 
$$f'_2(t)= \frac{1}{2\sqrt{t}}  \sin\left(\frac{\sqrt{3}}{2}\ln(t)\right)  + \frac{\sqrt{3}}{2\sqrt{t} } \cos\left(\frac{\sqrt{3}}{2}\ln(t)\right)$$

\item Pour $t= 1$ on obtient d'une part 
\begin{align*}
f(1)& = A \sqrt{1}  \cos\left(\frac{\sqrt{3}}{2}\ln(1)\right) +B\sqrt{1}  \sin\left(\frac{\sqrt{3}}{2}\ln(1)\right)\\
&=A
\end{align*}
et  d'autre part : 
\begin{align*}
f'(1) &= Af_1'(1) +  Bf_2'(1)\\
		&= \frac{A}{2} + \frac{B\sqrt{3}}{2}
\end{align*}
Comme $f'(1) =f(1/1)=f(1)$,
on obtient alors 
$$A= \frac{A+B\sqrt{3}}{2}$$
donc 
$2A=A+B\sqrt{3} $ et finalement 
\conclusion{$A-B\sqrt{3}=0$}
C'est la première équation du système $(S)$


Faisons la même chose pour $t=e^{\pi/\sqrt{3}}$. Remarquons tout d'abord que 
$$f'(e^{\pi/\sqrt{3}}) = f(1/e^{\pi/\sqrt{3}}) = f(e^{-\pi/\sqrt{3}})$$

Calculons alors les deux membres de cette égalité. 
\begin{align*}
f(e^{-pi/\sqrt{3}}) &=Ae^{-\pi/2\sqrt{3}}  \cos( -\frac{\pi}{2})+Be^{-\pi/2\sqrt{3}}  \sin( -\frac{\pi}{2})\\
&=-B e^{-\pi/2\sqrt{3}} 
\end{align*}

et
\begin{align*}
 f'_1(e^{\pi/\sqrt{3}})&= -\frac{\sqrt{3}}{2}e^{-\pi/2\sqrt{3}}
\end{align*}
\begin{align*}
 f'_2(e^{\pi/\sqrt{3}})&= \frac{1}{2}e^{-\pi/2\sqrt{3}}
\end{align*}
d'où 
$$f'(e^{\pi/\sqrt{3}}) =  -\frac{A\sqrt{3}}{2}e^{-\pi/2\sqrt{3}}+ \frac{B}{2}e^{-\pi/2\sqrt{3}}$$

Finalement on obtient 
$$B e^{-\pi/2\sqrt{3}}  =  -\frac{A\sqrt{3}}{2}e^{-\pi/2\sqrt{3}}+ \frac{B}{2}e^{-\pi/2\sqrt{3}}$$
Donc 
$$-B =  -\frac{A\sqrt{3}}{2} +\frac{B}{2}$$
Ce qui donne alors 
$-2B = -A\sqrt{3} +B$ et finalement 
\conclusion{$-3B+A\sqrt{3}=0$}
C'est la deuxième équation du système $(S)$

\item 
Le système $(S)$ est équivalent à 
$$\left\{ \begin{array}{ccc}
A-B\sqrt{3}&=&0\\
\sqrt{3}(A-\sqrt{3}B)&=&0
\end{array}\right. \equivaut A-B\sqrt{3}=0$$

Le système admet alors une infinité de solutions de la forme 
\conclusion{$\cS = \{ (B\sqrt{3}, B) \, |\, B\in R\}$}

\item On en déduit que $f$ est de la forme 
$$f (t) = B\sqrt{3} f_1(t) +Bf_2(t)$$
où 
$B$ est une constante réelle. 

Il faut maintenant vérifier que les fonctions de cette forme sont bien solutions de notre problème. 

$f$ est bien définie et dérivable sur $]0, +\infty[$
et $$f'(t) = \frac{B\sqrt{3}}{\sqrt{t}} \cos\left(\frac{\sqrt{3}}{2}\ln(t)\right)  - \frac{B}{\sqrt{t}} \sin\left(\frac{\sqrt{3}}{2}\ln(t)\right) $$

D'autre part $f(1/t) = B\sqrt{3}f_1(1/t) +Bf_2(1/t)$

Et  on a 
\begin{align*}
f_1(1/t) &=\frac{1}{\sqrt{t}} \cos\left(\frac{\sqrt{3}}{2}\ln(1/t)\right)  \\
			 &=\frac{1}{\sqrt{t}} \cos\left(-\frac{\sqrt{3}}{2}\ln(t)\right)  \\
			 			 &=\frac{1}{\sqrt{t}} \cos\left(\frac{\sqrt{3}}{2}\ln(t)\right)  
\end{align*}

De même on obtient 
$$f_2(1/t) = -\frac{1}{\sqrt{t}} \sin\left(\frac{\sqrt{3}}{2}\ln(t)\right)  $$
par imparité de la fonction $\sin$

Ainsi $$f(1/t) =  B\frac{\sqrt{3}}{\sqrt{t}} \cos\left(\frac{\sqrt{3}}{2}\ln(t)\right)  - B\frac{1}{\sqrt{t}} \sin\left(\frac{\sqrt{3}}{2}\ln(t)\right) =f'(t)$$
 


\end{enumerate}
\end{correction}