\subsection{Diagonalisation}





\begin{exercice}
Soit $A$ la matrice suivante : 
$\left(\begin{array}{ccc}
3&0&-1\\
2&4&2\\
-1&0&3
\end{array}
\right).$
\begin{enumerate}
\item Déterminer les réels $\lambda\in \R$ pour lesquels la matrice $A-\lambda \Id$ n'est pas  inversible. On appelle ces réels les \emph{valeurs propres} de $A$. 
\item Soit $\lambda\in \R$, Montrer que l'espace 
$E_\lambda =\{ X\in \cM_{3,1}(\R) \, |\, AX=\lambda X\}$ est un sev de $\cM_{3,1}(\R)$.
\item Pour chaque valeur propre de A  (les réels obtenus à la question 1), déterminez une base de $E_\lambda$. Vérifier qu'on obtien au total 3 vecteurs que vous noterez $u_1,u_2,u_3$. 
\item La famille $(u_1,u_2,u_3)$ est elle une base de $\cM_{3,1}(\R)$ ? 
\item \begin{enumerate}
\item Soit $P$ la matrice de $\cM_3(\R) $ dont la première colonne est constitutée des coordonnées de $u_1$, la seconde des coordonnées de $u_2$ et  la dernière des coordonnées de $u_3$. Déterminez explicitement $P$. 
\item Montrer que $P$ est inversible et calculer son inverse. 
\item Déterminez mla matrice $D=P^{-1} AP$. 
\item Déterminez une expression de la matrice $A^n$ pour tout entier naturel $n$. 
\end{enumerate}
\end{enumerate}
\end{exercice}

\begin{correction}
\begin{enumerate}
\item $A-\lambda \Id= \left(\begin{array}{ccc}
3-\lambda&0&-1\\
2&4-\lambda&2\\
-1&0&3-\lambda
\end{array}
\right).$
On va appliquer l'algorithme du pivot de Gauss : 
$\left(\begin{array}{ccc}
3-\lambda&0&-1\\
2&4-\lambda&2\\
-1&0&3-\lambda
\end{array}
\right) \sim_{L_1\longleftrightarrow L_3} \left(\begin{array}{ccc}
-1&0&3-\lambda\\
2&4-\lambda&2\\
3-\lambda&0&-1
\end{array}
\right) $\\
$\sim_{L_2\longleftarrow L_2+2L_1\\ L_3\longleftarrow L_3+(3-\lambda) L_1} \left(\begin{array}{ccc}
-1&0&3-\lambda\\
0&4-\lambda&8-2\lambda \\
0&0&8-6\lambda +\lambda^2
\end{array}
\right)  =\left(\begin{array}{ccc}
-1&0&3-\lambda\\
0&4-\lambda&8-2\lambda \\
0&0&(\lambda -2) (\lambda-4)
\end{array}
\right) $
Ainsi $A-\lambda\Id$ n'est pas inversible si et seulement si $\lambda \in \{2,4\}$

\item Remarquons tout d'abord que le vecteur nul appartient à $E_\lambda$ qui est donc non vide. Soit $X, Y\in E_\lambda$ et $\mu\in \R$, montrons que $X+\mu Y$ appartient à $E_\lambda$ :
$$A (X+\mu Y) = AX +\mu A Y$$ 
Comme $X\in E_\lambda$ $AX =\lambda X$ et de même $AY =\lambda Y$. 
On a donc 
$$A(X+\mu Y) =\lambda X +\mu \lambda Y = \lambda (X+\mu Y)$$
Ainsi, $X+\mu Y \in E_\lambda$. L'ensemble $E_\lambda$ est donc stable par combinaisons linéraires, c'est bien un sev de $\cM_{3,1} (\R)$. 
\item \begin{itemize}
\item \underline{$\lambda =2$ }.  Soit $\left(\begin{array}{c}
x\\
y\\
z
\end{array} \right)\in E_2$. 
On a donc 
$$AX = \left(\begin{array}{c}
3x-z\\
2x+4y+2z\\
-x+3z
\end{array} \right) = \left(\begin{array}{c}
2x\\
2y\\
2z
\end{array} \right)$$
On obtient 
\begin{align*}
\left\{ 
\begin{array}{lcc}
3x-z &=&2x\\
2x+4y+2z &=& 2y\\
-x+3z &=& 2z
\end{array} \right. &\equivaut \quad \left\{ 
\begin{array}{ccccc}
x& &-z &=&0\\
2x&+2y&+2z &=& 0\\
-x& &+z &=& 0
\end{array} \right. \\
&\equivaut \quad \left\{ 
\begin{array}{ccccc}
x& &-z &=&0\\
0&+2y&+4z &=& 0\\
0& &+0 &=& 0
\end{array} \right. \\
&\equivaut \quad \left\{ 
\begin{array}{cc}
x=&z\\
y =& -2z\\
\end{array} \right. 
\end{align*}
L'ensemble des solutions est donc $E_2= \{ (z,-2z,z) \, |\, z\in \R\} = \Vect( (1,-2,1))$ 
$u_1 =(1,-2,1)$  est une base de $E_2$


\item \underline{$\lambda =4$ }.  Soit $\left(\begin{array}{c}
x\\
y\\
z
\end{array} \right)\in E_4$. 
On a donc 
$$AX = \left(\begin{array}{c}
3x-z\\
2x+4y+2z\\
-x+3z
\end{array} \right) = \left(\begin{array}{c}
4x\\
4y\\
'z
\end{array} \right)$$
On obtient 
\begin{align*}
\left\{ 
\begin{array}{lcc}
3x-z &=&4x\\
2x+4y+2z &=& 4y\\
-x+3z &=& 4z
\end{array} \right. &\equivaut \quad \left\{ 
\begin{array}{ccccc}
-x& &-z &=&0\\
2x& &+2z &=& 0\\
-x& &-z &=& 0
\end{array} \right. \\
&\equivaut \quad \left\{ 
\begin{array}{ccccc}
x& &+z &=&0\\
2x& &+2z &=& 0\\
-x& &-z &=& 0
\end{array} \right. \\
&\equivaut \quad \left\{ 
\begin{array}{ccccc}
x& &+z &=&0\\
0& &0 &=& 0\\
0& &0 &=& 0
\end{array} \right.\\
&\equivaut \quad \left\{ 
\begin{array}{cc}
x=&-z\\
\end{array} \right. 
\end{align*}
L'ensemble des solutions est donc $E_2= \{ (-z,y,z) \, |\, (y,z)\in \R^2\} = \Vect( (-1,0,1), (0,1,0))$ 
On note $u_2 =(-1,0,1)$ et $u_3= (0,1,0)$. $(u_2, u_3) $ est une base de $E_4$. (Elle est génératrice par définition et elle est libre car les deux vecteurs ne sont pas proportionels) 
\end{itemize}
\item Vérifions que la famille $(u_1,u_2,u_3)$ est libre. Soit $(\mu_1, \mu_2, \mu_3) \in \R^3$ tel que $\mu_1u_1 +\mu_2 u_2 +\mu_3 u_3 =0$. 
En identifiant chaque coordonnées on obtient : 
\begin{align*}
\left\{ 
\begin{array}{ccccc}
\mu_1 &-\mu_2 &+0\mu_3 &=&0\\
-2 \mu_1& +0\mu_2 &+\mu_3 &=&0\\
\mu_1& +\mu_2 &+0\mu_3 &=&0\\
\end{array} \right. &\equivaut \quad \left\{ 
\begin{array}{ccccc}
\mu_1 &-\mu_2 &  &=&0\\
-2 \mu_1&  &+\mu_3 &=&0\\
\mu_1& +\mu_2 & &=&0\\
\end{array} \right.\\
&\equivaut \quad \left\{ 
\begin{array}{ccccc}
\mu_1 &-\mu_2 &  &=&0\\
0& -2\mu_2 &+\mu_3 &=&0\\
0& 2\mu_2 & &=&0\\
\end{array} \right. \\
&\equivaut \quad \left\{ 
\begin{array}{ccccc}
\mu_1 &-\mu_2 &  &=&0\\
0& -2\mu_2 &+\mu_3 &=&0\\
0& 0 & -\mu_3&=&0\\
\end{array} \right.\\
&\equivaut \quad \left\{ 
\begin{array}{c}
\mu_1=\mu_2 =\mu_3 =0\\
\end{array} \right. 
\end{align*}
Ainsi la famille est libre. Comme elle est de cardinal $3$ dans un ev de dimension $3$, c'est une base. 
\item
\begin{enumerate}
\item $P= \left(\begin{array}{ccc}
1&-1&0\\
-2&0&1\\
1&1&0
\end{array}
\right).$
\item Le rang de $P$ est égal au rang de la famille $(u_1, u_2,u_3) $, qui d'après la question précédente vaut $3$. Donc $rg(P)=3$, elle est donc inversible. 

L'algorithme du pivot permet de trouver son inverse, on trouve 
$$P^{-1} = \left(\begin{array}{ccc}
\frac{1}{2}&0&\frac{1}{2}\\
-\frac{1}{2}&0&\frac{1}{2}\\
1&1&1
\end{array}
\right).$$
\item Après calcul on obtient 
$$D=P^{-1} A P =\left(\begin{array}{ccc}
2&0&0\\
0&4&0\\
0&0&4
\end{array}
\right).$$
\item Par récurrence on montre $D^n = P^{-1} A^n P$ donc $A^n = P D^n P^{-1}$  et on  a par ailleurs $D^n =\left(\begin{array}{ccc}
2^n&0&0\\
0&4^n&0\\
0&0&4^n
\end{array}
\right), $ car $D$ est une matrice diagonale. 
Après calcul on obtient 
$$A^n  = \left(\begin{array}{ccc}
2^{n-1} +\frac{1}{2} 4^n&0&2^{n-1} -\frac{1}{2} 4^n\\
-2^{n} + 4^n&4^n&-2^{n} + 4^n\\
2^{n-1} -\frac{1}{2} 4^n&0&2^{n-1} +\frac{1}{2} 4^n
\end{array}
\right). $$  
\end{enumerate}
\end{enumerate}
\end{correction}