\subsection{Equation complexe}
\begin{exercice}
Résoudre dans $\bC$ l'équation d'inconnue $z$: 
$$\left(\frac{z-2i}{z+2i}\right)^3+\left(\frac{z-2i}{z+2i}\right)^2+\left(\frac{z-2i}{z+2i}\right)+1=0$$
\end{exercice}


\begin{correction}
On pose, $Z =\left(\frac{z-2i}{z+2i}\right)$, l'équation devient alors: 
$$Z^3+Z^2+Z+1=0.$$
On  remarque que $-1$ est une racine du polynôme,   $Z^3+Z^2+Z+1$, qui se factorise alors en 
$(Z+1)(Z^2+1)$. $Z^2+1 =(Z-i)(Z+i)$ et on  a donc 
$$Z^3+Z^2+Z+1 =(Z+1)(Z-i)(Z+i).$$

\begin{enumerate}
\item Pour $Z=-1 \Longleftrightarrow \left(\frac{z-2i}{z+2i}\right)=-1$, on obtient 
$z-2i =-z-2i$ soit $$z=0.$$
\item Pour $Z=i \Longleftrightarrow \left(\frac{z-2i}{z+2i}\right)=i$, on obtient 
$z-2i =iz-2$. Soit $z(1-i) = -2+2i$, donc 
$$z=-2$$


\item Pour $Z=-i \Longleftrightarrow \left(\frac{z-2i}{z+2i}\right)=-i$, on obtient 
$z-2i =-iz+2$ soit $z(1+i) =2+2i$ donc 
$$z=2$$



Les solutions de l'équation sont donc 
\conclusion{$\cS=\{ -2,0,  2 \}$}

\end{enumerate}



\end{correction}