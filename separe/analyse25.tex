\subsection{Calculs de limites}

\begin{exercice}
Calculer les limites suivantes 
\begin{enumerate}
\item $\lim_{x\tv 1} \frac{x-1}{\cos(\frac{\pi x}{2})}$
\item $\lim_{x\tv 0} \frac{x\ln(x)}{e^x-1}$
\item $\lim_{x\tv +\infty} \frac{\ln(x^2)}{\ln(x+1)}$
\item $\lim_{x\tv +\infty} \frac{\ln(x) e^{x^2}}{x^x}$
\end{enumerate}
\end{exercice}
\begin{correction}
\begin{enumerate}
\item  ( FI $\frac{0}{0}$ - pas nécessaire sur une copie) 
On fait un changement de variable : $y=x-1$.
$$\frac{x-1}{\cos(\frac{\pi x}{2})} = \frac{y}{\cos(\frac{\pi y}{2}+\frac{\pi}{2})}=-\frac{y}{\sin(\frac{\pi y}{2})}$$
Or $\sin(\frac{\pi y}{2})\sim_0 \frac{\pi y}{2}$. Donc 
$$\lim_{x\tv 1} \frac{x-1}{\cos(\frac{\pi x}{2})} = \lim_{y\tv 0} -\frac{y}{\sin(\frac{\pi y}{2})}= -\frac{2}{\pi}.$$
\item  ( FI $\frac{0}{0}$ - pas nécessaire sur une copie) 
D'après le cours $e^x-1\sim_0 x$ donc $ \frac{x\ln(x)}{e^x-1} \sim_0 \ln(x)$
Ainsi : $$\lim_{x\tv 0} \frac{x\ln(x)}{e^x-1}=-\infty.$$
\item  ( FI $\frac{+\infty}{+\infty}$ - pas nécessaire sur une copie) 
$\ln(x^2)= 2\ln(x)$ et  $\ln(x+1) = \ln(x) + \ln\left( 1+\frac{1}{x}\right)\sim_+\infty \ln(x)$. Donc 
$$\lim_{x\tv +\infty} \frac{\ln(x^2)}{\ln(x+1)}=2$$
\item  ( FI $\frac{+\infty}{+\infty}$ - pas nécessaire sur une copie) 
La puissance est une fonction de la variable $x$, on passe donc à la forme exponentielle : $x^x =\exp(x\ln(x))$.
\begin{align*}
\frac{\ln(x) e^{x^2}}{x^x }&= \frac{\ln(x) \exp(x^2)}{exp(x\ln(x)} \\
&= \ln(x) \exp(x^2-x\ln(x))\\ 
&=\ln(x) \exp(x(x-\ln(x))
\end{align*}
Or $x-\ln(x) \tv_{+\infty} +\infty$ par croissance comparée. Donc $\exp(x(x-\ln(x))\tv_{+\infty} +\infty$  et finalement 
$$\lim_{x\tv +\infty} \frac{\ln(x) e^{x^2}}{x^x} =+\infty.$$

\end{enumerate}

\end{correction}