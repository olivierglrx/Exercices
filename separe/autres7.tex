\subsection{Géométrie et complexe  }


\begin{exercice}
On considère $ S= \{ z\in \bC\, | \, |z|=2\}$.
\begin{enumerate}
\item Rappeler la nature géométrique de $S$.
Soit $f : \bC \tv \bC $ la fonction définie par $f(z) =\frac{2z +1}{z+1}$. Déterminer $D_f$ le domaine de définition de $f$. Est elle bien définie pour tous les points de $S$ ? 
\item 
\begin{enumerate}
\item Mettre $f(z) -\frac{7}{3}$ sous la forme d'une fraction. 
\item Montrer que pour tout $z$ dans l'ensemble de définition de $f$, $$\left| f(z) -\frac{7}{3}\right|^2 = \frac{|z|^2 +8\Re(z) +16 }{9 (|z|^2 +2\Re(z) +1)}$$
\item On note $S_2$ le cercle de centre $7/3$ et de rayon $r_0$. Montrer que $f(S) \subset S_2$
%En déduire qu'il existe $r_0\in \R$ tel que pour tout $z\in S$, 
%$$\left| f(z) -\frac{7}{3}\right| =r_0.$$
%\item  Montrer que pour tout $z$ dans l'ensemble de définition de $f$
%$$\left| f(z) -\frac{7}{3}\right|^2 - r_0^2 = \frac{-3 |z|^2 +12}{9 (|z|^2 +2\Re(z) +1)}$$
%\item On note $S_2$ le cercle de centre $7/3$ et de rayon $r_0$. Montrer que $f(S) \subset S_2$
%\item Conclure sur la nature géométrique de $f(S)$. 
\end{enumerate}
\item
\begin{enumerate}
\item  Soit $y =f(z)$, exprimer $z$ en fonction de $y$ quand cela a un sens. 
\item Déterminer l'ensemble $F$ tel que $f : D_f \tv F$ soit bijective. Déterminer l'expression de $f^{-1}$ 
\item (Difficile) Montrer que pour tout $y\in S_2$, $f^{-1}(y) \in S$. 
\item En déduire $f(S).$ 
\end{enumerate}

\end{enumerate}
\end{exercice}

\begin{correction}
\begin{enumerate}
\item $S$ est le cercle de centre $0$ et de rayon $2$. L'ensemble de définition de $f$ est $\bC\setminus \{ -1\}$. Comme $|-1|=1$, $-1\notin S$ donc $f$ est bien définie sur $S$. 
\item 
\begin{enumerate}
\item $$f(z)-\frac{7}{3}= \frac{6z+3 - 7(z+1)}{3(z+1)} = \frac{-z -4}{3(z+1)}$$
\item
\begin{align*}
\left| f(z) -\frac{7}{3}\right|^2 &= \left| \frac{-z -4}{3(z+1)}\right|^2\\
												&=  \frac{ |z +4|^2}{9|z+1|^2}\\
												&=  \frac{ (z +4)\overline{(z +4)}}{9(z+1)\overline{(z+1)}}\\
												&=  \frac{ (z +4)(\bar{z} +4)}{9(z+1)(\overline{z}+1)}\\
												&=  \frac{ z\bar{z}  +4(z+\bar{z} )+16}{9(z\bar{z} +(z+\overline{z})+1)}\\
												&=\frac{ |z|^2  +8\Re(z)+16}{9(|z|^2  +2\Re(z)+1)}
\end{align*}

\item  (La question était manifestement mal posée, il aurait par exemple fallu présicer le rayon qui vaut $\frac{2}{3}$) 

 Pour tout $z\in S$, on  a $|z|^2=4$ donc pour tout $z\in S$:
\begin{align*}
\left| f(z) -\frac{7}{3}\right|^2 &=\frac{ 4  +8\Re(z)+16}{9(4 +2\Re(z)+1)}\\
												&=\frac{ 8\Re(z)+20}{9( 2\Re(z)+5)}\\
												&=\frac{4 (2\Re(z)+5)}{9( 2\Re(z)+5)}\\
												&= \frac{4}{9}\\
												&=\left(\frac{2}{3}\right)^2
\end{align*}
On obtient $r_0=\frac{2}{3}$ car $\left| f(z) -\frac{7}{3}\right|>0$. 

Ainsi pour tout $z\in S$ on a $f(z) \in S_2$. D'où $f(S) \subset S_2$. 

\end{enumerate}
\item 
\begin{enumerate}
\item On résout $y = f(z) $. 
\begin{align*}
y&= \frac{2z+1}{z+1}\\
(z+1)y &= 2z+1\\
z(y-2) &= 1-y\\
z &= \frac{1-y}{y-2}\quad y\neq 2
\end{align*}
\item Ainsi $f : D_f \tv \bC\setminus \{ 2\} $ réalise une bijection et $f^{-1} (y) =\frac{1-y}{y-2}$

\item Soit $y\in S_2$ on va réaliser le même procédé que la question 2b) pour $f^{-1}$. Comme on va s'intéresser aux images de $y \in S_2$ on cherche à mettre en lumière le role de $|y-\frac{7}{3}|$
\begin{align*}
\left|f^{-1} (y) \right|^2 &=\frac{|1-y|^2}{|y-2|^2}\\
									&=\frac{|y-1|^2 }{|y-2|^2}	\\
									&=\frac{|(y-\frac{7}{3}) +\frac{4}{3}|^2 }{|(y-\frac{7}{3}) +\frac{1}{3}|^2}	\\
									&=\frac{|y-\frac{7}{3}|^2 +\frac{8}{3}\Re( y-\frac{7}{3}) + \frac{16}{9} }{|y-\frac{7}{3}|^2 +\frac{2}{3}\Re( y-\frac{7}{3}) + \frac{1}{9} }
\end{align*}
Maintenant, pour tout $y\in S_2$ on a $|y-\frac{7}{3}|^2 =\frac{4}{9}$ donc pour tout $y\in S_2$ on a 
\begin{align*}
\left|f^{-1} (y) \right|^2& = 
\frac{\frac{4}{9}+\frac{8}{3}\Re( y-\frac{7}{3}) + \frac{16}{9} }{\frac{4}{9} +\frac{2}{3}\Re( y-\frac{7}{3}) + \frac{1}{9} }\\
& = 
\frac{\frac{8}{3}\Re( y-\frac{7}{3}) + \frac{20}{9} }{\frac{2}{3}\Re( y-\frac{7}{3}) + \frac{5}{9} }\\
& = 
\frac{24\Re( y-\frac{7}{3}) + 20 }{6\Re( y-\frac{7}{3}) + 5 }\\
& = 
\frac{4(6\Re( y-\frac{7}{3}) + 5) }{6\Re( y-\frac{7}{3}) + 5 }\\
&=4
\end{align*}
Ainsi  pour tout $y\in S_2$  $f^{-1}(y)$ appartient au cercle de centre $0$ et de rayon $2$, c'est-à-dire $S$. 
On vient donc de montrer $f^{-1} (S_2)\subset S$. 
\item Les questions 2c) et 3c) impliquent que $f(S) =S_2$
\end{enumerate}
\end{enumerate}
\end{correction}