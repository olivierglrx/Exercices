\subsection{Urnes de Polya sans VAR + info (Pb)}
\begin{exercice}
On dispose d'une urne contenant initialement $b$ boules blanches et $r$ boules rouges. On fait des tirages successifs dans cette urne en respectant à chaque fois le protocole suivant : 
\begin{itemize}
\item Si la boule tirée est de couleur blanche, on la remet et on ajoute une boule blanche 
\item Si la boule tirée est de couleur rouge, on la remet et on ajoute une boule rouge.  
\end{itemize}

On appelle $B_i$ l'événement "tirer une boule blanche au $i$-iéme tirage" et on note $p_i =P(B_i)$. 

\begin{enumerate}
\item Calculer $p_1$ en fonction de $b$ et $r$.
\item Montrer que $p_2= \frac{b}{b+r}  $.
\item On a tiré une boule blanche au deuxième tirage. Donner alors la probabilité que l'on ait tiré une boule blanche au premier tirage  en fonction de $b$ et $r$. 
\item On appelle $E_n$  l'événément 
\begin{center}
$E_n$ : " On tire que des boules blanches sur les $n$ premiers tirages "
\end{center}

et $F_n$ l'événement
\begin{center}
$F_n$ : " On tire  pour la première fois une boule rouge au $n$-ième tirage"
\end{center}

\begin{enumerate}
\item Exprimer $E_n$  à l'aide des événements $(B_k)_{k\in \intent{1,n}} $ 
\item Exprimer $F_n$  à l'aide de $E_{n-1}$ et $B_n$ 
\end{enumerate}


\item Pour tout $k\geq 2$ calculer $P_{E_{k-1}}(B_k)$.
\item Calculer $P(E_n)$ en fonction de $b, r$ et $n$ puis $P(F_n)$.

\item On souhaite modéliser informatiquement cette expérience. On va utiliser la lettre 'B' pour désigner les boules blanches et 'R' pour les rouges. 
\begin{enumerate}
\item Créer une fonction \texttt{urne} qui prend en paramètres le nombre de boules blanches et rouges, et retourne une liste correspondant à l'urne initiale. (Cette  liste n'a pas à être "mélangée")
%\item Créer une fonction \texttt{shuffle} qui prend en argument une liste et retourne une autre liste contenant les mêmes élèments que le première mélangés aléatoirement. 
\item Créer une fonction \texttt{tirage} qui prend en argument une liste correspondant à une urne, modélise le tirage d'une boule alétoirement dans cette urne, affiche la couleur de la boule tirée et retourne une liste correspondant à l'urne aprés l'ajout de la boule de la couleur tirée. 
\item Créer une fonction \texttt{compte} qui prend une liste correspondant à une urne et retourne   le nombre de  boules blanches  contenues dans l'urne. 
\item Créer une fonction \texttt{expérience} qui prend en argument le nombre de boules blanches et rouges et $N$ le nombre de tirages effectués et retourne le nombre de boules blanches dans l'urne aprés $N$ tirages. 
\end{enumerate}
 
\end{enumerate}
\end{exercice}
\begin{correction}
\begin{enumerate}


\item On a $p_1=P(B_1)$. Comme il y a $b$ boules et $b+r$ boules en tout, on en déduit que $P(B_1) =\frac{b}{b+r}$ 
\conclusion{$p_1=\frac{b}{b+r}$}

\item On utilise le système complet d'événements $(B_1, \overline{B_1})$, la formule des probabilités totales donnent : 
$$p_2=P(B_2) =P(B_2|B_1) P(B_1) +P(B_2 |\overline{B_1})P(\overline{B_1})$$

Si on a tiré une boule blanche au tirage 1, il y a $b+1$ boules blanches dans l'urne et $r$ boules rouges, donc 
$$P(B_2|B_1)  =\frac{b+1}{b+r+1}$$

De même, si on a tiré une boule rouge au tirage 1, il y a $b$ boules blanches dans l'urne et $r+1$ boules rouges, donc 
$$P(B_2|\overline{B_1})  =\frac{b}{b+r+1}$$

D'après le calcul de $p_1$, on sait que $P(\overline{B_1})= 1-P(B_1) =1-\frac{b}{b+r}= \frac{r}{b+r}$

Ainsi 
\begin{align*}
p_2&=\frac{b+1}{b+r+1} \frac{b}{b+r} + \frac{b}{b+r+1} \frac{r}{b+r}\\
	&=\frac{(b+1)b}{(b+r+1)(b+r)} + \frac{br}{(b+r+1)(b+r)} \\
	&=\frac{(b+1+r)b}{(b+r+1)(b+r)}  \\
	&=\frac{b}{b+r}  
\end{align*}

\conclusion{ $p_2= \frac{b}{b+r}  $}

\item On cherche à calculer $P(B_1|B_2)$ et on utilise pour cela la formule de Bayes :
$$P(B_1|B_2) =\frac{P(B_2|B_1) P(B_1)}{P(B_2)}$$
Or on a vu que $P(B_1) =P(B_2)$ donc 
$$P(B_1|B_2) =P(B_2|B_1)$$

Ainsi \conclusion{ $P(B_1|B_2) = \frac{b+1}{b+r+1}$}
\item \begin{enumerate}
\item $E_n = B_1 \cap B_2 \cap \dots \cap B_{n} $. 
\item $F_n =E_{n-1} \cap \overline{B_n}$. 
\end{enumerate}
\item Si l'événement $E_{k-1} $ est réalisé, on a tiré que des boules blanches sur les $k-1$ premiers tirages. Il y a donc $b+k-1$ boules blanches et $b+k-1+r$ boules au total. 
Donc \conclusion{ $P(B_k |E_{k-1}) = \frac{b+k-1}{b+k-1+r}$} 

\item On utilise la formule des probabilités conditionnelles et on obtient : 
$$P(E_n) = P(B_1) P(B_2|B_1) P(B_3|B_2\cap B_1) \dots  P(B_{n}|B_{n-1}\cap \dots \cap B_2\cap B_1) $$
Remarquons que les termes du produit sont de la forme $P(B_k |E_{k-1})$ que l'on a calculé à la question précédente. On a donc 
\begin{align*}
P(E_n) &=P(B_1) \prod_{k=2}^{n} P(B_k |E_{k-1})\\
			&=\frac{b}{b+r}\frac{b+2-1}{b+2-1+r} \frac{b+3-1}{b+2-1+r}\dots \frac{b+n-1}{b+n-1+r}\\
			&=\frac{b}{b+r}\frac{b+1}{b+1+r} \frac{b+2}{b+2+r}  \dots \frac{b+n-1}{b+n-1+r}\\
			&= \frac{(b+n-1)!}{(b-1)!} \frac{(b+r-1)!}{(b+n-1+r)!} 
\end{align*}


\item On en déduit la valeur de $P(F_n) $ de nouveau en utilisant la formule des proababilités conditionnelles : 
$$P(F_n) =P(E_{n-1}\cap \overline{B_n}) =P(E_{n-1} ) P(\overline{B_n}|E_{n-1})$$

Si l'événement $E_{n-1}$ est réalisé, il y a  $r$ boules rouges et $b+n-1+r$ boules au total. 
Donc 
$$P(F_n) = \frac{r}{b+n-1+r}$$

\item \begin{enumerate}
\item
\begin{lstlisting}
def urne(b,r):
  L=['B' for i in range(b)] +['R' for i in range(n)]
  return(L)
\end{lstlisting}

\item
\begin{lstlisting}
from random import randint
def tirage(L):
	nouvel_urne=L[:]
	k=randint(0,len(L)-1)
	
	if L[k]=='B':
	  print('Boule blanche')
	  nouvel_urne=nouvel_urne+['B']
	else:
	  print('Boule Rouge')
	  nouvel_urne=nouvel_urne+['R']
	return(nouvel_urne)
\end{lstlisting}

\item 
\begin{lstlisting}

def compte(L):
  b=0
  for e in L:
    if e=='B':
      b=b+1
  return(b)
\end{lstlisting}

\item 
\begin{lstlisting}
def experience(b,r,N):
  U=urne(b,r)
  for k in range(N):
    U=tirage(U)
  boule_b=compte(U)
  return(boule_b)
    
\end{lstlisting}

\end{enumerate}

\end{enumerate}

\end{correction}