\subsection{Résolution de $\floor{2x-\sqrt{5x-1}} =0 $}
\begin{exercice}
On considère l'équation suivante d'inconnue $x\in \R : $ 
$$\floor{2x-\sqrt{5x-1}} =0 \quad \quad (E)$$
\begin{enumerate}
\item Déterminer le domaine de définition de $(E)$. 
\item Dire si les réels suivants sont solutions ou non de $(E)$
$$x_1 = \frac{1}{5}, \, x_2 = \frac{1}{2}, \, x_3 =1,\, x_4=12$$
\item Pour tout $a\in \R$, rappeler  un encadrement de la partie entière de $a$ en fonction de $a$. 
\item Montrer que résoudre $(E)$ est équivalent à résoudre le système :
$$\left\{\begin{array}{clc}
\sqrt{5x-1} >&2x-1 &\quad(E_1)\\ 
\sqrt{5x-1} \leq &2x&\quad(E_2)
\end{array}\right.
$$
\item Résoudre les deux inéquations obtenues à la question précédente. 
\item Résoudre $(E)$. 
\end{enumerate}
\end{exercice}

\begin{correction}
\begin{enumerate}
\item Seule la fonction $x\mapsto \sqrt{x}$ n'est pas définie sur $\R$ mais sur $\R_+$ ainsi $(E)$ est bien définie pour tout $x$ tel que $5x-1\geq 0$ c'est-à-dire 
\conclusion{ $D_E=]\frac{1}{5},+\infty[$}
\item Cours 
\conclusion{$\forall a\in \R\, \quad  a-1 <\floor{a}\leq a$}

\item Notons $f(x) = \floor{2x-\sqrt{5x-1} }$ 
On a $f(\frac{1}{5}) = \floor{2\frac{1}{5}-\sqrt{5\frac{1}{5}-1} }= \floor{2\frac{1}{5}} =0$
Donc \conclusion{$\frac{1}{5}$ est solution de $E$}

On a $f(\frac{1}{2} )  =\floor{2\frac{1}{2}-\sqrt{5\frac{1}{2}-1} } = \floor{1-\sqrt{\frac{3}{2} }}$ Or $\frac{3}{2}> 1 $ donc 
$\sqrt{\frac{3}{2}} >\sqrt{1}=1$ et donc 
$1-\sqrt{\frac{3}{2} }<0$ ainsi  
\conclusion{$\frac{1}{2}$ n'est pas solution de $E$}

On a $f(1) = \floor{2\times 1-\sqrt{5-1}} =\floor{2-2}=\floor{0}$
\conclusion{$1$ est solution de $E$}

On  a $f(12) = \floor{2\times 12-\sqrt{60-1}}=\floor{24-\sqrt{59}}$ 
Or$ 59<64=8^2$ donc  $\sqrt{59} < 8 $ et 
$24-\sqrt{59}> 24-8=16$ ainsi $f(2)>16 $ et 
\conclusion{$12$ n'est pas solution de $E$}


\item D'après ce qu'on vient de voir, pour tout $x\in D_E$ on a :
$$2x-\sqrt{5x-1} -1<\floor{2x-\sqrt{5x-1} } \leq 2x-\sqrt{5x-1} $$
Si $x$ est solution de $(E)$ on a $\floor{2x-\sqrt{5x-1} }=0$ et donc l'équation $(E)$ équivaut à $2x-\sqrt{5x-1} -1<0 \leq 2x-\sqrt{5x-1} $, soit 
\conclusion{$\left\{\begin{array}{clc}
\sqrt{5x-1} >&2x-1 &\quad(E_1)\\ 
\sqrt{5x-1} \leq &2x&\quad(E_2)
\end{array}\right.
$}

\item Résolvons ces deux inéquations. Tout d'abord la première :
$$\sqrt{5x-1} >2x-1 \quad(E_1)$$
On distingue deux cas : 
\begin{itemize}
\item[$\blacktriangleright$] \underline{Cas 1 :} $2x-1\geq 0$ c'est-à-dire $x\geq \frac{1}{2}$

Alors on peut passer au carré dans l'équation car les deux cotés sont du même signe. On a alors : 
\begin{align*}
(E_1)& \equivaut 5x-1 > (2x-1) ^2\\
& \equivaut 5x-1 > 4x^2-4x+1\\
& \equivaut 4x^2-9x+2<0
\end{align*}
Un petit discriminant comme on aime : 
$\Delta = 9^2 - 4*4*2 = 81- 32= 49 =7^2$. 
$4x^2-9x+2$ admet donc deux racines 
$$r_1 = \frac{9+7}{8}=2 \quadet r_2 =\frac{9-7}{8} = \frac{1}{4}$$

Ainsi les solutions de $(E_1) $ sur $[\frac{1}{2},+\infty[$ sont
\begin{align*}
\cS_1 &= ]\frac{1}{4},2[ \cap [\frac{1}{2},+\infty[ \cap D_E\\
		 &= [\frac{1}{2},2[	
\end{align*}


\conclusion{  Les solutions de $(E_1) $ sur $[\frac{1}{2},+\infty[$ sont  $\cS_1 = [\frac{1}{2},2[	$}

\item[$\blacktriangleright$] \underline{Cas 2 :} $2x-1<0$ c'est-à-dire $x<\frac{1}{2}$

Dans ce cas, tous les réels $x\in D_E$ sont solutions car le membre de gauche est positif et celui de droite négatif. 

\conclusion{  Les solutions de $(E_1) $ sur $]-\infty,\frac{1}{2}[$ sont  $\cS_1' = [\frac{1}{5},\frac{1}{2}]	$}

En conclusion :
\conclusion{  Les solutions de $(E_1) $ sur $D_E$ sont  $\cS = \cS_1\cup \cS_1' = [\frac{1}{5},2[	$}



\end{itemize}

On fait la même chose pour $(E_2) $ 
$$\sqrt{5x-1} \leq 2x \quad(E_2)$$

On distingue deux cas : 
\begin{itemize}
\item[$\blacktriangleright$] \underline{Cas 1 :} $2x\geq 0$ c'est-à-dire $x\geq 0$

Alors on peut passer au carré dans l'équation car les deux cotés sont du même signe. On a alors : 
\begin{align*}
(E_1)& \equivaut 5x-1 \leq  (2x) ^2\\
& \equivaut 5x-1 \leq 4x^2\\
& \equivaut 4x^2-5x+1\geq 0
\end{align*}
Un petit discriminant comme on aime : 
$\Delta = 5^2 - 4*4*1 = 25- 16= 9 =3^2$. 
$4x^2-5x+1$ admet donc deux racines 
$$r_1 = \frac{5+3}{8}=1 \quadet r_2 =\frac{5-3}{8} = \frac{1}{4}$$

Ainsi les solutions de $(E_2) $ sur $[0,+\infty[$ sont
\begin{align*}
\cE_2 &=( ]-\infty, \frac{1}{4}] \cup [1,+\infty[) \cap [0,+\infty[ \cap D_E\\
		 &= [\frac{1}{5} ,\frac{1}{4}] \cup[1,+\infty[	
\end{align*}


\conclusion{  Les solutions de $(E_2) $ sur $[0,+\infty[$ sont  $\cE_2 =  [\frac{1}{5} ,\frac{1}{4}] \cup[1,+\infty[	$}

\item[$\blacktriangleright$] \underline{Cas 2 :} $2x<0$ c'est-à-dire $x<0$

Dans ce cas, aucun réel n'est solution car le membre de gauche est positif et celui de droite négatif. 

\conclusion{  Les solutions de $(E_2) $ sur $]-\infty,0[$ sont  $\cE_2' =\emptyset$}

En conclusion :
\conclusion{  Les solutions de $(E_2) $ sur $D_E$ sont  $\cE = \cE_2\cup \cE_2' =  [\frac{1}{5} ,\frac{1}{4}] \cup[1,+\infty[		$}



\end{itemize}

\item $x$ est solution de $(E)$ si et seulement si il est solution de $(E_1) $ et $(E_2)$, l'ensemble des solutions correspond donc à l'intersection : $\cE \cap \cS =   ([\frac{1}{5} ,\frac{1}{4}] \cup[1,+\infty[	) \cap[\frac{1}{5},2[ = 
[\frac{1}{5} ,\frac{1}{4}] \cup[1,2[$ 

\conclusion{ Les solutions de $(E)$ sont $[\frac{1}{5} ,\frac{1}{4}] \cup[1,2[$ }


\end{enumerate}
\end{correction}