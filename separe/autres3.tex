\subsection{Equation du second degré à coeff complexes}


\begin{exercice}
On considère l'équation du second degré suivante : 
$$z^2+(3i-4)z+1-7i=0 \quad (E) $$

\begin{enumerate}
\item A la manière d'une équation réelle, calculer le discriminant $\Delta$ du polynôme complexe, et montrer que $\Delta=3+4i$
\item On se propose de résoudre $ (E_2) \, : \, u^2=\Delta \, $  d'inconnue complexe $u$. 
\begin{enumerate}
\item On écrit $u=x+iy$ avec $(x,y)\in \R^2$. Montrer que $(E_2)$ est équivalent à 
$$ x^4-3x^2-4=0 \quadet y =\frac{2}{x}.$$
\item En déduire que les solutions de $(E_2)$ sont 
$$u_1=3-i\quadet u_2=1-2i$$
\end{enumerate}
\item Soit $u_1$ une solution de l'équation précédente. 
On considère $r_1 = \frac{-3i+4 +u_1}{2}$. Montrer que $r_1$ est solutions de l'équation  $(E)$.
\item Quelle est à l'autre solution  de  $(E)$ ? 
\end{enumerate}

\end{exercice}

\begin{correction}
On suit les étapes indiquées dans l'énoncé. 
\begin{enumerate}
\item Le discriminant vaut 
$$\Delta = (3i-4)^2 -u^4 (1-7i) = -9-24i +16 -4+28i = 3+4i$$
\item Résolvons $u^2=3+4i$. \begin{enumerate}
\item On pose donc $u=x+iy$ avec $x,y\in \R$ 
On a  donc $(x+iy)^2 = 3+4i $, soit $x^2-y^2 +2xyi =3+4i$ En identifiant partie réelle et partie imaginaire on obtient : 
$$x^2 -y^2 =3 \quad 2xy=4$$

Comme $x\neq 0 $ (sinon $\Delta\in \R_-$ ), la deuxième équation devient 
\conclusion{$y=\frac{2}{x}.$} On remplace alors $y$ avec cette valeur dans la première équation, ce qui donne : 
$$x^2 -\frac{4}{x^2}=3$$ et  en multipliant par $x^2$ 
\conclusion{ $x^4 -3x-4=0$}

\item On fait un changement de variable $X=x^2$ dans l'équation $x^4-3x^2-4=0$. On obtient 
$$X^2 -3X-4=0$$
De discriminant $\Delta_2 = 9+4*4=25=5^2$. Cette équation admet ainsi deux solutions réelles : 
$$X_1= \frac{3-5}{2}= -1\quadet X_2 =\frac{3+5}{2}=4$$
Remarquons maintenant que $X$ doit être positif car $x^2=X$ ainsi, les solutions pour la variable $x$ sont 
$$x_1 =\sqrt{4}=2 \quadet x_2 =-\sqrt{4}=-2$$
Ce qui correspond respectivement à $y_1= 1$ et $y_2= -1$
On obtient finalement deux solutions pour $u^2=\Delta $ 
à savoir 
\conclusion{$u_1= 2+i \quadet u_2 =-2-i$}



\end{enumerate}
\item  On considère donc $r_1 = \frac{-3i+4+2+i}{2}= 3-i$. Montrons que $r_1$ est solution de $(E)$ 

$$r_1^2 = (3-i)^2 = 9-6i-1=8-6i$$
$$(3i-4)r_1 =(3i-4) (3-i) = 9i+3-12+4i = -9+13i$$
Donc 
$r_1^2 +(3i-4)r_1 = 8-6i -9+13i  =-1 +7i$
Soit 
$$r_1^2 +(3i-4)r_1 +1-7i=0$$
\conclusion{Donc $r_1$ est bien solution de $(E)$. }

\item L'autre solution est sans aucun doute 
\conclusion{ $r_2 = \frac{-3i+4+u_2}{2} = 1-2i$}

\end{enumerate}
\end{correction}