\subsection{Intégrale de Gauss (D'après G2E 2019]}

\begin{exercice}[G2E 2019]
Dans cet exercice $\sigma$ désigne un réel strictement positif. \\

On considère les trois fonctions définies respectivement sur $\R$, $\R$ et $\R^*$ par :
$$
f_\sigma (x) = \left\{\begin{array}{lr}
0 & \text{si $x\leq 0,$}\\
\frac{x}{\sigma^2}e^{-\frac{x^2}{2\sigma^2}} &\text{si $x> 0$}
\end{array}\right. \quad g(x) =xe^{-x}, \quad h(x) = \frac{2}{ex}
$$
$\cC_\sigma$ désigne la courbe représentative de $f_\sigma$ et $\cH$ la courbe représentative de $h$
\begin{enumerate}
\item Soit $\ddp I_\sigma(t) = \int_0^t f_\sigma(x) dx$. Calculer $I_\sigma(t) $ pour tout $t\geq 0$ et en déduire la limite $\ddp \lim_{t\tv \infty} I_\sigma(t) $ \\
{\footnotesize L'année prochaine, on dira que $f_\sigma$ est une fonction de densité.}
\item $f_\sigma$ est-elle continue ? 
\item \begin{enumerate}
\item Démontrer que $g$ admet un maximum que l'on déterminera. 
\item En déduire que :
$$\forall x\in \R_+^*, \quad f_\sigma (x) \leq h(x).$$
\item Etudier le cas d'égalité dans l'inégalité précédente puis montrer que pour tout $\sigma \in \R^*_+$, les courbes $\cC_\sigma$ et $\cH$ ont une tangente commune dont on donnera une équation cartésienne. 
\end{enumerate}
\end{enumerate}

\end{exercice}

\begin{correction}
\begin{enumerate}
\item Considérons $F_\sigma$ définie par $F_\sigma(t) =-e^{-\frac{x^2}{2\sigma^2}}$ pour tout $x\geq 0$, $F_\sigma$ est dérivalbe sur $\R_+$ et on a pour $x\geq 0$, $F_\sigma'(x) = f_\sigma(x)$ ainsi
$$I_\sigma( t) =\left[ F_\sigma(x) \right]_0^t = F_\sigma(t) -F_\sigma(0)= e^{-\frac{t^2}{2\sigma^2}}+1.$$

Or $\lim_{t\tv \infty} e^{-\frac{t^2}{2\sigma^2}}=0$ donc 
$$\ddp \lim_{t\tv \infty} I_\sigma(t) =1.$$
\item $f_\sigma $ est continue sur $\R_- $ et $\R_+$ comme composée de fonctions usuelles. En $0$, 
$$\lim_{x\tv 0^+} f_\sigma(x)  = \lim_{x\tv 0^+} \frac{x}{\sigma^2}e^{-\frac{x^2}{2\sigma^2}}  =0$$ 
et 
$$\lim_{x\tv 0^-} f_\sigma(x)  = \lim_{x\tv 0^+} 0 =0$$ 
Ainsi $f_\sigma $ est continue en $0$ et finalement continue sur $\R$. 
\item 
\begin{enumerate}
\item $g$ est définie et dérivable sur $\R$ et on a pour tout $x\in \R$: 
$$g'(x) = e^{-x} -xe^{-x} = (1-x) e^{-x}$$
Ainsi $g$ est croissante sur $]-\infty, 0] $ et décroissante sur $[1,+\infty[$. 

$g$ atteint son maximum en $1$ et vaut $e^{-1}$. 
\item Pour tout $x> 0$ on   
\begin{align*}
f_\sigma(x)& = \frac{2}{x}  \frac{x^2}{2\sigma^2}e^{-\frac{x^2}{2\sigma^2}} \\
				&= \frac{2}{x}   g(\frac{x^2}{2\sigma^2})\\
				&\leq \frac{2}{x}   e^{-1} \quad \text{ d'après la question précédente} 
\end{align*}
 Ainsi $f_\sigma(x) \leq \frac{2}{ex} =h(x)$.  
\item L'égalité a lieu en $x_0\geq 0$ vérifiant $g(\frac{x_0^2}{2\sigma^2})=e^{-1} =g(1)$. On a vu à la question 3)a) que $e^{-1}$ était atteint uniquement en $1$ par $g$. On a  donc $\frac{x_0^2}{2\sigma^2} =1$. 
C'est-à-dire, comme $x_0\geq 0$, $$x_0=\sqrt{2}\sigma$$

On a bien $$f_\sigma (x_0) = \frac{\sqrt{2}}{\sigma} e^{-1} = \frac{2}{e \sqrt{2} \sigma} = h(x_0).$$

Ainsi $\cC_\sigma $ et $\cH$ ont bien un point en commun. Vérifions que les tangentes sont identiques en ce point. 

Tout d'abord remarquons ques les deux courbes admettent bien  des tangentes car sont des courbes représentatives de fonctions $\cC^1$. L'équation de la tangente à $\cC_\sigma$ en $x_0$ est donnée par 
$Y - f_\sigma(x_0) = f'_\sigma(x_0) (X-x_0)$
et on a $f'_\sigma(x_0) = \frac{1}{\sigma^2} e^{-\frac{x_0^2}{2\sigma^2}}  - \frac{x_0^2}{\sigma^4}e^{-\frac{x_0^2}{2\sigma^2}} $ ce qui donne en  simplifiant :
$$f'_\sigma(x_0) =  \frac{1}{\sigma^2} e^{-1} - \frac{2}{\sigma^2}e^{-1}= - \frac{1}{\sigma^2}e^{-1} $$

On obtient ainsi comme équation pour la tangente à $\cC_\sigma$ en $x_0$:
$$Y- \frac{\sqrt{2}}{e\sigma} = - \frac{1}{e\sigma^2} (X- \sqrt{2}\sigma)$$

Faisons de même avec la tangente à $\cH$ en $x_0$ et calculons $h'(x_0)$. 
On  a $h'(x_0) =\frac{-2}{ex_0^2}= \frac{-2}{e 2 \sigma^2} =- \frac{1}{e\sigma^2} =f'_\sigma(x_0) $
Ainsi les deux courbes admettent bien la même tangente en $x_0$ à savoir :

$$Y- \frac{\sqrt{2}}{e\sigma} = - \frac{1}{e\sigma^2} (X- \sqrt{2}\sigma)$$

 
\end{enumerate}
\end{enumerate}
\end{correction}