%BECHATA Abdellah
%www.mathematiques.fr.st
\documentclass[a4paper, 11pt,reqno]{article}
\theoremstyle{definition}
\newtheorem{probleme}{Problème}
\theoremstyle{definition}


%%%%% box environement 
\newenvironment{fminipage}%
     {\begin{Sbox}\begin{minipage}}%
     {\end{minipage}\end{Sbox}\fbox{\TheSbox}}

\newenvironment{dboxminipage}%
     {\begin{Sbox}\begin{minipage}}%
     {\end{minipage}\end{Sbox}\doublebox{\TheSbox}}


%\fancyhead[R]{Chapitre 1 : Nombres}


\newenvironment{remarques}{ 
\paragraph{Remarques :}
	\begin{list}{$\bullet$}{}
}{
	\end{list}
}




\newtcolorbox{tcbdoublebox}[1][]{%
  sharp corners,
  colback=white,
  fontupper={\setlength{\parindent}{20pt}},
  #1
}







%Section
\pretocmd{\section}{%
  \ifnum\value{section}=0 \else\clearpage\fi
}{}{}



\sectionfont{\normalfont\Large \bfseries \underline }
\subsectionfont{\normalfont\Large\itshape\underline}
\subsubsectionfont{\normalfont\large\itshape\underline}



%% Format théoreme, defintion, proposition.. 
\newmdtheoremenv[roundcorner = 5px,
leftmargin=15px,
rightmargin=30px,
innertopmargin=0px,
nobreak=true
]{theorem}{Théorème}

\newmdtheoremenv[roundcorner = 5px,
leftmargin=15px,
rightmargin=30px,
innertopmargin=0px,
]{theorem_break}[theorem]{Théorème}

\newmdtheoremenv[roundcorner = 5px,
leftmargin=15px,
rightmargin=30px,
innertopmargin=0px,
nobreak=true
]{corollaire}[theorem]{Corollaire}

\usepackage{mdframed}
\newmdtheoremenv[%
roundcorner=5px,
innertopmargin=0px,
leftmargin=15px,
rightmargin=30px,
nobreak=true
]{defi}[theorem]{Définition}

\newmdtheoremenv[roundcorner = 5px,
leftmargin=15px,
rightmargin=30px,
innertopmargin=0px,
nobreak=true
]{prop}[theorem]{Proposition}

\newmdtheoremenv[roundcorner = 5px,
leftmargin=15px,
rightmargin=30px,
innertopmargin=0px,
]{prop_break}[theorem]{Proposition}

\newmdtheoremenv[roundcorner = 5px,
leftmargin=15px,
rightmargin=30px,
innertopmargin=0px,
nobreak=true
]{regles}[theorem]{Règles de calculs}


\newtheorem{exo}{Exercice}
\newtheorem{exercice}[theorem]{Exercice}
\newtheorem*{exemples}{Exemples}
\newtheorem{exemple}{Exemple}
\newtheorem*{rem}{Remarque}
\newtheorem*{rems}{Remarques}
% Warning sign

\newcommand\warning[1][4ex]{%
  \renewcommand\stacktype{L}%
  \scaleto{\stackon[1.3pt]{\color{red}$\triangle$}{\tiny\bfseries !}}{#1}%
}




\theoremstyle{definition}

%\newtheorem{prop}[theorem]{Proposition}
%\newtheorem{\defi}[1]{
%\begin{tcolorbox}[width=14cm]
%#1
%\end{tcolorbox}
%}


%--------------------------------------- 
% Document
%--------------------------------------- 



\hypersetup{
    colorlinks,
    citecolor=black,
    filecolor=blue,
    linkcolor=black,
    urlcolor=black
}


 \newcounter{correction}
% 

\lstset{numbers=left, numberstyle=\tiny, stepnumber=1, numbersep=5pt}





\geometry{hmargin=2.0cm, vmargin=3.5cm}

  \newif\ifshow
\showfalse
\usepackage{comment}

\ifshow
\newenvironment{correction}{{\par\medskip \refstepcounter{correction}
%\addcontentsline{toc}{section}{\footnotesize{Correction \thecorrection .}}
\color{red}
\textbf{ Correction \thecorrection .}\color{black} \ignorespaces
}
}{}
\else
  \excludecomment{correction}
\fi



\author{Olivier Glorieux}

\begin{document}



\section*{Exercice1}

$\alpha $ est un rel strictement positif. Pour tout $n\in \mathbb{N}$ on
pose : 
\begin{equation*}
u_{n}\left( \alpha \right) =\dfrac{n!}{\prod\limits_{k=0}^{n}\left( \alpha
+k\right) }
\end{equation*}

\begin{enumerate}
\item Etude de la convergence de la suite $\left( u_{n}\left( \alpha \right)
\right) _{n\in \mathbb{N}}$

\begin{enumerate}
\item Montrer que la suite $\left( u_{n}\left( \alpha \right) \right) _{n\in 
\mathbb{N}}$ est monotone et convergente. Que peut-on déduire pour la série
de terme général $\left( u_{n}\left( \alpha \right) -u_{n+1}\left( \alpha
\right) \right) $ ?

On note $\ell \left( \alpha \right) $ la limite de la suite $\left(
u_{n}\left( \alpha \right) \right) _{n\in \mathbb{N}}$

\item On suppose que $\ell \left( \alpha \right) $ est non nulle. Démontrer
que : 
\begin{equation*}
u_{n}\left( \alpha \right) -u_{n+1}\left( \alpha \right) \underset{%
n\rightarrow +\infty }{\thicksim }\dfrac{\alpha \ell \left( \alpha \right) }{%
n}
\end{equation*}

\item Déduire de ce qui précède que $\ell \left( \alpha \right) =0$
\end{enumerate}

\item Dans cette question : $\alpha \in ]0,1]$

\begin{enumerate}
\item Montrer que : 
\begin{equation*}
\forall n\in \mathbb{N\;\;}u_{n}\left( \alpha \right) \geqslant \dfrac{1}{%
n+\alpha }
\end{equation*}

\item Quelle est la nature de la série de terme général $u_{n}\left( \alpha
\right) $ ?
\end{enumerate}

\item On pose pour tout entier naturel $n$ : 
\begin{equation*}
I_{n}\left( \alpha \right) =\int\limits_{0}^{+\infty }e^{-\alpha t}\left(
1-e^{-t}\right) ^{n}dt
\end{equation*}

\begin{enumerate}
\item Etudier la convergence de l'intégrale généralisée $I_{n}\left( \alpha
\right) $ et calculer $I_{0}\left( \alpha \right) $

\item Soit un réel $x$ strictement positif. Intégrer par parties : 
\begin{equation*}
\int\limits_{0}^{x}e^{-\alpha t}\left( 1-e^{-t}\right) ^{n}dt
\end{equation*}%
et en déduire une relation simple entre $I_{n}\left( \alpha \right) $ et $%
I_{n-1}\left( \alpha +1\right) $, pour tout $n$ entier naturel non nul.

\item En déduire : $\forall n\in \mathbb{N\;\;}I_{n}\left( \alpha \right)
=u_{n}$
\end{enumerate}

\item On suppose désormais que $\alpha >1$

\begin{enumerate}
\item Montrer que, pour tout $N$ entier naturel : 
\begin{equation*}
\sum_{n=0}^{N}I_{n}\left( \alpha \right) =\dfrac{1}{\alpha -1}-I_{N+1}\left(
\alpha -1\right)
\end{equation*}

\item En déduire que la série de terme général $u_{n}\left( \alpha \right) $
est convergente, et donner en fonction de $\alpha $ la valeur de $%
\sum\limits_{n=0}^{+\infty }u_{n}\left( \alpha \right) $.
\end{enumerate}
\end{enumerate}

\section*{Exercice 2}

$\mathfrak{M}_{3}\left( \mathbb{R}\right) $ désigne l'ensemble des matrices
carrées d'ordre 3 à coefficients réels.\newline
$\mathfrak{M}_{3,1}\left( \mathbb{R}\right) $ est l'ensemble des matrices
colonnes à trois lignes dont les coefficients sont réels.\newline
On pose : 
\begin{equation*}
A=\left( 
\begin{array}{lll}
1 & 0 & 2 \\ 
\dfrac{3}{2} & -2 & 6 \\ 
\dfrac{1}{2} & -1 & \dfrac{5}{2}%
\end{array}%
\right) \;\text{et }B=\left( 
\begin{array}{c}
x \\ 
y \\ 
z%
\end{array}%
\right) 
\end{equation*}%
où $x,\;y$ et $z$ sont des nombres réels.\newline
On définit alors une suite de matrices colonnes $\left( X_{n}\right) _{n\in 
\mathbb{N}}$ de la manière suivante :

\begin{equation*}
\left\{ 
\begin{array}{l}
X_{0}\in \mathfrak{M}_{3,1}\left( \mathbb{R}\right) \\ 
\forall n\in \mathbb{N\;X}_{n+1}=AX_{n}+B%
\end{array}%
\right.
\end{equation*}

\begin{enumerate}
\item Montrer que 0 , $\dfrac{1}{2}$ et $1$ sont les valeurs propres de A,
et préciser des vecteurs propres $u,\;v$ et $w$ qui leur sont respectivement
associés.

\item Justifier les affirmations suivantes :

\begin{itemize}
\item il existe un unique triplet $\left( \alpha ,\beta ,\gamma \right) $ de 
$\mathbb{R}^{3}$ tel que : 
\begin{equation*}
B=\alpha u+\beta v+\gamma w
\end{equation*}

\item Pour tout entier naturel $n,$ il existe un unique triplet $\left(
\alpha _{n},\beta _{n},\gamma _{n}\right) $ de $\mathbb{R}^{3}$ tel que : 
\begin{equation*}
X_{n}=\alpha _{n}u+\beta _{n}v+\gamma _{n}w
\end{equation*}
\end{itemize}

\item Etablir par récurrence que 
\begin{equation*}
n\in \mathbb{N}^{\ast }\;\;\left\{ 
\begin{array}{l}
\alpha _{n}=\alpha \\ 
\beta _{n}=\left( \dfrac{1}{2}\right) ^{n}\left( \beta _{0}-2\beta \right)
+2\beta \\ 
\gamma _{n}=\gamma _{0}+n\gamma%
\end{array}%
\right.
\end{equation*}

\item Soit $\left( a_{n}\right) _{n\in \mathbb{N}},\;\left( b_{n}\right)
_{n\in \mathbb{N}}$ et $\left( c_{n}\right) _{n\in \mathbb{N}}$ les suites ré%
elles telles que : 
\begin{equation*}
\forall n\in \mathbb{N\;}X_{n}=\left( 
\begin{array}{c}
a_{n} \\ 
b_{n} \\ 
c_{n}%
\end{array}%
\right)
\end{equation*}%
On dit que la suite de matrices colonnes $\left( X_{n}\right) _{n\in \mathbb{%
N}}$ converge si les suites réelles $\left( a_{n}\right) _{n\in \mathbb{N}%
},\;\left( b_{n}\right) _{n\in \mathbb{N}}$ et $\left( c_{n}\right) _{n\in 
\mathbb{N}}$ convergent. Dans ce cas on écrit : 
\begin{equation*}
\lim X_{n}=\left( 
\begin{array}{c}
\lim a_{n} \\ 
\lim b_{n} \\ 
\lim c_{n}%
\end{array}%
\right)
\end{equation*}

\begin{enumerate}
\item Prouver que $\left( X_{n}\right) _{n\in \mathbb{N}}$ converge si et
seulement si le réel $\gamma $ (introduit en 2.) est nul.

\item En déduire que $\left( X_{n}\right) _{n\in \mathbb{N}}$ converge si et
seulement si : 
\begin{equation*}
3x-4y+12z=0
\end{equation*}
\end{enumerate}

\item On dit que le couple $(A,B)$ admet une position d'équilibre stable si
la suite $\left( X_{n}\right) _{n\in \mathbb{N}}$ converge vers la même
limite quelle que soit la valeur de $X_{0}$.
\end{enumerate}

Expliquer pourquoi, quelle que soit la valeur de $B$, le couple $\left(
A,B\right) $ n'admet pas de position d'équilibre stable.

\section*{Exercice 3}

Dans tout le problème (qui comporte deux parties indépendantes), on suppose
que la durée, exprimée en minutes, d'une communication téléphonique est une
variable aléatoire réelle $D$ qui suit la loi exponentielle de paramètre $%
\alpha $

\subsection*{I Comparaison de deux tarifications}

Pour ses communications, on propose à l'utilisateur d'une ligne téléphonique
deux tarifications $T_{1}$ et $T_{2}$, exprimées en francs, définies de la fa%
çon suivante :

\begin{itemize}
\item $T_{1}=aD$, où $a$ est un nombre réel strictement supérieur à 1 qui
représente le prix d'une minute de communication

\item $T_{2}$ est à valeurs dans $\mathbb{N}^{\ast }$ et, pour tout $n$
entier naturel non nul : $\{T_{2}=n\}=\{n-1<D\leqslant n\}$
\end{itemize}

\begin{enumerate}
\item Calculer $E(T_{1})$ en fonction de $a$ et de $\alpha $.

\item Déterminer la loi de $T_{2}$. De quelle loi s'agit-il ? Exprimer $%
E(T_{2})$ en fonction de $\alpha $

\item On pose : 
\begin{equation*}
\left\{ 
\begin{array}{c}
\forall t\in \mathbb{R}_{+}^{\ast }:\varphi \left( t\right) =\dfrac{t}{%
1-e^{-t}} \\ 
\varphi \left( 0\right) =1%
\end{array}%
\right.
\end{equation*}

\begin{enumerate}
\item Montrer que $\varphi $ est une fonction de classe $C^{1}$ sur $%
[0,+\infty \lbrack $

\item On définit de plus la fonction $\psi $ sur $[0,+\infty \lbrack $ par : 
\begin{equation*}
\forall t\in \mathbb{R}:\psi \left( t\right) =1-\left( 1+t\right) e^{-t}
\end{equation*}%
Utiliser cette fonction pour en déduire que $\varphi $ réalise une bijection
de$[0,+\infty \lbrack $ vers $[1,+\infty \lbrack $
\end{enumerate}

\item Comparaison des tarifications

\begin{enumerate}
\item Montrer qu'il existe un unique réel $\alpha _{0}$ strictement positif
tel que $\varphi \left( \alpha _{0}\right) =a$

\item Préciser quelle est, en moyenne, la tarification la plus avantageuse
suivant la valeur de la durée moyenne d'une communication.
\end{enumerate}

\item Pour $a=1,25$ donner, en utilisant votre calculatrice. une valeur
approchée de $\dfrac{1}{\alpha _{0}}$

(on ne donnera que les deux premières décimales fournies par la
calculatrice).
\end{enumerate}

\subsection*{II Étude d'un standard téléphonique}

Dans toute cette partie, $\theta $ est un nombre réel strictement positif
représentant un temps exprimé en minutes. Un standard téléphonique de capacit%
é illimitée reçoit des communications téléphoniques entre l'instant 0 et
l'instant $\theta $ inclus.

\subsubsection*{II. A) Cas d'une seule communication}

On désigne par n un entier naturel non nul. L'instant où débute la
communication est une variable aléatoire réelle $I_{n}$ telle que : 
\begin{equation*}
\left\{ 
\begin{array}{c}
I_{n}\left( \Omega \right) =\left\{ \dfrac{\theta }{n},\dfrac{2\theta }{n}%
,\dots ,\dfrac{\left( n-1\right) \theta }{n},\dfrac{n\theta }{n}\right\}  \\ 
\forall k\in \left[ \left[ 1,n\right] \right] :p\left( I_{n}=\dfrac{k\theta 
}{n}\right) =\dfrac{1}{n}%
\end{array}%
\right. 
\end{equation*}%
où $p$ désigne la probabilité. De plus $I_{n}$ et $D$ (la durée aléatoire de
la communication) sont indépendantes.

\begin{enumerate}
\item Pour tout réel positif $t,$ rappeler quelle est l'expression de $%
p\left( D>t\right) $ en fonction de $t$ et de $\alpha $.

\item En déduire, pour $k$ élément de $\left[ \left[ 1,n\right] \right] $ la
probabilité conditionnelle de $\left\{ D+I_{n}>\theta \right\} $ sachant $%
\left\{ I_{n}=\dfrac{k\theta }{n}\right\} .$

\item Démontrer l'égalité suivante : 
\begin{equation*}
p\left( D+I_{n}>\theta \right) =\dfrac{1}{n}\left( \dfrac{1-e^{-\alpha
\theta }}{1-e^{-\dfrac{\alpha \theta }{n}}}\right)
\end{equation*}

\item Déterminer : 
\begin{equation*}
\lim_{n\rightarrow +\infty }p\left( D+I_{n}>\theta \right)
\end{equation*}
\end{enumerate}

\subsubsection*{II. B) Étude de l'encombrement du standard à l'instant $%
\protect\theta $}

Dans cette partie on définit les nombres réels $p$ et $q$ par : 
\begin{equation*}
p=\dfrac{1-e^{-\alpha \theta }}{\alpha \theta }\text{ et }q=1-p
\end{equation*}%
On suppose désormais que la probabilité qu'une communication reçue dans
l'intervalle de temps $\left[ 0,\theta \right] $ se poursuive au-delà de
l'instant $\theta $ est égale à $p$.\newline
On note $N_{\theta }$ la variable aléatoire réelle égale au nombre de
communications reçues dans l'intervalle de temps $\left[ 0,\theta \right] $
et l'on suppose que $N_{\theta }$ suit une loi de Poisson de paramètre $%
\theta $.\newline
On note$C_{\theta }$ la variable aléatoire réelle égale au nombre de
communications reçues dans l'intervalle de temps $\left[ 0,\theta \right] $
qui se poursuivent au-delà de l'instant $\theta $\newline
Les instants aléatoires où les communications se terminent sont mutuellement
indépendant.

\begin{enumerate}
\item Loi de probabilité de $C_{\theta }$

\begin{enumerate}
\item Soit $r$ un entier naturel. Quelle est la loi conditionnelle de $%
C_{\theta }$ sachant que $\left\{ N_{\theta }=r\right\} $ ?

\item Démontrer que l'on a : 
\begin{equation*}
\forall r\in \mathbb{N\;\;\forall }k\in \left[ \left[ 0,r\right] \right]
,\quad p\left( \left\{ C_{\theta }=k\right\} \cap \left\{ N_{\theta
}=r\right\} \right) =\dfrac{e^{-\theta }\left( p\theta \right) ^{k}\left(
q\theta \right) ^{r-k}}{k!\left( r-k\right) !}
\end{equation*}

\item En déduire, pour tout entier naturel $k$, une expression simple de $%
p\left( C_{\theta }=k\right) $ en fonction de $k$, $p,$ et $\theta $. Quelle
est la loi de probabilité de $C_{\theta }?$
\end{enumerate}

\item Étude de l'espérance de $C_{\theta }$

\begin{enumerate}
\item Déterminer l'expression de $E\left( C_{\theta }\right) $ en fonction
de $\theta $ et de $\alpha $.

\item Quelle est la limite de $E\left( C_{\theta }\right) $ lorsque $\theta $
tend vers$+\infty $ ? Vérifier qu'elle majore $E\left( C_{\theta }\right) $.
\end{enumerate}
\end{enumerate}




\section{correction}


$\alpha $ est un r\'{e}el strictement positif. Pour tout $n\in \mathbb{N}$
on pose : 
\begin{equation*}
u_{n}\left( \alpha \right) =\frac{n!}{\prod\limits_{k=0}^{n}\left( \alpha
+k\right) }
\end{equation*}

\begin{enumerate}
\item Etude de la convergence de la suite $\left( u_{n}\left( \alpha \right)
\right) _{n\in \mathbb{N}}$

\begin{enumerate}
\item Pour tout $n,$ $u_{n}\left( \alpha \right) \geq 0$ et 
\begin{eqnarray*}
u_{n+1}\left( \alpha \right) -u_{n}\left( \alpha \right)  &=&\frac{\left(
n+1\right) !}{\prod\limits_{k=0}^{n+1}\left( \alpha +k\right) }-\frac{n!}{%
\prod\limits_{k=0}^{n}\left( \alpha +k\right) } \\
&=&\frac{n!}{\prod\limits_{k=0}^{n+1}\left( \alpha +k\right) }\left[
n+1-\left( \alpha +n+1\right) \right] <0
\end{eqnarray*}%
Donc la suite est d\'{e}croissante et moinor\'{e}e par $0$.

\textsl{Conclusion : }\fbox{ $\left( u_{n}\left( \alpha \right) \right)
_{n\in \mathbb{N}}$ est monotonne et convergente}

La somme partielle $\sum_{n=0}^{N}u_{n}\left( \alpha \right) -u_{n+1}\left(
\alpha \right) =u_{0}\left( \alpha \right) -u_{N+1}\left( \alpha \right) $ a
donc un e limite finie quand $N$ tende vers $+\infty .$

\textsl{Conclusion : }\fbox{la s\'{e}rie de terme g\'{e}n\'{e}ral $\left(
u_{n}\left( \alpha \right) -u_{n+1}\left( \alpha \right) \right) $ converge}

On note $\ell \left( \alpha \right) $ la limite de la suite $\left(
u_{n}\left( \alpha \right) \right) _{n\in \mathbb{N}}$

\item On suppose que $\ell \left( \alpha \right) $ est non nulle. 

(par factorisation des pr\'{e}pond\'{e}rants) On a vu que 
\begin{eqnarray*}
u_{n}\left( \alpha \right) -u_{n+1}\left( \alpha \right)  &=&\frac{n!\alpha 
}{\prod\limits_{k=0}^{n+1}\left( \alpha +k\right) } \\
&=&\frac{\alpha u_{n}\left( \alpha \right) }{\alpha +n+1} \\
&=&\frac{\alpha \ell \left( \alpha \right) }{n}\underset{\rightarrow 1}{%
\frac{u_{n}\left( \alpha \right) /\ell \left( \alpha \right) }{1+\left(
\alpha +1\right) /n}} \\
&&\underset{n\rightarrow +\infty }{\thicksim }\frac{\alpha \ell \left(
\alpha \right) }{n}
\end{eqnarray*}

\item Or la s\'{e}rie (Riemann) $\sum_{n\geq 1}\frac{1}{n}$ diverge. et par 
\'{e}quivalence de temres positifs, la s\'{e}rie $\sum_{n\geq 1}\left(
u_{n}\left( \alpha \right) -u_{n+1}\left( \alpha \right) \right) $ diverge 
\'{e}galement. FAUX ! DOnc $\ell \left( \alpha \right) $ n'est pas non nul.

\textsl{Conclusion : }\fbox{$\ell \left( \alpha \right) =0$}
\end{enumerate}

\item Dans cette question : $\alpha \in ]0,1]$

\begin{enumerate}
\item On a alors 
\begin{eqnarray*}
\mathbb{\;}u_{n}\left( \alpha \right)  &=&\frac{n!}{\prod\limits_{k=0}^{n}%
\left( \alpha +k\right) } \\
&=&\frac{1\cdot 2\cdots \cdot n}{\alpha \cdot \left( \alpha +1\right) \cdots
\left( \alpha +n\right) } \\
&=&\frac{1}{\alpha }\frac{2}{\alpha +1}\cdots \frac{n}{\alpha +n-1}\frac{1}{%
n+\alpha } \\
&\geq &\frac{1}{n+\alpha }
\end{eqnarray*}%
car $\alpha +i\leq 1+i$ et donc $\dfrac{i+1}{\alpha +i}\geq 1$

$N.B.$ il est plus naturel de le faire par r\'{e}currence avec 
\begin{eqnarray*}
u_{n+1}\left( \alpha \right)  &=&\frac{n+1}{\alpha +n+1}u_{n}\left( \alpha
\right)  \\
&\geq &\frac{1}{n+\alpha }\frac{n+1}{\alpha +n+1} \\
&\geq &\frac{1}{\alpha +n+1}
\end{eqnarray*}%
car $n+\alpha \leq n+1$

\textsl{Conclusion : }\fbox{$\forall n\in \mathbb{N\;\;}u_{n}\left( \alpha
\right) \geq \dfrac{1}{n+\alpha }$}

\item On a finalement $u_{n}\left( \alpha \right) \geq \dfrac{1}{n+\alpha }%
\geq \dfrac{1}{n+1}$ dont la s\'{e}rie (de Riemann apr\`{e}s r\'{e}%
indexation) diverge.

et par minoration de teremes positifs, 

\textsl{Conclusion : }\fbox{la s\'{e}rie de terme g\'{e}n\'{e}ral $%
u_{n}\left( \alpha \right) $ diverge}
\end{enumerate}

\item On pose pour tout entier naturel $n$ : 
\begin{equation*}
I_{n}\left( \alpha \right) =\int_{0}^{+\infty }e^{-\alpha t}\left(
1-e^{-t}\right) ^{n}dt
\end{equation*}

\begin{enumerate}
\item $\left( 1-e^{-t}\right) ^{n}\rightarrow 1$ quand $t\rightarrow +\infty 
$ donc $e^{-\alpha t}\left( 1-e^{-t}\right) ^{n}\sim e^{-\alpha t}$ dont la s%
\'{e}rie converge ($\alpha >1$ )

et par \'{e}quivalence de termes positifs, 

\textsl{Conclusion : }\fbox{ l'int\'{e}grale g\'{e}n\'{e}ralis\'{e}e $%
I_{n}\left( \alpha \right) $ converge}

\begin{eqnarray*}
I_{0}\left( \alpha \right)  &=&\int_{0}^{+\infty }e^{-\alpha t}dt \\
\int_{0}^{M}e^{-\alpha t}dt &=&\left[ \frac{-1}{\alpha }e^{-\alpha t}\right]
_{0}^{M}=\frac{-1}{\alpha }e^{-\alpha M}+\frac{1}{\alpha } \\
&\rightarrow &\frac{1}{\alpha }
\end{eqnarray*}%
\textsl{Conclusion : }\fbox{$I_{0}\left( \alpha \right) =\dfrac{1}{\alpha }$}

\item Soit un r\'{e}el $x$ strictement positif et $n\in \mathbb{N}^{\ast }$
(pour la d\'{e}riv\'{e}e et pour $\left[ {}\right] $ en $0$ )

$u\left( t\right) =\left( 1-e^{-t}\right) ^{n}:u^{\prime }\left( t\right)
=n\left( 1-e^{-t}\right) ^{n-1}e^{-t}$

$v^{\prime }\left( t\right) =e^{-\alpha t}:v\left( t\right) =\dfrac{-1}{%
\alpha }e^{-\alpha t}$ avec $u$ et $v$ $C^{1}$%
\begin{eqnarray*}
\int_{0}^{x}e^{-\alpha t}\left( 1-e^{-t}\right) ^{n}dt &=&\left[ \dfrac{-1}{%
\alpha }e^{-\alpha t}\left( 1-e^{-t}\right) ^{n}\right] _{0}^{x}-\int_{0}^{x}%
\dfrac{-n}{\alpha }e^{-t}e^{-\alpha t}\left( 1-e^{-t}\right) ^{n-1}dt \\
&=&\dfrac{-1}{\alpha }e^{-\alpha x}\left( 1-e^{-x}\right) ^{n}+\dfrac{n}{%
\alpha }\int_{0}^{x}e^{-\left( \alpha +1\right) t}\left( 1-e^{-t}\right)
^{n-1}dt \\
&\rightarrow &\dfrac{n}{\alpha }I_{n-1}\left( \alpha +1\right) 
\end{eqnarray*}

\textsl{Conclusion : }\fbox{$I_{n}\left( \alpha \right) =\dfrac{n}{\alpha }%
I_{n-1}\left( \alpha +1\right) $ pour tout $n\in \mathbb{N}^{\ast }$}

\item On a alors $I_{n+1}\left( \alpha \right) =\dfrac{n+1}{\alpha }%
I_{n}\left( \alpha +1\right) $ pour tout $n\in \mathbb{N}$ et par r\'{e}%
currence 

\textsl{Conclusion : }\fbox{$\forall n\in \mathbb{N\;\;}I_{n}\left( \alpha
\right) =u_{n}$} car $u_{n+1}\left( \alpha \right) =\dfrac{n+1}{\alpha }%
u_{n}\left( \alpha +1\right) $ et $u_{0}=I_{0}$
\end{enumerate}

\item On suppose d\'{e}sormais que $\alpha >1$

\begin{enumerate}
\item Par r\'{e}currence (avec $I_{n}\left( \alpha \right) =\dfrac{n}{\alpha 
}I_{n-1}\left( \alpha +1\right) $  pour tout $n\in \mathbb{N}$ )

Pour $N=0$ on a 
\begin{eqnarray*}
\sum_{n=0}^{0}I_{n}\left( \alpha \right)  &=&I_{0}\left( \alpha \right) =%
\frac{1}{\alpha } \\
I_{1}\left( \alpha -1\right)  &=&\dfrac{1}{\alpha -1}I_{0}\left( \alpha
\right) =\dfrac{1}{\alpha -1}\frac{1}{\alpha }\text{ donc } \\
\frac{1}{\alpha -1}-I_{1}\left( \alpha -1\right)  &=&\frac{1}{\alpha -1}-%
\dfrac{1}{\alpha -1}\frac{1}{\alpha } \\
&=&\frac{1}{\alpha -1}\left( 1-\frac{1}{\alpha }\right) =\frac{1}{\alpha }
\end{eqnarray*}
donc $\displaystyle\sum_{n=0}^{0}I_{n}\left( \alpha \right) =\frac{1}{\alpha
-1}-I_{1}\left( \alpha -1\right) $

Soit $n$ $\in \mathbb{N}$ tel que 
\begin{equation*}
\sum_{n=0}^{N}I_{n}\left( \alpha \right) =\frac{1}{\alpha -1}-I_{N+1}\left(
\alpha -1\right) 
\end{equation*}%
alors 
\begin{eqnarray*}
\sum_{n=0}^{N+1}I_{n}\left( \alpha \right)  &=&\sum_{n=0}^{N}I_{n}\left(
\alpha \right) +I_{N+1}\left( \alpha \right)  \\
&=&\frac{1}{\alpha -1}-I_{N+1}\left( \alpha -1\right) +I_{N+1}\left( \alpha
\right) 
\end{eqnarray*}%
la relation de r\'{e}currence ci-dessus ne m\`{e}ne \`{a} rien... mais sous
forme d'int\'{e}grale :

\begin{eqnarray*}
I_{N+1}\left( \alpha \right) -I_{N+1}\left( \alpha -1\right) 
&=&\int_{0}^{1}e^{-\alpha t}\left( 1-e^{-t}\right)
^{N+1}dt-\int_{0}^{1}e^{-\left( \alpha -1\right) t}\left( 1-e^{-t}\right)
^{N+1}dt \\
&=&\int_{0}^{1}e^{-\left( \alpha -1\right) t}\left( e^{-t}-1\right) \left(
1-e^{-t}\right) ^{n}dt \\
&=&-I_{N+2}\left( \alpha -1\right) 
\end{eqnarray*}

\textsl{Conclusion : }\fbox{$\displaystyle\sum_{n=0}^{N}I_{n}\left( \alpha
\right) =\frac{1}{\alpha -1}-I_{N+1}\left( \alpha -1\right) $ pouer tout
eniter $n$}

\item Comme $u_{n}\left( \alpha \right) =I_{n}\left( \alpha \right) $ et que 
$u_{n}\left( \alpha \right) \rightarrow \ell \left( \alpha \right) =0$ alors 
$\displaystyle\sum_{n=0}^{N}I_{n}\left( \alpha \right) \rightarrow \dfrac{1}{%
\alpha -1}$

\textsl{Conclusion : }\fbox{ $\sum\limits_{n=0}^{+\infty }u_{n}\left( \alpha
\right) $ converge et vaut $\dfrac{1}{\alpha -1}$.}
\end{enumerate}
\end{enumerate}

\begin{center}
{\LARGE EXERCICE 2}
\end{center}

\begin{enumerate}
\item Pour montrer que 0 , $\displaystyle
\frac{1}{2}$ et $1$ sont \textbf{des} valeurs propres de $A$, il suffit de
trouver (ou de donner) des vecteurs porpres associ\'{e}s.

$A\left( 
\begin{array}{c}
-2 \\ 
\frac{3}{2} \\ 
1%
\end{array}
\right) =0=0\left( 
\begin{array}{c}
-2 \\ 
\frac{3}{2} \\ 
1%
\end{array}
\right) $ prouve par exemple que $0$ est valeur propre de $A$ car $u=\left( 
\begin{array}{c}
-2 \\ 
\frac{3}{2} \\ 
1%
\end{array}
\right) $

On peut aussi r\'{e}diger la recherche de vecteurs propres :

$\displaystyle
\left( A-\frac{1}{2}I\right) U=0\Leftrightarrow \left\{ 
\begin{array}{c}
\frac{1}{2}x+2z=0 \\ 
\frac{3}{2}x-\frac{5}{2}y+6z=0 \\ 
\frac{1}{2}x-y+2z=0%
\end{array}
\right. $ $\Leftrightarrow \left\{ 
\begin{array}{c}
x=-4z \\ 
y=0 \\ 
y=0%
\end{array}
\right. \Leftrightarrow \left\{ 
\begin{array}{c}
x=-4z \\ 
y=0%
\end{array}
\right. $

donc $1/2$ est valeur propre car $v=\left( 
\begin{array}{c}
-4 \\ 
0 \\ 
1%
\end{array}
\right) $ est un vecteur propre associ\'{e}.

$\left( A-1I\right) U=0\Leftrightarrow \left\{ 
\begin{array}{c}
2z=0 \\ 
\frac{3}{2}x-3y+6z=0 \\ 
\frac{1}{2}x-y+\frac{3}{2}z=0%
\end{array}
\right. \Leftrightarrow \left\{ 
\begin{array}{c}
z=0 \\ 
x=2y \\ 
0=0%
\end{array}
\right. \Leftrightarrow \left\{ 
\begin{array}{c}
z=0 \\ 
x=2y%
\end{array}
\right. $

Donc $1$ est une valeur propre car $w=\left( 
\begin{array}{c}
2 \\ 
1 \\ 
0%
\end{array}
\right) $ est un vecteur propre associ\'{e}.

Donc $\displaystyle0,\ \frac{1}{2}$ et $1$ sont \textbf{des} valeurs propres
de $A$.

Pour prouver que ce sont \textbf{les }valeurs propres de $A,$ il reste \`{a}
prouver que $A$ n'en a pas d'autres.

Or $A$ est d'ordre 3 donc il a au plus 3 valeurs propres distinctes.

Finalement $\displaystyle0,~\frac{1}{2}$ et $1$ sont \textbf{les} valeurs
propres de $A$.

\item Justifier les affirmations suivantes :

\begin{itemize}
\item Comme on a trois valeurs propres distinctes pour une matrice d'ordre
3, alors la famille des colonnes propres associ\'{e}es $\left( u,v,w\right) $
est une base.

Donc pour toute colonne (et en particulier pour $B$ ) il existe un unique
triplet $\left( \alpha ,\beta ,\gamma \right) $ de coordonn\'{e}es de $%
\mathbb{R}^{3}$ tel que $B=\alpha u+\beta v+\gamma w$

\item et de m\^{e}me, pour toute colonne $X_{n}$ il existe un unique triplet 
$\left( \alpha _{n},\beta _{n},\gamma _{n}\right) $ de $\mathbb{R}^{3}$ tel
que $X_{n}=\alpha _{n}u+\beta _{n}v+\gamma _{n}w$
\end{itemize}

\item Remarquer que la formule est donn\'{e}e pour $n\in \mathbb{N}^{\ast }$
donc

\begin{itemize}
\item On commence pour $n=1:$On a $\alpha _{1},\;\beta _{1}$ et $\gamma _{1} 
$ qui sont d\'{e}finis par : $X_{1}=\alpha _{1}u+\beta _{1}v+\gamma _{1}w$

Or $X_{1}=AX_{0}+B=A\left( \alpha _{0}u+\beta _{0}v+\gamma _{0}w\right)
+\alpha u+\beta v+\gamma w.$

Et comme $u,$ $v$ et $w$ sont des colonnes propres :

$\displaystyle
X_{1}=\alpha _{0}0u+\beta _{0}\frac{1}{2}v+\gamma _{0}w+\alpha u+\beta
v+\gamma w=\alpha u+\frac{1}{2}\left( \beta _{0}+2\beta \right) v+\left(
\gamma _{0}+\gamma \right) w$

Donc par unicit\'{e} des coordonn\'{e}e :$\left\{ 
\begin{array}{l}
\alpha _{1}=\alpha \\ 
\beta _{1}=\left( \frac{1}{2}\right) \left( \beta _{0}-2\beta \right) +2\beta
\\ 
\gamma _{n}=\gamma _{0}+1\gamma%
\end{array}
\right. $

\item Soit $n\in \mathbb{N}^{*}$ tel que $\left\{ 
\begin{array}{l}
\alpha _{n}=\alpha \\ 
\beta _{n}=\left( \frac{1}{2}\right) ^{n}\left( \beta _{0}-2\beta \right)
+2\beta \\ 
\gamma _{n}=\gamma _{0}+n\gamma%
\end{array}
\right. $ alors

$X_{n+1}=AX_{n}+B=\beta _{n}\frac{1}{2}v+\gamma _{n}1w+\alpha u+\beta
v+\gamma w=\alpha u+\frac{1}{2}\left( \beta _{n}+2\beta \right) v+\left(
\gamma _{n}+\gamma \right) w$

Donc$\;\left\{ 
\begin{array}{l}
\alpha _{n+1}=\alpha \\ 
\beta _{n+1}=\frac{1}{2}\left( \beta _{n}+2\beta \right) =\left( \frac{1}{2}%
\right) ^{n+1}\left( \beta _{0}-2\beta \right) +2\beta \\ 
\gamma _{n}=\gamma _{n}+\gamma =\gamma _{0}+n\gamma%
\end{array}
\right. $

\item Donc pour tout entier $n\ge 1$, les formules de r\'{e}currence sont
bien v\'{e}rifi\'{e}e.
\end{itemize}

\item Soit $\left( a_{n}\right) _{n\in \mathbb{N}},\;\left( b_{n}\right)
_{n\in \mathbb{N}}$ et $\left( c_{n}\right) _{n\in \mathbb{N}}$ les suites r%
\'{e}elles telles que :

\begin{enumerate}
\item On a $X_{n}=\alpha _{n}u+\beta _{n}v+\gamma _{n}w=\alpha _{n}\left( 
\begin{array}{c}
-2 \\ 
\frac{3}{2} \\ 
1%
\end{array}
\right) +\beta _{n}\left( 
\begin{array}{c}
-4 \\ 
0 \\ 
1%
\end{array}
\right) +\gamma _{n}\left( 
\begin{array}{c}
2 \\ 
1 \\ 
0%
\end{array}
\right) =\allowbreak \left( 
\begin{array}{c}
-2\alpha _{n}-4\beta _{n}+2\gamma _{n} \\ 
\frac{3}{2}\alpha _{n}+\gamma _{n} \\ 
\alpha _{n}+\beta _{n}%
\end{array}
\right) $

Donc $a_{n}=-2\alpha _{n}-4\beta _{n}+2\gamma _{n}:b_{n}=\frac{3}{2}\alpha
_{n}+\gamma _{n}$ et $c_{n}=\alpha _{n}+\beta _{n}$

Et comme $\alpha _{n}\rightarrow \alpha $ que $\beta _{n}\rightarrow 2\beta $
et $\gamma _{n}=\gamma _{0}+n\gamma \bar{\rightarrow}\pm \infty $ si $\gamma
\ne 0$ et $\gamma _{0}$ si $\gamma =0$

alors $a_{n}$ et $b_{n}$ donc $X_{n}$ convergent si et seulement si $\gamma
=0$

\item On r\'{e}sout donc $B=\alpha u+\beta v+\gamma w\Leftrightarrow \alpha
\left( 
\begin{array}{c}
-2 \\ 
\frac{3}{2} \\ 
1%
\end{array}
\right) +\beta \left( 
\begin{array}{c}
-4 \\ 
0 \\ 
1%
\end{array}
\right) +\gamma \left( 
\begin{array}{c}
2 \\ 
1 \\ 
0%
\end{array}
\right) =\left( 
\begin{array}{c}
x \\ 
y \\ 
z%
\end{array}
\right) $

$\Leftrightarrow \left\{ 
\begin{array}{cc}
-2\alpha -4\beta +2\gamma =x & L_{1}+2L_{3} \\ 
\frac{3}{2}\alpha +\gamma =y & L_{2}-\frac{3}{2}L_{3} \\ 
\alpha +\beta =z & L_{3}%
\end{array}
\right. \Leftrightarrow \left\{ 
\begin{array}{cc}
-2\beta +2\gamma =x+2z & L_{1}-\frac{4}{3}L_{2} \\ 
-\frac{3}{2}\beta +\gamma =y-\frac{3}{2}z &  \\ 
\alpha +\beta =z & 
\end{array}
\right. $

$\Leftrightarrow \left\{ 
\begin{array}{cc}
\frac{1}{3}\gamma =x+2z-\frac{4}{3}\left( y-\frac{3}{2}z\right) & =x-\frac{4%
}{3}y+4z \\ 
-\frac{3}{2}\beta +\gamma =y-\frac{3}{2}z &  \\ 
\alpha +\beta =z & 
\end{array}
\right. $

Donc $\gamma =3x-4y+12z$ et donc $\left( X_{n}\right) _{n\in \mathbb{N}}$
converge si et seulement si $3x-4y+12z=0$
\end{enumerate}

\item Si le couple $(A,B)$ admet une position d'\'{e}quilibre stable alors
la suite $\left( X_{n}\right) _{n\in \mathbb{N}}$ converge. Donc $\gamma =0$

La limite de la suite $\left( a_{n}\right) $ est donc $-2\alpha +4\beta
+2\gamma _{0},$ celle de la suite $\left( b_{n}\right) $ est $\frac{3}{2}%
\alpha +\gamma _{0}$ et celle de la suite $\left( c_{n}\right) $ est $\alpha
+\beta $

Comme $x,\;y$ et $z$ sont fix\'{e}es par $B$, alors $\alpha $, $\beta $ et $%
\gamma $ le sont aussi. Les limites de $a$ et $b$ d\'{e}pendent donc de $%
\gamma _{0}.$

Donc, quelle que soit la valeur de $B$, le couple $\left( A,B\right) $
n'admet pas de position d'\'{e}quilibre stable.
\end{enumerate}

\begin{center}
{\Large Exercice 3}
\end{center}

Dans tout le probl\`{e}me (qui comporte deux parties ind\'{e}pendantes), on
suppose que la dur\'{e}e, exprim\'{e}e en minutes, d'une communication t\'{e}%
l\'{e}phonique est une variable al\'{e}atoire r\'{e}elle $D$ qui suit la loi
exponentielle de param\`{e}tre $\alpha $

\textbf{I Comparaison de deux tarifications}

\begin{enumerate}
\item Pour ses communications, on propose \`{a} l'utilisateur d'une ligne t%
\'{e}l\'{e}phonique deux tarifications $T_{1}$ et $T_{2}$, exprim\'{e}es en
francs, d\'{e}finies de la fa\c{c}on suivante :
\end{enumerate}

\begin{itemize}
\item $T_{1}=aD$, o\`{u} $a$ est un nombre r\'{e}el strictement sup\'{e}%
rieur \`{a} 1 qui repr\'{e}sente le prix d'une minute de communication

\item $T_{2}$ est \`{a} valeurs dans $\mathbb{N}^{*}$ et, pour tout $n$
entier naturel non nul : $\{T_{2}=n\}=\{n-1<D\le n\}$
\end{itemize}

\begin{enumerate}
\item Comme $D$ suit une loi exponentielle, On a $E\left( D\right) =1/\alpha 
$ donc $E(T_{1})=aE\left( D\right) =a/\alpha $

\item On a pour tout entier $n\in \mathbb{N}^{*}:$%
\begin{eqnarray*}
p\left( T_{2}=n\right) &=&p\left( n-1<D\le n\right) \\
&=&\int_{n-1}^{n}\alpha e^{-\alpha t}dt=\left[ -e^{-\alpha t}\right]
_{n-1}^{n} \\
&=&e^{-\alpha \left( n-1\right) }-e^{-\alpha n} \\
&=&\left( e^{-\alpha }\right) ^{n-1}\left( 1-e^{-\alpha }\right)
\end{eqnarray*}
Donc $T_{2}\hookrightarrow \mathcal{G}\left( 1-e^{-\alpha }\right) $ et $%
E(T_{2})=\displaystyle
\frac{1}{1-e^{-\alpha }}$

\item On pose : 
\begin{equation*}
\left\{ 
\begin{array}{c}
\forall t\in \mathbb{R}_{+}^{*}:\displaystyle \varphi \left( t\right) =\frac{%
t}{1-e^{-t}} \\ 
\varphi \left( 0\right) =1%
\end{array}
\right.
\end{equation*}

\begin{enumerate}
\item $\varphi $ est de calsse $C^{1}$ sur $]0,+\infty [$ comme quotient de
fonction $C^{1}$

En $0$ : on peut passer par le th\'{e}or\`{e}me de prolongement qui demande
la continuit\'{e} de $\varphi $ d'abord puis la limite de la d\'{e}riv\'{e}e.

ou revenir au taux d'accroissement puis prouver la continuit\'{e} de la d%
\'{e}riv\'{e}e.

par le prolongement :

comme $e^{x}-1\thicksim x$ quand $x\rightarrow 0$ alors $e^{-t}-1\thicksim
-t $ donc $\displaystyle
\frac{-t}{e^{-t}-1}\rightarrow 1$ et $\varphi \left( t\right) \rightarrow 1$
donc $\varphi $ est continue sur $\mathbb{R}^{+}$

$\varphi $ est d\'{e}rivable sur $\left] 0,+\infty \right[ $ et pour avoir
la limite de $\varphi ^{\prime }$ en $0$ on efffectue un d\'{e}veloppement
limit\'{e} :

(onsubsitue $-t$ \`{a} x dans le DL $e^{x}=1+x+x^{2}/2+x^{2}\varepsilon
\left( x\right) $ avec $\varepsilon \left( x\right) \rightarrow 0.$ On prend
le DL d'ordre 2 car celui d'ordre 1 au num\'{e}rateur laissaitune forme ind%
\'{e}termin\'{e}e $\varepsilon \left( t\right) /t\dots )$%
\begin{eqnarray*}
\varphi ^{\prime }\left( t\right) &=&\frac{1-e^{-t}-t\left( -e^{-t}\left(
-1\right) \right) }{\left( 1-e^{-t}\right) ^{2}} \\
&=&\frac{1-\left( 1+t\right) e^{-t}}{\left( -1+e^{-t}\right) ^{2}} \\
&=&\frac{1-\left( 1+t\right) \left( 1-t+\displaystyle\frac{t^{2}}{2}%
+t^{2}\varepsilon \left( t\right) \right) }{\left( 1-1-t-t\varepsilon
_{1}\left( t\right) \right) ^{2}} \\
&=&\frac{1-\left( 1-t^{2}/2+t^{2}\varepsilon _{2}\left( t\right) \right) }{%
\left( -t-t\varepsilon _{1}\left( t\right) \right) ^{2}} \\
&=&\frac{t^{2}/2-t^{2}\varepsilon _{2}\left( t\right) }{t^{2}\left(
1+\varepsilon _{1}\left( t\right) \right) ^{2}}=\frac{\displaystyle\frac{1}{2%
}-\varepsilon _{2}\left( t\right) }{\left( 1+\varepsilon _{1}\left( t\right)
\right) ^{2}} \\
&\rightarrow &\frac{1}{2}
\end{eqnarray*}

avec $\varepsilon ,\;\varepsilon _{1}$ et $\varepsilon _{2}$ qui tendent
vars $0$ en $0.$

Comme $\varphi $ est continue en $0$ et que $\varphi ^{\prime }$ $%
\rightarrow 1/2$ alors $\varphi $ est d\'{e}rivable en $0$, $\varphi
^{\prime }\left( 0\right) =1/2$ et $\varphi ^{\prime }$ est continue en $0.$

Donc $\varphi $ est de classe $C^{1}$ en $0$ donc sur $[0,+\infty [$

\item On a pour $t>0:\displaystyle
\varphi ^{\prime }\left( t\right) =\frac{1-\left( 1+t\right) e^{-t}}{\left(
-1+e^{-t}\right) ^{2}}=\frac{\psi \left( t\right) }{\left( -1+e^{-t}\right)
^{2}}$ qui est donc du signe de $\psi \left( t\right) $

$\psi $ est d\'{e}rivable sur $\mathbb{R}$ et $\psi ^{\prime }\left(
t\right) =-e^{-t}-\left( 1+t\right) e^{-t}\left( -1\right) =te^{-t}$ du
signe de $t$

Donc $\psi $ est strictement croissante sur $\mathbb{R}^{+}$ et comme $\psi
\left( 0\right) =0$ on a $\psi >0$ sur $\left] 0,+\infty \right[ $

\textbf{Remarque : }$f$ est strictement croissante si $f$ est continue sur
un intervalle $I$ et $f^{\prime }>0$ sauf en un nombre fini de points de $I$
. Il n'est pas n\'{e}cessaire qu'elle soit d\'{e}rivable ou que la d\'{e}riv%
\'{e}e soit strictement positive sur tout l'intervalle)

Finalement $\varphi $ est strictement croissante et continue donc bijective
de $[0,+\infty [$ dans

$[\varphi \left( 0\right) ,\lim_{+\infty }\varphi [=[1,+\infty [$

($\varphi $ ne donne pas de forme ind\'{e}termin\'{e}e en $+\infty $)
\end{enumerate}

\item Comparaison des tarifications

\begin{enumerate}
\item Comme $a>1$ on a $a\in [1,+\infty [$ donc il existe un unique $\alpha
_{0}\in [0,+\infty [$ tel que $\varphi \left( \alpha _{0}\right) =a$

\item Il faut ici comparer les couts moyens des deux facturations :

\begin{equation*}
\frac{E\left( T_{2}\right) }{E\left( T_{1}\right) }=\frac{\frac{1}{%
1-e^{-\alpha }}}{\frac{a}{\alpha }}=\frac{1}{a}\frac{\alpha }{1-e^{-\alpha }}%
=\frac{\varphi \left( \alpha \right) }{a}
\end{equation*}

Comme la dur\'{e}e moyenne de communication est $1/\alpha $,

\begin{itemize}
\item si cette dur\'{e}e moyenne ($1/\alpha $) est sup\'{e}rieure \`{a} $%
1/\alpha _{0}$ alors $\alpha <\alpha _{0}$ et comme $\varphi $ est
strictement croissante sur $\mathbb{R}^{+}$ on a $\varphi \left( \alpha
\right) >\varphi \left( \alpha _{0}\right) =a$ donc $E\left( T_{2}\right)
>E\left( T_{1}\right) $ et la premi\`{e}re tarifivcation est la plus
avantageuse (la moins ch\`{e}re)

\item Inversement, si elle est sup\'{e}rieure \`{a} $1/\alpha _{0}$ la
seconde tarification est la plus avantageuse.
\end{itemize}

La tarificatin \`{a} la seconde estdonc interressante pour les
communications longues
\end{enumerate}
\end{enumerate}

\textbf{II \'{E}tude d'un standard t\'{e}l\'{e}phonique }

Dans toute cette partie, $\theta $ est un nombre r\'{e}el strictement
positif repr\'{e}sentant un temps exprim\'{e} en minutes. Un standard t\'{e}l%
\'{e}phonique de capacit\'{e} illimit\'{e}e re\c{c}oit des communications t%
\'{e}l\'{e}phoniques entre l'instant 0 et l'instant $\theta $ inclus.

\textbf{II A Cas d'une seule communication}

On d\'{e}signe par $n$ un entier naturel non nul. L'instant o\`{u} d\'{e}%
bute la communication est une variable al\'{e}atoire r\'{e}elle $I_{n}$
telle que : 
\begin{equation*}
\left\{ 
\begin{array}{c}
\displaystyle I_{n}\left( \Omega \right) =\left\{ \frac{\theta }{n},\frac{%
2\theta }{n},\dots ,\frac{\left( n-1\right) \theta }{n},\frac{n\theta }{n}%
\right\} \\ 
\displaystyle \forall k\in \left[ \left[ 1,n\right] \right] :p\left( I_{n}=%
\frac{k\theta }{n}\right) =\frac{1}{n}%
\end{array}
\right.
\end{equation*}
o\`{u} $p$ d\'{e}signe la probabilit\'{e}. De plus $I_{n}$ et $D$ (la dur%
\'{e}e al\'{e}atoire de la communication) sont ind\'{e}pendantes.

\begin{enumerate}
\item Soit $f$ la densit\'{e} de la loi exponentielle de param\`{e}tre $%
\alpha $.

On a $p\left( D>t\right) =\int_{t}^{+\infty }f\left( x\right) dx$ int\'{e}%
grale impropre en $+\infty ;$ Et comme $t\ge 0$%
\begin{eqnarray*}
\int_{t}^{A}\alpha e^{-\alpha x}dx=\left[ -e^{-\alpha x}\right]
_{x=t}^{A}=e^{-\alpha t}-e^{-\alpha A} \\
\rightarrow e^{-\alpha t}
\end{eqnarray*}
donc $p\left( D>t\right) =e^{-\alpha t}$

\item On a donc 
\begin{eqnarray*}
p\left( D+I_{n}>\theta \;/\;I_{n}=\frac{k\theta }{n}\right) =p\left(
D>\theta -\frac{k\theta }{n}\;/\;I_{n}=\frac{k\theta }{n}\right) \\
=p\left( D>\theta -\frac{k\theta }{n}\right) \\
=p\left( D>\theta \left( 1-\frac{k}{n}\right) \right) \\
=e^{-\alpha \theta \left( 1-\displaystyle\frac{k}{n}\right) }
\end{eqnarray*}
on peut supprimer le conditinnment car $D$ et $I_{n}$ sont ind\'{e}pendantes.

\item On utilise alors la formule des probabilit\'{e} totales avec pour syst%
\`{e}me complet d'\'{e}v\'{e}nements : $\displaystyle
\left( I_{n}=\frac{k\theta }{n}\right) _{k\in \left[ \left[ 1,n\right] %
\right] }$

on a alors 
\begin{eqnarray*}
p\left( D+I_{n}>\theta \right) =\sum_{k=1}^{n}p\left( D+I_{n}>\theta
\;/\;I_{n}=\frac{k\theta }{n}\right) \cdot p\left( I_{n}=\frac{k\theta }{n}%
\right) \\
=\sum_{k=1}^{n}\;e^{-\alpha \theta +\alpha \theta \displaystyle\frac{k}{n}}%
\frac{1}{n}=\frac{e^{-\alpha \theta }}{n}\sum_{k=1}^{n}\;\left( e^{\alpha
\theta /n}\right) ^{k} \\
=\frac{e^{-\alpha \theta }}{n}\left( \frac{e^{\left( n+1\right) \alpha
\theta /n}-1}{e^{\alpha \theta /n}-1}-1\right) \\
=\frac{e^{-\alpha \theta }}{n}\left( \frac{e^{\left( n+1\right) \alpha
\theta /n}-e^{\alpha \theta /n}}{e^{\alpha \theta /n}-1}\right) =\frac{%
e^{-\alpha \theta }e^{\alpha \theta }}{n}\left( \frac{e^{n\alpha \theta /n}-1%
}{e^{\alpha \theta /n}-1}\right) \\
=\frac{1}{n}\left( \frac{\displaystyle 1-e^{-\alpha \theta }}{\displaystyle %
1-e^{-\frac{\alpha \theta }{n}}}\right)
\end{eqnarray*}

\item Comme $e^{x}-1\thicksim x$ quand $x$ $\rightarrow 0$ et que $-\frac{%
\alpha \theta }{n}\rightarrow 0$ quand $n\rightarrow +\infty $ alors $%
e^{-\alpha \theta /n}-1\thicksim -\displaystyle\frac{\alpha \theta }{n}$

et 
\begin{equation*}
\frac{e^{-\displaystyle \alpha \theta /n}-1}{-\alpha \theta /n}=\frac{1-e^{-%
\displaystyle \alpha \theta /n}}{\alpha \theta /n}\rightarrow 1
\end{equation*}
donc 
\begin{eqnarray*}
p\left( D+I_{n}>\theta \right) =\frac{1}{n}\left( \frac{\displaystyle %
1-e^{-\alpha \theta }}{\displaystyle 1-e^{-\frac{\alpha \theta }{n}}}\right)
\\
=\frac{1-e^{-\displaystyle \alpha \theta }}{\alpha \theta }\frac{\alpha
\theta /n}{1-e^{-\displaystyle \alpha \theta /n}} \\
\rightarrow \frac{1-e^{-\displaystyle \alpha \theta }}{\alpha \theta }
\end{eqnarray*}
quand $n\rightarrow +\infty $

c'est la limite de la probabilit\'{e} que l'appel se finisse apr\`{e}s $%
\theta $ pour des d\'{e}buts d'appels \'{e}quir\'{e}partis sur $\left[
0,\theta \right] $ (quand $n$ tend vers $+\infty $ chacun des segments $%
\left[ \frac{k}{n},\frac{k+1}{n}\right] $ \'{e}tait \'{e}quiprobable)
\end{enumerate}

\textbf{II B \'{E}tude de l'encombrement du standard \`{a} l'instant }$%
\theta $

Dans cette partie on d\'{e}finit les nombres r\'{e}els $p$ et $q$ par : 
\begin{equation*}
p=\frac{1-e^{-\alpha \theta }}{\alpha \theta }\text{ \ et \ \ }q=1-p
\end{equation*}
On suppose d\'{e}sormais que la probabilit\'{e} qu'une communication re\c{c}%
ue dans l'intervalle de temps $\left[ 0,\theta \right] $ se poursuive au-del%
\`{a} de l'instant $\theta $ est \'{e}gale \`{a} $p$.

On note $N_{\theta }$ la variable al\'{e}atoire r\'{e}elle \'{e}gale au
nombre de communications re\c{c}ues dans l'intervalle de temps $\left[
0,\theta \right] $ et l'on suppose que $N_{\theta }$ suit une loi de Poisson
de param\`{e}tre $\theta $.

On note$C_{\theta }$ la variable al\'{e}atoire r\'{e}elle \'{e}gale au
nombre de communications re\c{c}ues dans l'intervalle de temps $\left[
0,\theta \right] $ qui se poursuivent au-del\`{a} de l'instant $\theta $

Les instants al\'{e}atoires o\`{u} les communications se terminent sont
mutuellement ind\'{e}pendant.

\begin{enumerate}
\item Loi de probabilit\'{e} de $C_{\theta }$

\begin{enumerate}
\item Quand $N_{\theta }=r:C_{\theta }$ est le \emph{nombre de }comunication%
\textbf{\ }qui se poursuivent au del\`{a} de $\theta $ \emph{parmi }$r$
communications re\c{c}ues, de fin \emph{ind\'{e}pendantes} et ayants toutes
la m\^{e}mes probabilit\'{e} $p$ de se poursuivre au del\`{a} de $\theta .$

Donc la loi conditionnelle de $C_{\theta }$ sachant que $\left\{ N_{\theta
}=r\right\} $ est une loi bin\^{o}miale de param\`{e}tres $\left( r,p\right) 
$

\item On a pour $r\in \mathbb{N\;}$et $\mathbb{\;\forall }k\in \left[ \left[
0,r\right] \right] :$ 
\begin{eqnarray*}
p\left( \left\{ C_{\theta }=k\right\} \cap \left\{ N_{\theta }=r\right\}
\right) =p\left( C_{\theta }=k\;/\;N_{\theta }=r\right) \cdot p\left(
N_{\theta }=r\right) \\
=C_{r}^{k}p^{k}q^{r-k}\frac{\theta ^{r}e^{-\theta }}{r!} \\
=\frac{r!}{k!\left( r-k\right) !}p^{k}q^{r-k}\theta ^{r-k+k}e^{-\theta }%
\frac{1}{r!} \\
=\displaystyle \frac{e^{-\theta }\left( p\theta \right) ^{k}\left( q\theta
\right) ^{r-k}}{k!\left( r-k\right) !}
\end{eqnarray*}
la loi bin\^{o}miale \'{e}tant donn\'{e}e par cette formule car $0\le k\le r$%
\begin{equation*}
\;\;p\left( \left\{ C_{\theta }=k\right\} \cap \left\{ N_{\theta }=r\right\}
\right) =\displaystyle \frac{e^{-\theta }\left( p\theta \right) ^{k}\left(
q\theta \right) ^{r-k}}{k!\left( r-k\right) !}\;\;\;
\end{equation*}

\item On a alors par la formule des probabilti\'{e}s totales, avec comme syst%
\`{e}me complet d'\'{e}v\'{e}nements $\left( N_{\theta }=r\right) _{r\in 
\mathbb{N}}$

\begin{equation*}
p\left( C_{\theta }=k\right) =\sum_{r=0}^{+\infty }p\left( \left\{ C_{\theta
}=k\right\} \cap \left\{ N_{\theta }=r\right\} \right)
\end{equation*}

On calcule la somme partielle de cette s\'{e}rie en s\'{e}parant les valeurs 
$r\ge k$ et $r<k$%
\begin{eqnarray*}
\sum_{r=0}^{M}p\left( \left\{ C_{\theta }=k\right\} \cap \left\{ N_{\theta
}=r\right\} \right) =\sum_{r=0}^{k-1}p\left( \left\{ C_{\theta }=k\right\}
\cap \left\{ N_{\theta }=r\right\} \right) \\
+\sum_{r=k}^{M}p\left( \left\{ C_{\theta }=k\right\} \cap \left\{ N_{\theta
}=r\right\} \right) \\
=0+\sum_{r=k}^{M}\frac{e^{-\theta }\left( p\theta \right) ^{k}\left( q\theta
\right) ^{r-k}}{k!\left( r-k\right) !} \\
=\frac{e^{-\theta }\left( p\theta \right) ^{k}}{k!}\sum_{r=k}^{M}\frac{%
\left( q\theta \right) ^{r-k}}{\left( r-k\right) !} \\
=\frac{e^{-\theta }\left( p\theta \right) ^{k}}{k!}\sum_{i=0}^{M-k}\frac{%
\left( q\theta \right) ^{i}}{i!} \\
\rightarrow \frac{e^{-\theta }\left( p\theta \right) ^{k}}{k!}e^{q\theta }=%
\frac{e^{\left( q-1\right) \theta }\left( p\theta \right) ^{k}}{k!} \\
=\frac{e^{-p\theta }\left( p\theta \right) ^{k}}{k!}
\end{eqnarray*}

quand $M\rightarrow +\infty $

Donc $p\left( C_{\theta }=k\right) =\displaystyle
\frac{e^{-p\theta }\left( p\theta \right) ^{k}}{k!}$ pour tout $k\ge 0$ et
on reconnait l\`{a} une loi de Poisson de param\^{e}tre $p\theta $
\end{enumerate}

\item \'{E}tude de l'esp\'{e}rance de $C_{\theta }$

\begin{enumerate}
\item On a donc $E\left( C_{\theta }\right) =p\theta =\displaystyle
\frac{1-e^{-\alpha \theta }}{\alpha \theta }\theta =\frac{1-e^{-\alpha
\theta }}{\alpha }$

\item Quand $\theta $ tend vers $+\infty $ on a $e^{-\alpha \theta
}\rightarrow 0$ car $\alpha >0$ et donc $E\left( C_{\theta }\right)
\rightarrow 1/\alpha $ lorsque $\theta $ tend vers$+\infty $

On a bien toujours $E\left( C_{\theta }\right) <1/\alpha .$
\end{enumerate}
\end{enumerate}


\end{document}
