% Titre : fonctions
% Filiere : BCPST
% Difficulte : 
% Type : TD 
% Categories :fonctions
% Subcategories : 
% Keywords : fonctions




\begin{exercice}  \; Donner l'ensemble de d\'efinition et de d\'erivabilit\'e des fonctions suivantes, puis calculer leur d\'eriv\'ees:
\begin{enumerate}
\begin{minipage}[t]{0.27\textwidth}
\item $f(x)=x^2e^{-\frac{1}{x}}$
\item $f(x)=\dfrac{\sin x}{\sqrt{x^2+1}}$
\item $f(x)=\sqrt{e^x}$
\item $f(x)=e^{x\cos{(x)}}$
\item $f(x)=(1-x)e^{\sqrt{x-x^2}}$\item $f(x)=\ddp\frac{\sin^3{(2x)}}{2+\cos{(5x)}}$
\item $f(x)=\sin{(\ln{x})}$
\end{minipage}
\begin{minipage}[t]{0.3\textwidth}
\item $f(x) =\ln(e^x+x^2)$
\item $f(x)=\ddp\frac{x-e^x}{e^x+1}$
\item $f(x)=\ln{\left(  \ddp\frac{x+2}{\sqrt{9x^2-4}} \right)}$
\item $f(x)=\ddp\frac{1}{(\cos{(x)})^4}$
\item $f(x)=\ddp\frac{1}{2^{x+1}}$
\item $f(x)=(e^{2x}-1)^{\pi}$
\end{minipage}
\begin{minipage}[t]{0.3\textwidth}
\item $f(x)=\left(  \ddp\frac{\sqrt{x^2+3x}}{3^x} \right)^4$
\item $f(x)=2^{\ln{x}}$
\item $f(x)=\ddp\frac{\sqrt{\ln{x}}}{x}$
\item $f(x)=\ln{(\ln{x})}$
\item $f(x)=\ln{(\sqrt{x^2-1}+x)}$
\item $f(x)=\ddp\frac{3^{x-1}\cos{x}}{x^x}$
\end{minipage}
\end{enumerate}
\end{exercice}


\%\%\%\%\%\%\%\%\%\%\%\%\%\%\%\%\%\%\%\%
\%\%\%\%\%\%\%\%\%\%\%\%\%\%\%\%\%\%\%\%
\%\%\%\%\%\%\%\%\%\%\%\%\%\%\%\%\%\%\%\%



\begin{correction}  \;
\begin{enumerate}
\item \textbf{Ensemble de d\'efinition, de d\'erivabilit\'e et d\'eriv\'ee de $\mathbf{f}$ d\'efinie par: $\mathbf{f(x)=x^2e^{-\frac{1}{x}}}$:}
\begin{itemize}
\item[$\bullet$] \underline{Ensemble de d\'efinition:} La fonction $f$ est bien d\'efinie si et seulement si $x\not= 0$. Donc $\mathcal{D}_f=\R^{\star}$.
\item[$\bullet$] \underline{Ensemble de d\'erivabilit\'e:} La fonction $f$ est d\'erivable sur $\mathcal{D}_f$ comme compos\'ee et produit de fonctions d\'erivables.
\item[$\bullet$] \underline{D\'eriv\'ee:} Pour tout $x\in\mathcal{D}_f$, on a: \fbox{$f^{\prime}(x)=e^{-\frac{1}{x}}(2x+1)$.}
 \end{itemize}
%-------------
\item  \textbf{Ensemble de d\'efinition, de d\'erivabilit\'e et d\'eriv\'ee de $\mathbf{f}$ d\'efinie par: $\mathbf{f(x)=\ddp \frac{\sin x}{\sqrt{x^2+1}}}$:}
\begin{itemize}
\item[$\bullet$] \underline{Ensemble de d\'efinition:} La fonction $f$ est d\'efinie si et seulement si $x^2+1\geq 0$ et $\sqrt{x^2+1}\not= 0$. Ainsi elle est bien d\'efinie si et seulement si $x^2+1>0$ ce qui est toujours vrai. Ainsi $\mathcal{D}_{f}=\R$. 
\item[$\bullet$] \underline{Ensemble de d\'erivabilit\'e:} La fonction $f$ est d\'erivable sur $\R$ comme compos\'ee et quotient de fonctions d\'erivables et car $x^2+1>0$ pour tout $x\in\R$. 
\item[$\bullet$] \underline{D\'eriv\'ee:} Pour tout $x\in\R$, on a: $f^{\prime}(x)=\ddp\frac{ \cos{x}\sqrt{x^2+1}-\sin{x}\frac{2x}{2\sqrt{x^2+1}}  }{(\sqrt{x^2+1})^2}= \ddp\frac{ (x^2+1)\cos{x}-x\sin{x} }{(x^2+1)\sqrt{x^2+1}}$.
 \end{itemize}
  %-------------
\item  \textbf{Ensemble de d\'efinition, de d\'erivabilit\'e et d\'eriv\'ee de $\mathbf{f}$ d\'efinie par: $\mathbf{f(x)=\ddp \sqrt{e^x}}$:}
\begin{itemize}
\item[$\bullet$] \underline{Ensemble de d\'efinition:} La fonction $f$ est d\'efinie si et seulement si $e^x\geq 0$: toujours vrai. Ainsi $\mathcal{D}_{f}=\R$. 
\item[$\bullet$] \underline{Ensemble de d\'erivabilit\'e:} Comme pour tout $x\in\R$: $e^x>0$, la fonction $f$ est d\'erivable sur $\R$ comme compos\'ee de fonctions d\'erivables.
\item[$\bullet$] \underline{D\'eriv\'ee:} Pour tout $x\in\R$: $f^{\prime}(x)=\ddp\frac{e^x}{2\sqrt{e^x}}$.
 \end{itemize} 
 %-------------
\item  \textbf{Ensemble de d\'efinition, de d\'erivabilit\'e et d\'eriv\'ee de $\mathbf{f}$ d\'efinie par: $\mathbf{f(x)=e^{x\cos{(x)}}}$:}
\begin{itemize}
\item[$\bullet$] \underline{Ensemble de d\'efinition:} La fonction $f$ est toujours bien d\'efinie. Donc $\mathcal{D}_f=\R$.
\item[$\bullet$] \underline{Ensemble de d\'erivabilit\'e:} La fonction $f$ est d\'erivable sur $\mathcal{D}_f$ comme produit et compos\'ee de fonctions d\'erivables.
\item[$\bullet$] \underline{D\'eriv\'ee:} Pour tout $x\in\mathcal{D}_f$, on a: \fbox{$f^{\prime}(x)=\left \lbrack \cos{(x)}-x\sin{(x)}  \right\rbrack e^{x\cos{x}}$.}
 \end{itemize}
 %-------------
\item  \textbf{Ensemble de d\'efinition, de d\'erivabilit\'e et d\'eriv\'ee de $\mathbf{f}$ d\'efinie par: $\mathbf{f(x)=(1-x)e^{\sqrt{x-x^2}}}$:}
\begin{itemize}
\item[$\bullet$] \underline{Ensemble de d\'efinition:} La fonction $f$ est bien d\'efinie si et seulement si $x-x^2\geq 0$. C'est un polyn\^{o}me de der\'e 2 dont les racies sont 0 et 1. Donc $\mathcal{D}_f=\lbrack 0,1\rbrack$.
\item[$\bullet$] \underline{Ensemble de d\'erivabilit\'e:} La fonction $f$ est d\'erivable si $x-x^2>0$. Ainsi la fonction $f$ est d\'erivable sur $\rbrack 0,1\lbrack$ comme somme, copos\'ee et produti de fonctions d\'erivables.
\item[$\bullet$] \underline{D\'eriv\'ee:} Pour tout $x\in\rbrack 0,1\lbrack$, on a: \fbox{$f^{\prime}(x)=\left\lbrack \ddp\frac{ (1-x)(1-2x) }{2\sqrt{x-x^2}} -1 \right\rbrack e^{\sqrt{x-x^2}}$.}
 \end{itemize}
  %-------------
\item  \textbf{Ensemble de d\'efinition, de d\'erivabilit\'e et d\'eriv\'ee de $\mathbf{f}$ d\'efinie par: $\mathbf{f(x)=\ddp\frac{\sin^3{(2x)}}{2+\cos{(5x)}}}$:}
\begin{itemize}
\item[$\bullet$] \underline{Ensemble de d\'efinition:} La fonction $f$ est bien d\'efinie si et seulement si $2+\cos{(5x)}\not= 0\Leftrightarrow \cos{(5x)}\not= -2$: impossible car un cosinus est toujours compris entre -1 et 1. Donc $\mathcal{D}_f=\R$
\item[$\bullet$] \underline{Ensemble de d\'erivabilit\'e:} La fonction $f$ est d\'erivable sur $\mathcal{D}_f$ comme produit, compos\'ee, somme et quotient de fonctions d\'erivables.
\item[$\bullet$] \underline{D\'eriv\'ee:}  Pour tout $x\in\mathcal{D}_f$, on a:

 \fbox{$f^{\prime}(x)=\ddp\frac{\sin^2{(2x)}}{(2+\cos{(x)})^2}\left\lbrack 12\cos{(2x)}+6\cos{(2x)}\cos{(5x)}+5\sin{(2x)}\sin{(5x)} \right\rbrack$.}
 \end{itemize}
  %-------------
\item  \textbf{Ensemble de d\'efinition, de d\'erivabilit\'e et d\'eriv\'ee de $\mathbf{f}$ d\'efinie par: $\mathbf{f(x)=\sin{(\ln{x})}}$:}
\begin{itemize}
\item[$\bullet$] \underline{Ensemble de d\'efinition:} La fonction $f$ est bien d\'efinie si et seulement si $x>0$. Donc $\mathcal{D}_f=\R^{+\star}$.
\item[$\bullet$] \underline{Ensemble de d\'erivabilit\'e:} La fonction $f$ est d\'erivable sur $\mathcal{D}_f$ comme compos\'ee de fonctions d\'erivables.
\item[$\bullet$] \underline{D\'eriv\'ee:} Pour tout $x\in\mathcal{D}_f$, on a: \fbox{$f^{\prime}(x)=\ddp\frac{\cos{(\ln{x})}}{x}$.} 
 \end{itemize}
   %-------------
\item  \textbf{Ensemble de d\'efinition, de d\'erivabilit\'e et d\'eriv\'ee de $\mathbf{f}$ d\'efinie par: $\mathbf{f(x)=\ddp \ln{(e^x+x^2)}}$:}
\begin{itemize}
\item[$\bullet$] \underline{Ensemble de d\'efinition:} La fonction $f$ est d\'efinie si et seulement si $e^x+x^2>0$: toujours vrai comme somme de deux nombres positifs dont l'un  est strictement positif. Ainsi $\mathcal{D}_{f}=\R$. 
\item[$\bullet$] \underline{Ensemble de d\'erivabilit\'e:} La fonction $f$ est d\'erivable sur $\R$ comme somme et compos\'ee de fonctions d\'erivables.
\item[$\bullet$] \underline{D\'eriv\'ee:}  Pour tout $x\in\R$, on a: $f^{\prime}(x)=\ddp\frac{e^x+2x}{e^x+x^2}$.
 \end{itemize}

 %-------------
\item  \textbf{Ensemble de d\'efinition, de d\'erivabilit\'e et d\'eriv\'ee de $\mathbf{f}$ d\'efinie par: $\mathbf{f(x)=\ddp\frac{x-e^x}{e^x+1}}$:}
\begin{itemize}
\item[$\bullet$] \underline{Ensemble de d\'efinition:} La fonction $f$ est bien d\'efinie si et seulement si $e^x+1\not= 0$: toujours vrai comme somme de deux termes tous les deux strictement positifs, une exponentielle \'etant toujours strictement positive. Donc $\mathcal{D}_f=\R$.
\item[$\bullet$] \underline{Ensemble de d\'erivabilit\'e:} La fonction $f$ est d\'erivable sur $\mathcal{D}_f$ comme somme et quotient de fonctions d\'erivables.
\item[$\bullet$] \underline{D\'eriv\'ee:} Pour tout $x\in\mathcal{D}_f$, on a: \fbox{$f^{\prime}(x)=\ddp\frac{1-xe^x}{(e^x+1)^2}$.}
 \end{itemize}
 %-------------
\item  \textbf{Ensemble de d\'efinition, de d\'erivabilit\'e et d\'eriv\'ee de $\mathbf{f}$ d\'efinie par: $\mathbf{f(x)=\ln{\left(  \ddp\frac{x+2}{\sqrt{9x^2-4}} \right)}}$:}
\begin{itemize}
\item[$\bullet$] \underline{Ensemble de d\'efinition:} La fonction $f$ est bien d\'efinie si et seulement si $9x^2-4>0$ et $ \ddp\frac{x+2}{\sqrt{9x^2-4}} >0$. Comme une racine carr\'ee est toujours positive, la fonction $f$ est bien d\'efinie si et seulement si $9x^2-4>0$ et $x+2>0$. La première condition est un polyn\^{o}me de degr\'e 2 dont les racines sont $-\ddp\frac{2}{3}$ et $\ddp\frac{2}{3}$. Donc $\mathcal{D}_f=\left\rbrack -2,-\ddp\frac{2}{3}\right\lbrack\cup\left\rbrack \ddp\frac{2}{3},+\infty\right\lbrack$.
\item[$\bullet$] \underline{Ensemble de d\'erivabilit\'e:} La fonction $f$ est d\'erivable sur $\mathcal{D}_f$ (car ce qui est sous la racine est d\'ej\`{a} strictement positif) comme produit, sommes, compos\'ees et quotient de fonctions d\'erivables.
\item[$\bullet$] \underline{D\'eriv\'ee:} Pour tout $x\in\mathcal{D}_f$, on a: \fbox{$f^{\prime}(x)=\ddp\frac{-2(9x+2)}{(x+2)(9x^2-4)}$.}
 \end{itemize}
 %-------------
\item  \textbf{Ensemble de d\'efinition, de d\'erivabilit\'e et d\'eriv\'ee de $\mathbf{f}$ d\'efinie par: $\mathbf{f(x)=\ddp\frac{1}{(\cos{(x)})^4}}$:}
\begin{itemize}
\item[$\bullet$] \underline{Ensemble de d\'efinition:} La fonction $f$ est bien d\'efinie si et seulement si $\cos^4{(x)}\not= 0 \Leftrightarrow \cos{x}\not= 0$. Donc $\mathcal{D}_f=\R\setminus\left\lbrace \ddp\frac{\pi}{2}+k\pi,\ k\in\Z   \right\rbrace$.
\item[$\bullet$] \underline{Ensemble de d\'erivabilit\'e:} La fonction $f$ est d\'erivable sur $\mathcal{D}_f$ comme compos\'ee et quotient de fonctions d\'erivables.
\item[$\bullet$] \underline{D\'eriv\'ee:} Pour tout $x\in\mathcal{D}_f$, on a: \fbox{$f^{\prime}(x)=\ddp\frac{4\sin{(x)}}{(\cos{(x)})^5}$.}
 \end{itemize}
 %-------------
 \item  \textbf{Ensemble de d\'efinition, de d\'erivabilit\'e et d\'eriv\'ee de $\mathbf{f}$ d\'efinie par: $\mathbf{f(x)=\ddp\frac{1}{2^{x+1}}}$:}
\begin{itemize}
\item[$\bullet$] \underline{Ensemble de d\'efinition:} La fonction $f$ est bien d\'efinie si et seulement si $2^{x+1}\not= 0\Leftrightarrow e^{(x+1)\ln{2}}\not= 0$: toujours vrai car une exponentielle est toujours strictement positive. Donc $\mathcal{D}_f=\R$.
\item[$\bullet$] \underline{Ensemble de d\'erivabilit\'e:} La fonction $f$ est d\'erivable sur $\mathcal{D}_f$ comme produit, compos\'ee et quotient de fonctions d\'erivables.
\item[$\bullet$] \underline{D\'eriv\'ee:} Pour tout $x\in\mathcal{D}_f$, on a: \fbox{$f^{\prime}(x)=\ddp\frac{-\ln{2}}{2^{x+1}}$.} On peut pour cela remarquer que $f(x)=e^{-(x+1)\ln{2}}$.
 \end{itemize}
 %-------------
\item  \textbf{Ensemble de d\'efinition, de d\'erivabilit\'e et d\'eriv\'ee de $\mathbf{f}$ d\'efinie par: $\mathbf{f(x)=(e^{2x}-1)^{\pi}}$:}
\begin{itemize}
\item[$\bullet$] \underline{Ensemble de d\'efinition:} La fonction $f$ est bien d\'efinie si et seulement si $e^{2x}-1>0$ car on a: $(e^{2x}-1)^{\pi}=e^{\pi\ln{(e^{2x}-1)}}$. Or $e^{2x}-1>0\Leftrightarrow e^{2x}>1  \Leftrightarrow x>0$ par passage au logarithme n\'ep\'erien qui est strictement croissante sur $\R^{+\star}$. Donc $\mathcal{D}_f=\R^{+\star}$.
\item[$\bullet$] \underline{Ensemble de d\'erivabilit\'e:} La fonction $f$ est d\'erivable sur $\mathcal{D}_f$ comme compos\'ee, somme et produit de focntions d\'erivables.
\item[$\bullet$] \underline{D\'eriv\'ee:} Pour tout $x\in\mathcal{D}_f$, on a: \fbox{$f^{\prime}(x)=\ddp\frac{2\pi e^{2x}}{e^{2x}-1}(e^{2x}-1)^{\pi}.$}
 \end{itemize}
 %-------------
\item  \textbf{Ensemble de d\'efinition, de d\'erivabilit\'e et d\'eriv\'ee de $\mathbf{f}$ d\'efinie par: $\mathbf{f(x)=\left(  \ddp\frac{\sqrt{x^2+3x}}{3^x} \right)^4}$:}
\begin{itemize}
\item[$\bullet$] \underline{Ensemble de d\'efinition:} La fonction $f$ est bien d\'efinie si et seulement si $x^2+3x\geq 0$ et $3^x=e^{x\ln{3}}\not= 0$. La deuxi\`{e}me condition est toujours v\'erifi\'ee car une exponentielle est toujours strictement positive. Pour la premi\`{e}re condition, on reconna\^{i}t un polyn\^{o}me de degr\'e 2 dont les racines sont 0 et -3. Donc $\mathcal{D}_f=\rbrack -\infty,-3\rbrack\cup\lbrack 0,+\infty\lbrack$.
\item[$\bullet$] \underline{Ensemble de d\'erivabilit\'e:} La fonction $f$ est d\'erivable sur $\rbrack -\infty,-3\lbrack\cup\rbrack 0,+\infty\lbrack$ car on doit avoir $x^2+3x>0$ puis comme sommes, produit, compos\'ees et quotient de fonctions d\'erivables.
\item[$\bullet$] \underline{D\'eriv\'ee:} Pour tout $x\in\mathcal{D}_f$, on a: \fbox{$f^{\prime}(x)=\ddp\frac{2(x^2+3x)}{3^{4x}}\left\lbrack  2x+3-2(x^2+3x)\ln{3}  \right\rbrack$.}
 \end{itemize}
 %-------------
 \item  \textbf{Ensemble de d\'efinition, de d\'erivabilit\'e et d\'eriv\'ee de $\mathbf{f}$ d\'efinie par: $\mathbf{f(x)=2^{\ln{x}}}$:}
\begin{itemize}
\item[$\bullet$] \underline{Ensemble de d\'efinition:} La fonction $f$ est bien d\'efinie si et seulement si $x>0$ en \'ecrivant que $2^{\ln{x}}=e^{\ln{(x)}\ln{2}}$. Donc $\mathcal{D}_f=\R^{+\star}$.
\item[$\bullet$] \underline{Ensemble de d\'erivabilit\'e:} La fonction $f$ est d\'erivable sur $\mathcal{D}_f$ comme produit et compos\'ee de fonctions d\'erivables.
\item[$\bullet$] \underline{D\'eriv\'ee:} Pour tout $x\in\mathcal{D}_f$, on a: \fbox{$f^{\prime}(x)=\ddp\frac{\ln{2}}{x}2^{\ln{x}}$.}
 \end{itemize}
 %-------------
\item  \textbf{Ensemble de d\'efinition, de d\'erivabilit\'e et d\'eriv\'ee de $\mathbf{f}$ d\'efinie par: $\mathbf{f(x)=\ddp\frac{\sqrt{\ln{x}}}{x}}$:}
\begin{itemize}
\item[$\bullet$] \underline{Ensemble de d\'efinition:} La fonction $f$ est bien d\'efinie si et seulement si $x>0$, $\ln{x}\geq 0$ et $x\not= 0$. La deuxi\`{e}me condition donne: $\ln{x}\geq 0\Leftrightarrow x\geq 1$. Donc $\mathcal{D}_f=\lbrack 1,+\infty\lbrack$.
\item[$\bullet$] \underline{Ensemble de d\'erivabilit\'e:} La fonction $f$ est d\'erivable sur $\rbrack 1,+\infty\lbrack$ car on doit avoir $\ln{x}>0$ comme compos\'ee et quotient de fonctions d\'erivables.
\item[$\bullet$] \underline{D\'eriv\'ee:} Pour tout $x\in\mathcal{D}_f$, on a: \fbox{$f^{\prime}(x)=\ddp\frac{1-2\ln{x}}{2x^2\sqrt{\ln{x}}}$.}
 \end{itemize}
 %-------------
 \item  \textbf{Ensemble de d\'efinition, de d\'erivabilit\'e et d\'eriv\'ee de $\mathbf{f}$ d\'efinie par: $\mathbf{f(x)=\ln{(\ln{x})}}$:}
\begin{itemize}
\item[$\bullet$] \underline{Ensemble de d\'efinition:} La fonction $f$ est bien d\'efinie si et seulement si $x>0$ et $\ln{(x)}>0$. Or on a: $\ln{(x)}>0\Leftrightarrow x>1$. Donc $\mathcal{D}_f=\rbrack 1,+\infty\lbrack$.
\item[$\bullet$] \underline{Ensemble de d\'erivabilit\'e:} La fonction $f$ est d\'erivable sur $\mathcal{D}_f$ comme compos\'ee de fonctions d\'erivables. 
\item[$\bullet$] \underline{D\'eriv\'ee:} Pour tout $x\in\mathcal{D}_f$, on a: \fbox{$f^{\prime}(x)=\ddp\frac{1}{x\ln{(x)}}$.}
 \end{itemize}
 %-------------
\item  \textbf{Ensemble de d\'efinition, de d\'erivabilit\'e et d\'eriv\'ee de $\mathbf{f}$ d\'efinie par: $\mathbf{f(x)=\ln{(\sqrt{x^2-1}+x)}}$:}
\begin{itemize}
\item[$\bullet$] \underline{Ensemble de d\'efinition:} La fonction $f$ est bien d\'efinie si et seulement si $\sqrt{x^2-1}+x>0\Leftrightarrow \sqrt{x^2-1}>-x$ et $x^2-1\geq 0\Leftrightarrow x\in\rbrack -\infty,-1\rbrack\cup\lbrack 1,+\infty\lbrack$. On doit donc \'etudier deux cas afin de r\'esoudre la premi\`{e}re in\'equation:
\begin{itemize}
\item[$\star$] Si $x\geq 1$, alors $-x\leq -1$ et l'in\'equation est toujours v\'erifi\'ee car une racine carr\'ee est toujours sup\'erieure \`{a} un nombre n\'egatif.
\item[$\star$] Si $x\leq -1$ alors $-x\geq 1$ et on peut donc passer au carr\'ee tout en conservant l'\'equivalence, la fonction carr\'ee \'etant strictement croissante sur $\R^+$. On obtient alors: $\sqrt{x^2-1}>-x\Leftrightarrow x^2-1>x^2\Leftrightarrow -1>0$. Toujours faux. 
\end{itemize}
Donc $\mathcal{D}_f=\lbrack 1,+\infty\lbrack$.
\item[$\bullet$] \underline{Ensemble de d\'erivabilit\'e:} La fonction $f$ est d\'erivable sur $\rbrack 1,+\infty\lbrack$ car on doit avoir en plus $x^2-1>0$ comme somme et compos\'ee de fonctions d\'erivables.
\item[$\bullet$] \underline{D\'eriv\'ee:} Pour tout $x\in\mathcal{D}_f$, on a: \fbox{$f^{\prime}(x)=\ddp\frac{1}{\sqrt{x^2-1}}$.}
 \end{itemize}
 %-------------
\item  \textbf{Ensemble de d\'efinition, de d\'erivabilit\'e et d\'eriv\'ee de $\mathbf{f}$ d\'efinie par: $\mathbf{f(x)=\ddp\frac{3^{x-1}\cos{x}}{x^x}}$:}
\begin{itemize}
\item[$\bullet$] \underline{Ensemble de d\'efinition:} La fonction $f$ est bien d\'efinie si et seulement si $x>0$ et $x^x=e^{x\ln{x}}\not= 0$. La deuxi\`{e}me in\'equation est toujours v\'erifi\'ee, une exponentielle \'etant toujours strictement n\'egative. Donc $\mathcal{D}_f=\R^{+\star}$ (on commence par \'ecrire que: $\ddp\frac{3^{x-1}\cos{x}}{x^x}=\ddp\frac{\cos{(x)}e^{(x-1)\ln{(3)}}   }{e^{x\ln{x}}}$).
\item[$\bullet$] \underline{Ensemble de d\'erivabilit\'e:} La fonction $f$ est d\'erivable sur $\mathcal{D}_f$ comme produit, compos\'ees et quotient de fonctions d\'erivables.
\item[$\bullet$] \underline{D\'eriv\'ee:} Pour tout $x\in\mathcal{D}_f$, on a: \fbox{$f^{\prime}(x)=\ddp\frac{e^{(x-1)\ln{(3)}}}{x^x}\left\lbrack  -\sin{(x)} +\ln{(3)}\cos{(x)} -\cos{(x)} (\ln{(x)}+1) \right\rbrack$.}
 \end{itemize}
 %-------------
\end{enumerate}
\end{correction}


%--------------------------------------------------
%------------------------------------------------
