% Titre : nombre
% Filiere : BCPST
% Difficulte : 
% Type : TD 
% Categories :nombre
% Subcategories : 
% Keywords : nombre




\begin{exercice}
On consid\`ere l'expression $R(a)=\sqrt{a+2\sqrt{a-1}}+\sqrt{a-2\sqrt{a-1}}$.
\begin{enumerate}
 \item Pour quels valeurs de $a$, $R(a)$ est-elle bien d\'efinie ? 
\item Pour ces valeurs, simplifier l'expression $R(a)$. Tracer la fonction $a\mapsto R(a)$.
\end{enumerate}
\end{exercice}


\%\%\%\%\%\%\%\%\%\%\%\%\%\%\%\%\%\%\%\%
\%\%\%\%\%\%\%\%\%\%\%\%\%\%\%\%\%\%\%\%
\%\%\%\%\%\%\%\%\%\%\%\%\%\%\%\%\%\%\%\%



\begin{correction}   \;
\begin{enumerate}
\item \textbf{Valeurs de $\mathbf{a}$ pour que $\mathbf{R(a)}$ soit bien d\'efini:}\\
\noindent Pour que $R(a)$ soit bien d\'efinie, il faut d\'ej\`a que $a-1\geq 0$, c'est-\`a-dire que $a\geq 1$. On suppose donc que $a\geq 1$.
Sous cette hypoth\`ese, on a donc que $a+2\sqrt{a-1}>0$ comme somme d'un terme strictement positif et d'un autre terme positif. Il reste \`a \'etudier $a-2\sqrt{a-1}$.
$$a-2\sqrt{a-1}\geq 0\Leftrightarrow a\geq 2\sqrt{a-1}\Leftrightarrow a^2-2a+1\geq 0.$$
On est pass\'e au carr\'e tout en conservant l'\'equivalence car les deux termes sont bien positifs. Le discriminant de la derni\`ere in\'equation est strictement n\'egatif ($\Delta=-4$) et ainsi, on a $a^2-2a+1>0$, d'o\`u $a-2\sqrt{a-1}>0$. Finalement, on obtient
$$\fbox{$ \mathcal{D}_R=\lbrack 1,+\infty\lbrack. $}$$
\item \textbf{Simplifions $\mathbf{R(a)}$:} \\
\noindent On suppose donc que $a\geq 1$. Ainsi, $R(a)$ a bien un sens et on peut calculer $R(a)^2$. On obtient
$$R(a)^2=2a+2\sqrt{a^2-4a+4}=2a+2\sqrt{(a-2)^2}=2a+2|a-2|.$$
Ainsi, si $1\leq a\leq 2$, on obtient
$$R(a)^2=2a+2(-a+2)=4\quad \hbox{donc}\quad R(a)=2$$
car $R(a)=-2$ est impossible car $R(a)$ est un nombre positif comme somme de deux nombres positifs (somme de deux racines carr\'ees).
Et si $a\geq 2$, on obtient
$$R(a)^2=2a+2(a-2)=4(a-1)\quad \hbox{donc}\quad R(a)=2\sqrt{a-1}$$
car $R(a)=-2\sqrt{a-1}$ est impossible car $R(a)$ est un nombre positif comme somme de deux nombres positifs (somme de deux racines carr\'ees). \\
\noindent On a donc obtenu:
$$\fbox{ $
\forall a\geq 1,\ R(a)=\left\lbrace
\begin{array}{ll}
2 & \hbox{si}\ 1\leq x\leq 2\vsec\\
2\sqrt{a-1} & \hbox{si}\ x\geq 2.
\end{array}
\right.
$}$$
%---


\end{enumerate}
\end{correction}





