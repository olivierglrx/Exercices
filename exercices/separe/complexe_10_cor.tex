
\begin{correction}   \;
\begin{enumerate}
%----------------------------------------------------
\item  \textbf{R\'esolution de $\mathbf{(z+1)^2+(2z+3)^2=0}$ :}
On reconna\^it un trin\^ome : 
$$(z+1)^2+(2z+3)^2=0 \Leftrightarrow z^2 + 2z +1 +4z^2 + 12 z  + 9 =0 \Leftrightarrow 5 z^2 + 14 z + 10 = 0.$$
Le discriminant vaut $\Delta = 14^2 - 4 \times 5 \times 10 = 4 ( 49 - 50) = -4$. Les solutions sont donc $z_1 = \ddp \frac{-14 - 2i}{10} = \frac{-7 - i}{5}$ et $z_2 = \ddp \frac{-7+i}{5}$.\\
Ainsi, $\fbox{$ \mathcal{S}=\left\lbrace \ddp\frac{-7-i}{5},\ddp\frac{-7+i}{5}  \right\rbrace. $}$
%----------------------------------------------------
\item \textbf{R\'esolution de $\mathbf{2z^2(1-\cos{(2\theta)})-2z\sin{(2\theta)}+1=0}$ :} on fait deux cas, car le coefficient du $z^2$ peut s'annuler.
\begin{itemize} 
\item[$\bullet$] Si $1-\cos(2\theta) = 0 \Leftrightarrow \cos(2\theta) = 1 \Leftrightarrow  \; \exists k \in \Z, 2\theta = 2k\pi  \Leftrightarrow  \; \exists k \in \Z, \theta = k\pi $.\\
On a alors $\sin(2\theta) = 0$, et on doit donc r\'esoudre : $0+0+1 = 0$, ce qui est impossible. Donc $\mathcal{S}_{1} = \emptyset$.
\item[$\bullet$] Si $1-\cos(2\theta) = 0 \Leftrightarrow  \forall k \in \Z, \theta \not = k\pi $.\\
C'est une \'equation du second degr\'e en $z$, on calcule donc le discriminant et on obtient
\begin{align*}
\Delta&=4\sin^2{(2\theta)}-8(1-\cos{(2\theta)})\\
&=4\left( 2\sin{(\theta)}\cos{(\theta)}  \right)^2-8\times 2\sin^2{(\theta)}\\&=16\sin^2{(\theta)} (\cos^2{(\theta)} -1)\\
&=-16\sin^4{(\theta)} .
\end{align*}

Ainsi $\Delta<0$ et $\sqrt{-\Delta}=4\sin^2{(\theta)}$.
On obtient alors $z_1=\ddp\frac{ 2\sin{(2\theta)} +4i\sin^2{(\theta)}  }{ 4\times 2\sin^2{(\theta)}}=\ddp\demi\left( \cot{(\theta)}+i \right)$
en utilisant le fait que $\sin{(2\theta)}=2\cos{(\theta)}\sin{(\theta)}$. Et les racines \'etant alors complexes conjugu\'ees, on obtient: $z_2=\ddp\demi\left( \cot{(\theta)}-i \right)$. Ainsi $\fbox{$ \mathcal{S}=\left\lbrace \ddp\demi\left( \cot{(\theta)}-i \right),\ddp\demi\left( \cot{(\theta)}+i \right) \right\rbrace $}$.
\end{itemize}
%%----------------------------------------------------
%\item Exercice tr\`{e}s classique: L'id\'ee ici est de se ramener \`{a} la r\'esolution d'une \'equation type racine n-i\`{e}me de l'unit\'e.
%\begin{itemize}
%\item[$\bullet$] Comme 1 n'est pas solution de l'\'equation, on peut supposer que $z\not= 1$. Ainsi, on peut bien diviser par $(z-1)^n$ qui est bien non nul. Ainsi, on a
%$$(z+1)^n=(z-1)^n\Leftrightarrow \left( \ddp\frac{z+1}{z-1}\right)^n=1\Leftrightarrow Z^n=1$$
%en posant $Z=\ddp\frac{z+1}{z-1}$.
%\item[$\bullet$] R\'esolution des racines n-i\`{e}me de l'unit\'e: \`{a} savoir faire: voir cours:\\
%\noindent On obtient donc que les solutions sont les $Z$ de la forme
%$$Z_k=e^{\frac{2ik\pi}{n}},\quad k\in\intent{ 0,n-1}.$$
%\item[$\bullet$] On repasse alors \`{a} $z$ et on cherche donc les $z$ tels que: $\ddp\frac{z+1}{z-1}=e^{\frac{2ik\pi}{n}}$ avec $k\in\intent{ 0,n-1}$ fix\'e. On obtient alors
%$$\ddp\frac{z+1}{z-1}=e^{\frac{2ik\pi}{n}}\Leftrightarrow z+1=e^{\frac{2ik\pi}{n}} (z-1)\Leftrightarrow z\left(1- e^{\frac{2ik\pi}{n}} \right)=-e^{\frac{2ik\pi}{n}}-1\Leftrightarrow z\left(e^{\frac{2ik\pi}{n}} -1\right)=e^{\frac{2ik\pi}{n}}+1.$$
%Ici, il faut faire attention car on ne peut JAMAIS diviser par un nombre sans v\'erifier qu'il est bien NON nul. Or on a:
%$$e^{\frac{2ik\pi}{n}} -1=0\Leftrightarrow e^{\frac{2ik\pi}{n}} =1\Leftrightarrow \ddp\frac{2k\pi}{n}=2k^{\prime}\pi\Leftrightarrow k=nk^{\prime}$$
%avec $k^{\prime}\in\Z$. Or $k\in\intent{ 0,n-1}$ donc le seul $k$ qui v\'erifie cela est $k=0$. 
%\begin{itemize}
%\item[$\star$] Pour $k=0$, on obtient: $0=2$ donc il n'y a pas de solution pour $k=0$.
%\item[$\star$] Pour $k\not= 0$, \`{a} savoir pour $k\in\intent{ 1,n-1}$, on sait que $1- e^{\frac{2ik\pi}{n}}\not= 0$ et on peut donc bien diviser. On obtient
%$$z=\ddp\frac{e^{\frac{2ik\pi}{n}}+1}{e^{\frac{2ik\pi}{n}} -1}=-i\cot{\left( \ddp\frac{k\pi}{n} \right)}$$
%en utilisant la m\'ethode de l'angle moiti\'e.
%\end{itemize}
%\item[$\bullet$] Conclusion: $\fbox{$ \mathcal{S}=\left\lbrace  z\in\bC,\exists k\in\intent{ 1,n-1},\ z=-i\cot{\left( \ddp\frac{k\pi}{n} \right)}  \right\rbrace $}$
%\end{itemize}
%%----------------------------------------------------
%----------------------------------------------------
\item \textbf{R\'esolution de $\mathbf{\exp(z)=3+\sqrt{3}i}$ :}\\
On commence par mettre $3+\sqrt{3}i$ sous forme exponentielle : on obtient $3+\sqrt{3}i=2\sqrt{3} e^{i \frac{\pi}{6}}$. On pose $z=a+ib$, avec $(a,b)\in \R^2$. On doit alors r\'esoudre :
$$e^a e^{ib} = 2\sqrt{3} e^{i \frac{\pi}{6}}.$$
Par identification du module et de l'argument, on obtient $e^a=2\sqrt{3}$, soit $a = \ln(2\sqrt{3})$, et $b=\ddp \frac{\pi}{6} + 2k \pi$, avec $k \in \Z$. On a donc comme solutions : 

$$\fbox{$\ddp \mathcal{S} = \left\{ \ln(2\sqrt{3}) + i \left(\frac{\pi}{6} + 2k \pi\right),k \in \Z\right\}$}.$$
\end{enumerate}
\end{correction}