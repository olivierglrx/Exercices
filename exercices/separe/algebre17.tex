% Titre : Tirages conditionnels  
% Filiere : BCPST
% Difficulte :
% Type : DS, DM
% Categories : algebre
% Subcategories : 
% Keywords : algebre




\begin{exercice}
Une urne contient $b\in \N^*$ boules blanches et $n\in \N^*$ boules noires. On effectue des tirages successifs. Après chaque tirage, on remet la boule tirée dans l'urne en y rajoutant $a\in \N^*$ boules de la même couleur. \\
On répète l'expèrience à chaque tirage. \\
On note $B_i$ l'événement \og Obtenir une boule blanche au tirage $i$\fg \, et $N_i$    \og Obtenir une boule noire au tirage $i$\fg.
\begin{enumerate}
\item Calculer $\bP(B_1)$ et $\bP(B_2)$.
\item Pour $p,q\in \N$, déterminer la probabilité d'obtenir d'abord $p$ boules blanches puis $q$ boules noires. 
\item Préciser cette valeur pour $b=n=a$. 
\item Déterminer la probabilité d'obtenir exactement $p$ boules blanches en $p+q$ tirages. 
\item Préciser cette valeur pour $p=n=a$. 
\item 
\begin{enumerate}
\item Ecrire une fonction Python qui modélise  le tirage d'une boule dans une urne avec $b$ boules blanches et $n$ boules noires.
\item Ecrire une fonction Python qui modélise $N$ tirages successifs (comme décris dans l'énoncé) et retourne le nombre de boules blanches tirées. 
\item Ecrire une fonction Python qui effectue l'expérience décrite et retourne le numéro du premier tirage où l'on tire une boule blanche. 
\end{enumerate}

\end{enumerate}
\end{exercice}

\begin{correction}
\begin{enumerate}
\item Pour $B_1$ il s'agit d'une probabilité uniforme, sur un ensemble de $b+n$ éléments.  $\bP(B_1) =\frac{b}{b+n}$.

Pour $B_2$ on utilise le systême complet d'événements $B_1, N_1$, la formule des probabilités totales donne : 
\begin{align*}
\bP(B_2) &= \bP(B_2|B_1) \bP(B_1) + \bP(B_2|N_1) \bP(N_1)\\
			&= \frac{b+a}{b+n+a}\frac{b}{b+n}+\frac{b}{b+n+a}\frac{n}{b+n}\\
			&= \frac{b^2+ab +bn }{(b+n+a)(b+n)}
\end{align*}
\item Soit $E_{p,q}$ l'événements "obtenir $p$ boules blanches puis $q$ boules noires".
On a $$E_{p,q} = \cap_{i=1}^p B_i \cap_{i=p+1}^{p+q} N_{i}$$
Ce qui donne en utilisant la formule des probabilités composées :
\begin{align*}
\bP(E_{p,q}) &= \bP(B_1) \bP_{B_1} (B_2) \dots \bP_{\cap_{i=1}^{p-1} B_i} (B_p)  \bP_{\cap_{i=1}^{p} B_i} (N_p)  \bP_{\cap_{i=1}^{p} B_i \cap N_p} (N_{p+1})  \dots  \bP_{\cap_{i=1}^{p}B_i \cap_{i=p+1}^{p+q-1} N_{i}} (N_{p+q}) \\
&= \frac{b}{b+n } \frac{b+a}{b+n+a }  \dots \frac{b+(p-1)a}{b+n +(p-1)a} \frac{n}{b+n+p a }  \frac{n+a}{b+n+(p+1) a }  \dots \frac{n+(q-1)a}{b+n+(p+q-1) a }  \\
\end{align*}
\item Si $a=b=n$
\begin{align*}
\frac{b}{b+n } \frac{b+a}{b+n+a }  \dots \frac{b+(p-1)a}{b+n +(p-1)a}  &=  \frac{n}{n+n } \frac{n+n}{n+n+n }  \dots \frac{n+(p-1)n}{n+n +(p-1)n} \\
&= \frac{1}{2 } \frac{2}{3}  \dots \frac{np}{(p+1)n} \\
&=  \frac{1}{(p+1)} 
\end{align*} 

et 
\begin{align*}
\frac{n}{b+n+p a }  \frac{n+a}{b+n+(p+1) a }  \dots \frac{n+(q-1)a}{b+n+(p+q-1) a }  
&= \frac{n}{(p+2)n }  \frac{n+n}{(p+3)n }  \dots \frac{n+(q-1)n}{n+n+(p+q-1) n } \\
&=\frac{1}{(p+2) }  \frac{2}{(p+3) }  \dots \frac{q}{(p+q+1)  }\\
&= \frac{q! (p+1)!}{(p+q+1)!}\\
&= \frac{1}{\binom{p+q+1}{p+1}}
\end{align*}

On obtient :
$$\bP(E_{p,q}) =  \frac{1}{\binom{p+q+1}{p+1}(p+1)} $$

\item Soit $F_{p,q}$ l'événement 'obtenir exactement $p$ boules blanches en $p+q$ tirages'. On peut ainsi choisir indépendamment le numéro des $p$  tirages parmi les $p+q$ où l'on tire les blanches. On obtient donc $\binom{p+q}{p}\Card(E_{p,q})  =\Card(F_{p,q})$
$$\bP(F_{p,q}) = \binom{p+q}{p}\bP(E_{p,q})$$

\item 
 On a donc pour $b=n=a$ d'après les deux questions précédentes : 
 \begin{align*}
 \bP(F_{p,q})  & =  \frac{\binom{p+q}{p}}{\binom{p+q+1}{p+1}(p+1)}  \\
 					&= \frac{1}{p+q+1 }
 \end{align*}
 
 \item \begin{enumerate}
 \item 
  
 \end{enumerate}
\end{enumerate}
\end{correction}