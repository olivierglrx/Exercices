
\begin{exercice} \;
On d\'efinit la suite $\suiteu$ par $u_0\in\R$ et $\forall n \geq 1, u_{n+1}=\ddp\frac{3}{4}u_n^2-2u_n+3$.
%$\left\lbrace\begin{array}{l}
%u_0\in\R\vsec\\
%u_{n+1}=\ddp\frac{3}{4}u_n^2-2u_n+3
%\end{array}\right.$
\begin{enumerate}
\item \'Etudier la fonction $f$ associ\'ee.
\item \'Etudier le signe de $g: x\mapsto f(x)-x$.
\item Calculer les limites \'eventuelles de la suite $\suiteu$.
\item On suppose que $u_0>2$.
\begin{enumerate}
\item Montrer que la suite est bien d\'efinie et que pour tout $n\in\N$: $u_n>2$.
\item \'Etudier la monotonie de la suite $\suiteu$.
\item \'Etudier le comportement \`{a} l'infini de la suite $\suiteu$.
\end{enumerate}
\item On suppose que $u_0\in\left\rbrack \ddp\frac{2}{3},2\right\lbrack $.
\begin{enumerate}
\item Montrer que la suite est bien d\'efinie et que pour tout $n\in\N$: $u_n\in\left\rbrack \ddp\frac{2}{3},2\right\lbrack$.
\item \'Etudier la monotonie de la suite $\suiteu$.
\item \'Etudier le comportement \`{a} l'infini de la suite $\suiteu$.
\end{enumerate} 
\end{enumerate}
\end{exercice}