% Titre : Logique et trigo
% Filiere : BCPST
% Difficulte :
% Type : DS, DM
% Categories : autres
% Subcategories : 
% Keywords : autres




\begin{exercice}
\begin{enumerate}
\item A quelle condition sur $X,Y\in \R$ a-t-on 
$$X=Y \Longleftrightarrow X^2=Y^2  $$
\item Résoudre dans $\R$ puis dans $[-\pi, \pi[$ l'équation :
\end{enumerate}

%\begin{equation}
%\cos(x+\frac{\pi}{3})=\sin(2x).\\
%\end{equation}

\begin{equation}
|\cos(x)|=|\sin(x)|.\\
\end{equation}

\end{exercice}

\begin{correction}
\begin{enumerate}
\item On a  $X=Y \Longleftrightarrow X^2=Y^2  $ si $X$ et $Y$ sont de même signe.
\item Comme $|\cos(x)|\geq 0$ et $|\sin(x)|\geq 0$ l'équation est équivalente à $\cos^2(x) =\sin^2(x)$, soit encorrectione 
$$\cos(2x)=0.$$
On a donc $2x\equiv \frac{\pi}{2}\quad [\pi]$ ou encorrectione 
$$x\equiv \frac{\pi}{4}\quad [\frac{\pi}{2}]$$
 Les solutions sur $\R$ sont 
 $$\cS =\bigcup_{k\in Z} \{ \frac{\pi}{4}+\frac{\pi k}{2}\}$$
 Sur $[-\pi, \pi[$  les solutions sont :
 $$\cS\cap [-\pi, \pi[ =  \{ \frac{\pi}{4}, \frac{3\pi}{4}, \frac{-\pi}{4}, \frac{-3\pi}{4}\}$$
 




\end{enumerate}
\end{correction}