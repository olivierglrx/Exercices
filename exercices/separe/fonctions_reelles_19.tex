
\begin{exercice}  \;
Dans chacun des cas suivants, \'etudier la parit\'e et l'imparit\'e de la fonction $f$. Indiquer aussi la p\'eriodicit\'e lorsqu'elle est manifeste:
\begin{enumerate}
\begin{minipage}[t]{0.4\textwidth}
 \item $f(x)=\sqrt{x^2}$ 
\item $f(x)=x^2+x^4+x^6+x^8$ 
\item $f(x)=x+x^3+x^5+2x^7$ 
\item $f(x)=\ddp\sqrt{\ddp\frac{1-|x|}{2-|x|}}$ 
\end{minipage}
\begin{minipage}[t]{0.4\textwidth}
\item $f(x)=\ddp\frac{x^3+3x}{x^2+|x|}$ 
\item $f(x)=|x+1|-|x-1|$ 
\item $f(x)=\sin{x}+\cos{x}$ 
\item $f(x)=\cos{x}+\cos{(2x)}$ 
\end{minipage}
\end{enumerate}
\end{exercice}