
\begin{correction}  \;
\begin{itemize}
\item[$\bullet$] \textbf{Calcul de $\mathbf{\cos{\left(\ddp\frac{\pi}{12}\right)}}$:} on a $\ddp \frac{pi}{3}-\frac{\pi}{4} = \frac{\pi}{12}$, donc :
$$\ddp \cos\left(\ddp\frac{\pi}{12}\right) = \cos\left(\frac{\pi}{3}-\frac{\pi}{4}\right) = \cos \left(\frac{\pi}{3}\right)\cos \left(\frac{\pi}{4}\right) + \sin \left(\frac{\pi}{3}\right)\sin \left(\frac{\pi}{4}\right),$$
soit apr\`es simplification : \fbox{$\ddp \cos \left(\frac{\pi}{12}\right) = \frac{\sqrt{2}(1+\sqrt{3})}{4}$}.
\item[$\bullet$] \textbf{Calcul de $\mathbf{\sin{\left(\ddp\frac{\pi}{12}\right)}}$:}  de m\^eme, on a 
$$\ddp \sin\left(\ddp\frac{\pi}{12}\right) = \sin\left(\frac{\pi}{3}-\frac{\pi}{4}\right) = \sin \left(\frac{\pi}{3}\right)\cos \left(\frac{\pi}{4}\right) - \cos \left(\frac{\pi}{3}\right)\sin \left(\frac{\pi}{4}\right),$$
soit apr\`es simplification : \fbox{$\ddp \sin \left(\frac{\pi}{12}\right) = \frac{\sqrt{2}(\sqrt{3}-1)}{4}$}.
\item[$\bullet$] \textbf{Calcul de $\mathbf{\cos{\left(\ddp\frac{7\pi}{12}\right)}}$:} \`a partir du calcul pr\'ec\'edent, on a :
$$\ddp \cos{\left(\ddp\frac{7\pi}{12}\right)} = \cos{\left(\ddp\frac{\pi}{12}+\frac{\pi}{2}\right)} = -\sin{\left(\ddp\frac{\pi}{12}\right)} = \fbox{$\ddp \frac{\sqrt{2}(1-\sqrt{3}}{4}$}$$
%-----
\item[$\bullet$] \textbf{Calcul de $\mathbf{\cos{\left(\ddp\frac{\pi}{8}\right)}}$ et $\mathbf{\sin{\left(\ddp\frac{\pi}{8}\right)}}$:}  Il suffit de remarquer que: $\ddp\frac{\pi}{8}=\frac{\pi}{4}\times \demi$. On pose alors $\theta=\ddp\frac{\pi}{8}$ et on obtient par la formule de duplication des angles : $\cos{(2\theta)}=2\cos^2{(\theta)}-1.$
Ainsi, 
$$\cos^2{\left(\ddp\frac{\pi}{8}\right)}=\ddp\frac{\cos{\left(\frac{\pi}{4}\right)}+1}{2}=\ddp\frac{\sqrt{2}+2}{4}.$$
Le r\'eel $\ddp\frac{\pi}{8}$ est dans l'intervalle $\left\lbrack 0,\ddp\frac{\pi}{2}\right\rbrack$ et le cosinus est positif sur cet intervalle, ainsi: 
\begin{equation*}
\fbox{
$\cos{\left(\ddp\frac{\pi}{8}\right)}=\ddp\sqrt{\frac{\sqrt{2}+2}{4}}.$
}
\end{equation*}
Une fois le cosinus connu, le sinus se d\'eduit par la formule $\cos^2{(x)}+\sin^2{(x)}=1$. On obtient ainsi, 
$$\sin^2{\left(\ddp\frac{\pi}{8}\right)}=1-\ddp \frac{\sqrt{2}+2}{4} = \frac{2-\sqrt{2}}{4}.$$
L\`a encore, le sinus \'etant positif sur l'intervalle $\left\lbrack 0,\ddp\frac{\pi}{2}\right\rbrack$, on obtient: 
\begin{equation*}
\fbox{
$\sin{\left(\ddp\frac{\pi}{8}\right)}=\ddp\sqrt{\frac{2-\sqrt{2}}{4}}.$
}
\end{equation*}
\end{itemize}
\end{correction}