% Titre : suites
% Filiere : BCPST
% Difficulte : 
% Type : TD 
% Categories :suites
% Subcategories : 
% Keywords : suites



\begin{exercice}
On consid\`ere les suites $\suiteu$ et $\suitev$ d\'efinies par
$$v_0>u_0>0, \; \textmd{ et } \; \forall n\in\N,\ u_{n+1}=\ddp\frac{u_n^2}{u_n+v_n}, \textmd{ et } \ v_{n+1}=\ddp\frac{v_n^2}{u_n+v_n}.$$
%$$\left\lbrace\begin{array}{l}
%v_0>u_0>0\vsec\\
%\forall n\in\N,\ u_{n+1}=\ddp\frac{u_n^2}{u_n+v_n}\vsec\\
%\forall n\in\N,\ v_{n+1}=\ddp\frac{v_n^2}{u_n+v_n}
%\end{array}\right.$$
\begin{enumerate}
 \item
Montrer que les deux suites $\suiteu$ et $\suitev$ sont bien d\'efinies et qu'elles sont strictement positives. 
\item 
Montrer que $\suiteu$ et $\suitev$ convergent.
\item 
Calculer leur limite respective.

\end{enumerate}
\end{exercice}


\%\%\%\%\%\%\%\%\%\%\%\%\%\%\%\%\%\%\%\%
\%\%\%\%\%\%\%\%\%\%\%\%\%\%\%\%\%\%\%\%
\%\%\%\%\%\%\%\%\%\%\%\%\%\%\%\%\%\%\%\%



\begin{correction}
On donne seulement les indications de solutions. 
	\begin{enumerate}
		\item On prouve par récurrence la propriété $P(n): " u_n>0 \text{ et } v_n>0"$. Ce qui prouve que $\suite{u}$ et $\suite{v}$ sont bien définies. 
		\item On montre  que les suites $\suite{u}$et $\suite{v}$ sont  décroissantes, il suffit pour cela d'étudier le signe de $u_{n+1}-u_n$  et de $v_{n+1}-v_n$. Les suites sont donc décroissantes et minorées par 0, le théorème  des suite monotones assurent la convergence. 
		\item On note $\ell$ et $\ell'$ les limites respectives de $\suite{u}$ et $\suite{v} $. On obtient 
		\item En raisonnant par l'absurde, on montre que les deux limites sont nulles. 

	\end{enumerate}
\end{correction}