% Titre : Etude dérivabilité 
% Filiere : BCPST
% Difficulte :
% Type : DS, DM
% Categories : analyse
% Subcategories : 
% Keywords : analyse




\begin{exercice}   \;
Etudier la continuité et la dérivabilité de 
$$f(x)=\left\{\begin{array}{ll}
x^2\sin\left(\frac 1x\right)&x\neq 0\\
0&x=0
\end{array}\right.\quad\quad\quad
g(x)=\left\{\begin{array}{ll}
x^3\sin\left(\frac 1x\right)&x\neq 0\\
0&x=0.
\end{array}\right.$$

Ces fonctions sont-elles de classe $\cC^1$ ? 
\end{exercice}

\begin{correction}

$f$ et $g$ sont $\cC^\infty$ sur $\R^*$ comme composée et produit de fonctions usuelles et on a $\forall x\in \R^*$

$$f'(x) = 2x \sin\left(\frac 1x\right) -\cos\left(\frac 1x\right)$$
et 

$$g'(x) = 3x^2 \sin\left(\frac 1x\right) -x\cos\left(\frac 1x\right)$$


Etudions la dérivabilité en $0$. 
On  a $\tau_{f,0} (x)=\frac{f(x)-f(0)}{x-0}= x\sin\left(\frac 1x\right)$ 
On montre comme dans le TD que $\lim_{x\tv0}\tau_{f,0} (x) =0$
Donc $f$ est dérivable en $0$ et $f'(0)=0$

De même, avec 
$\tau_{g,0} (x)=\frac{g(x)-g(0)}{x-0}= x^2\sin\left(\frac 1x\right)$ 
On montre que $\lim_{x\tv0}\tau_{g,0} (x) =0$
Donc $g$ est dérivable en $0$ et $g'(0)=0$


Il faut maintenant étudier la continuité de  la dérivée. 
$$\lim_{x\tv 0} g'(x) = \lim_{x\tv 0}  3x^2 \sin\left(\frac 1x\right) -x\cos\left(\frac 1x\right) = 0$$
Donc $g'$ est continue en $0$  ainsi $g$ est de classe $\cC^1$ sur $\R$. 

En revanche $f'(x) $ n'admet pas de limite en $0$ et en particulier $f'$ n'est pas continue en $0$. Ainsi $f$ n'est  pas de classe $\cC^1$. 

\end{correction}