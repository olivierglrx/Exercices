
\begin{correction} \;
\begin{enumerate}
 \item Comme toujours pour ce genre de question, on fait une r\'ecurrence.
\begin{itemize}
\item[$\bullet$] On montre par r\'ecurrence sur $n\geq 3$ la propri\'et\'e $\mathcal{P}(n):\quad u_n \textmd{ d\'efini et  } u_n> 1.$
\item[$\bullet$]  Initialisation: pour $n=3$:\\
\noindent On a: $u_1=-1$ puis $u_2=-7$ et $u_3=\ddp\frac{37}{5}>1$. Ainsi, $\mathcal{P}(3)$ est vraie.
\item[$\bullet$]  H\'er\'edit\'e: soit $n\geq 3$, on suppose la propri\'et\'e vraie \`a l'ordre $n$, montrons que $\mathcal{P}(n+1)$ est vraie. Par hypoth\`ese de r\'ecurrence, on sait que $u_n>1$, donc $u_n-1\not=0$ et $u_{n+1}$ est bien d\'efini. De plus, on a
$$u_{n+1}>1 \; \Leftrightarrow \; \ddp\frac{5u_n-2}{u_n+2}>1 \; \Leftrightarrow \; 5u_n -2>u_n+2 \; \Leftrightarrow \; u_n>1.$$
Ici on a utilis\'e le fait que $u_n>1$ d'apr\`es $\mathcal{P}(n)$, et donc que $u_n+2>0$. On arrive $u_n>1$ qui est bien vrai, donc par \'equivalences, $u_{n+1}>1$ est vrai aussi. Ainsi, $\mathcal{P}(n+1)$ est vraie.
\item[$\bullet$]  Conclusion: il r\'esulte du principe de r\'ecurrence que $\forall n\geq 3,\ u_n>1.$
\end{itemize}
%--
\item La suite $(v_n)_{n\in\N}$ est bien d\'efinie  car $u_0,\ u_1,\ u_2$ ne sont pas \'egaux \`a 1 et ensuite on a $\forall n \geq 3, u_n > 1$. Ainsi pour tout $n\in\N$, on a bien $u_n-1\not= 0$ et $v_n$ bien d\'efini. 
%--
\item Soit $n\in\N$:
$$\begin{array}{lll}
v_{n+1}&=& \ddp\frac{u_{n+1}-2}{u_{n+1}-1}
= \ddp\frac{\frac{5u_n-2-2u_n-4}{u_n+2}}{\frac{5u_n-2-u_n-2}{u_n+2}}
= \ddp\frac{3u_n-6}{4u_n-4}
= \ddp\frac{3}{4}\ddp\frac{u_n-2}{u_n-1}
= \ddp\frac{3}{4}v_n.
\end{array}$$
Ainsi la suite $(v_n)_{n\in\N}$ est une suite g\'eom\'etrique de raison $\ddp\frac{3}{4}$ et de premier terme $2$.
\item On en d\'eduit la formule explicite de $v_n$:
$$\forall n\in\N,\quad v_n=2\left( \ddp\frac{3}{4} \right)^n.$$
En remarquant que: $u_n(v_n-1)=v_n-2$ et que la suite $(v_n)_{n\in\N}$ \'etait toujours diff\'erente de 1, on obtient que
$$\forall n\in\N,\ u_n=\ddp\frac{v_n-2}{v_n-1}\Rightarrow u_n=\ddp\frac{2(\frac{3}{4})^n-2}{2(\frac{3}{4})^n-1}.$$
\item Comme $-1<\ddp\frac{3}{4}<1$, la suite $(v_n)_{n\in\N}$ converge vers 0 et ainsi, on a: $\lim\limits_{n\to\N} u_n=2$.
\end{enumerate}
\end{correction}