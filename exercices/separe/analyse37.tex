% Titre : Somme double $\max$ + info
% Filiere : BCPST
% Difficulte :
% Type : DS, DM
% Categories : analyse
% Subcategories : 
% Keywords : analyse




\begin{exercice}
\begin{enumerate}
\item Montrer par récurrence que pour tout $n\in \N$, 
$$ \sum_{k=1}^n  k^2= \frac{n(n+1)(2n+1)}{6}$$

\item Soit $i\in \N$ et $n\in \N$ tel que $i\leq n$. Caculer en fonction de $i$ et $n$ :
$$\sum_{j=i+1}^n j$$

\item 
On rappelle que l'on note $\max(i,j) =  \left\{
    \begin{array}{ll}
       i & \text{si\, } i \geq j \\
        j & \mbox{sinon.}
    \end{array}
\right.
$
Montrer que pour tout $n\in \N$, $$\ddp \sum_{i,j\in \intent{1,n}} \max(i,j)  = \sum_{i=1}^n \frac{n^2+i^2 +n-i}{2}$$ 
\item En déduire que 
$$\ddp \sum_{i,j\in \intent{1,n}} \max(i,j) \left(   \frac{n(n+1)(4n-1)}{6} \right)	 $$

\item On note $$S_k =  \sum_{i,j\in \intent{1,1000}} \max(i^k,j^k).$$ 
\begin{enumerate}
\item Rappeler ce que renvoie l'instruction Python $\texttt{range(a,b)}$ avec deux entiers $a,b\in \N$ tel que $a\leq b$.
\item Ecrire un script Python qui demande à l'utilsateur la valeur de $k$, calcul $S_k$ et affiche le résultat. 
\end{enumerate}
\end{enumerate}
\end{exercice}



\begin{correction}
Pour $n=0 $ on a d'une part $\sum_{k=1}^0 k^2 =0$ et 
$\frac{0*1*(2*0+1)}{6}=0$ la propriété est vraie au rang 0

Montrons l'hérédité de la formule et supposons qu'il existe $n\in \N$ tel que $ \sum_{k=1}^n  k^2= \frac{n(n+1)(2n+1)}{6}$
On a 
$$\sum_{k=1}^{n+1} k^2 =\sum_{k=1}^{n} k^2 +  (n+1)^2$$
et donc par hypothèse : 
\begin{align*}
\sum_{k=1}^{n+1} k^2 &=\frac{n(n+1)(2n+1)}{6} +  (n+1)^2\\
									&= \frac{(n+1) [ n(2n+1) +6(n+1)}{(n+1)^2}\\
									&= \frac{(n+1) [ 2n^2+ +7n+6)}{(n+1)^2}		\\												&= \frac{(n+1) [ (2n+3)(n+2)}{(n+1)^2}				\\											&= \frac{(n+1) ((n+1)+1) (2(n+1)+1)}{(n+1)^2}			
\end{align*}
La formule est  bien héréditaire elle est donc vraie pour tout $n\in \N$. 



D'après le cours :
$\ddp \sum_{j=i+1}^n j = \frac{(n+i+1)(n-i)}{2}$
\begin{align*}
\sum_{i,j \in \intent{1,n}} \max(i,j) &= \sum_{i=1}^n \sum_{j=1}^n\max(i,j)\\
												&= \sum_{i=1}^n \sum_{j=1}^i\max(i,j) + \sum_{i=1}^n \sum_{j=i+1}^n\max(i,j)\\
												&= \sum_{i=1}^n \sum_{j=1}^i i  + \sum_{i=1}^n \sum_{j=i+1}^n j\\
													&= \sum_{i=1}^n i^2+ \sum_{i=1}^n \frac{(n+i+1)(n-i)}{2} \\
													&= \sum_{i=1}^n i^2 + \frac{n^2-i^2 +n-i}{2}\\		
													&= \sum_{i=1}^n \frac{n^2+i^2 +n-i}{2}\\	
\end{align*}
\begin{align*}
\sum_{i,j \in \intent{1,n}} \max(i,j)	&= \frac{1}{2} \left( n (n^2+n) + \sum_{i=1}^n i^2  -  \sum_{i=1}^n i\right)\\
													&= \frac{1}{2} \left(  n^2 (n+1) +  \frac{n(n+1)(2n+1)}{6} -   \frac{n(n+1)}{2}\right)\\
													&=\frac{1}{2} \left(   \frac{n(n+1)(6n +(2n+1) -3 }{6} \right)\\
													&=\frac{1}{2} \left(   \frac{n(n+1)(8n-2)}{6} \right)\\		
													&= \left(   \frac{n(n+1)(4n-1)}{6} \right)											
\end{align*}

\texttt{range(a,b)} renvoie la suite d'entiers de $a$ à $b-1$. 


\begin{lstlisting}
k = int(input ('quelle est la valeur de k ? ))
S=0
for i in range(1,1001):  #on fait une boucle for pour obtenir la somme sur i 
  for j in range(1,1001):  #et une deuxieme pour la somme sur j
    if i<=j:
      S=S+j**k
    else:
      S=S+i**k
print(S)
\end{lstlisting}


\end{correction}