% Titre : Sujet Révision - Shuffle Ipod
% Filiere : BCPST
% Difficulte :
% Type : DS, DM
% Categories : probabilite
% Subcategories : 
% Keywords : probabilite





\begin{probleme}
Dans tout le problème, $n$ sera un entier naturel supérieur ou égal à $2$. 

Un I-Pod contient $n$ pistes (numérotées de $1$ à $n$) et fonctionne en mode aléatoire  selon le protocole suivant : 

\begin{itemize}
\item La première psite lie est choisie de façon aléatoire et uniforme parmi les $n$ pistes. 
\item A la fin de la lecture d'une piste, la suivante est choisie de façon aléatoire et uniforme parmi les $n$ pistes. (Il est donc possible que la même piste soit lue plusieurs fois de suite...) 

Ce problème étudie différents aspects de vette lecture aléatoire. Les différentes partites sont dans une grande mesure indépendantes les unes des autres. 
\end{itemize}


\paragraph{Partie A}
\noindent

Dans cette partie on fixe un entier naturel $k$ supéreiur ou égal à $1$ et on s'intéresse aux $k$ premières lectures effectuées. Pour tout $i\in \intent{1, n}$, on note $X_i$ le nombre de fois où la piste numéro $i$ est lue au cours des $k$ premières lectures. 

\begin{enumerate}
\item Déterminer la loi de $X_i$ et donner son espérance est sa variance. 


\item Les variables aléatoires $X_1, X_2, \dots , X_n $ sont-elles indépendantes ? 


\item \begin{enumerate}
\item Que vaut $X_1 +X_2 +\dots +X_n $ ? 


\item En déduire que la covariance de $X_i$ et $X_j $ pour tout $i\neq j$ vaut $\frac{-k}{n^2}.$


\end{enumerate}
\item \begin{enumerate}
\item Déterminer la loi conjointe de $X_i $ et $X_j$ pour tout $i\neq j$. 

\item  Retrouver alors le résultat du 3b. 

\end{enumerate}
\item Commenter le signe de la covariance de $X_i$ et $X_j$ pour $i\neq j$. 



\item Soient $a_1, a_2, \dots, a_n$, $n$ entiers naturels. 
\begin{enumerate}
\item On suppose que $\sum_{i=1}^n a_i \neq k$. Que vaut la probabilité 
$\bP(X_1 =a_1 \cap X_2 =a_2 \cap \dots X_n =a_n)$ ? 

\item On suppose maintenant  que $\sum_{i=1}^n a_i = k$. Montrer que probabilité 
$$\bP(X_1 =a_1 \cap X_2 =a_2 \cap \dots X_n =a_n) = \frac{k!}{a_1! a_2! \dots a_n ! } \left( \frac{1}{n}\right)^k$$ 

\end{enumerate}
\end{enumerate}


\paragraph{Partie B}
\noindent 

Pour tout $k\in \N^*$, on note $Z_k$ le nombre de pistes différentes qui ont été lues au moins une fois au cours des $k$ premières lectures. 
\begin{enumerate}
\item Décrire avec soin l'ensemble des valeurs que prend $Z_k$ en fonction de $n$ et $k$. 


\item Quelle est la loi de $Z_1$ ? Donner son espérance et sa variance. 


\item \begin{enumerate}
\item Soient $i$  et $j$  entre $1$ et $n$. Détermminer $P_{(Z_k =i) } (Z_{k+1} =j)$ en distinguant les cas $j=i$, $j=i+1$ et $j\notin \{ i, i+1\}$


\item En déduire que pour tout $k\in \N^*$ et pour tout $i\in \intent{1, n}$, on a : 
$$P(Z_{k+1}  =i )= \frac{i}{n} P(Z_k =i ) + \frac{n-i+1}{n} P(Z_k =i-1)$$

\item Calculer $P(Z_k=1)$ pour tout $k\in \N^*$.

\item On pose, pour tout $k\in \N^*$, $\alpha_k =n^{k-1} P(Z_k =2)$. Exprimer $\alpha_{k+1} $ en fonction de $\alpha_k$ et $n$, puis en déduire l'expression de $\alpha_k$ en fonction de $k$ et $n$. 

\item Déduire de ce qui précède  la valeur de $P(Z_k=2) $ pour tout $k\in \N^*$. 
\end{enumerate}
\item \begin{enumerate}
\item A l'aide de la question $3b$ montrer que :
$$E(Z_{k+1}) = \frac{n-1}{n} E(Z_k) +1$$
\item En déduire l'expression de $E(Z_k)$ en fonction de $n$ et $k$. 
\item Calculer la limite quand $k\tv +\infty$ de $E(Z_k)$. Ce résultat  était-il prévisible ? 
\item  Calculer la limite quand $n\tv +\infty$ de $E(Z_k)$. Ce résultat  était-il prévisible ? 


\end{enumerate}
\item On va dans cette question montrer par récurrence sur $k$ que pour tout $k\in \N^*$:
$$\forall i\in \intent{ 1, n}, \quad P(Z_k =i) =\frac{\binom{n}{i}}{n^k} \sum_{j=0}^{i-1} (-1)^j \binom{i}{j}(i-j)^k$$
\begin{enumerate}
\item Montrer que la propriété est vraie pour $k=1$ (traiter séparément les cas $i=1$ et $i>1$.) 


On suppose la propriété vraie pour un certain rang $k\in \N^*$ et on va montrer qu'elle est vraie pour le rang $k+1$. 
\item Montrer que la relation au rang $k+1$ est vraie pour $i=1$. 



\item A l'aide du résultat de la question 3b, montrer que la relation au rang $k+1$ est vraie pour tout $i\geq 2$. 



\item Conclure. 

\item Soient $k$ et $i$ deux entiers tels que $1\leq k < i $ . Que vaut $\ddp \sum_{j=0}^{i-1} (-1)^j \binom{i}{j}(i-j)^k$ ?



\end{enumerate}

\end{enumerate}
\end{probleme}






\begin{correction}
\begin{enumerate}
\item Par symmétrie du problème, tous les $X_i$ suivent la même loi. On a $X_i (\Omega) =\intent{0,n}$ et ce sont des sommes de  $k$ variables de Bernouilli de paramètre $\frac{1}{n}$ ce sont donc des binomiales $\cB(k,\frac{1}{n})$
\item $\bP (X_1 = k , X_2=1)=0$ en effet, si la piste $1$ a été jouée  $k$ fois, il n'est pas possible que la piste $2$ ait été jouée. En revanche $\bP(X_1=k)\bP(X_2=1) \neq 0$. Ainsi les variables ne sont pas 2 à $2$ indépendantes, donc a fortiori elles ne sont pas mutuellement indépendantes. 
\item $\sum_{i=1}^n X_i = k$
\item Remarquons que les covariances $\cov(X_i,X_j) $ sont toutes égales pour tout $i, j \in \intent{1, n}^2$ pour $i\neq  j$  par symmétrie du problème. On note ce nombre $\alpha$. 

Par ailleurs, $\cov(X_i,X_i) = V(X_i)=\frac{1}{n}\frac{n-1}{n} k $ (formule du cours) 

Maintenant on calcule $\ddp \cov(\sum_{i=1}^n X_i,X_j)$ de deux manières : 
Par linéarité vis-à-vis de la première variable   on obtient 
$$ \cov(\sum_{i=1}^n X_i,X_j) =\sum_{i=1}^n \cov(X_i,X_j) =(n-1) \alpha +\frac{1}{n}\frac{n-1}{n} k$$

On calcule directement à l'aide de la formule de Koenig Huygens, comme $\sum_{i=1}^n X_i = k$ on obtient : 
$$\cov(\sum_{i=1}^n X_i,X_j) = \cov(k,X_j) = E(kX_j)-E(k)E(X_j)  = 0$$


Ainsi 
$$ (n-1) \alpha +\frac{1}{n}\frac{n-1}{n} k  =0$$ et donc 
$$ \cov(X_i,X_j)   = -\frac{k}{n^2}$$

\item $(X_i,X_j) (\Omega) =\intent{1,k}^2$ 

Soit $u,v \in \intent{1,k}^2$. 
 Si $u+v\geq k$, on a alors $\bP(X_i = u ,X_j= v) = 0 $

Maintenant si $u+v\leq k$, il faut choisir les places des $u$ fois où la piste $i$ est jouée : il y a 
$\binom{k}{u} $ possibilités, chacune arrivant avec la probabilité $(\frac{1}{n})^{u}$. Ensuite il faut placer les $v$ fois où la psite $j$ est jouée parmi les lectures restantes : il y a  $\binom{k-u}{v} $ possibilités, chacune arrivant avec la probabilité $(\frac{1}{n})^{v}$. 

On obtient : 

$$\bP(X_i = u , X_j = v) =\binom{k}{u}\binom{k-u}{v} \left(\frac{1}{n}\right)^{u+v} \left(1-\frac{2}{n}\right)^{k-u-v}$$


\item 
On sait que $E(X_i) =E(X_j)=\frac{k}{n}$

Calculons maintenant $E(X_iX_j)$. 
\begin{align*}
E(X_iX_j) &= \sum_{u=0}^k\sum_{v=0}^k uv \bP(X_i=u, X_j =v)\\
			&=  \sum_{u=0}^k \sum_{v=0}^{k-u} uv \binom{k}{u}\binom{k-u}{v} \left(\frac{1}{n}\right)^{u+v} \left(1-\frac{2}{n}\right)^{k-u-v}\\
			&=\sum_{u=0}^k u \binom{k}{u} \left(\frac{1}{n}\right)^{u}  \sum_{v=0}^{k-u} v \binom{k-u}{v} \left(\frac{1}{n}\right)^{v} \left(1-\frac{2}{n}\right)^{k-u-v} \\
\end{align*}

Analysons la  somme  intérieure : 
\begin{align*}
\sum_{v=0}^{k-u} v \binom{k-u}{v} \left(\frac{1}{n}\right)^{v} \left(1-\frac{2}{n}\right)^{k-u-v} &=
\sum_{v=1}^{k-u} v \binom{k-u}{v} \left(\frac{1}{n}\right)^{v} \left(1-\frac{2}{n}\right)^{k-u-v} \\
&= \sum_{v=1}^{k-u} (k-u) \binom{k-u-1}{v-1} \left(\frac{1}{n}\right)^{v} \left(1-\frac{2}{n}\right)^{k-u-v}\\
&= (k-u) \sum_{v=0}^{k-u-1} \binom{k-u-1}{v} \left(\frac{1}{n}\right)^{v+1} \left(1-\frac{2}{n}\right)^{k-u-1-v}\\
&= (k-u)\frac{1}{n} \sum_{v=0}^{k-u-1} \binom{k-u-1}{v} \left(\frac{1}{n}\right)^{v} \left(1-\frac{2}{n}\right)^{k-u-1-v}
\end{align*}
On reconnait un binome de Newton : 
\begin{align*}
\sum_{v=0}^{k-u} v \binom{k-u}{v} \left(\frac{1}{n}\right)^{v} \left(1-\frac{2}{n}\right)^{k-u-v}&=
(k-u) \frac{1}{n}  \left(\frac{1}{n}+1-\frac{2}{n}\right)^{k-u-1}
\end{align*}

Revenons en au calcul de $E(X_iX_j)$, en remplacant la  somme intérieure par le terme que l'on vient de trouver. 
\begin{align*}
E(X_iX_j) &= \sum_{u=0}^k u \binom{k}{u} \left(\frac{1}{n}\right)^{u}    (k-u) \frac{1}{n}  \left(1-\frac{1}{n}\right)^{k-u-1}\\
&=  \sum_{u=1}^k  (k-u)  k\binom{k-1}{u-1} \left(\frac{1}{n}\right)^{u+1}    \left(1-\frac{1}{n}\right)^{k-u-1}\\
\end{align*}
On utilise alors le fait que $(k-u) \binom{k-1}{u-1} =(k-1) \binom{k-2}{u-1}$ ( on peut le vériifer en passant par les factorielles) on obtient alors :
\begin{align*}
E(X_iX_j) &= \sum_{u=1}^{k-1}  k (k-1) \binom{k-2}{u-1} \left(\frac{1}{n}\right)^{u+1}    \left(1-\frac{1}{n}\right)^{k-u-1}\\
&= k (k-1)  \sum_{u=0}^{k-2}  \binom{k-2}{u} \left(\frac{1}{n}\right)^{u+2}    \left(1-\frac{1}{n}\right)^{k-u-2}\\
&= k (k-1) (\frac{1}{n} )^2 \sum_{u=0}^{k-2}  \binom{k-2}{u} \left(\frac{1}{n}\right)^{u}    \left(1-\frac{1}{n}\right)^{k-2- u}\\
&= k (k-1) (\frac{1}{n} )^2 \left( \frac{1}{n} +1 -\frac{1}{n}\right)^{k-2}\\
&=k(k-1) (\frac{1}{n} )^2
\end{align*}


On obtient bien alors 
$$Cov(X_i,X_j)  =E(X_iX_j) -E(X_i)E(X_j) = k(k-1) (\frac{1}{n} )^2 - (\frac{k}{n})^2= \frac{-k}{n^2}$$


\item La covariance est négative. En effet les deux variables aléatoires ont un comportement opposées : si le morceau $i$ est lu beaucoup de fois, le morceau $j$ a tendance à être moins lu. 


\item 
Dans ce cas, on a nécessairement
$\bP(X_1 =a_1 \cap X_2 =a_2 \cap \dots X_n =a_n)=0$ car il y a $k$ morceaux joués. 


\item Il faut placer les $a_1$ lectures du morceau $1$ parmi les $k$ lectures, puis les $a_2$ lecture du morceau $2$ parmi les $k-a$ restante et ainsi de suite. 

Chacune de ces séquence a pour probabilité $\left(\frac{1}{n} \right)^{a_1}\left(\frac{1}{n} \right)^{a_2}...\left(\frac{1}{n} \right)^{a_n}=\left(\frac{1}{n} \right)^{k}$

Ainsi $$\bP(X_1 =a_1 \cap X_2 =a_2 \cap \dots X_n =a_n) = 
\binom{k}{a_1}\binom{k-a_1}{a_2}...\binom{k-a_1-a_2 -...-a_{n-1}}{a_n}
\left(\frac{1}{n} \right)^{k} $$

Simplifions le produit des coefficients binomiaux : 

\begin{align*}
\binom{k}{a_1}\binom{k-a_1}{a_2}...\binom{k-a_1-a_2 -...-a_{n-1}}{a_n} &= 
\frac{k!}{a_1 ! (k-a_1)!} \frac{(k-a_1)!}{a_2 ! (k-a_2)!} ... \frac{(k-a_1-a_2 -...-a_{n-1})!}{a_n ! (k-a_1-a_2 -...-a_{n-1}-a_n)!}\\
&= \frac{k!}{a_1  ! a_2! ... a_n! }
\end{align*}

On obtient bien le résultat demandé. 
\end{enumerate}

\begin{enumerate}
\item $Z_k (\Omega) =\intent{1, \min(k,n)}$
\item $Z_1  (\Omega) =\{1\}$. $Z_1$ est variable aléatoire égale certaine égale à $1$. 
$\bP(Z_1 =1 ) = 1$ et $E(Z_1) = 1$, $V(Z_1) =0$

\item \underline{Si $j=i$}
$\bP_{(Z_k =i)} (Z_{k+1} = i ) $ est la probabilité que le morceau joué au rang $k+1$ soit compris dans les $i$ premiers morceaux joués. On a donc 
$$\bP_{(Z_k =i)} (Z_{k+1} = i )= \frac{i}{n} $$

\underline{Si $j=i+1$}
$\bP_{(Z_k =i)} (Z_{k+1} = i+1 ) $ est la probabilité que le morceau joué au rang $k+1$ ne  soit pas compris dans les $i$ premiers morceaux joués. On a donc 
$$\bP_{(Z_k =i)} (Z_{k+1} = i )= \frac{n-i}{n} $$

\underline{Si $j\notin \{i,i+1\}$}
$\bP_{(Z_k =i)} (Z_{k+1} = j ) $ est un événement impossible et on a 
$$\bP_{(Z_k =i)} (Z_{k+1} = j ) =0$$

\item On utilise la formule des probabilités totales. 

On obtient 
\begin{align*}
\bP(Z_{k+1} = i ) &= \sum_{j=1}^{\min(n,k)} \bP( Z_{k+1} = i  \text{ et }  Z_k = j)\\
						&= \bP( Z_{k+1} = i  \text{ et }  Z_k = i) +\bP( Z_{k+1} = i  \text{ et }  Z_k = i-1)\\
						&= \bP_{( Z_k = i)}( Z_{k+1} = i  ) P( Z_k = i) +\bP_{( Z_k = i-1)}( Z_{k+1} = i ) \bP(Z_k = i-1)\\
						&= \frac{i}{n} P(Z_k =i ) + \frac{n-i+1}{n} P(Z_k =i-1)	
\end{align*}
\item $Z_k=1$ correspond à l'événement : " 1 seule piste a été jouée". On a donc 
$$P(Z_k = 1) =(\frac{1}{n})^{k-1}$$

\item \begin{align*}
\alpha_{k+1} &= n^{k} P(Z_{k+1} = 2) \\
					&= n^k  \left(  \frac{2}{n} P(Z_k =2 ) + \frac{n-1}{n} P(Z_k =1)	 \right)\\
					&= 2 n^{k-1}   P(Z_k =2 ) + (n-1)n^{k-1} \frac{1}{n^{k-1}} 	 \\
					&= 2 \alpha_k+ n-1
\end{align*}

C'est donc une suite arithmético géométrique. On obtient 
$$\alpha_k = (n-1) (2^{k-1} -1)$$

\item $$P(Z_k=2) = \frac{1}{n^{k-1}}\alpha_k = \frac{ (n-1) (2^{k-1} -1)}{n^{k-1}}$$
\item \begin{align*}
E(Z_{k+1})&= \sum_{i=1}^n i P(Z_{k+1} =i )\\
				&= \sum_{i=1}^n  i\left(\frac{i}{n} P(Z_k =i ) + \frac{n-i+1}{n} P(Z_k =i-1)\right)	\\
				&= \sum_{i=1}^n \frac{i^2}{n} P(Z_k =i ) + \sum_{i=1}^n i    \frac{n-i+1}{n} P(Z_k =i-1)\\
				&= \sum_{i=0}^n \frac{i^2}{n} P(Z_k =i ) + \sum_{i=0}^{n-1} (i+1)    \frac{n-i}{n} P(Z_k =i) \quad \text{ changement de variable}\\
					&= \sum_{i=0}^n \frac{i^2}{n} P(Z_k =i ) + \sum_{i=0}^{n-1}  -\frac{i^2}{n} 
					P(Z_k =i) + (i-\frac{i}{n}) P(Z_k=i) + (P(Z_k=i)\\
					&= nP(Z_k=n) + (1-\frac{1}{n}) (E(Z_k) - P(Z_k=n) ) + (1 - P(Z_k=n))\\
					&= (1-\frac{1}{n})E(Z_k)  - 1 
\end{align*}



\item C'est de nouveau une suite arithmético géométrique. On trouve : 
$$E(Z_k) = n \left( 1 - \left(1- \frac{1}{n}\right)^k\right)$$

\item $$\lim_{k\tv +\infty } E(Z_k) =  n$$
En effet quand le nombre de lectures est grand, toutes les pistes ont tendance à avoir été lue au moins une fois et $Z_k$ tends vers la constante $n$, sont espérance en particulier aussi. 

\item On  va faire un DL, on a $\left(1- \frac{1}{n}\right)^k = 1- \frac{k}{n} +o(\frac{1}{n})$
et donc 
$$E(Z_k) = k +o(1)$$
donc $$\lim_{n\tv +\infty } E(Z_k) =  k$$

Si le nombre de pistes est très grand devant le nombre de lecture, on aura à chaque écoute un nouveau morceau et donc le nombre de morceaux différents joués sera égale au nombre de lectures, autrement dit $Z_k $ aura tendance à être égal à $k$.


\item On est dans le cas $k=1$. On cherche à montrer que 
$$\forall i\in \intent{ 1, n}, \quad P(Z_1 =i) =\frac{\binom{n}{i}}{n} \sum_{j=0}^{i-1} (-1)^j \binom{i}{j}(i-j)$$


\underline{ Si $i =1$}
On cherche à montrer que 
$$P(Z_1 =1) =\frac{\binom{n}{1}}{n} \sum_{j=0}^{0} (-1)^j \binom{1}{j}(1-j)$$
On sait que $P(Z_1= 1) =1$  et par ailleurs 
$\frac{\binom{n}{1}}{n} \sum_{j=0}^{0} (-1)^j \binom{1}{j}(1-j) =  \frac{n}{n}((-1)^0 1 (1-0) = 1$

\underline{ Si $i >1$}
On sait que $P(Z_1= i) =0$. Par ailleurs 
\begin{align*}
\sum_{j=0}^{i-1} (-1)^j \binom{i}{j}(i-j) &=  \sum_{j=0}^{i-1} (-1)^j \frac{i!}{j! (i-j)!}
(i-j) \\
&=  \sum_{j=0}^{i-1} (-1)^j \frac{i!}{j! (i-j-1)!}\\
&=  i\sum_{j=0}^{i-1} (-1)^j \frac{(i-1)!}{j! (i-1-j)!}\\
&=  i\sum_{j=0}^{i-1} (-1)^j \binom{i-1}{j}\\
&=  i(1-1)^{i-1}\\
&=0
\end{align*}

La propriété est donc vérifiée au rang $k=1$. 



\item Pour $i=1$ on cherche à prouver que 
$$P(Z_{k+1} =1) =\frac{\binom{n}{1}}{n^{k+1}} \sum_{j=0}^{0} (-1)^j \binom{1}{j}(1-j)^{k+1}$$

On sait d'une part que $P(Z_{k+1} =1) = (\frac{1}{n} )^{k}$

et d'autre part on  a 
\begin{align*}
\frac{\binom{n}{1}}{n^{k+1}} \sum_{j=0}^{0} (-1)^j \binom{1}{j}(1-j)^{k+1} &= \frac{1}{n^k} (-1) ^0 1 (1-0){k+1}\\
&= \frac{1}{n^k}
\end{align*}



\item Soit $i\geq 2$ 

\begin{align*}
P(Z_{k+1} = i) &=  \frac{i}{n} P(Z_k =i ) + \frac{n-i+1}{n} P(Z_k =i-1)	\\
					&= \frac{i}{n}   \frac{\binom{n}{i}}{n^k} \sum_{j=0}^{i-1} (-1)^j \binom{i}{j}(i-j)^k +  \frac{n-i+1}{n}  \frac{\binom{n}{i}}{n^k} \sum_{j=0}^{i-2} (-1)^j \binom{i-1}{j}(i-1-j)^k
\end{align*}
REmarquons que 
$\frac{n-i+1}{n}  \frac{\binom{n}{i}}{n^k}  =  \frac{i\binom{n}{i}}{n^{k+1}}$
et en faisant un changement de variable on obtient 
$$\sum_{j=0}^{i-2} (-1)^j \binom{i-1}{j}(i-1-j)^k = -\sum_{j=1}^{i-1} (-1)^j \binom{i-1}{j-1}(i-j)^k$$
Donc 

\begin{align*}
P(Z_{k+1} = i) &=   \frac{i\binom{n}{i}}{n^{k+1}} \left(\sum_{j=0}^{i-1} (-1)^j \binom{i}{j}(i-j)^k  +
- \sum_{j=1}^{i-1} (-1)^j \binom{i-1}{j-1}(i-j)^k
\right) \\
&= \frac{i\binom{n}{i}}{n^{k+1}} \left( 
 \sum_{j=1}^{i-1} \left( (-1)^j (i-j)^k\left[ \binom{i}{j} - \binom{i-1}{j-1} \right]\right) +i^k
\right) \\
&= \frac{\binom{n}{i}}{n^{k+1}} \left( 
 \sum_{j=1}^{i-1} \left((-1)^j (i-j)^k\left[i \binom{i-1}{j}\right] \right)+i^k
\right) \\
&= \frac{\binom{n}{i}}{n^{k+1}} \left( 
 \sum_{j=1}^{i-1} \left((-1)^j (i-j)^k (i-j) \binom{i}{j} \right)+i^k
\right) \\
&= \frac{\binom{n}{i}}{n^{k+1}} \left( 
 \sum_{j=1}^{i-1} \left((-1)^j (i-j)^{k+1}  \binom{i}{j} \right)+i^k
\right) \\
&= \frac{\binom{n}{i}}{n^{k+1}} \left( 
 \sum_{j=0}^{i-1} \left((-1)^j (i-j)^{k+1}  \binom{i}{j} \right)
\right) 
\end{align*}
Ouchhh !
 



\item Initialisé en 6a et héréditaire en 6c) la propriété est donc vraie pour tout $k\in \N$. 
\item Cette somme vaut 0. 
\end{enumerate}
\end{correction}