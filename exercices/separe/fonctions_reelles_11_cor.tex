\begin{correction}  \;
\begin{enumerate}
\item La fonction $f$ est bien d\'efinie si et seulement si $x\not= 0$. Ainsi on a: $\mathcal{D}_f=\R^{\star}$.
\item On commence par donner l'expression de $f(x)$ selon les valeurs de $x$. 
\begin{itemize}
\item[$\bullet$] Si $x>0$, on a: $f(x)=xe^{  |  \ln{x} | }$. Il s'agit alors d'\'etudier le signe de $\ln{x}$. 
\begin{itemize}
\item[$\star$] Si $x\geq 1$, on obtient: $f(x)=xe^{\ln{x}}=x^2$.
\item[$\star$] Si $0<x<1$, on obtient: $f(x)=xe^{ -\ln{x} }=xe^{\ln{ \frac{1}{x} }}=x\times \ddp\frac{1}{x}=1$.
\end{itemize}
\item[$\bullet$] Si $x<0$, on a: $f(x)=xe^{ | \ln{(-x)} | }$. L\`{a} encore, il s'agit d'\'etudier le signe de $\ln{(-x)}$:
\begin{itemize}
\item[$\star$] Si $-1 \leq x<0$ alors $0<-x\leq 1$ et on obtient: $f(x)=xe^{-\ln{(-x)}}=xe^{\ln{ \frac{-1}{x} }}=x\times \ddp\frac{-1}{x}=-1$.
\item[$\star$] Si $x<-1$ alors $-x>1$, on obtient: $f(x)=xe^{ \ln{(-x)} }=-x^2$.
\end{itemize}
\end{itemize}
Ainsi, on obtient les valeurs suivantes pour $f$ selon les valeurs de $x$:
\begin{center}
 \begin{tikzpicture}
 \tkzTabInit{ $x$          /1,%
            $f(x)$       /2}%
     { $-\infty$,$-1$,$0$,$1$,$+\infty$}%   
   \tkzTabLine{ ,-x^2,t,-1,d,1,t,x^2,}                 
\end{tikzpicture}
\end{center}

\end{enumerate}
\end{correction}



%%--------------------------------------------------
%%------------------------------------------------
