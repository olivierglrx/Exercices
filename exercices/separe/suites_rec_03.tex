% Titre : suites
% Filiere : BCPST
% Difficulte : 
% Type : TD 
% Categories :suites
% Subcategories : 
% Keywords : suites




\begin{exercice} \;
Soit une suite $\suiteu$ qui v\'erifie la relation de r\'ecurrence
$$\left\lbrace\begin{array}{l}
u_0\in\R\vsec\\
\forall n\in\N,\ u_{n+1}=-u_n^2+2u_n
\end{array}\right.$$
\begin{enumerate}
 \item 
Calculer $1-u_{n+1}$ en fonction de $1-u_n$.
\item 
D\'eterminer la limite de la suite $\suiteu$, si elle existe, en fonction du premier terme $u_0$.
\end{enumerate}
\end{exercice}


\%\%\%\%\%\%\%\%\%\%\%\%\%\%\%\%\%\%\%\%
\%\%\%\%\%\%\%\%\%\%\%\%\%\%\%\%\%\%\%\%
\%\%\%\%\%\%\%\%\%\%\%\%\%\%\%\%\%\%\%\%




\begin{correction} \;
\begin{enumerate}
 \item Soit $n\in\N$, on a: $1-u_{n+1}=1+u_n^2-2u_n=(1-u_n)^2.$
 \item On pose $v_n = 1-u_n$. On a alors $v_{n+1}=v_n^2$. Essayons de calculer $v_n$ : on a $v_1=v_0^2$, $v_2=v_0^4$, $v_3=v_0^8$. On conjecture donc : $\forall n\in\N,\ v_n=v_0^{2^n}$. \\
Montrons par r\'ecurrence sur $n\in\N$ la propri\'et\'e : $\mathcal{P}(n):\quad v_n=v_0^{2^n}.$
\begin{itemize}
\item[$\bullet$]  Initialisation: pour $n=0$:\\
\noindent On a: $v_0^{2^0}=v_0$. Donc $\mathcal{P}(0)$ est vraie.
\item[$\bullet$]  H\'er\'edit\'e: Soit $n\in\N$. On suppose la propri\'et\'e vraie \`a l'ordre $n$, montrons qu'elle est vraie \`a l'ordre $n+1$. On a vu que: $v_{n+1}=v_n^2$. On utilise alors l'hypoth\`ese de r\'ecurrence et on obtient
$$v_{n+1}=\left(v_0^{2^n}  \right)^2=v_0^{2^{n+1}}.$$
Donc $\mathcal{P}(n+1)$ est vraie.
\item[$\bullet$] Conclusion: il r\'esulte du principe de r\'ecurrence que
$$\forall n\in\N,\quad v_n=v_0^{2^n}.$$
\end{itemize}
On obtient donc pour tout $n\in\N$: $u_n=1-(1-u_0)^{2^n}$.
\begin{itemize}
\item[$\bullet$] Si $1-u_0>1 \; \Leftrightarrow \; u_0<0$, alors : $\lim\limits_{n\to +\infty} (1-u_0)^{2^n}=+\infty$, donc $\lim\limits_{n\to +\infty} u_n=-\infty$.
\item[$\bullet$] Si $u_0=0$, alors $1-u_0=1$ et ainsi: $\forall n\in\N,\quad u_n=0$ et donc $\lim\limits_{n\to +\infty} u_n=0$.
\item[$\bullet$] Si $-1<1-u_0<1 \; \Leftrightarrow \; 0<u_0<2$, alors  : $\lim\limits_{n\to +\infty} u_n=1$.
\item[$\bullet$] Si $u_0=2$, alors $1-u_0=-1$ et $(1-u_0)^{2^n} = 1$, et ainsi: $\forall n\in\N,\quad u_n=0$ et donc $\lim\limits_{n\to +\infty} u_n=0$.
\item[$\bullet$] Si $1-u_0 < -1 \; \Leftrightarrow \; u_0 > 2$, alors $(1-u_0)^2 > 1$, et donc $\lim\limits_{n\to +\infty} (1-u_0)^{2^n}=+\infty$, soit $\lim\limits_{n\to +\infty} u_n=-\infty$.
\end{itemize}
\end{enumerate}
\end{correction}