% Titre : somme
% Filiere : BCPST
% Difficulte : 
% Type : TD 
% Categories :somme
% Subcategories : 
% Keywords : somme




\begin{exercice}   \; \textbf{Sommes d'indices pairs et impairs}\\
\noindent Soit $n$ un entier naturel non nul. On d\'efinit les sommes suivantes: $S_n=\ddp \sum\limits_{k=0}^{n} \binom{2n}{2k}\quad \hbox{et}\quad T_n=\ddp \sum\limits_{k=0}^{n-1} \binom{2n}{2k+1}.$
\begin{enumerate}
\item Montrer que $S_n+T_n=2^{2n}$ et $S_n-T_n=0$.
\item En d\'eduire une expression de $S_n$ et de $T_n$ en fonction de $n$.
\end{enumerate} 
\end{exercice}


\%\%\%\%\%\%\%\%\%\%\%\%\%\%\%\%\%\%\%\%
\%\%\%\%\%\%\%\%\%\%\%\%\%\%\%\%\%\%\%\%
\%\%\%\%\%\%\%\%\%\%\%\%\%\%\%\%\%\%\%\%




\begin{correction}   
\begin{enumerate}
\item 
\begin{itemize}
\item[$\bullet$] \textbf{Calcul de $\mathbf{S_n+T_n}$:}\\
\noindent Si on ne voit pas comment d\'ebuter, on commence par \'ecrire la somme $S_n+T_n$ sous forme d\'evelopp\'ee. On obtient alors que: $S_n+T_n=\ddp \sum\limits_{k=0}^{2n} \binom{2n}{k}$ car on se rend compte en \'ecrivant les sommes sous forme d\'evelopp\'ees que l'on obtient au final la somme de tous les coefficients binomiaux: $S_n$ correctionrespond en effet \`{a} la somme des coefficients binomiaux $\binom{2n}{k}$ avec $k$ pair et $T_n$ correctionrespond \`{a} la somme des coefficients binomiaux $\binom{2n}{k}$ avec $k$ impair donc en sommant les deux on a bien la somme de tous les coefficients binomiaux pour $k$ allant de 0 \`{a} $2n$. Ainsi, d'apr\`{e}s le bin\^{o}me de Newton, on obtient que: \fbox{$S_n+T_n=2^{2n}=4^n$.}
\item[$\bullet$] \textbf{Calcul de $\mathbf{S_n-T_n}$:}\\
\noindent De m\^{e}me, on peut commencer par \'ecrire la somme $S_n-T_n$ sous forme d\'evelopp\'ee. On obtient alors que: $S_n-T_n=\ddp \sum\limits_{k=0}^{2n} \binom{2n}{k}(-1)^k$ car on se rend compte en \'ecrivant les sommes sous forme d\'evelopp\'ees que l'on obtient au final la somme de tous les coefficients binomiaux coefficient\'es par 1 ou par -1: les coefficients binomiaux $\binom{2n}{k}$ avec $k$ pair sont coefficient\'e par 1 et les coefficients binomiaux $\binom{2n}{k}$ avec $k$ impair sont coefficient\'e par -1. Ainsi cela revient bien \`{a} sommer tous les nombres $\binom{2n}{k}(-1)^k$ pour $k$ allant de 0 \`{a} 
$2n$. 
Ainsi, d'apr\`{e}s le bin\^{o}me de Newton, on obtient que: \fbox{$S_n+T_n=(1-1)^n=0$. }
\end{itemize}
%\begin{itemize}
%\item[$\bullet$] On montre par r\'ecurrence sur $n\in\N^{\star}$ la propri\'et\'e 
%$$\mathcal{P}(n):\quad S_n+T_n=2^{2n}\quad \hbox{et}\quad S_n-T_n=0.$$
%\item[$\bullet$] Initialisation: pour $n=1$:
%\begin{itemize}
%\item[$\star$] $S_1+T_1=\ddp \sum\limits_{k=0}^{1} \binom{2}{2k}+\ddp \sum\limits_{k=0}^{0} \binom{2}{2k+1}=\binom{2}{0}+\binom{2}{2}+\binom{2}{1}=1+1+2=4$. Or $2^{2}=4$ donc $S_1+T_1=2^{2}$.
%\item[$\star$] $S_1-T_1=\ddp \sum\limits_{k=0}^{1} \binom{2}{2k}-\ddp \sum\limits_{k=0}^{0} \binom{2}{2k+1}=\binom{2}{0}+\binom{2}{2}-\binom{2}{1}=1+1-2=0$.
%\end{itemize}
%Ainsi $\mathcal{P}(1)$ est vraie.
%\item[$\bullet$] H\'er\'edit\'e: soit $n\in\N^{\star}$ fix\'e, on suppose la propri\'et\'e vraie \`{a} l'ordre $n$, montrons qu'elle est vraie \`{a} l'ordre $n+1$.
%\begin{itemize}
%\item[$\star$] $S_{n+1}+T_{n+1}=\ddp \sum\limits_{k=0}^{n+1} \binom{2(n+1)}{2k}+\ddp \sum\limits_{k=0}^{n} \binom{2(n+1)}{2k+1}=\ddp \sum\limits_{k=0}^{n} \binom{2n+2}{2k}+\ddp \sum\limits_{k=0}^{n} \binom{2(n+1)}{2k+1}$
%\item[$\star$]
%\end{itemize}

%\item[$\bullet$]
%\end{itemize}
\item Il s'agit alors juste de r\'esoudre le syst\`{e}me $\left\lbrace \begin{array}{lll}  S_n+T_n&=&2^{2n}\vsec \\ S_n-T_n&=& 0. \end{array}\right.$. On obtient alors: $2S_n=2^{2n}$  donc $S_n= 2^{2n-1}$ et $T_n=S_n=2^{2n-1}$.
\end{enumerate}
\end{correction}