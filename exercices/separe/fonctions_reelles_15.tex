
\begin{exercice}  \;
\textbf{(Param\`etres)}
Pour tout $m\in\R^{\star}$, on d\'esigne par $f_m$ la fonction d\'efinie par $f_m(x)=e^{mx\ln{(1+x)}}$.
\begin{enumerate}
\item D\'eterminer l'ensemble de d\'efinition commun $\mathcal{D}$ de toutes les fonctions $f_m$.
\item Discuter selon $m$ les limites de $f_m$ aux bornes de cet ensemble.
\item Montrer que $f_m$ est d\'erivable sur $\mathcal{D}$ pour tout $m\in\R^{\star}$. Montrer que pour tout $m\in\R^{\star}$, il existe une fonction $g_m$ telle que: $f^{\prime}_m=mg_mf_m$.
\item Montrer que $g_m$ est une fonction strictement croissante sur $\mathcal{D}$.
\item D\'eduire des questions pr\'ec\'edentes les variations de $f_m$ suivant $m$.
\item Tracer dans un m\^{e}me rep\`{e}re $\mathcal{C}_{-1}$, $\mathcal{C}_{-\demi}$ et $\mathcal{C}_{1}$.
\end{enumerate}
\end{exercice}