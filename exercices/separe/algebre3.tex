% Titre : Système linéaire $AX=\lambda X$
% Filiere : BCPST
% Difficulte :
% Type : DS, DM
% Categories : algebre
% Subcategories : 
% Keywords : algebre




\begin{exercice}
Soit $\lambda \in \R$. On considère le système suivant 
$$(S_\lambda)\quad  \left\{ \begin{array}{ccc}
2x +2y & =& \lambda x\\
x +3y  & =& \lambda y 
\end{array}\right. $$

\begin{enumerate}
\item Déterminer $\Sigma$ l'ensemble des réels $\lambda$ pour lequel ce système \underline{n'est pas} de Cramer. 
\item Pour $\lambda \in \Sigma$, résoudre $S_\lambda$
\item Quelle est la solution si $\lambda \notin \Sigma$. 
\end{enumerate}
\end{exercice}

\begin{correction}
\begin{enumerate}
\item  On met le système sous forme échelonné
$$(S_\lambda)\equivaut  \left\{ \begin{array}{ccc}
(2-\lambda)x +2y & =& 0\\
x +(3-\lambda)y  & =& 0
\end{array}\right.
\equivaut 
\left\{ \begin{array}{ccc}
x +(3-\lambda)y  & =& 0\\
(2-\lambda)x +2y & =& 0
\end{array}\right.
\equivaut 
\left\{ \begin{array}{rcc}
x +(3-\lambda)y  & =& 0\\
+2y -  (3-\lambda) (2-\lambda) y& =& 0
\end{array}\right.
 $$
 D'où 
 
 $$(S_\lambda)\equivaut  \left\{ \begin{array}{rcc}
x +(3-\lambda)y  & =& 0\\
  (-\lambda^2 +5\lambda -4) y& =& 0
\end{array}\right.$$
 
Le système n'est pas de Cramer si  $ (\lambda^2 +5\lambda -4) =0$, soit 
$$\Sigma =\{ 1, 4\}$$
\item \begin{itemize}
\item $\underline{\lambda=1}$
On obtient $S_1 \equivaut x+ 2y =0$
$$\cS_1 =\{ (-2y, y) | y \in \R\}$$

\item $\underline{\lambda=4}$
On obtient $S_4 \equivaut x-y =0$
$$\cS_4 =\{ (x, x) | x \in \R\}$$
\end{itemize}

\item Si $\lambda$ n'est pas dans $ \Sigma$, le système est de Cramer, il admet donc une unique solution. Comme $(0,0)$ est solution, c'est la seule. 
$$\cS_\lambda =\{ (0,0)\}$$
\end{enumerate}
\end{correction}