
\begin{exercice}  \; \textbf{Sommes t\'elescopiques}
\begin{enumerate}
\item D\'eterminer $(a,b)\in\bR^2$ tels que $\forall k\in\N^{\star}, \ \ddp\frac{1}{(k+1)(k+2)}=\ddp\frac{a}{k+1}+\ddp\frac{b}{k+2}$. En d\'eduire : $\ddp \sum\limits_{k=1}^n \ddp\frac{1}{(k+1)(k+2)} $. 
\item D\'eterminer trois r\'eels $a$, $b$ et $c$ tels que : $\forall k\in\N^{\star},\ \ddp\frac{k-1}{k(k+1)(k+3)}=\ddp\frac{a}{k}+\ddp\frac{b}{k+1}+\ddp\frac{c}{k+3}.$ En d\'eduire la valeur de $\ddp \sum\limits_{k=1}^{n}  \ddp\frac{k-1}{k(k+1)(k+3)}$.
\item D\'eterminer trois r\'eels $a$, $b$ et $c$ tels que : $\forall k\in\N^{\star},\ \ddp\frac{1}{k(k+1)(k+2)}=\ddp\frac{a}{k}+\ddp\frac{b}{k+1}+\ddp\frac{c}{k+2}.$ En d\'eduire la valeur de $\ddp \sum\limits_{k=1}^{n}  \ddp\frac{1}{k(k+1)(k+2)}$.\\
Retrouver ce r\'esultat par r\'ecurrence : montrer que $\forall n\geq 1$: $\ddp \sum\limits_{k=1}^{n} \ddp\frac{1}{k(k+1)(k+2)}=\ddp\frac{n(n+3)}{4(n+1)(n+2)}$.
\end{enumerate}
\end{exercice}