% Titre : somme
% Filiere : BCPST
% Difficulte : 
% Type : TD 
% Categories :somme
% Subcategories : 
% Keywords : somme




\begin{exercice}  \; \textbf{Sommes et d\'erivation:} Soit $n\in\N^{\star}$. Pour tout $x\in\bR\setminus\lbrace 1\rbrace$, on pose $f(x)=\ddp \sum\limits_{k=0}^n x^k$.
\begin{enumerate}
\item Calculer $f(x)$.
\item En d\'erivant, calculer $\ddp \sum\limits_{k=1}^n kx^{k-1}$, et en d\'eduire $\ddp \sum\limits_{k=1}^n kx^k$.
\item Calculer de la m\^eme fa\c con : $\ddp \sum\limits_{k=2}^n k(k-1)x^{k-2}$.
\end{enumerate}

\end{exercice}


\%\%\%\%\%\%\%\%\%\%\%\%\%\%\%\%\%\%\%\%
\%\%\%\%\%\%\%\%\%\%\%\%\%\%\%\%\%\%\%\%
\%\%\%\%\%\%\%\%\%\%\%\%\%\%\%\%\%\%\%\%




\begin{correction}  \; Il s'agit ici du m\^{e}me type de m\'ethode que pour l'exercice pr\'ec\'edent sauf que cette fois ci, on l'applique \`{a} la somme des termes d'une suite g\'eom\'etrique et plus au bin\^{o}me de Newton.
\begin{enumerate}
\item On reconna\^{i}t la somme des termes d'une suite g\'eom\'etrique et ainsi, on obtient, comme $x\not= 1$:\\
\noindent  \fbox{$\forall x\in\R\setminus\lbrace 1\rbrace,\ f(x)=\ddp\frac{1-x^{n+1}}{1-x}$.}
\item La fonction $f$ est d\'erivable sur $\R\setminus\lbrace 1\rbrace$ comme produit, somme et quotient dont le d\'enominateur ne s'annule pas de fonctions d\'erivables. 
\begin{itemize}
\item[$\bullet$] D'un c\^{o}t\'e, la fonction $f$ vaut: $f(x)=\ddp\frac{1-x^{n+1}}{1-x}$. Ainsi, en d\'erivant, on obtient que:\\
\noindent \fbox{$\forall x\in\R\setminus\lbrace 1 \rbrace,\ f^{\prime}(x)=\ddp\frac{  1+nx^{n+1} -(n+1)x^n }{(1-x)^2}$.}
\item[$\bullet$] De l'autre c\^{o}t\'e, la fonction $f$ vaut $f(x)=\ddp \sum\limits_{k=0}^{n} x^k=1+\ddp \sum\limits_{k=1}^{n} x^k$. La d\'eriv\'ee d'une somme \'etant \'egale \`{a} la somme des d\'eriv\'ees, on obtient que:\\
\noindent \fbox{$\forall x\in\R\setminus\lbrace 1 \rbrace,\  f^{\prime}(x)=\ddp \sum\limits_{k=1}^{n} k x^{k-1}$.} \\
\noindent \warning La somme commence bien \`{a} $k=1$ car le terme pour $k=0$ dans $f(x)$ est le terme constant $1$ qui est nul lorsqu'on d\'erive.
\end{itemize} 
On obtient donc que: \fbox{$\forall x\in\R\setminus\lbrace 1\rbrace,\ \ddp \sum\limits_{k=1}^{n} k x^{k-1}=\ddp\frac{  1+nx^{n+1} -(n+1)x^n }{(1-x)^2}$.}\\
On a : $\ddp \sum\limits_{k=1}^{n} k x^{k}=\ddp \sum\limits_{k=1}^{n} k x\times x^{k-1}=x\ddp \sum\limits_{k=1}^{n} k x^{k-1}$. D'apr\`{e}s la question pr\'ec\'edente, on obtient donc:\begin{center}
 \fbox{$\forall x\in\R\setminus\lbrace 1\rbrace,\ \ddp \sum\limits_{k=1}^{n} k x^{k}=x\times \ddp\frac{  1+nx^{n+1} -(n+1)x^n }{(1-x)^2}$.}
\end{center}
\item Il faut ici remarquer que la somme correctionrespond \`{a} d\'eriver deux fois la somme $f(x)=\ddp \sum\limits_{k=0}^{n} x^k=1+x+\ddp \sum\limits_{k=2}^{n} x^k$. La fonction $f$ est bien deux fois d\'erivables comme fonction polynomiale.
Et en d\'erivant deux fois, on obtient bien: $\forall x\in\R\setminus\lbrace 1\rbrace,\ f^{\prime\prime}(x)=\ddp \sum\limits_{k=2}^{n} k(k-1)x^{k-2}$. Cette somme commence bien \`{a} $k=2$ car quand on d\'erive deux fois les termes $1$ et $x$, ils deviennent nuls. En d\'erivant deux fois l'autre expression de $f$, on obtient la valeur de la somme:\\ 
\vsec
 \begin{center}
\fbox{$\forall x\in\R\setminus\lbrace 1\rbrace,\ \ddp \sum\limits_{k=2}^{n} k(k-1)x^{k-2}=\ddp\frac{  2-n(n+1)x^{n-1} +2(n^2-1)x^n-n(n-1)x^{n+1}    }{(1-x)^3}$. }
\end{center}
\end{enumerate}
\end{correction}