% Titre : nombre
% Filiere : BCPST
% Difficulte : 
% Type : TD 
% Categories :nombre
% Subcategories : 
% Keywords : nombre




\begin{exercice}%\footnote{http://www.mathieu-mansuy.fr/pdf/PCSI5-TD9.pdf}
Montrer que la fonction partie entière est croissante, ie montrer que pour tout $x, y\in \R^2,$ : 
$$x\leq y \implique \floor{x} \leq \floor{y}.$$

Montrer que pour tout $x, y\in \R^2,$ : 
$$\floor{x}+\floor{y} \leq \floor{x+y} \leq \floor{x} +\floor{y}+1.$$
\end{exercice}


\%\%\%\%\%\%\%\%\%\%\%\%\%\%\%\%\%\%\%\%
\%\%\%\%\%\%\%\%\%\%\%\%\%\%\%\%\%\%\%\%
\%\%\%\%\%\%\%\%\%\%\%\%\%\%\%\%\%\%\%\%




\begin{correction}

Soit $x,y\in \R^2$ et  $k=\floor{x}$. On a donc $x\in [k,k+1[$. Il y a maintenant deux cas possibles 
\paragraph{Cas 1 : $y\in  [k,k+1[$}
alors $\floor{y}=k$ et donc $\floor{x}=\floor{y}\leq \floor{y}$. 

\paragraph{Cas 2 : $y\notin  [k,k+1[$}
Comme $y\geq x$, on a $y>k+1$ et comme $\floor{y}>y-1$ on a 
$\floor{y} >k =\floor{x}$

On a ainsi montré que la fonction était croissante. 






\end{correction}