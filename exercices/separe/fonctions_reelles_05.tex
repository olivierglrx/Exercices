% Titre : fonctions
% Filiere : BCPST
% Difficulte : 
% Type : TD 
% Categories :fonctions
% Subcategories : 
% Keywords : fonctions




\begin{exercice}  \;
On pose $P(x)=x^2+3x$, $Q(x)=x^2+x+1$, $S(x)=x^2-1$.
\begin{enumerate}
 \item Calculer $P^2(x)$, $P(x)-Q(x)$ et $P^2(x)-Q^2(x)$.
%\item Calculer $P(Q)$, $Q(P)$ et $P(R)-R(P)$.
\item Calculer $P(x+1)$.
\item Calculer $S\circ f(t)$ avec $f: t\mapsto \cos{(t)}$.
\end{enumerate}
\end{exercice}


\%\%\%\%\%\%\%\%\%\%\%\%\%\%\%\%\%\%\%\%
\%\%\%\%\%\%\%\%\%\%\%\%\%\%\%\%\%\%\%\%
\%\%\%\%\%\%\%\%\%\%\%\%\%\%\%\%\%\%\%\%



\begin{correction}  \;
\begin{enumerate}
\item Les calculs donnent $P^2=X^4+6X^3+9X^2$, $P-Q=2X-1$, $P^2-Q^2=4X^3+6X^2-2X-1$.
%\item Il s'agit ici de composer des polyn\^{o}mes. Les calculs donnent: $P(Q)=Q^2+3Q=X^4+2X^3+6X^2+5X+4$, $Q(P)=P^2+P+1=X^4+6X^3+10X^2+3X+1$ et $P(R)-R(P)=-9X^5-29X^4-24X^3+2X^2$.
\item On obtient $P(X+1)=X^2+5X+4$.
\item $S\circ f : t \mapsto -\sin^2(t) $.
\end{enumerate}
\end{correction}