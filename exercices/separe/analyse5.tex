% Titre : Racine de $x^3-6x-9$
% Filiere : BCPST
% Difficulte :
% Type : DS, DM
% Categories : analyse
% Subcategories : 
% Keywords : analyse




\begin{exercice}
On cherche les racines réelles du polynôme $P(x) =x^3-6x-9$. 
\begin{enumerate}
\item Donner en fonction du paramètre $x$ réel, le nombre de solutions réelles de l'équation $x=y+\frac{2}{y}$ d'inconnue $y\in \R^*$. 
\item Soit $x\in \R$ vérifiant $|x|\geq 2\sqrt{2}$. Montrer en posant le changement de variable $x=y+\frac{2}{y}$ que : 
$$ P(x) =0 \equivaut y^6 -9y^3 +8=0$$
\item Résoudre l'équation $z^2-9z+8=0$ d'inconnue $z\in \R$. 
\item En déduire une racine du polynôme $P$.
\item Donner toutes les racines réelles du polynôme $P$. 
\end{enumerate}
\end{exercice}

\begin{correction}
\begin{enumerate}
\item Résolvons l'équation proposée en fonction du paramètre $x$. 
On a $$
\begin{array}{lrl}
&y+\frac{2}{y}&=x\\
\equivaut & y^2+2&=yx\\
\equivaut & y^2-xy+2&=0
\end{array}
$$

On calcule le discriminant de ce polynome de degré $2$ on obtient
$$\Delta = x^2-8$$

Donc : 
\begin{itemize}
\item si $x^2-8>0$  c'est-à-dire si  $|x| >2\sqrt{2}$, 
l'équation admet $2$ solutions. 
\item si $x^2-8=0$  c'est-à-dire si  $x=2\sqrt{2}$ ou $x=-2\sqrt{2}$ 
l'équation admet $1$ seule solution. 
\item si $x^2-8<0$  c'est-à-dire si  $x\in ]-2\sqrt{2},22\sqrt{2}[$ 
l'équation admet $0$ solution. 
\end{itemize}

\item 
Soit $x=y+\frac{2}{y}$, on a :
$$\begin{array}{lrl}
&P(x)&=0 \\
\equivaut&\left(y+ \frac{2}{y}\right)^3-6\left(y+ \frac{2}{y}\right)-9&=0
\end{array}
$$

Développons à part 
$\left(y+ \frac{2}{y}\right)^3$. On obtient tout calcul fait
$$\left(y+ \frac{2}{y}\right)^3=y^3 +6y+\frac{12}{y}+\frac{8}{y^3}$$
Donc 

$$\begin{array}{lrl}
&\left(y+ \frac{2}{y}\right)^3-6\left(y+ \frac{2}{y}\right)-9&=0\\
\equivaut& y^3 +\frac{8}{y^3}-9&=0\\
\equivaut& y^6 +8-9y^3&=0
\end{array}
$$
où la dernière équivalence s'obtient en multipliant par $y^3$ non nul. 

\item On résout $z^2-9z+8=0$ à l'aide du discriminant du polynôme $z^2-9z+8$ qui vaut 
$\delta = 81-32= 49=7^2$. On  a donc deux solutions 
\conclusion{
$z_1 =\frac{9+7}{2}=8 \quadet z_2 =\frac{9-7}{2}=1$
}

\item La question d'avant montre que $\sqrt[3]{1}=1$ est solution de l'équation $y^6-9y^3+8=0$ (on peut le vérifier à la main si on veut, mais c'était le but de la question précédente.) 

Comme  on a effectué le changement de variable  $x=y+\frac{2}{y}$ et à l'aide de la question $2$, on voit que $x=1+\frac{2}{1}=3$ est solution de l'équation $P(x)=0$ c'est-à-dire que 
\conclusion{$3$ est une racine de $P$.} 

(de nouveau on pourrait le revérifier en faisant le calcul, mais ceci n'est psa nécéssaire)

\item Comme $3$ est racine de $P$, on peut écrire $P(x)$ sous la forme $(x-3)(ax^2+bx+c)$, avec $(a,b,c)\in \R^3$. 

En développant on obtient 
$P(x)= ax^3 +(-3a+b)x^2+(c-3b)x-3c.$ Maintenant par identification on obtient 
$$\left\{\begin{array}{ccc}
a&=&1\\
-3a+b&=&0\\
c-3b&=&-6\\
-3c&=&-9
\end{array}\right.$$
Ce qui donne 
$$\left\{\begin{array}{ccc}
a&=&1\\
b&=&3\\
c&=&3\\
c&=&3
\end{array}\right.$$
Et finalement 
$$P(x) = (x-3) (x^2+3x+3)$$

Il nous reste plus qu'à trouver les racines de $x^2+3x+3$ que l'on fait grâce à son discriminant qui vaut $\Delta =9-12<-3$. 

\conclusion{
L'unique racine réelle de $P$ est $3$}


Je rajoute le graphique de la courbe représentative de $P$ avec le programme Python qui permet de le tracer. 
\begin{lstlisting}
import matplotlib.pyplot as plt
import numpy as np
def P(x):
    return(x**3-6*x+9)
X=np.linspace(-5,5,100)
Y=P(X)
Z=np.zeros(100)
plt.plot(X,Y)
plt.plot(X,Z)
plt.show()
\end{lstlisting}
%\begin{center}
%\includegraphics[scale=0.4]{../../../graph.png} 
%\end{center}

 
\end{enumerate}

\end{correction}