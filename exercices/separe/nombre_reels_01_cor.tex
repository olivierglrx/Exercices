
\begin{correction}  \; \textbf{R\'esolution d'\'equations et d'in\'equations avec des polyn\^{o}mes.}
\begin{enumerate}
%\item \textbf{R\'esolution dans $\mathbf{\R}$ de $\mathbf{4x^3-8x^2-5x+7=0}$:}\\
%\noindent  -1 est racine \'evidente et on obtient: $4x^3-8x^2-5x+7=0\Leftrightarrow (x+1)(4x^2-12x+7)=0$. De plus, $4x^2-12x+7$ admet $\ddp \frac{3-\sqrt{2}}{2}$ et $\ddp \frac{3+\sqrt{2}}{2}$ pour racines, donc \fbox{$\mathcal{S}=\ddp \left\lbrace -1, \frac{3-\sqrt{2}}{2}, \frac{3+\sqrt{2}}{2}  \right\rbrace$.}
%---
\item \textbf{R\'esolution dans $\mathbf{\R}$ de $\mathbf{x^3+4x^2+x-6\geq 0 }$:}\\
\noindent 1 est racine \'evidente et on obtient : $x^3+4x^2+x-6\geq 0\Leftrightarrow (x-1)(x^2+5x+6)\geq 0$. Un tableau de signe donne \fbox{$\mathcal{S}=\lbrack -3,-2\rbrack\cup\lbrack 1,+\infty\lbrack$.}
%---
\item \textbf{R\'esolution dans $\mathbf{\R}$ de $\mathbf{x^3-x^2-x-2<0}$:}\\
\noindent 2 est racine \'evidente et on obtient:$x^3-x^2-x-2<0\Leftrightarrow (x-2)(x^2+x+1)<0$ et le discriminant de $x^2+x+1$ est n\'egatif donc \fbox{$\mathcal{S}= \; \rbrack -\infty,2\lbrack$.}
%---
\item \textbf{R\'esolution dans $\mathbf{\R}$ de $\mathbf{(3x-1)(x+2)+(2-6x)(4x+3)>0}$:}\\
\noindent On factorise par $3x-1$ et on obtient: 
$$(3x-1)(x+2)+(2-6x)(4x+3)>0\Leftrightarrow (3x-1)\left\lbrack x+2-2(4x+3)\right\rbrack >0\Leftrightarrow (3x-1)(-7x-4)>0.$$ 
Un tableau de signe donne \fbox{$\mathcal{S}=\left\rbrack -\ddp\frac{4}{7},\ddp\frac{1}{3}\right\lbrack $.}
\item \textbf{R\'esolution dans $\mathbf{\R}$ de $\mathbf{32x^6-162x^2<0}$:}\\
\noindent On factorise par $2x^2$ puis on utilise l'identit\'e remarquable $a^2-b^2$ et on obtient: 
$$32x^6-162x^2<0\Leftrightarrow 2x^2(16x^4-81)<0\Leftrightarrow 2x^2( 4x^2-9 )(4x^2+9)<0.$$
Un tableau de signe donne \fbox{$\mathcal{S}=\left\rbrack -\ddp\frac{3}{2},\ddp\frac{3}{2}\right\lbrack\setminus\lbrace 0\rbrace$.}
%\item \textbf{R\'esolution dans $\mathbf{\R}$ de $\mathbf{(2x-3)(x+2)-(3-2x)(x^2-1)+(4x-6)(x-1)\geq 0}$:}\\
%\noindent On factorise par $2x-3$ et on obtient: $(2x-3)(x+2)-(3-2x)(x^2-1)+(4x-6)(x-1)\geq 0\Leftrightarrow (2x-3)\left\lbrack x+2+x^2-1+2(x-1)\right\rbrack\geq 0\Leftrightarrow 
%(2x-3)(x^2+3x-1)\geq 0$. Un tableau de signe donne \fbox{$\mathcal{S}=\left\lbrack \ddp\frac{-3-\sqrt{13}}{2},\ddp\frac{-3+\sqrt{13}}{2}  \right\rbrack\cup \left\lbrack \ddp\frac{3}{2},+\infty\right\lbrack$.}
\item \textbf{R\'esolution dans $\mathbf{\R}$ de $\mathbf{\ddp\frac{2x}{4x^2-1}\leq \ddp\frac{2x+1}{4x^2-4x+1}}$:}\\
\noindent On commence par le domaine de r\'esolution. L'in\'equation est bien d\'efinie si et seulement si $4x^2-1\not= 0$ et $4x^2-4x+1\not= 0$. Ainsi $\mathcal{D}=\R\setminus\left\lbrace -\ddp\demi,\ddp\demi\right\rbrace$.\\
\noindent On passe tout du m\^{e}me c\^{o}t\'e et on met tout au m\^{e}me d\'enominateur. 
On a :
$$\ddp\frac{2x}{(2x+1)(2x-1)}-\ddp\frac{2x+1}{(2x-1)^2}\leq 0 \Leftrightarrow \ddp\frac{ 2x(2x-1)-(2x+1)(2x+1)  }{(2x-1)^2(2x+1)}\leq 0 \Leftrightarrow \ddp\frac{ -6x-1  }{(2x-1)^2(2x+1)}\leq 0.$$ 
Un tableau de signe donne 
\fbox{$\mathcal{S}=\left\rbrack -\infty,-\ddp\demi\right\lbrack\cup \left\lbrack -\ddp\frac{1}{6},\ddp\demi\right\lbrack\cup\left\rbrack \ddp\demi,+\infty\right\lbrack$.}
\item \textbf{R\'esolution dans $\mathbf{\R}$ de $\mathbf{ \ddp\frac{x^4+x}{x^4-5x^2+4}<1}$:}\\
\noindent On commence par le domaine de r\'esolution. L'in\'equation est bien d\'efinie si et seulement si $x^4-5x^2+4\not= 0\Leftrightarrow (x^2-4)(x^2-1)\not= 0$. Ainsi $\mathcal{D}=\R\setminus\left\lbrace -2,-1,1,2\right\rbrace$.\\
\noindent On passe tout du m\^{e}me c\^{o}t\'e et on met tout au m\^{e}me d\'enominateur. On a: 
$$\ddp\frac{x^4+x}{x^4-5x^2+4}-1<0\Leftrightarrow \ddp\frac{5x^2+x-4}{(x^2-4)(x^2-1)}<0.$$ 
Un tableau de signe donne \fbox{$\mathcal{S}=\rbrack -2,-1\lbrack \; \cup \left\rbrack -1,\ddp\frac{4}{5}\right\lbrack \cup \; \rbrack 1,2\lbrack$.}

\item \textbf{R\'esolution dans $\mathbf{\R}$ de $\mathbf{2x^2-4x+2=1-x}$:}\\
\noindent $2x^2-4x+2=1-x\Leftrightarrow 2x^2-3x+1=0$ donc \fbox{$\mathcal{S}=\left\lbrace \ddp\demi,1  \right\rbrace$}.
%---
\item \textbf{R\'esolution dans $\mathbf{\R}$ de $\mathbf{(x-1)^2\leq 1}$:}\\
\noindent $(x-1)^2\leq 1\Leftrightarrow x(x-2)\leq 0$ donc \fbox{$\mathcal{S}=\lbrack 0,2\rbrack$}.
%---
\item \textbf{R\'esolution dans $\mathbf{\R}$ de $\mathbf{\ddp\frac{1}{x-2}\leq \ddp\frac{1}{2x}}$:}\\
\noindent On commence par le domaine de r\'esolution. L'in\'equation est bien d\'efinie si et seulement si $x-2\not= 0$ et $2x \not= 0$. Ainsi $\mathcal{D}=\R\setminus\left\lbrace 0,2 \right\rbrace$.\\
\noindent De plus, on a : $\ddp\frac{1}{x-2}\leq \ddp\frac{1}{2x}\Leftrightarrow \ddp\frac{x+2}{2x(x-2)}\leq 0$ et un tableau de signe donne 
\fbox{$\mathcal{S}= \; \rbrack -\infty,-2\rbrack \; \cup \; \rbrack 0,2\lbrack$}.
%---
\item \textbf{R\'esolution dans $\mathbf{\R}$ de $\mathbf{\ddp\frac{2x+1}{1+x}\geq \ddp\frac{3x-2}{1+x}}$:}\\
\noindent On commence par le domaine de r\'esolution. L'in\'equation est bien d\'efinie si et seulement si $x+1\not= 0$. Ainsi $\mathcal{D}=\R\setminus\left\lbrace -1 \right\rbrace$.\\
\noindent $\ddp\frac{2x+1}{x+1}\geq \ddp\frac{3x-2}{1+x}\Leftrightarrow \ddp\frac{-x+3}{1+x}\geq 0$ et un tableau de signe donne 
\fbox{$\mathcal{S}= \; \rbrack -1,3\rbrack$}.
%---
\item \textbf{R\'esolution dans $\mathbf{\R}$ de $\mathbf{\ddp\frac{x^2+10x-4}{x-2}\leq \ddp\frac{16x+2}{x+1}}$:}\\
\noindent On commence par le domaine de r\'esolution. L'in\'equation est bien d\'efinie si et seulement si $x-2\not= 0$ et $x+1 \not= 0$. Ainsi $\mathcal{D}=\R\setminus\left\lbrace -1,2 \right\rbrace$.\\
\noindent $\ddp\frac{x^2+10x-4}{x-2}\leq \ddp\frac{16x+2}{x+1}\Leftrightarrow \ddp\frac{x(x^2-5x+36)}{(x-2)(x+1)}\leq 0$ donc un tableau de signe donne \fbox{$\mathcal{S}= \; \rbrack -\infty, -1\lbrack\cup \lbrack 0,2\lbrack$}.
%\item \textbf{R\'esolution dans $\mathbf{\R}$ de $\mathbf{ \ddp\frac{2x-2}{3x+1}>1}$:}\\
%\noindent On commence par le domaine de r\'esolution. L'in\'equation est bien d\'efinie si et seulement si $3x+1\not= 0$. Ainsi $\mathcal{D}=\R\setminus\left\lbrace -\ddp\frac{1}{3} \right\rbrace$.\\
%\noindent $\ddp\frac{2x-2}{3x+1}-1>0\Leftrightarrow \ddp\frac{-x-3}{3x+1}>0$. Un tableau de signe donne 
%\fbox{$\mathcal{S}=\left\rbrack -3,\ddp\frac{-1}{3}\right\lbrack$.}
\end{enumerate}
\end{correction}