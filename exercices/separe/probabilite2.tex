% Titre : Chaine de markov - puce sur un triangle (événement, pas de diag)
% Filiere : BCPST
% Difficulte :
% Type : DS, DM
% Categories : probabilite
% Subcategories : 
% Keywords : probabilite







\begin{exercice}

On considère trois points distincts du plan nommés $A, B$ et $C$. Nous allons étudier le déplacement aléatoire d'un pion se déplaçant sur ces trois points. A l'étape $n=0$, on suppose que le pion se trouve sur le point $A$. Ensuite, le mouvement aléatoire du pion respecte les deux règles suivantes :
\begin{itemize}
\item  le mouvement du pion de l'étape $n$ à l'étape $n+1$ ne dépend que de la position du pion à l'étape $n$;
\item pour passer de l'étape $n$ à l'étape $n+1$, on suppose que le pion a une chance sur deux de rester sur place, sinon il se déplace de manière équiprobable vers l'un des deux autres points.

\end{itemize}

Pour tout $n \in \mathbb{N}$, on note $A_{n}$ l'évènement "le pion se trouve en $A$ à l'étape $n$ ", $B_{n}$ l'évènement "le pion se trouve en $B$ à l'étape $n$ " et $C_{n}$ l'évènement "le pion se trouve en $C$ à l'étape $n$ ". On note également, pour tout $n \in \mathbb{N}$,
$$
a_{n}=P\left(A_{n}\right), b_{n}=P\left(B_{n}\right), c_{n}=P\left(C_{n}\right) \text { et } V_{n}=\left(\begin{array}{l}
a_{n} \\
b_{n} \\
c_{n}
\end{array}\right)
$$
\begin{enumerate}
\item  Calculer les nombres $a_{n}, b_{n}$ et $c_{n}$ pour $n=0,1$.
\item  Pour $n \in \mathbb{N}$, exprimer $a_{n+1}$ en fonction de $a_{n}, b_{n}$ et $c_{n} .$ Faire de même pour $b_{n+1}$ et $c_{n+1}$.
\item  Donner une matrice $M$ telle que, pour tout $n \in \mathbb{N}$, on a $V_{n+1}=M V_{n}$.
\item  On admet que, pour tout $n \in \mathbb{N}$, on a
$$
M^{n}=\frac{1}{3 \cdot 4^{n}}\left(\begin{array}{ccc}
4^{n}+2 & 4^{n}-1 & 4^{n}-1 \\
4^{n}-1 & 4^{n}+2 & 4^{n}-1 \\
4^{n}-1 & 4^{n}-1 & 4^{n}+2
\end{array}\right)
$$
En déduire une expression de $a_{n}, b_{n}$ et $c_{n}$ pour tout $n \in \mathbb{N}$.
\item  Déterminer les limites respectives des suites $\left(a_{n}\right),\left(b_{n}\right)$ et $\left(c_{n}\right)$. Interpréter le résultat.
\end{enumerate}


\end{exercice}



\begin{correction}
\begin{enumerate}


\item  Puisqu'en $n=0$ le pion est en $A$, on a $a_{0}=1, b_{0}=0$ et $c_{0}=0 .$ A l'étape $n=1$, d'après les informations de l'énoncé, $a_{1}=1 / 2, b_{1}=c_{1}$. Puisque $a_{1}+b_{1}+c_{1}=1$, on a $b_{1}=c_{1}=1 / 4$.
\item  Les événements $A_{n}, B_{n}$ et $C_{n}$ forment un système complet d'événements. D'après la formule des probabilités totales,
$$
P\left(A_{n+1}\right)=P_{A_{n}}\left(A_{n+1}\right) P\left(A_{n}\right)+P_{B_{n}}\left(A_{n+1}\right) P\left(B_{n}\right)+P_{C_{n}}\left(A_{n+1}\right) P\left(C_{n}\right) .
$$
Comme à la question précédente, on a $P_{A_{n}}\left(A_{n+1}\right)=1 / 2, P_{B_{n}}\left(A_{n+1}\right)=1 / 4$ et $P_{C_{n}}\left(A_{n+1}\right)=1 / 4$. On en déduit que
$$
a_{n+1}=\frac{1}{2} a_{n}+\frac{1}{4} b_{n}+\frac{1}{4} c_{n}
$$
En raisonnant de la même façon, ou par symétrie,
$$
\begin{gathered}
b_{n+1}=\frac{1}{4} a_{n}+\frac{1}{2} b_{n}+\frac{1}{4} c_{n} \\
c_{n+1}=\frac{1}{4} a_{n}+\frac{1}{4} b_{n}+\frac{1}{2} c_{n}
\end{gathered}
$$
\item  D'après la question précédente, la matrice
$$
M=\frac{1}{4}\left(\begin{array}{lll}
2 & 1 & 1 \\
1 & 2 & 1 \\
1 & 1 & 2
\end{array}\right)
$$
convient.
\item  On a $V^{n}=M^{n} V_{0}$, ce qui donne
$$
\left\{\begin{array}{l}
a_{n}=\frac{4^{n}+1}{3 \cdot 4^{n}} \\
b_{n}=\frac{4^{n}-2}{3 \cdot 4^{n}} \\
c_{n}=\frac{4^{n}-2}{3 \cdot 4^{n}}
\end{array}\right.
$$
On remarque qu'on a bien $a_{n}+b_{n}+c_{n}=1$.
\end{enumerate}
\end{correction}