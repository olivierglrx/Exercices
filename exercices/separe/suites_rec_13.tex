% Titre : suites
% Filiere : BCPST
% Difficulte : 
% Type : TD 
% Categories :suites
% Subcategories : 
% Keywords : suites




\begin{exercice} \;
On d\'efinit deux suites $(u_n)_{n\in\N^{\star}}$ et $(v_n)_{n\in\N^{\star}}$ par 
$$u_1=1\quad v_1=12\quad \forall n\in\N^{\star},\ u_{n+1}=\ddp\frac{u_n+2v_n}{3}\quad v_{n+1}=\ddp\frac{u_n+3v_n}{4}.$$
\begin{enumerate}
 \item
On pose, pour tout $n\in\N^{\star}$, $w_n=v_n-u_n$. Donner l'expression de $(w_n)_{n\in\N^{\star}}$. 
\item 
Montrer que $(u_n)_{n\in\N^{\star}}$ et $(v_n)_{n\in\N^{\star}}$ sont adjacentes.
\item 
On pose pour tout $n\in\N^{\star}$, $t_n=3u_n+8v_n$. \\
Donner l'expression de $(t_n)_{n\in\N^{\star}}$ et en d\'eduire la limite de $(u_n)_{n\in\N^{\star}}$ et $(v_n)_{n\in\N^{\star}}$.
\end{enumerate}
\end{exercice}


\%\%\%\%\%\%\%\%\%\%\%\%\%\%\%\%\%\%\%\%
\%\%\%\%\%\%\%\%\%\%\%\%\%\%\%\%\%\%\%\%
\%\%\%\%\%\%\%\%\%\%\%\%\%\%\%\%\%\%\%\%



\begin{correction}
\begin{enumerate}
 \item Soit $n\in\N^{\star}$: $w_{n+1}=v_{n+1}-u_{n+1}=\ddp\frac{3u_n+9v_n-4u_n-8v_n}{12}=\ddp\frac{v_n-u_n}{12}=
\ddp\frac{w_n}{12}$. Ainsi, la suite 
$(w_n)_{n\in\N^{\star}}$ est g\'eom\'etrique de raison $\ddp\frac{1}{12}$ et de premier terme $w_1=11$. Ainsi,
$$\forall n\in\N,\ w_n=11\left( \ddp\frac{1}{12} \right)^{n-1}.$$
\item \begin{itemize}
\item[$\bullet$] D'apr\`es ce qui pr\'ec\`ede, on a: $\lim\limits_{n\to +\infty} v_n-u_n=0$ car la suite g\'eom\'etrique $(w_n)_{n\in\N^{\star}}$ converge vers 0 car $0<\ddp\frac{1}{12}<1$.
\item[$\bullet$]  \'Etude de la monotonie de $(u_n)_{n\in\N^{\star}}$:\\
\noindent Soit $n\in\N$, on a
$$u_{n+1}-u_n=\ddp\frac{2(v_n-u_n)}{3}=\ddp\frac{2}{3}w_n.$$
Or la suite $(w_n)_{n\in\N^{\star}}$ est \`a termes positifs donc la suite $(u_n)_{n\in\N^{\star}}$ est croissante.
\item[$\bullet$]  \'Etude de la monotonie de $(v_n)_{n\in\N^{\star}}$:\\
\noindent Soit $n\in\N$, on a
$$v_{n+1}-v_n=\ddp\frac{u_n-v_n}{4}=\ddp\frac{-1}{4}w_n.$$
Or la suite $(w_n)_{n\in\N^{\star}}$ est \`a termes positifs donc la suite $(v_n)_{n\in\N^{\star}}$ est d\'ecroissante.
      \end{itemize}
Les deux suites sont donc adjacentes et le th\'eor\`eme sur les suites adjacentes assure ainsi qu'elles convergent vers la m\^eme limite finie $l\in\R$.
\item Soit $n\in\N^{\star}$, on a:
$$t_{n+1}=3u_{n+1}+8v_{n+1}= u_n+2v_n+2u_n+6v_n=3u_n+8v_n=t_n.$$
Ainsi, la suite $(t_n)_{n\in\N^{\star}}$ est constante et donc:
$$\forall n\in\N^{\star},\ t_n=t_1=99.$$
De plus, la suite $(t_n)_{n\in\N^{\star}}$ est convergente et converge d'apr\`es les propri\'et\'es sur les sommes de limite vers la limite $3l+8l=11l$. Comme elle est constante \'egale \`a 99, on a, d'apr\`{e}s l'unicit\'e de la limite:
$$11l=99\Leftrightarrow l=9.$$
Ainsi, les suites $(u_n)_{n\in\N^{\star}}$ et $(v_n)_{n\in\N^{\star}}$ convergent toutes les deux vers $9$.
\end{enumerate}
\end{correction}