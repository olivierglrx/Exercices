% Titre : complexe
% Filiere : BCPST
% Difficulte : 
% Type : TD 
% Categories :complexe
% Subcategories : 
% Keywords : complexe




\begin{exercice}  \;
Soit $x$ un r\'eel fix\'e. Calculer la partie r\'eelle et imaginaire de 
$(x+i)^2$ et de $\ddp\frac{x-3i}{x^2+1-2ix}$.
\end{exercice}


\%\%\%\%\%\%\%\%\%\%\%\%\%\%\%\%\%\%\%\%
\%\%\%\%\%\%\%\%\%\%\%\%\%\%\%\%\%\%\%\%
\%\%\%\%\%\%\%\%\%\%\%\%\%\%\%\%\%\%\%\%




\begin{correction}   \;
\begin{itemize}
\item[$\bullet$] \textbf{Calculer la partie r\'eelle et la partie imaginaire de $\mathbf{(x+i)^2}$:}
En d\'eveloppant $(x+i)^2$, on obtient $(x+i)^2=(x^2-1)+2i x$. Ainsi
$$\fbox{$\reel{\left\lbrack(x+i)^2\right\rbrack}=x^2-1\quadet \imag{\left\lbrack(x+i)^2\right\rbrack}=2x.$}$$
\item[$\bullet$]  \textbf{Calculer la partie r\'eelle et la partie imaginaire de $\mathbf{\ddp\frac{x-3i}{x^2+1-2i x}}$:}
On a: $\ddp\frac{x-3i}{x^2+1-2i x}=\ddp\frac{(x-3i)(x^2+1+2i x)}{x^4+6x^2+1}$.
Ainsi, on obtient
$$\fbox{$\reel{ \left(\ddp\frac{x-3i}{x^2+1-2i x}\right)}=\ddp\frac{x(x^2+7)}{x^4+6x^2+1}\quad\hbox{et}\quad \imag{\left(\ddp\frac{x-3i}{x^2+1-2i x}\right)}=\ddp\frac{-x^2-3}{x^4+6x^2+1}.$}$$
\end{itemize}
\end{correction}