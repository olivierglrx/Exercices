% Titre : Etude famille de fonction, intégrale, et somme (ECRICOME 2002)
% Filiere : BCPST
% Difficulte :
% Type : DS, DM
% Categories : analyse
% Subcategories : 
% Keywords : analyse




\begin{exercice}



On consid\`ere la famille de fonctions $(f_n)_{n\in\mathbb{N}^*}$ d\'efinies
sur $]-1,+\infty[$ par~: 
\begin{equation*}
f_n(x)=x^n\ln (1+x).
\end{equation*}


\paragraph{A- \'Etude des fonctions $f_n$.\\}




Soit $n\in \mathbb{N}^*$. On note $h_n$ la fonction d\'efinie sur $%
]-1,+\infty[$ par~: 
\begin{equation*}
h_n(x)=n\ln(1+x)+\frac{x}{1+x}.
\end{equation*}

\begin{enumerate}
\item \'{E}tudier le sens de variation des fonctions $h_{n}$.

\item Calculer $h_{n}(0)$, puis en d\'{e}duire le signe de $h_{n}$.

\item \'{E}tude du cas particulier $n=1$.

\begin{enumerate}
\item Apr\`{e}s avoir justifi\'{e} la d\'{e}rivabilit\'{e} de $f_{1}$ sur $%
]-1,+\infty [$, exprimer $f_{1}^{\prime }(x)$ en fonction de $h_{1}(x)$.

\item En d\'{e}duire les variations de la fonction $f_{1}$ sur $]-1,+\infty
[ $.
\end{enumerate}

\item Soit $n\in \mathbb{N}^{*}\setminus \{1\}$.

\begin{enumerate}
\item Justifier la d\'{e}rivabilit\'{e} de $f_{n}$ sur $]-1,+\infty [$ et
exprimer $f_{n}^{\prime }(x)$ en fonction de $h_{n}(x)$.

\item En d\'{e}duire les variations de $f_{n}$ sur $]-1,+\infty [$. (On
distinguera les cas $n$ pair et $n$ impair). On pr\'{e}cisera les limites
aux bornes sans \'{e}tudier les branches infinies.
\end{enumerate}
\end{enumerate}

\paragraph{B- \'Etude d'une suite.\\}

On consid\`ere la suite $\left(U_n\right)_{n\in\mathbb{N}^*}$ d\'efinie
par~: 
\begin{equation*}
U_n=\int_0^1f_n(x)\,dx.
\end{equation*}
\begin{enumerate}
\item Calcul de $U_1$.


\begin{enumerate}
\item Prouver l'existence de trois r\'{e}els $a$, $b$, $c$ tels que~: 
\begin{equation*}
\forall x\in [0,1],\quad \frac{x^{2}}{x+1}=ax+b+\frac{c}{x+1}.
\end{equation*}

\item En d\'{e}duire la valeur de l'int\'{e}grale~: 
\begin{equation*}
\int_{0}^{1}\frac{x^{2}}{x+1}\,dx.
\end{equation*}

\item Montrer que $\displaystyle U_{1}=\frac{1}{4}$.
\end{enumerate}

\item Convergence de la suite $\left(U_n\right)_{n\in\mathbb{N}^*}$\\

\begin{enumerate}
\item Montrer que la suite $\left( U_{n}\right) _{n\in \mathbb{N}^{*}}$ est
monotone.

\item Justifier la convergence de la suite $\left( U_{n}\right) _{n\in 
\mathbb{N}^{*}}$. (On ne demande pas sa limite.)

\item D\'{e}montrer que~: 
\begin{equation*}
\forall n\in \mathbb{N}^{*},\quad 0\leqslant U_{n}\leqslant \frac{\ln 2}{n+1}%
.
\end{equation*}

\item En d\'{e}duire la limite de la suite $\left( U_{n}\right) _{n\in 
\mathbb{N}^{*}}$.
\end{enumerate}


\item Calcul de $U_n$ pour $n\geqslant 2$


Pour $x\in [0,1]$ et $n\in \mathbb{N}^*\setminus \{1\}$, on pose~: 
\begin{equation*}
S_n(x)=1-x+x^2+\cdots+(-1)^nx^n=\sum_{k=0}^n(-1)^kx^k.
\end{equation*}

\begin{enumerate}
\item Montrer que~: 
\begin{equation*}
S_{n}(x)=\frac{1}{1+x}+\frac{(-1)^{n}x^{n+1}}{1+x}.
\end{equation*}

\item En d\'{e}duire que~: 
\begin{equation*}
\sum_{k=0}^{n}\frac{(-1)^{k}}{k+1}=\ln 2+(-1)^{n}\int_{0}^{1}\frac{x^{n+1}}{%
1+x}\,dx.
\end{equation*}

\item En utilisant une int\'{e}gration par parties dans le calcul de $U_{n}$%
, montrer que~: 
\begin{equation*}
U_{n}=\frac{\ln 2}{n+1}+\frac{(-1)^{n}}{n+1}\left[ \ln 2-\left( 1-\frac{1}{2}%
+\cdots +\frac{(-1)^{k}}{k+1}+\cdots +\frac{(-1)^{n}}{n+1}\right) \right] .
\end{equation*}
\end{enumerate}

\end{enumerate}

\end{exercice}





\begin{correction}
On consid\`ere la famille de fonctions $(f_n)_{n\in\mathbb{N}^*}$ d\'efinies
sur $]-1,+\infty[$ par~: 
\begin{equation*}
f_n(x)=x^n\ln (1+x).
\end{equation*}

\paragraph{A - \'Etude des fonctions $f_n$.}

Soit $n\in \mathbb{N}^{*}$. On note $h_{n}$ la fonction d\'{e}finie sur $%
]-1,+\infty [$ par~: 
\begin{equation*}
h_{n}(x)=n\ln (1+x)+\frac{x}{1+x}.
\end{equation*}

\begin{enumerate}
\item $h_{n}$ est d\'{e}rivable sur $]-1,+\infty [$ comme compos\'{e}e et
quotient de fonctions d\'{e}rivables et 
\begin{equation*}
h_{n}^{\prime }\left( x\right) =\frac{n}{1+x}+\frac{1+x-x}{\left( 1+x\right)
^{2}}=\frac{n+1+nx}{\left( 1+x\right) ^{2}}
\end{equation*}

Et comme $x>-1$ on a $nx>-n$ et $h_{n}^{\prime }\left( x\right) >0.$

Donc $h_{n}$ est strictement croissante sur $]-1,+\infty [$

\item On a : $h_{n}(0)=0$, et comme $h_{n}$ est strictement croissante, sur $%
]-1,0[$ on a :

\fbox{$h_{n}<0$ sur $]-1,0[$ et sur $]0,+\infty [$ on a $h_{n}>0$}

\item \'{E}tude du cas particulier $n=1$.

\begin{enumerate}
\item $f_{1}(x)=x\ln (1+x).$

La compos\'{e}e de $x\rightarrow 1+x$ d\'{e}rivable sur $]-1,+\infty [$ \`{a}
valeurs dans $]0,+\infty [$ o\`{u} $\ln $ est d\'{e}rivable.

Et $x\rightarrow x$ est d\'{e}rivable sur $\mathbb{R}$ donc $f_{n}$ est d%
\'{e}rivable sur $]-1,+\infty [$

\begin{equation*}
f_{1}^{\prime }(x)=\ln \left( 1+x\right) +\frac{x}{1+x}=h_{1}\left( x\right)
\end{equation*}

\item Donc $f_{1}$ est strictement d\'{e}croissante sur $]-1,0[$ et
strictement croissante sur $]0,+\infty [.$
\end{enumerate}

\item Soit $n\in \mathbb{N}^{*}\setminus \{1\}$.

\begin{enumerate}
\item Comme $n\in \mathbb{N}^{*},$ la fonction $x\rightarrow x^{n}$ est d%
\'{e}rivable sur $\mathbb{R}$ (la formule pour d\'{e}river serait diff\'{e}%
rente pour la puissance 0) donc (produit et somme ) $f_{n}$ est d\'{e}%
rivable sur $]-1,+\infty [$

\begin{eqnarray*}
f_{n}^{\prime }(x) &=&nx^{n-1}\ln (1+x)+\frac{x^{n}}{1+x}=x^{n-1}\left( n\ln
\left( 1+x\right) +\frac{x}{1+x}\right) \\
&=&x^{n-1}h_{n}\left( x\right)
\end{eqnarray*}

\item Donc si $n$ est pair, $n-1$ est impair donc

$n$ pair: 
\begin{tabular}{|c|ccccc|}
\hline
$x$ & -1 &  & 0 &  &  \\ \hline
$h_{n}\left( x\right) $ &  & $-$ & 0 & $+$ &  \\ \hline
$x^{n-1}$ &  & $-$ & 0 & $+$ &  \\ \hline
$f_{n}^{\prime }$ &  & $+$ & 0 & $+$ &  \\ \hline
&  &  &  & $\nearrow $ & $+\infty $ \\ 
$f_{n}\left( x\right) $ &  &  & 0 &  &  \\ 
& $-\infty $ & $\nearrow $ &  &  &  \\ \hline
\end{tabular}
$n$ impair : 
\begin{tabular}{|c|ccccc|}
\hline
$x$ & $-1$ &  & $0$ &  &  \\ \hline
$h_{n}\left( x\right) $ &  & $-$ & $0$ & $+$ &  \\ \hline
$x^{n-1}$ &  & $+$ & $0$ & $+$ &  \\ \hline
$f_{n}^{\prime }\left( x\right) $ &  & $-$ & $0$ & $+$ &  \\ \hline
$f_{n}\left( x\right) $ & $+\infty $ & $\searrow $ &  & $\nearrow $ & $%
+\infty $ \\ 
&  &  & $0$ &  &  \\ \hline
\end{tabular}

En -1, $x^{n}\rightarrow +1$ si $n$ est pair et $x^{n}\ln (1+x)\rightarrow
-\infty $ et $x^{n}\ln (1+x)\rightarrow +\infty $ si $n$ impair

En $+\infty :x^{n}\ln (1+x)\rightarrow +\infty $
\end{enumerate}
\end{enumerate}

\paragraph{B - \'{E}tude d'une suite.}

On consid\`ere la suite $\left(U_n\right)_{n\in\mathbb{N}^*}$ d\'efinie
par~: 
\begin{equation*}
U_n=\int_0^1f_n(x)\,dx.
\end{equation*}

\begin{enumerate}
\item Calcul de $U_1$.
\begin{enumerate}
\item Pour comparer, on met les deux expressions sous la m\^{e}me forme (m%
\^{e}me d\'{e}nominateur) en r\'{e}ordonnant par rapport aux puisances de $x$
: 
\begin{equation*}
ax+b+\frac{c}{x+1}=\frac{ax^{2}+\left( b+a\right) x+b+c}{x+1}
\end{equation*}

On a donc l'\'{e}galit\'{e} \textbf{si} $a=1$ et $b+a=0$ et $b+c=0$ soit $%
a=1 $, $b=-1$ et $c=1$

Une autre r\'{e}daction est de chercher ces coefficients au brouillon et de
constater que : 
\begin{equation*}
x-1+\frac{1}{x+1}=\frac{x^{2}}{x+1}
\end{equation*}%
et donc que $a=1$, $b=-1$ et $c=1$ conviennent

\item On peut alors d\'{e}terminer une primitive (la fonction int\'{e}gr\'{e}%
e est continue sur l'intervalle d'int\'{e}gration) et $x+1>0$ 
\begin{eqnarray*}
\int_{0}^{1}\frac{x^{2}}{x+1}\,dx &=&\int_{0}^{1}\left( x-1+\frac{1}{x+1}%
\right) dx=\left[ \frac{x^{2}}{2}-x+\ln \left( x+1\right) \right] _{x=0}^{1}
\\
&=&\ln \left( 2\right) -\frac{1}{2}.
\end{eqnarray*}

\item On a 
\begin{equation*}
U_{1}=\int_{0}^{1}f_{1}(x)\,dx=\int_{0}^{1}x\ln \left( 1+x\right) dx
\end{equation*}%
et en int\'{e}grant par partie (on d\'{e}rive le $\ln $ pour le faire dispara%
\^{\i}tre)

$u\left( x\right) =\ln \left( 1+x\right) ,$ $u$ de classe $C^{1}$ sur $\left[
0,1\right] $, $u^{\prime }\left( x\right) =\displaystyle
\frac{1}{1+x}$

$v^{\prime }\left( x\right) =x,$ $v^{\prime }$ est continue $v\left(
x\right) =x^{2}/2$%
\begin{eqnarray*}
U_{1} &=&\left[ \frac{x^{2}}{2}\ln \left( 1+x\right) \right]
_{0}^{1}-\int_{0}^{1}\frac{x^{2}}{2\left( 1+x\right) }dx=\frac{\ln \left(
2\right) }{2}-\frac{1}{2}\int_{0}^{1}\frac{x^{2}}{1+x}dx \\
&=&\frac{1}{4}
\end{eqnarray*}
\end{enumerate}

\item Convergence de la suite $\left( U_{n}\right) _{n\in \mathbb{N}%
^{*}}$.

\begin{enumerate}
\item Pour monter que la suite $\left( U_{n}\right) _{n\in \mathbb{N}^{*}}$
est monotone, il suffit de comparer $U_{n}$ et $U_{n+1}$.

Comme ce sont des int\'{e}grales, on compare leurs contenus sur $\left[ 0,1%
\right] $ :

$x^{n+1}-x^{n}=x^{n}\left( x-1\right) \le 0$ pour tout $x\in \left[ 0,1%
\right] $ et comme $\ln \left( 1+x\right) \ge 0$ sur $\left[ 0,1\right] $
(car $1+x\ge 1$) donc $x^{n+1}\ln \left( 1+x\right) \le x^{n}\ln \left(
1+x\right) $ et comme $0\le 1$ (ordre des bornes) on a alors 
\begin{equation*}
\int_{0}^{1}x^{n+1}\ln \left( 1+x\right) dx\le \int_{0}^{1}x^{n}\ln \left(
1+x\right) dx
\end{equation*}
et $U_{n+1}\le U_{n}.$

\textsl{Conclusion : }\fbox{la suite $U$ est d\'{e}croissante}

\item Toutes ces int\'{e}grales sont positive ou nulles car le contenu est
positif et les bornes sont en ordre croissant

Donc $U$ est d\'{e}croissante et minor\'{e}e par $0$donc convergente.

\item Pour encadrer l'int\'{e}grale, on encadre l\`{a} encore le contenu.
Pour obtenir $\frac{1}{n+1}$ on conserve le $x^{n}$ dans cet encadrement. On
se contente donc d'encadre le $\ln :$

Si $0\le x\le 1$ alors $1\le 1+x\le 2$ et comme $\ln $ est strictement
croissante sur $]0,+\infty [$ et que 1, $1+x$ et $2$ en sont \'{e}l\'{e}%
ments, $\ln \left( 1\right) \le \ln \left( 1+x\right) \le \ln \left(
2\right) .$

Comme $x^{n}\ge 0$ alors $0\le x^{n}\ln \left( 1+x\right) \le x^{n}\ln
\left( 2\right) $

Enfin comme $0\le 1:$%
\begin{equation*}
0\le \int_{0}^{1}x^{n}\ln \left( 1+x\right) dx\le \int_{0}^{1}x^{n}\ln
\left( 2\right) dx=\ln \left( 2\right) \left[ \frac{x^{n+1}}{n+1}\right]
_{0}^{1}=\frac{\ln \left( 2\right) }{n+1}
\end{equation*}
\textsl{Conclusion : }\fbox{$\forall n\in \mathbb{N}^{*}:\displaystyle
0\leqslant U_{n}\leqslant \frac{\ln 2}{n+1}$}

\item Et comme $\displaystyle
\frac{\ln 2}{n+1}\rightarrow 0,$ \fbox{par encadrement $U_{n}\underset{%
n\rightarrow +\infty }{\rightarrow }0$}
\end{enumerate}


\item Calcul de $U_n$ pour $n\geqslant 2$.

Pour $x\in [0,1]$ et $n\in \mathbb{N}^{*}\setminus \{1\}$, on pose~: 

\begin{equation*}
S_{n}(x)=1-x+x^{2}+\cdots +(-1)^{n}x^{n}=\sum_{k=0}^{n}(-1)^{k}x^{k}.
\end{equation*}
\begin{enumerate}
\item Comme $-x\ne 1$ on a : 
\begin{eqnarray*}
S_{n}(x) &=&\sum_{k=0}^{n}(-1)^{k}x^{k}=\sum_{k=0}^{n}(-x)^{k}=\frac{\left(
-x\right) ^{n+1}-1}{-x-1}. \\
&=&\frac{1}{1+x}-\frac{(-1)^{n+1}x^{n+1}}{1+x} \\
&=&\frac{1}{1+x}+\frac{(-1)^{n}x^{n+1}}{1+x}
\end{eqnarray*}

\item On a donc en int\'{e}grant l'\'{e}galit\'{e} pr\'{e}c\'{e}dente sur $%
\left[ 0,1\right] :$%
\begin{eqnarray*}
\int_{0}^{1}\sum_{k=0}^{n}(-1)^{k}x^{k}dx
&=&\sum_{k=0}^{n}(-1)^{k}\int_{0}^{1}x^{k}dx=\sum_{k=0}^{n}(-1)^{k}\left[ 
\frac{x^{k+1}}{k+1}\right] _{x=0}^{1} \\
&=&\sum_{k=0}^{n}\frac{(-1)^{k}}{k+1}
\end{eqnarray*}%
d'une part et d'autre part 
\begin{eqnarray*}
\int_{0}^{1}\sum_{k=0}^{n}(-1)^{k}x^{k}dx &=&\int_{0}^{1}\left( \frac{1}{1+x}%
+\frac{(-1)^{n}x^{n+1}}{1+x}\right) dx \\
&=&\int_{0}^{1}\frac{1}{1+x}dx+(-1)^{n}\int_{0}^{1}\frac{x^{n+1}}{1+x}dx \\
&=&\left[ \ln \left( 1+x\right) \right] _{0}^{1}+(-1)^{n}\int_{0}^{1}\frac{%
x^{n+1}}{1+x}dx \\
&=&\ln 2+(-1)^{n}\int_{0}^{1}\frac{x^{n+1}}{1+x}\,dx
\end{eqnarray*}%
et finalement 
\begin{equation*}
\sum_{k=0}^{n}\frac{(-1)^{k}}{k+1}=\ln 2+(-1)^{n}\int_{0}^{1}\frac{x^{n+1}}{%
1+x}\,dx.
\end{equation*}

\item On reconna\^{\i}t dans la formule propos\'{e}e $\displaystyle%
\sum_{k=0}^{n}\frac{(-1)^{k}}{k+1}$. On fait donc appara\^{\i}tre dans
l'expression de $U_{n}$ la quantit\'{e} $\displaystyle\int_{0}^{1}\frac{%
x^{n+1}}{1+x}\,dx:$

On a 
\begin{equation*}
U_{n}=\int_{0}^{1}x^{n}\ln \left( 1+x\right) dx
\end{equation*}
avec $u\left( x\right) =\ln \left( 1+x\right) $, $u$ est de classe $C^{1}$
sur $\left[ 0,1\right] $ et $u^{\prime }\left( x\right) =\displaystyle
\frac{1}{1+x}$

et avec $v^{\prime }\left( x\right) =x^{n}$ continue on a $v\left( x\right) =%
\displaystyle
\frac{x^{n+1}}{n+1}$ donc en int\'{e}grant par parties : 
\begin{eqnarray*}
U_{n} &=&\left[ \frac{x^{n+1}\ln \left( 1+x\right) }{n+1}\right]
_{x=0}^{1}-\int_{0}^{1}\frac{x^{n+1}}{\left( n+1\right) \left( 1+x\right) }dx
\\
&=&\frac{\ln 2}{n+1}-\frac{1}{n+1}\int_{0}^{1}\frac{x^{n+1}}{1+x}dx
\end{eqnarray*}
et de 
\begin{equation*}
\sum_{k=0}^{n}\frac{(-1)^{k}}{k+1}=\ln 2+(-1)^{n}\int_{0}^{1}\frac{x^{n+1}}{%
1+x}\,dx
\end{equation*}
on tire 
\begin{equation*}
\int_{0}^{1}\frac{x^{n+1}}{1+x}\,dx=(-1)^{n}\left[ \sum_{k=0}^{n}\frac{%
(-1)^{k}}{k+1}-\ln 2\right]
\end{equation*}

d'o\`{u} finalement 
\begin{equation*}
U_{n}=\frac{\ln 2}{n+1}+\frac{(-1)^{n}}{n+1}\left[ \ln 2-\left( 1-\frac{1}{2}%
+\cdots +\frac{(-1)^{k}}{k+1}+\cdots +\frac{(-1)^{n}}{n+1}\right) \right] .
\end{equation*}
\end{enumerate}

\end{enumerate}


\end{correction}



%%%%%---
\subsection{Binome de Newton}


\begin{exercice}


\begin{enumerate}
\item Vérifier que la formule du binôme est vraie pour $n=0$, $n=1$, $n=2$ (et sur votre brouillon faite $n=3$).
On va prouver la formule par récurrence. On détaille les différentes étapes dans les prochaines questions: 
\item Montrer que $\forall (a,b)\in \bC^2,\,  \forall n \in \N, $:
$$\sum_{k=0}^n \binom{n}{k}a^{k+1} b^{n-k} = a^{n+1}+\sum_{k=1}^{n} \binom{n}{k-1}a^{k} b^{n-k+1}.$$


\item Montrer que $\forall (a,b)\in \bC^2,\,  \forall n \in \N, $
$$(a+b)\left( \sum_{k=0}^n \binom{n}{k}a^k b^{n-k}\right) = a^{n+1}+b^{n+1}+\sum_{k=1}^{n} \left( \binom{n}{k-1}+\binom{n}{k}\right)a^{k} b^{n-k+1}$$

\item En déduire que 
$$(a+b)\left( \sum_{k=0}^n \binom{n}{k}a^k b^{n-k}\right) = \sum_{k=0}^{n+1}  \binom{n+1}{k}a^{k} b^{n+1-k}$$

\item Conclure. 

Application : 
Soit $n,m\in \N^2$ 
\begin{enumerate}
\item Calculer $(1+x)^n(1+x)^m$  et $(1+x)^{n+m}$ à l'aide du binome de Newton. 
\item En déduire que pour tout $r\leq n+m$ on a : 
$$\sum_{j=0}^r  \binom{n}{j} \binom{m}{r-j}=\binom{n+m}{r}$$
%\item En déduire que pour tout $N\in \N^*$ et tout $n\in \intent{0,N}$ 
%$$\sum_{k=n}^N \binom{k}{n}=\binom{N+1}{n+1}$$
%(jouer avec les indices, changement de variables etc... ) 
\end{enumerate}


\end{enumerate}




\end{exercice}




\begin{correction}



\begin{enumerate}

\item Vérifier que la formule du binôme est vraie pour $n=0$, $n=1$, $n=2$ (et sur votre brouillon faite $n=3$).



\begin{enumerate}
\item $n=0$

On a 
$(a+b)^0 =1$ et 
$\sum_{k=0}^0 \binom{0}{k}a^k b^{0-k} = a^0b^0=1$

\item $n=1$

On a 
$(a+b)^1 =a+b$ et 
$\ddp \sum_{k=0}^1 \binom{1}{k}a^k b^{1-k} = \binom{1}{0}a^0 b^{1-0}+\binom{1}{1}a^1 b^{1-1}= a+b$


\item $n=2$

On a 
$(a+b)^2 =a^2+2ab+b^2$ et 
$\ddp \sum_{k=0}^2 \binom{2}{k}a^k b^{2-k} = \binom{2}{0}a^0 b^{2-0}+\binom{2}{1}a^1 b^{2-1}+\binom{2}{2}a^2 b^{2-2} =b^2+2ab+b^2$


\end{enumerate}
 









On va prouver la formule par récurrence. On détaille les différentes étapes dans les prochaines questions: 
\item Montrer que $\forall (a,b)\in \bC^2,\,  \forall n \in \N, $:
$$\sum_{k=0}^n \binom{n}{k}a^{k+1} b^{n-k} = a^{n+1}+\sum_{k=1}^{n} \binom{n}{k-1}a^{k} b^{n-k+1}.$$


\begin{align*}
\sum_{k=0}^n \binom{n}{k}a^{k+1} b^{n-k}  &= \sum_{k=0}^{n-1} \binom{n}{k}a^{k+1} b^{n-k}  + \binom{n}{n}a^{n+1} b^{n-n} \\
&= \sum_{k=0}^{n-1} \binom{n}{k}a^{k+1} b^{n-k}  + a^{n+1} 
\end{align*}
On fait le changement devariable $k+1=j$ sur la somme. On obtient 
$j\in \intent{1,n}$, donc 
$$\sum_{k=0}^{n-1} \binom{n}{k}a^{k+1} b^{n-k}  = \sum_{j=1}^{n} \binom{n}{j-1}a^{j} b^{n-j+1} $$
Comme $j$ est un indice muet, on peut le changer en $k$. On a donc la formule demandée.




\item Montrer que $\forall (a,b)\in \bC^2,\,  \forall n \in \N, $
$$(a+b)\left( \sum_{k=0}^n \binom{n}{k}a^k b^{n-k}\right) = a^{n+1}+b^{n+1}+\sum_{k=1}^{n} \left( \binom{n}{k-1}+\binom{n}{k}\right)a^{k} b^{n-k+1}$$



\begin{align*}
(a+b)\left( \sum_{k=0}^n \binom{n}{k}a^k b^{n-k}\right) &=a  \sum_{k=0}^n \binom{n}{k}a^k b^{n-k} +b\sum_{k=0}^n \binom{n}{k}a^k b^{n-k}\\
&= \sum_{k=0}^n \binom{n}{k}a^{k+1} b^{n-k} +\sum_{k=0}^n \binom{n}{k}a^k b^{n+1-k} 
\end{align*}
Maintenant on fait un changement de variable sur la première somme en posant $j =k+1$. On obtient : 
$$ \sum_{k=0}^n \binom{n}{k}a^{k+1} b^{n-k}= \sum_{j=1}^{n+1} \binom{n}{j-1}a^{j} b^{n-j+1}$$
On a donc, en se rappelant que $j$ est muet et donc remplacable par $k$
\begin{align*}
(a+b)\left( \sum_{k=0}^n \binom{n}{k}a^k b^{n-k}\right) &= \sum_{k=1}^{n+1} \binom{n}{k-1}a^{k} b^{n-k+1}+ \sum_{k=0}^{n} \binom{n}{k}a^{k} b^{n+1-k}
\end{align*}
On applique la relation de Chasles au  dernier terme de la première somme et au premier terme de la deuxième somme. On obtient : 

\begin{align*}
(a+b)\left( \sum_{k=0}^n \binom{n}{k}a^k b^{n-k}\right) &= \sum_{k=1}^{n} \binom{n}{k-1}a^{k} b^{n-k+1}+\binom{n}{n+1-1}a^{n+1} b^{n-(n+1)+1} \\& \hspace{2cm} +\sum_{k=1}^{n} \binom{n}{k}a^{k} b^{n+1-k} + \binom{n}{0}a^{0} b^{n+1-0} \\
&=  a^{n+1}+b^{n+1}+\sum_{k=1}^{n} \left( \binom{n}{k-1}+\binom{n}{k}\right)a^{k} b^{n-k+1}
\end{align*}




\item En déduire que 
$$(a+b)\left( \sum_{k=0}^n \binom{n}{k}a^k b^{n-k}\right) = \sum_{k=0}^{n+1}  \binom{n+1}{k}a^{k} b^{n+1-k}$$



On applique la relation obtenue dans la question 2 (relation de Pascal) à ce qu'on vient de trouver. 
$$\sum_{k=1}^{n} \left( \binom{n}{k-1}+\binom{n}{k}\right)a^{k} b^{n-k+1} = \sum_{k=1}^{n} \binom{n+1}{k}a^{k} b^{n-k+1}$$
Par ailleurs, 
$$a^{n+1} = \binom{n+1}{n+1}a^{n+1} b^{n-(n+1)+1}$$
et 
$$b^{n+1} = \binom{n+1}{0}a^{0} b^{n-0+1}$$
Ce sont donc les deux termes qui manquent à la somme de $0$ à $(n+1)$. ON a ainsi 
$$a^{n+1}+b^{n+1}+\sum_{k=1}^{n} \left( \binom{n}{k-1}+\binom{n}{k}\right)a^{k} b^{n-k+1} =\sum_{k=0}^{n+1}  \binom{n+1}{k}a^{k} b^{n+1-k}$$
Ce qui prouve le résultat grace à la question 5




\item Conclure. 



On fait une récurrence. On pose pour tout $n\in \N$:
$$\cP :' \forall a, b\in \bC, (a+b)^n =\sum_{k=0}^n \binom{n}{k}a^k b^{n-k}.$$
L'initialisation a été faite à la question 3.

L'hérédité correspond à la question 6. 

\paragraph{Application }

D'après le binome :
$$(1+x)^n(1+x)^m = \sum_{k=0}^n \binom{n}{k}x^k \times  \sum_{l=0}^m \binom{m}{l}x^l $$
Et par ailleurs 
$$(1+x)^{n+m} =  \sum_{j=0}^{n+m} \binom{n+m}{j}x^j$$

Comme $(1+x)^n(1+x)^m =(1+x)^{n+m}$ on peut identifier les  coefficients des deux polynomes. On obtient pour tout $r\in \intent{0,n+m}$
$$ \binom{n+m}{r} = \sum_{k,l, k+l=r}  \binom{n}{k} \binom{m}{l}$$
et 
$$ \sum_{k,l, k+l=r}  \binom{n}{k} \binom{m}{l} = \sum_{k=0}^n  \binom{n}{k} \binom{m}{r-k}$$




\end{enumerate}










\end{correction}