% Titre : suites
% Filiere : BCPST
% Difficulte : 
% Type : TD 
% Categories :suites
% Subcategories : 
% Keywords : suites




\begin{exercice}  \; Suites homographiques.\\
\noindent On consid\`ere les suites $\suiteu$ et $\suitev$ d\'efinies par
$$u_0=0\ \hbox{et}\ \forall n\in\N,\ u_{n+1}=\ddp\frac{5u_n-2}{u_n+2} \quad \textmd{ et } \quad \forall \, n \in \N, \ v_n = \ddp\frac{u_n-2}{u_n-1}.$$
\begin{enumerate}
 \item
Montrer que la suite $\suiteu$ est bien d\'efinie et que pour tout $n\geq 3$, $u_n>1$. 
\item 
En d\'eduire que la suite $\suitev$  est bien d\'efinie sur $\N$.
\item 
Montrer que $\suitev$ est g\'eom\'etrique.
\item 
En d\'eduire l'expression explicite de $\suitev$ puis de $\suiteu$.
\item Etudier la convergence de la suite $\suiteu$.
\end{enumerate}
\end{exercice}


\%\%\%\%\%\%\%\%\%\%\%\%\%\%\%\%\%\%\%\%
\%\%\%\%\%\%\%\%\%\%\%\%\%\%\%\%\%\%\%\%
\%\%\%\%\%\%\%\%\%\%\%\%\%\%\%\%\%\%\%\%




\begin{correction} \;
Toutes ces suites sont des suites lin\'eaires r\'ecurrentes d'ordre deux, on les r\'esout en \'etudiant l'\'equation caract\'eristique. Je ne donne ici que le r\'esultat.
\begin{enumerate}
 \item $\forall n\in\N,\quad u_n=\ddp\frac{1}{4}\left( 3^{n+1}+(-1)^n \right)$
\item $\forall n\in\N,\quad u_n=(1-n) 2^n  $
%\item $\forall n\in\N,\quad u_n=\cos{\left( \ddp\frac{n\pi}{3} \right)}+\ddp\frac{1}{\sqrt{3}}\sin{\left( \ddp\frac{n\pi}{3} \right)}  $
%\item A ne pas faire.
\item $\forall n\in\N,\quad u_n=\left(2-\ddp\frac{5}{4}n\right)(-4)^{n}  $
\item Suite de Fibonacci, $\ddp \forall n\in\N,\quad u_n=\frac{1}{\sqrt{5}}\left(\frac{1+\sqrt{5}}{2}\right)^n-\frac{1}{\sqrt{5}}\left(\frac{1-\sqrt{5}}{2}\right)^n$.
\item $\forall n\in\N,\quad u_n=2^n\left( \cos{\left( \ddp\frac{n\pi}{2} \right)}+\sin{\left( \ddp\frac{n\pi}{2} \right)}  \right) $.
\end{enumerate}
\end{correction}