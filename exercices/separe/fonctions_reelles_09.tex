% Titre : fonctions
% Filiere : BCPST
% Difficulte : 
% Type : TD 
% Categories :fonctions
% Subcategories : 
% Keywords : fonctions




\begin{exercice}  \;
\begin{enumerate}
\item Soit $\ddp x \in \R\backslash \left\{ \pi + 2 k \pi, k \in \Z \right\}$. On pose : $u =  \tan \left(\ddp \frac{x}{2} \right).$ \'Etablir les relations suivantes, et indiquer pour quelles valeurs de $x$ elles sont valides :\\
\begin{minipage}[t]{0.3\textwidth}
(a) $\ddp \cos x = \frac{1-u^2}{1+u^2}$
\end{minipage}
\begin{minipage}[t]{0.3\textwidth}
(b)  $\ddp \sin x = \frac{2 u}{1+u^2}$
\end{minipage}
\begin{minipage}[t]{0.3\textwidth}
(c) $\ddp \tan x = \frac{2 u}{1-u^2}$
\end{minipage}
\item En utilisant ces relations, r\'esoudre sur $\R$ l'\'equation : $\ddp \cos x - 3 \sin x + 2\tan \left( \frac{x}{2} \right) - 1 = 0.$
\end{enumerate}
\end{exercice}


\%\%\%\%\%\%\%\%\%\%\%\%\%\%\%\%\%\%\%\%
\%\%\%\%\%\%\%\%\%\%\%\%\%\%\%\%\%\%\%\%
\%\%\%\%\%\%\%\%\%\%\%\%\%\%\%\%\%\%\%\%



\begin{correction}  
\begin{enumerate}
\item Tout d'abord, $u=\tan \left( \frac{x}{2}\right)$ est bien d\'efini pour $\ddp x \in \R\backslash \left\{ \pi + 2 k \pi, k \in \Z \right\}$.
\begin{itemize}
\item[(a)] On a $1+u^2>0$ donc $\ddp \frac{1-u^2}{1+u^2}$ est bien d\'efini. De plus, on a 
$$\frac{1-u^2}{1+u^2} = \frac{1-\frac{\sin^2 \left( \frac{x}{2}\right) }{\cos^2 \left( \frac{x}{2}\right) }}{1+\frac{\sin^2 \left( \frac{x}{2}\right) }{\cos^2 \left( \frac{x}{2}\right) }} = \frac{\cos^2 \left( \frac{x}{2}\right) - \sin^2 \left( \frac{x}{2}\right)}{\cos^2 \left( \frac{x}{2}\right) + \sin^2 \left( \frac{x}{2}\right)} = \frac{\cos x}{1} = \cos x.$$
\item[(b)] De m\^eme, $1+u^2>0$ donc $\ddp \frac{2 u}{1+u^2}$ est bien d\'efini. De plus, on a 
$$\frac{2 u}{1+u^2} = \frac{ 2 \frac{\sin \left( \frac{x}{2}\right) }{\cos \left( \frac{x}{2}\right) }}{1+\frac{\sin^2 \left( \frac{x}{2}\right) }{\cos^2 \left( \frac{x}{2}\right) }} = \frac{ 2 \cos \left( \frac{x}{2}\right) \sin \left( \frac{x}{2}\right)}{\cos^2 \left( \frac{x}{2}\right) + \sin^2 \left( \frac{x}{2}\right)} = \frac{\sin x}{1} = \sin x.$$
\item[(c)] On a $1-u^2 \not=0$ si et seulement si $\ddp \tan^2 \left( \frac{x}{2}\right) \not=1$, c'est-\`a-dire $\ddp \tan \left( \frac{x}{2}\right) \not\in \{-1,1\}$. On doit donc avoir $x \in \ddp \R\backslash \left( \left\{ \pi + 2 k \pi, k \in \Z \right\} \cup \left\{ \frac{\pi}{2} + k \pi, k \in \Z \right\} \right)$. On a alors :
$$\frac{2 u}{1- u^2} = \frac{ 2 \frac{\sin \left( \frac{x}{2}\right) }{\cos \left( \frac{x}{2}\right) }}{1-\frac{\sin^2 \left( \frac{x}{2}\right) }{\cos^2 \left( \frac{x}{2}\right) }} = \frac{ 2 \cos \left( \frac{x}{2}\right) \sin \left( \frac{x}{2}\right)}{\cos^2 \left( \frac{x}{2}\right) - \sin^2 \left( \frac{x}{2}\right)} = \frac{\sin x}{\cos x} = \tan x.$$
 \end{itemize}
\item L'\'equation est d\'efinie pour $\ddp \frac{x}{2} \in \R\backslash \left\{ \frac{\pi}{2} + k \pi, k \in \Z \right\}$. Le domaine de d\'efinition est donc  
 \conclusion{$\mathcal{D}=\ddp \R\backslash \left\{ \pi + 2 k \pi, k \in \Z \right\}$}.\\
On pose alors $u =  \tan \left(\ddp \frac{x}{2} \right)$, et on utilise les formules de la question pr\'ec\'edentes pour transformer l'\'equation. On est ramen\'es \`a r\'esoudre
$$\frac{1-u^2}{1+u^2} - 3 \frac{2u}{1+u^2} + 2u - 1 = 0 \Leftrightarrow \frac{1-u^2- 6u + 2u +2u^3-1-u^2}{1+u^2} = 0   \Leftrightarrow \frac{2u(u^2-u-2)}{1+u^2}  = 0.$$
On doit donc trouver les $u$ tels que le num\'erateur s'annule. On obtient $u \in \{0,-1, 2\}$. On doit ensuite revenir \`a la variable $x$, on r\'esout donc 
\begin{itemize}
\item[$\bullet$] $\tan\left( \ddp \frac{x}{2} \right) = 0 \Leftrightarrow \frac{x}{2} = k \pi, k \in \Z \Leftrightarrow  x = 2k\pi, k \in \Z$ 
\item[$\bullet$] $\tan\left( \ddp \frac{x}{2} \right) = -1\Leftrightarrow \frac{x}{2} = - \frac{\pi}{4} + k \pi, k \in \Z \Leftrightarrow  x = - \frac{\pi}{2} +2k\pi, k \in \Z$ 
\item[$\bullet$] $\tan\left( \ddp \frac{x}{2} \right) = 2 \Leftrightarrow \frac{x}{2} = \arctan(2) + k \pi, k \in \Z \Leftrightarrow  x = \arctan(2)  +2k\pi, k \in \Z$.
\end{itemize}
$$\fbox{ $ S = \ddp \left\{2 k \pi, k \in \Z \right\} \cup  \left\{-\frac{\pi}{2} + 2 k\pi, k \in \Z \right\} \cup \left\{2\arctan 2 + 2 k \pi, k \in \Z \right\} $}$$
\end{enumerate}
\end{correction}

\vspace*{1cm}
%------------------------------------------------
%-------------------------------------------------
%-------------------------------------------------
%--------------------------------------------------
%----------------------------------------------------------------------------------------------
%-----------------------------------------------------------------------------------------------

