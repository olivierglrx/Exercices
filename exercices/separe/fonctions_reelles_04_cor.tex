\begin{correction}  \;
\begin{enumerate}
 \item
$f(x)=x\ln{\ddp\sqrt{e^{\frac{x}{2}}}}+\left( \ddp\sqrt{e^{2\ln{(2x-1)}}} \right)^3$: La fonction $f$ est bien d\'efinie si et seulement si $\sqrt{e^{\frac{x}{2}}}>0$, $e^{\frac{x}{2}}\geq 0$, $2x-1>0$ et $e^{2\ln{(2x-1)}}\geq 0$. Comme toute exponentielle est strictement positive, la fonction $f$ est bien d\'efinie si et seulement si $2x-1>0\Leftrightarrow x>\ddp\demi$. Ainsi \fbox{$\mathcal{D}_f=\rbrack \ddp\demi,+\infty\lbrack$}.\\
\noindent  Pour tout $x>\ddp\demi$, on a: 
$$f(x)=x\ln{  \left( e^{\frac{x}{2}} \right)^{\demi} } + \left( e^{2\ln{(2x-1)}}\right)^{\frac{3}{2}}= x\ln{( e^{\frac{x}{4}} )}  +  \left( e^{\ln{((2x-1)^2)}}\right)^{\frac{3}{2}}= x\times \ddp\frac{x}{4}+ ((2x-1)^2)^{\frac{3}{2}}= \ddp\frac{x^2}{4}+(2x-1)^3$$
\item 
$g(x)=e^{\ddp\sqrt{\ln{x}}}+e^{(\ln{x})^2}$. La fonction $g$ est bien d\'efinie si et seulement si $x>0$ et $\ln{x}\geq 0\Leftrightarrow x\geq 1$. Ainsi \fbox{$\mathcal{D}_g=\lbrack 1,+\infty\lbrack$}.\\
\noindent On ne peux RIEN simplifier car $\sqrt{\ln{x}}=\left( \ln{x} \right)^{\demi}\not= \ddp\demi\ln{(x)}$... De m\^{e}me, on a: $(\ln{x})^2\not= \ln{(x^2)}$... et on ne peux rien faire avec $(\ln{x})^2$.
\end{enumerate}
\end{correction}