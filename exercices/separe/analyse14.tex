% Titre : Equations trigonométriques
% Filiere : BCPST
% Difficulte :
% Type : DS, DM
% Categories : analyse
% Subcategories : 
% Keywords : analyse





\begin{exercice}
Résoudre dans $\R$ puis dans $[-\pi, \pi[$:
\begin{equation}
\cos(3x-1)=\sin(2x)
\end{equation}


\begin{equation}
\cos(3x)+\cos(2x)+\cos(-x)=0
\end{equation}
\end{exercice}

\begin{correction}
$$\cos(3x-1)=\sin(2x)$$
Equivaut à 
$$\cos(3x-1)=\cos(\frac{\pi}{2}-2x)$$
Donc 
$$\left\{ \begin{array}{cc}
3x-1&\equiv \frac{\pi}{2}-2x\quad [2\pi]\\
ou &\\
3x-1&\equiv -\frac{\pi}{2}+2x\quad [2\pi]
\end{array}\right.$$
C'est à dire : 
$$\left\{ \begin{array}{cc}
5x&\equiv \frac{\pi}{2}+1\quad [2\pi]\\
ou &\\
x&\equiv 1-\frac{\pi}{2}\quad [2\pi]
\end{array}\right.$$

$$\left\{ \begin{array}{cc}
x&\equiv \frac{\pi}{10}+\frac{1}{5}\quad [\frac{2\pi}{5}]\\
ou &\\
x&\equiv 1-\frac{\pi}{2}\quad [2\pi]
\end{array}\right.$$






Sur $\R $ : les solutions sont $$\cS=\bigcup_{k\in \Z} \{ \frac{\pi}{10}+\frac{1}{5} + \frac{2k\pi}{5},1-\frac{\pi}{2}+2k\pi \}$$
Sur $[-\pi, \pi[$:
$$\cS\cap [-\pi, \pi[ = \{  \frac{\pi}{10}+\frac{1}{5},  \frac{\pi}{2}+\frac{1}{5} ,  \frac{-3\pi}{10}+\frac{1}{5},  \frac{-7\pi}{10}+\frac{1}{5} ,  \frac{9\pi}{10}+\frac{1}{5}, 1-\frac{\pi}{2}\}$$


(On a $\frac{1}{5}< \frac{\pi}{10} \Longleftrightarrow 1<\frac{\pi}{2} $, qui est vrai, donc $ \frac{9\pi}{10}+\frac{1}{5}<\pi$  )

\begin{equation}\tag{(2)}
\cos(3x)+\cos(2x)+\cos(-x)=0
\end{equation}
$\cos(3x) =\Re((e^{ix})^3)=\cos^3(x) -3\cos(x)\sin^2(x)=4\cos^3(x)-3\cos(x) $\\
$\cos(2x) = \cos^2(x)-\sin^2(x)=2\cos^2(x)-1$\\
L'équation est équivalente  à 
$$4\cos^3(x)+2\cos^2(x)-2\cos(x)-1=0$$
Notons $X=\cos(x)$, on obtient l'équation 
$$4X^3+2X^2-2X-1 =0.$$
Or $4X^3+2X^2-2X-1  =4(X+\frac{1}{2}) (X-\frac{\sqrt{2}}{2})(X+\frac{\sqrt{2}}{{2}})$
On a donc 

$\left\{ \begin{array}{cc}
\cos(x)&=\frac{-1}{2}\\
ou&\\
\cos(x)&=\frac{\sqrt{2}}{2}\\
ou&\\
\cos(x)&=-\frac{\sqrt{2}}{2}\\
\end{array}\right.$

\conclusion{$\cS = \{ \frac{2\pi}{3}+2k\pi, -\frac{2\pi}{3}+2k\pi, \frac{\pi}{4}+2k\pi, -\frac{\pi}{4}+2k\pi,\frac{3\pi}{4}+2k\pi,-\frac{3\pi}{4}+2k\pi\,  |\, k\in \Z\}$}

et 
\conclusion{$\cS \cap[-\pi, \pi[ =  \{ \frac{2\pi}{3}, -\frac{2\pi}{3}, \frac{\pi}{4}, -\frac{\pi}{4},\frac{3\pi}{4},-\frac{3\pi}{4} \}$}

\end{correction}