\begin{correction}  \;
La fonction $f$ est bien d\'efinie si et seulement si $x^2-(m+1)x+m\geq 0$. Le discriminant donne: $\Delta=(m+1)^2-4m=(m-1)^2$.
\begin{itemize}
\item[$\bullet$] Cas 1: si $m=1$: On obtient alors $\Delta=0$ et ainsi, pour tout $x\in\R$, on a: $x^2-(m+1)x+m\geq 0$. Ainsi : \fbox{$\mathcal{D}_{m=1}=\R$}. 
\item[$\bullet$] Cas 2: si $m\not= 1$: On obtient alors $\Delta>0$ et les deux racines distinctes sont alors: $\ddp\frac{m+1+|m-1|}{2}$ et $\ddp\frac{m+1-|m-1|}{2}$. Afin de calculer la valeur absolue, on doit encore distinguer deux cas:
\begin{itemize}
\item[$\star$] Si $m>1$: les deux racines sont alors $1$ et $m$ et on obtient ainsi: \fbox{$\mathcal{D}_{m>1}=\rbrack -\infty,1\rbrack\cup\lbrack m,+\infty\lbrack$}.
\item[$\star$] Si $m<1$: les deux racines sont alors $m$ et $1$ et on obtient ainsi: \fbox{$\mathcal{D}_{m<1}=\rbrack -\infty,m\rbrack\cup\lbrack 1,+\infty\lbrack$}.
\end{itemize}
\end{itemize}
\end{correction}