% Titre : Suite récurrente et césaro PB(long)
% Filiere : BCPST
% Difficulte :
% Type : DS, DM
% Categories : analyse
% Subcategories : 
% Keywords : analyse



\begin{exercice}
Le but de cet exerice est l'étude de  la suite $\left(a_{n}\right)$ définie par $a_{1}=1$ et $\forall n \in \mathbb{N}^{*}, a_{n+1}=$ $\frac{a_{n}\left(1+a_{n}\right)}{1+2 a_{n}} .$

\begin{enumerate}
\item Etude de la limite de $\suiteun{a}$.
\begin{enumerate}

\item  Calculer $a_{2}$ et $a_{3}$.
\item Etudier la fonction $f$ définie par  $f(x) =\frac{x(x+1)}{1+2x}$
\item Déterminer l'image directe de $]0,1[$ par $f$. 
\item  Démontrer que, $\forall n \geqslant 2$, on a $0<a_{n}<1$.
\item Montrer que la suite $\suiteun{a}$ est décroissante.
\item Rédoudre l'équation $f(x)=x$ sur $[0,1]$. 
\item En déduire la limite de $\suiteun{a}$.
\end{enumerate}
\item Un résultat intermédiaire. 

Soit $\suiteun{u} $ une suite croissante, admettant une limite $\ell$ en $+\infty$ et $\suiteun{C}$ définie par 
$$C_n=\frac{1}{n}\sum_{k=1}^n  u_n$$
\begin{enumerate}
\item Montrer que pour tout $n\in \N^*$, $C_n\leq u_n$. 
\item Montrer que pour $\suiteun{C}$ est croissante. 
\item Montrer que pour tout $n\in \N^*$, $2C_{2n}-C_n \geq u_{n+1}$. 
\item En déduire que $\suiteun{C}$ converge et donner la valeur de sa  la limite en fonction de celle de $\suiteun{u}$. 

\end{enumerate}
\item Etude d'un équivalent de $\suiteun{a}$.
\begin{enumerate}
\item Montrer que $\frac{1}{a_{n+1}}-\frac{1}{a_{n}}=\frac{1}{1+a_{n}}$.
\item On pose $u_{n}=\frac{1}{a_{n+1}}-\frac{1}{a_{n}} .$ Déterminer la limite de $\suiteun{u}$.
\item Montrer que $\suite{u}$ est croissante. 
\item En posant $C_{n}=\ddp \frac{1}{n} \sum_{k=1}^{n} u_{k}$, exprimer $C_{n}$ en fonction de $a_{n+1}$ et de $a_{1}$.
\item Conclure à l'aide de la question 2.e que $a_n \equivalent{+\infty} \frac{1}{n}$.
\end{enumerate}
\end{enumerate}


\end{exercice}

\begin{correction}
\begin{enumerate}
\item \begin{enumerate}
\item $a_2 = \frac{1(1+1)}{1+2\times1}= \frac{2}{3}$

 $a_3 =  \frac{ \frac{2}{3}(1+ \frac{2}{3})}{1+2\times \frac{2}{3}}= \frac{ \frac{10}{9}}{\frac{7}{3}} = \frac{10}{21}$
 
\conclusion{  $a_2 = \frac{2}{3}$ et $a_3 =  \frac{10}{21}$}
 
\item $f$ est continue et dérivable sur $\R\setminus \{ \frac{-1}{2}\}$ et $\forall x\in  \R\setminus \{ \frac{-1}{2}\}$ 
$$ f'(x) = \frac{(2x+1)(1+2x) -x(x+1)2  }{(1+2x)^2 } = \frac{2x^2 +2x +1}{(1+2x)^2}$$

Le discriminant du numérateur vaut $\Delta = 4 - 8=-4<0$ donc $f'$ est strictement positif sur $\R\setminus \{ \frac{-1}{2}\}$
Ainsi $f$ est strictement croissante sur $]-\infty, \frac{-1}{2}[$ et sur $]\frac{-1}{2},+\infty[$.

\item  $f(0)=0$ et $f(1)=\frac{2}{3}$, comme $f$ est continue et strictement croissante sur $[0,1]$, le thoérème de la bijection assure que 
\conclusion{$f(]0,1[) = ]0,\frac{2}{3}[$}

\item On montre le résultat par récurrence. Soit $P(n)$ la propriété 
$$P(n): " 0<a_n<1"$$

\paragraph{Initialisation : }
$P(2)$ est vraie d'après la question 1a)

\paragraph{Hérédité : }
On suppose qu'il existe $n\geq 2$ tel que $P(n)$ soit vraie, on a alors $0<a_n <1$. D'après l'étude de $f$ on a alors que $f(a_n) \in 0,\frac{2}{3}
\subset ]0,1[$, donc 
$$a_{n+1} =f(a_n) \in ]0,1[$$

\paragraph{Conclusion : }
La propriété $P(n)$ est héréditaire donc pour tout $n\geq 2$, on  a 
\conclusion{ $0<a_n<1$}



\item  Pour tout $n\in N^*$ on a 
\begin{eqnarray*}
a_{n+1} -a_n &= f(a_n) -a_n\\
					&= \frac{a_n(1+a_n)}{1+2a_n)} - a_n\\
					&=  \frac{a_n(1+a_n) - a_n -2a_n^2}{1+2a_n)} \\
					&=  \frac{-a_n^2}{1+2a_n)} 
\end{eqnarray*}
Or on a a prouvé que $a_n\in ]0,1[$ donc $1+2a_n>0$ et $-a_n^2<0$ donc 
$a_{n+1}-a_n<0$. Ainsi: 
\conclusion{ $\suite{a}$ est décroissante }

\item
\begin{eqnarray}
f(x) &= x\\
\equivaut \frac{x(x+1)}{1+2x}&=x\\
\equivaut \frac{-2x^2)}{1+2x}&=0\\ 
\equivaut x=0 
\end{eqnarray}
Donc 

\conclusion{  La seule solution de $f(x)=x$ est $x=0$}

\item La suite $\suiteun{a}$ est décroissante et minorée donc elle converge, notons $\ell$ sa limite. Par unicité de la limite $a_{n+1} $ converge vers $\ell$ et par continuité de $f$ la suite $f(a_n) $ converge vers $f(\ell)$ 
Ainsi $f(\ell)=\ell$ et finalement d'après la question précédente:  
\conclusion{ $\ddp \lim_{n\tv +\infty } a_n = 0$}

\end{enumerate}
\item \begin{enumerate}
\item Par croissance de $\suiteun{u}$ on a pour  tout $k\in \intent{1,n}$, 
$$u_k \leq u_n$$
Donc $$\sum_{k=1}^n u_k \leq \sum_{k=1}^n u_n,$$
c'est-à-dire $\sum_{k=1}^n u_k \leq n u_n$
En divisant par $n\in \N^*$ on obtient :
\conclusion{ $C_n \leq u_n$}

\item 
$C_{n+1} =\frac{1}{n+1} \sum_{k=1}^{n+1} u_k = \frac{1}{n+1} \sum_{k=1}^{n} u_k +\frac{1}{n+1} u_{n+1}$
Or $C_n \leq u_n \leq u_{n+1} $ où la deuxième inégalité vient de la croissance de $\suiteun{u}$. Donc

\begin{eqnarray*}
C_{n+1} &\geq \frac{1}{n+1} \sum_{k=1}^{n} u_k + \frac{1}{n+1} C_n\\
			&\geq \frac{1}{n+1} nC_n + \frac{1}{n+1} C_n\\
			&\geq \frac{n+1}{n+1} C_n \\
			&\geq C_n	
\end{eqnarray*}

Ainsi :
\conclusion{ $\suiteun{C}$  est croissante.}


\item \begin{eqnarray}
2 C_{2n}-C_n &= 2 \frac{1}{2n }\sum_{k=1}^{2n} u_k - \frac{1}{n }\sum_{k=1}^{n} u_k\\
					&= \frac{1}{n }\sum_{k=1}^{2n} u_k - \frac{1}{n }\sum_{k=1}^{n} u_k\\
				&= \frac{1}{n }\sum_{k=n+1}^{2n} u_k 
\end{eqnarray}
Or par croissance de $\suiteun{u}$, pour tout $k\geq n+1$, $u_k\geq u_{n+1}$ Donc  
$$\sum_{k=n+1}^{2n} u_k  \geq \sum_{k=n+1}^{2n} u_{n+1} = nu_{n+1}$$

Finalement 
\begin{eqnarray*}
2 C_{2n}-C_n &\geq \frac{1}{n }nu_{n+1}\\ 
					&\geq u_{n+1}
\end{eqnarray*}


\item D'aprés $2a)$ $C_n \leq u_n$ et comme $u_n $ est croissante $u_n\leq \ell$. Donc $C_n\leq \ell$. 

D'après 2b) $\suiteun{C} $ est majorée, donc $\suiteun{C}$ converge en vertu du théorème de la limite monotone.  Soit $\ell'$ sa limite. 

D'après 2a) $$\ell' \leq \ell$$

Et d'après 2c) $2 \ell' - \ell' \geq \ell $ d'où $$\ell' \geq \ell$$

Finalement
\conclusion{ $\suiteun{C}$ converge et $\lim_{n\tv +\infty } C_n =\ell$.}




\end{enumerate}
\item \begin{enumerate}
\item On a pour tout $n\geq 1$
\begin{eqnarray*}
\frac{1}{a_{n+1}} -\frac{1}{a_n} &= \frac{1+2a_n}{a_n(1+a_n)} -\frac{1}{a_n}\\
&= \frac{1+2a_n - (1+a_n)}{a_n(1+a_n)} \\
&= \frac{a_n}{a_n(1+a_n)} \\
&= \frac{1}{(1+a_n)} 
\end{eqnarray*}
Ce qui est bien l'égalité demandée. 
\item Pour tout $n\geq 1$:  $u_n =\frac{1}{1+a_n}$, or $\suiteun{a}$ converge et $\lim_{n\tv+\infty} a_n =0$ donc 
\conclusion{ $\lim_{n\tv+\infty} u_n =\frac{1}{1+0}=1$ }

\item $u_{n+1}-u_n= \frac{1}{a_{n+2}}- \frac{1}{a_n}$
Comme $\suiteun{a}$  est décroissante $a_n\geq a_{n+2}$ et donc 
$\frac{1}{a_{n+2}}- \frac{1}{a_n}\geq  0$ 
\conclusion{$\suiteun{u}$ est croissante}


\item $C_n =\frac{1}{n }\sum_{k=1}^n \frac{1}{a_{k+1}} -\frac{1}{a_k} $
On reconnait une somme télescopique : on a donc 
\conclusion{ $C_n = \frac{1}{n}\left(\frac{1}{a_{n+1}} - \frac{1}{a_1}\right)$ }



\item D'après la question précédente : 

$$a_{n+1} = \frac{1}{C_n +\frac{1}{a_1}}= \frac{1}{nC_n +1}$$

D'après la question $2d)$ Comme $\suiteun{u}$ est croissante et converge  vers $1$, $C_n$ converge aussi vers $1$. 
On a donc $$a_n \equivalent{+\infty} \frac{1}{n+1}\equivalent{+\infty} \frac{1}{n}$$

Au final 



\item 
\item 
\item 
\end{enumerate}
\end{enumerate}
\end{correction}