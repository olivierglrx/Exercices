% Titre : suites
% Filiere : BCPST
% Difficulte : 
% Type : TD 
% Categories :suites
% Subcategories : 
% Keywords : suites




\begin{exercice} \;
\'Etudier la suite $\suiteu$ d\'efinie par $u_0\in\lbrack 0,1\rbrack$ et $\forall n \geq 1$, $u_{n+1}=\ddp\demi(u_n^2+1).$
%$\left\lbrace\begin{array}{l}
%u_0\in\lbrack 0,1\rbrack\vsec\\
%u_{n+1}=\ddp\demi(u_n^2+1).
%\end{array}\right.$\\
\noindent Que peut-on dire si on choisit maintenant $u_0\in\lbrack -1,0\rbrack$ ?
\end{exercice}


\%\%\%\%\%\%\%\%\%\%\%\%\%\%\%\%\%\%\%\%
\%\%\%\%\%\%\%\%\%\%\%\%\%\%\%\%\%\%\%\%
\%\%\%\%\%\%\%\%\%\%\%\%\%\%\%\%\%\%\%\%




\begin{correction} \;
\begin{enumerate}
\item \textbf{\'Etudions la suite $\mathbf{\suiteu}$ d\'efinie par $\mathbf{\left\lbrace\begin{array}{l}
u_0\in\lbrack 0,1\rbrack\vsec\\
u_{n+1}=\ddp\demi(u_n^2+1)
\end{array}\right.}$:}
\begin{itemize}
\item[$\bullet$] \textbf{\'Etude des variations de la fonction $\mathbf{f: x\mapsto \ddp\demi(x^2+1)}$ associ\'ee:}
\begin{itemize}
\item[$\star$] La fonction $f$ est d\'efinie sur $\R$ comme fonction polynomiale.
\item[$\star$] la fonction $f$ est d\'erivable sur $\R$ comme fonction polynomiale et pour tout $x\in\R$: $f^{\prime}(x)=x$.
\item[$\star$] On obtient ainsi les variations suivantes:
\begin{center}
\begin{tikzpicture}
 \tkzTabInit{ $x$          /1,%
       $f'(x)$      /1,%
       $f$       /2}%
     { $-\infty$, $0$ ,$+\infty$ }%
  \tkzTabLine{,-,0,+,}%
  \tkzTabVar{
       +/ $+\infty$        /,
        -/$\ddp\demi$           /,%
       +/$+\infty$           /,
                      }
 \tkzTabVal[draw]{2}{3}{0.3}{$1$}{$1$}
  \tkzTabVal[draw]{1}{2}{0.6}{$-1$}{$1$}
\end{tikzpicture}
\end{center}
\item[$\star$] Montrons que l'intervalle $\lbrack 0,1\rbrack$ est stable par $f$. On a:
\begin{itemize}
\item[$\circ$] La fonction $f$ est continue sur $\lbrack 0,1\rbrack$.
\item[$\circ$] La fonction $f$ est strictement croissante sur $\lbrack 0,1\rbrack$.
\item[$\circ$] $f(0)=\ddp\demi$ et $f(1)=1$.
\end{itemize}
Ainsi d'apr\`{e}s le th\'eor\`{e}me de la bijection, on a en particulier que $f(\lbrack 0,1\rbrack)=\left\lbrack \ddp\demi,1\right\rbrack$. Et comme $\left\lbrack \ddp\demi,1\right\rbrack \subset \lbrack 0,1\rbrack$, \fbox{l'intervalle $\lbrack 0,1\rbrack$ est stable par $f$.} 
\end{itemize}
\item[$\bullet$] \textbf{\'Etude du signe de la fonction $\mathbf{g: x\mapsto \ddp\demi(x^2+1)-x}$:}\\
\noindent Pour tout $x\in\R$: $g(x)=\ddp\frac{x^2+1-2x}{2}=\ddp\frac{(x-1)^2}{2}$. Ainsi

 \fbox{la fonction $g$ est positive sur $\R$ et ne s'annule qu'en 1.}
\item[$\bullet$]  \textbf{Calcul des limites \'eventuelles:}\\
\noindent On suppose que la suite $\suiteu$ converge vers un r\'eel $l\in\mathcal{D}_f=\R$.
\begin{itemize}
\item[$\star$] On a donc:
\begin{itemize}
\item[$\circ$] La suite converge vers $l$.
\item[$\circ$] La fonction $f$ est continue sur $\R$ comme fonction polynomiale donc elle est en particulier continue en $l$.
\end{itemize}
Donc d'apr\`{e}s le th\'eor\`{e}me sur les suite et fonction, on obtient que: $\lim\limits_{n\to +\infty} f(u_n)=f(l)$.
\item[$\star$] De plus on a: $\lim\limits_{n\to +\infty} u_{n+1}=l$.
\item[$\star$] On peut donc passer \`{a} la limite dans l'\'egalit\'e: $u_{n+1}=f(u_n)$ et on obtient que: $l=f(l)$/
\item[$\star$] On a donc: $l=f(l)\Leftrightarrow g(l)=0\Leftrightarrow l=1$. \\
\noindent \fbox{La seule limite \'eventuelle est 1.}
\end{itemize}
\item[$\bullet$]  \textbf{Montrons que la suite est bien d\'efinie et que pour tout $n\in\N$, $u_n\in \lbrack 0,1\rbrack$:}
\begin{itemize}
\item[$\star$] On montre par r\'ecurrence sur $n\in\N$ la propri\'et\'e
$$\mathcal{P}(n):\ u_n\ \hbox{existe et}\ u_n\in\lbrack 0,1\rbrack.$$
\item[$\star$] Initialisation: pour $n=0$:\\
\noindent Par d\'efinition de la suite, on a bien que $u_0$ existe et $u_0\in\lbrack 0,1\rbrack$. Donc $\mathcal{P}(0)$ est vraie.
\item[$\star$] H\'er\'edit\'e: soit $n\in\N$ fix\'e, on suppose que la propri\'et\'e vraie \`{a} l'ordre $n$, montrons que $\mathcal{P}(n+1)$ est vraie.
\begin{itemize}
\item[$\circ$] Par hypoth\`{e}se de r\'ecurrence, on sait que $u_n$ existe et que $u_n\in\lbrack 0,1\rbrack$. En particulier $u_n$ existe et $u_n\in\mathcal{D}_f$. Donc $f(u_n)$ existe c'est-\`{a}-dire $u_{n+1}$ existe.
\item[$\circ$] Par hypoth\`{e}se de r\'ecurrence, on sait que $u_n$ existe et que $u_n\in\lbrack 0,1\rbrack$. Or l'intervalle $\lbrack 0,1\rbrack$ est stable par $f$. Donc $f(u_n)\in\lbrack 0,1\rbrack$ c'est-\`{a}-dire $u_{n+1}\in\lbrack 0,1\rbrack$.
\end{itemize}
Donc $\mathcal{P}(n+1)$ est vraie.
\item[$\star$] Conclusion: il r\'esulte du principe de r\'ecurrence que

 \fbox{la suite $\suiteu$ est bien d\'efinie et que pour tout $n\in\N,\ u_n\in\lbrack 0,1\rbrack.$}
\end{itemize}
\item[$\bullet$]  \textbf{\'Etude de la monotonie de la suite:}\\
\noindent Soit $n\in\N$, on a: $u_{n+1}-u_n=f(u_n)-u_n=g(u_n)$. Ainsi comme le signe de $g$ est positif sur $\R$, on obtient que pour tout $n\in\N$: $u_{n+1}-u_n\geq 0$. Ainsi \fbox{la suite $\suiteu$ est croissante.}
\item[$\bullet$]  \textbf{\'Etude de la convergence de la suite:}
\begin{itemize}
\item[$\star$] La suite $\suiteu$ est croissante et major\'ee par 1 donc d'apr\`{e}s le th\'eor\`{e}me sur les suites monotones, elle converge.
\item[$\star$] Comme la seule limite \'eventuelle est 1, \fbox{la suite $\suiteu$ converge vers 1.}
\end{itemize}
\end{itemize}
\item \textbf{Que peut-on dire si on choisit maintenant $\mathbf{u_0\in\lbrack -1,0\rbrack}$ ?}
\begin{itemize}
\item[$\bullet$] Montrons que $u_1\in\lbrack 0,1\rbrack$:\\
\noindent On a:
\begin{itemize}
\item[$\star$] La fonction $f$ est continue sur $\lbrack -1,0\rbrack$ comme fonction polynomiale.
\item[$\star$] La fonction $f$ est strictement d\'ecroissante sur $\lbrack -1,0\rbrack$.
\item[$\star$] $f(-1)=1$ et $f(0)=\ddp\demi$
\end{itemize}
Ainsi d'apr\`{e}s le th\'eor\`{e}me de la bijection, on a en particulier que $f(\lbrack -1,0\rbrack)=\left\lbrack 0,\ddp\demi\right\rbrack$. Or on a suppos\'e que $u_0\in\lbrack -1,0\rbrack$ donc $f(u_0)\in \left\lbrack 0,\ddp\demi\right\rbrack$, \`{a} savoir $u_1\in \left\lbrack 0,\ddp\demi\right\rbrack$. Comme $\left\lbrack 0,\ddp\demi\right\rbrack\subset \lbrack 0,1\rbrack$, on a en particulier que: \fbox{$u_1\in\lbrack 0,1\rbrack.$}
\item[$\bullet$] La suite $(u_n)_{n\geq 1}$ a donc un terme initial $u_1\in\lbrack 0,1\rbrack$. Ainsi la suite $(u_n)_{n\geq 1}$ se comporte comme la suite de la question pr\'ec\'edente et en particulier elle converge vers 1. Mais le comportement \`{a} l'infini d'une suite ne d\'epend pas de ses premiers termes donc \fbox{la suite $\suiteu$ converge aussi vers 1.}
\end{itemize}
\end{enumerate}
\end{correction}