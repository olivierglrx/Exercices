% Titre : nombre
% Filiere : BCPST
% Difficulte : 
% Type : TD 
% Categories :nombre
% Subcategories : 
% Keywords : nombre




\begin{exercice} 
Montrer que pour tout $x\in \R_+$:
$$\floor{x} =\floor{\frac{x}{2} } +\floor{\frac{x+1}{2} }.$$

\end{exercice}


\%\%\%\%\%\%\%\%\%\%\%\%\%\%\%\%\%\%\%\%
\%\%\%\%\%\%\%\%\%\%\%\%\%\%\%\%\%\%\%\%
\%\%\%\%\%\%\%\%\%\%\%\%\%\%\%\%\%\%\%\%




\begin{correction}
Distinguons les cas selon la parité de $\floor{x}$. 
\paragraph{Cas 1 : $\floor{x}$ est paire}
Dans ce cas, il existe $k\in \N$ tel que $\floor{x} \in [2k,2k+1[$, où $\floor{x}=2k$. On a alors 
$\frac{x}{2}\in [k,k+\frac{1}{2}[$ donc $\floor{\frac{x}{2}} =k$
et $\frac{x+1}{2 }\in [k+\frac{1}{2}, k+1[$, donc de nouveau $\floor{\frac{x+1}{2 }}=k$

On a bien l'égalité demandée. 

\paragraph{Cas 2 : $\floor{x}$ est impaire}
Dans ce cas, il existe $k\in \N$ tel que $\floor{x} \in [2k+1,2k+2[$, où $\floor{x}=2k+1$.  On a alors 
$\frac{x}{2}\in [k+\frac{1}{2},k+1[$ donc $\floor{\frac{x}{2}} =k$
et $\frac{x+1}{2 }\in [k+1, k+\frac{3}{2}$, donc cette fois $\floor{\frac{x+1}{2 }}=k+1$

On a bien l'égalité demandée. 


\end{correction}