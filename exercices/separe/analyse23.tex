% Titre : Wallis  - Calcul $\lim_{n\tv \infty} \sum_{k=1}^n \frac{1}{k^2}$ (Pb)
% Filiere : BCPST
% Difficulte :
% Type : DS, DM
% Categories : analyse
% Subcategories : 
% Keywords : analyse





Le but de ce DM est de calculer la valeur de 
$$\lim_{n\tv \infty} \sum_{k=1}^n \frac{1}{k^2}$$

\subsubsection{Convergence}
On note $S_n=  \sum_{k=1}^n \frac{1}{k^2}$.

\begin{enumerate}
\item Montrer que la suite $\suite{S}$  est monotone. 
\item Montrer que pour tout $k\geq 2$
$$\frac{1}{k^2} \leq \frac{1}{k-1}-\frac{1}{k}$$
\item En déduire que la suite $\suite{S}$ converge vers une limite $\ell \in [0,2]$. 
\end{enumerate}

\begin{correction}
\begin{enumerate}
\item Pour tout $n\in \N$ on a  $S_{n+1} = S_n + \frac{1}{(n+1)^2}$, donc 
$S_{n+1}-S_n = \frac{1}{(n+1)^2}\geq 0$. Ainsi, $\suite{S}$ est croissante. 
\item Pour tout $k\geq 2$ : $\frac{1}{k-1}-\frac{1}{k}= \frac{k-(k-1)}{k(k-1)}= \frac{1}{k(k-1)}$
Or pour tout $k\geq 2$,  $0<k(k-1) \leq k^2$. Comme la fonction inverse est décroissante sur $\R_+^*$ on a donc : 
$$\frac{1}{k-1}-\frac{1}{k}=\frac{1}{k(k-1)}\geq \frac{1}{k^2}$$

\item  D'après la question précédente, on peut majorer tous les termes de $S_n$ à partir du rang $k=2$. On  a alors : 
$$S_n = 1+ \sum_{k=2}^n\frac{1}{k^2} \leq 1 + \sum_{k=2}^n\frac{1}{k-1} -\frac{1}{k}$$
On reconnait alors dans le  membre de droite une somme téléscopique qui se simplifie de la manière suivante : 
$$\sum_{k=2}^n\frac{1}{k-1} -\frac{1}{k} = \frac{1}{2-1} -\frac{1}{n} = 1-\frac{1}{n}$$
On obtient alors $S_n  \leq 2-\frac{1}{n}$. 
La suite $\suite{S}$ est croissante et majorée, d'après le théorème des limlites monotones la suite $\suite{S}$ converge, notons $\ell$ sa limite. 

Comme $0\leq S_n\leq 2-\frac{1}{n}$ et $\lim_{n\tv \infty} 2-\frac{1}{n}=2$, le théorème d'encadrement assure que $\ell\in [0,2]$. 
 


\end{enumerate}
\end{correction}