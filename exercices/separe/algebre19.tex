% Titre : Matrice, endomorphisme, vp, (Ecricome 2002)
% Filiere : BCPST
% Difficulte :
% Type : DS, DM
% Categories : algebre
% Subcategories : 
% Keywords : algebre



\begin{exercice}

Dans l'ensemble $\mathcal{M}_{3}(\mathbb{R})$ des matrices carr\'{e}es
d'ordre $3$ \`{a} coefficients r\'{e}els, on consid\`{e}re le sous-ensemble $%
E$ des matrices $M(a,b)$ d\'{e}finies par~: 
\begin{equation*}
M(a,b)=\left( 
\begin{array}{ccc}
b & a & b \\ 
a & b & b \\ 
b & b & a%
\end{array}
\right) .
\end{equation*}
Ainsi~: 
\begin{equation*}
E=\left\{ M(a,b)\quad a,b\in \mathbb{R}\right\} .
\end{equation*}
On note $f_{a,b}$ l'endomorphisme de $\mathbb{R}^{3}$ repr\'{e}sent\'{e} par
la matrice $M(a,b)$ dans la base canonique $\mathcal{B}=(e_{1},e_{2},e_{3})$
de $\mathbb{R}^{3}$.

\begin{enumerate}


\item Structure de $E$


\begin{enumerate}
\item Montrer que $E$ est un sous-espace vectoriel de $\mathcal{M}_{3}(%
\mathbb{R})$.

\item Donner une base de $E$, ainsi que sa dimension.
\end{enumerate}







\item \'Etude d'un cas particulier.

On pose $A=M(1,0)$.

\begin{enumerate}
\item Calculer $A^{2}$. En d\'{e}duire que $A$ est une matrice inversible et
exprimer $A^{-1}$ en fonction de $A$.

\item D\'{e}terminer les valeurs propres de $A$.

\item Trouver une base de $\mathbb{R}^{3}$ dans laquelle la matrice de $%
f_{1,0}$ est~: 
\begin{equation*}
\left( 
\begin{array}{ccc}
1 & 0 & 0 \\ 
0 & 1 & 0 \\ 
0 & 0 & -1%
\end{array}
\right) .
\end{equation*}
\end{enumerate}

\item Diagonalisation des \'{e}l\'{e}ments de $E$ et application.

On consid\`ere les vecteurs de $\mathbb{R}^3$ suivants~: 
\begin{equation*}
\vec{u}=(1,1,1), \quad \vec{v}=(1,-1,0), \quad \vec{w}=(1,1,-2).
\end{equation*}

\begin{enumerate}
\item Justifier que les matrices de l'ensemble $E$ sont diagonalisables.

\item Montrer que $\mathcal{C}=\left( \vec{u},\vec{v},\vec{w}\right) $ est
une base de $\mathbb{R}^{3}$.

\item On note $P$ la matrice de passage de la base $\mathcal{B}$ \`{a} la
base $\mathcal{C}$. \'{E}crire $P$.

\item D\'{e}terminer $P^{-1}$.

\item Exprimer les vecteurs $f_{a,b}\left( \vec{u}\right) $, $f_{a,b}\left( 
\vec{v}\right) $, $f_{a,b}\left( \vec{w}\right) $ en fonction de $\vec{u}$, $%
\vec{v}$, $\vec{w}$.

\item En d\'{e}duire l'expression de la matrice $D_{a,b}$ de $f_{a,b}$ dans
la base $\mathcal{C}$.

\item Justifier l'\'{e}galit\'{e}~: 
\begin{equation*}
P^{-1}M_{a,b}P=D_{a,b}.
\end{equation*}

\item Donner une condition n\'{e}cessaire et suffisante portant sur $a$ et $%
b $ pour que $D_{a,b}$ soit inversible.

\item Cette condition \'{e}tant r\'{e}alis\'{e}e, d\'{e}terminer la matrice
inverse de $D_{a,b}$.

\item Donner une condition n\'{e}cessaire et suffisante portant sur $a$ et $%
b $ pour que $M_{a,b}$ soit inversible.
\end{enumerate}

\end{enumerate}

\end{exercice}





\begin{correction}

Dans l'ensemble $\mathcal{M}_{3}(\mathbb{R})$ des matrices carr\'{e}es
d'ordre $3$ \`{a} coefficients r\'{e}els, on consid\`{e}re le sous-ensemble $%
E$ des matrices $M(a,b)$ d\'{e}finies par~: 
\begin{equation*}
M(a,b)=\left( 
\begin{array}{rrr}
b & a & b \\ 
a & b & b \\ 
b & b & a%
\end{array}
\right)
\end{equation*}
Ainsi~: 
\begin{equation*}
E=\left\{ M(a,b)\quad a,b\in \mathbb{R}\right\} .
\end{equation*}
On note $f_{a,b}$ l'endomorphisme de $\mathbb{R}^{3}$ repr\'{e}sent\'{e} par
la matrice $M(a,b)$ dans la base canonique $\mathcal{B}=(e_{1},e_{2},e_{3})$
de $\mathbb{R}^{3}$.

\begin{enumerate}
\item Structure de $E$. 
\begin{enumerate}
\item On a $E=\left\{ \left( 
\begin{array}{rrr}
b & a & b \\ 
a & b & b \\ 
b & b & a%
\end{array}
\right) /a,b\in \mathbb{R}\right\} =\left\{ a\left( 
\begin{array}{rrr}
0 & 1 & 0 \\ 
1 & 0 & 0 \\ 
0 & 0 & 1%
\end{array}
\right) +b\left( 
\begin{array}{rrr}
1 & 0 & 1 \\ 
0 & 1 & 1 \\ 
1 & 1 & 0%
\end{array}
\right) /a,b\in \mathbb{R}\right\} $

on reconna\^{\i}t $\left( 
\begin{array}{rrr}
0 & 1 & 0 \\ 
1 & 0 & 0 \\ 
0 & 0 & 1%
\end{array}%
\right) =M\left( 1,0\right) $ et $\left( 
\begin{array}{rrr}
1 & 0 & 1 \\ 
0 & 1 & 1 \\ 
1 & 1 & 0%
\end{array}%
\right) =M\left( 0,1\right) $

Donc $E=Vect\left( M\left( 1,0\right) ,M\left( 0,1\right) \right) $ sous
espace de $\mathcal{M}_{3}(\mathbb{R})$.

\item On a du m\^{e}me coup pour famille g\'{e}n\'{e}ratrice : $\left(
M\left( 1,0\right) ,M\left( 0,1\right) \right) $

Pour montrer qu'elle est libre on montre que \textbf{si} une combinaison lin%
\'{e}aire est nulle \textbf{alors }les coefficients sont nuls :

Soient $a$ et $b$ deux r\'{e}els. Si $aM\left( 1,0\right) +bM\left(
0,1\right) =0$ alors $\left( 
\begin{array}{rrr}
b & a & b \\ 
a & b & b \\ 
b & b & a%
\end{array}
\right) =\left( 
\begin{array}{rrr}
0 & 0 & 0 \\ 
0 & 0 & 0 \\ 
0 & 0 & 0%
\end{array}
\right) $,

donc $a=0$ et $b=0.$

Cette famille est donc libre et g\'{e}n\'{e}ratrice. C'est donc une base de $%
E$ qui est donc de dimension 2.
\end{enumerate}

\item \'Etude d'un cas particulier.\\

On pose $A=M(1,0)$.

\begin{enumerate}
\item $A^{2}=\left( 
\begin{array}{rrr}
0 & 1 & 0 \\ 
1 & 0 & 0 \\ 
0 & 0 & 1%
\end{array}
\right) \left( 
\begin{array}{rrr}
0 & 1 & 0 \\ 
1 & 0 & 0 \\ 
0 & 0 & 1%
\end{array}
\right) =\allowbreak \left( 
\begin{array}{ccc}
1 & 0 & 0 \\ 
0 & 1 & 0 \\ 
0 & 0 & 1%
\end{array}
\right) =I$

Donc $A\cdot A=I$ (et $A\cdot A=I$) $A$ est inversible et son inverse est $%
A^{-1}=A$

\item On peut utiliser la relation pr\'{e}c\'{e}dente pour en d\'{e}duire
les seules valeurs propres \textbf{possibles} de $A$ :

\begin{itemize}
\item \textbf{Si }$\alpha $ est une valeur propre de $A$ et $X$ une colonne
propre (non nulle) alors $A^{2}X=A\left( AX\right) =\alpha AX=\alpha ^{2}X$
et comme $A^{2}=I,$ on a aussi $A^{2}X=X$ donc $\alpha ^{2}X=X$ et comme $%
X\neq 0$ on a $\alpha ^{2}=1.$ \textbf{Alors }$\alpha =1$ ou $\alpha =-1$

\item \textbf{Est-ce que }1 est valeur propre de $A?$ (on aura besoi du sous
espace propre ensuite). Avec $X=\left( 
\begin{array}{c}
x \\ 
y \\ 
z%
\end{array}%
\right) $

$\left( A-I\right) \cdot X=0\Longleftrightarrow \left\{ 
\begin{array}{c}
-x+y=0 \\ 
x-y=0 \\ 
0=0%
\end{array}%
\right. \Leftrightarrow x=y$ donc $1$ est bien valeur propre et a pour sous
espace propre associ\'{e} : \newline
$\mathcal{S}_{1}=\left\{ \left( y,y,z\right) \ /\ y,z\in \mathbb{R}\right\} =%
\mathrm{Vect}\left( \left( 1,1,0\right) ,\left( 0,0,1\right) \right) $

Ces \textbf{deux }vecteurs sont non colin\'{e}aires. Ils forment donc une
famille libre.

C'est donc une famille libre et g\'{e}n\'{e}ratrice i.e. une base du sous
espace propre.

\item De la m\^{e}me fa\c{c}on \textbf{est-ce que }-1 est valeur propre de $%
A?$

$A\cdot X=-1\cdot X$ $\Leftrightarrow \left\{ 
\begin{array}{c}
y=-x \\ 
x=-y \\ 
z=-z%
\end{array}%
\right. \Leftrightarrow \left\{ 
\begin{array}{c}
y=-x \\ 
z=0%
\end{array}%
\right. $

L'ensemble des solutions est $\mathcal{S}_{-1}=\left\{ \left( x,-x,0\right)
\ /\ x\in \mathbb{R}\right\} =\mathrm{Vect}\left( \left( 1,-1,0\right)
\right) $

Donc $-1$ est bien valeur propre et le sous espace propre associ\'{e} \`{a}
-1 est engendr\'{e} par $\left( \left( 1,-1,0\right) \right) $ qui en est
donc une base .

\item Ce sont donc les deux seules valeurs propres de $A$
\end{itemize}

\item Soient $\vec{i}=\left( 1,1,0\right) ,\quad \vec{j}=\left( 0,0,1\right) 
$ et $\vec{k}=\left( 1,-1,0\right) .$

Ils ont pour coordonn\'{e}es dans la base canonique $\left( 1,1,0\right)
,\quad \left( 0,0,1\right) $ et $\left( 1,-1,0\right) $

Comme $A$ est la matrice de $f_{1,0}$ dans la base $\mathcal{B}$ les
vecteurs $\left( \vec{i},\vec{j}\right) $ forment une base du sous espace
propre de $f_{1,0}$ associ\'{e}s \`{a} la valeur propre 1 et le vecteur $%
\left( \vec{k}\right) $ est une base du sous espace propre associ\'{e} \`{a}
la valeur propre $-1$.

Comme la somme des dimensions des sous espaces propres est \'{e}gal \`{a} 3
(dimension de $\mathbb{R}^{3}$ ) la concat\'{e}nation $\left( \vec{i},\vec{j}%
,\vec{k}\right) $ de ces bases forme une base de $\mathbb{R}^{3}$.

Et la matrice de $f_{1,0}$ dans cette base est $\left( 
\begin{array}{rrr}
1 & 0 & 0 \\ 
0 & 1 & 0 \\ 
0 & 0 & -1%
\end{array}%
\right) $
\end{enumerate}



\item Diagonalisation des \'el\'ements de $E$ et application.

On consid\`ere les vecteurs de $\mathbb{R}^3$ suivants~: 
\begin{equation*}
\vec{u}=(1,1,1), \quad \vec{v}=(1,-1,0), \quad \vec{w}=(1,1,-2).
\end{equation*}


\begin{enumerate}
\item Les matrices de $E$ sont sym\'{e}triques donc diagonalisables.

\item Comme $\mathcal{C}=\left( \vec{u},\vec{v},\vec{w}\right) $ a 3 \'{e}l%
\'{e}ments, il suffit de montrer que la famille est libre pour montrer
qu'elle est une base :

\textbf{Si }$x\vec{u}+y\vec{v}+z\vec{w}=\vec{0}$ \textbf{alors} $\dots
x=y=z=0$

\textbf{Mais comme on demande ensuite la matrice inverse} de la matrice de
passage, il est plus \'{e}conome de tout faire en m\^{e}me temps.

$\left( \vec{u},\vec{v},\vec{w}\right) $ est une base si et seulement si la
matrice de leurs coordonn\'{e}es (en colonne) est inversible.

Soit $P=\left( 
\begin{array}{rrr}
1 & 1 & 1 \\ 
1 & -1 & 1 \\ 
1 & 0 & -2%
\end{array}
\right) $ cette matrice. On applique la m\'{e}thode de Gauss

$\left( \left. 
\begin{array}{rrr}
1 & 1 & 1 \\ 
1 & -1 & 1 \\ 
1 & 0 & -2%
\end{array}
\right| 
\begin{array}{rrr}
1 & 0 & 0 \\ 
0 & 1 & 0 \\ 
0 & 0 & 1%
\end{array}
\right) 
\begin{array}{r}
L_{1} \\ 
L_{2}-L_{1} \\ 
L_{3}-L_{1}%
\end{array}
\Leftrightarrow \left( \left. 
\begin{array}{rrr}
1 & 1 & 1 \\ 
0 & -2 & 0 \\ 
0 & -1 & -3%
\end{array}
\right| 
\begin{array}{rrr}
1 & 0 & 0 \\ 
-1 & 1 & 0 \\ 
-1 & 0 & 1%
\end{array}
\right) 
\begin{array}{c}
L_{1}+L_{2}/2 \\ 
-L_{2}/2 \\ 
L_{3}-L_{2}/2%
\end{array}
$

$\Leftrightarrow \left( \left. 
\begin{array}{rrr}
1 & 0 & 1 \\ 
0 & 1 & 0 \\ 
0 & 0 & -3%
\end{array}
\right| 
\begin{array}{rrr}
1/2 & 1/2 & 0 \\ 
1/2 & -1/2 & 0 \\ 
-1/2 & -1/2 & 1%
\end{array}
\right) 
\begin{array}{c}
L_{1}+L_{3}/3 \\ 
L_{2} \\ 
-L_{3}/3%
\end{array}
\Leftrightarrow \left( \left. 
\begin{array}{rrr}
1 & 0 & 0 \\ 
0 & 1 & 0 \\ 
0 & 0 & 1%
\end{array}
\right| 
\begin{array}{rrr}
1/3 & 1/3 & 1/3 \\ 
1/2 & -1/2 & 0 \\ 
1/6 & 1/6 & -1/3%
\end{array}
\right) 
\begin{array}{c}
L_{1}+L_{3}/3 \\ 
L_{2} \\ 
-L_{3}/3%
\end{array}
$

Donc $P$ est inversible et $P^{-1}=\left( 
\begin{array}{rrr}
1/3 & 1/3 & 1/3 \\ 
1/2 & -1/2 & 0 \\ 
1/6 & 1/6 & -1/3%
\end{array}
\right) $

\textbf{Un raccourci} de la r\'{e}daction (le calcul d'inverse est \`{a}
faire au brouillon) est de constater (sans expliquer d'o\`{u} vient cette
matrice) que :

$\left( 
\begin{array}{rrr}
1 & 1 & 1 \\ 
1 & -1 & 1 \\ 
1 & 0 & -2%
\end{array}
\right) \left( 
\begin{array}{rrr}
1/3 & 1/3 & 1/3 \\ 
1/2 & -1/2 & 0 \\ 
1/6 & 1/6 & -1/3%
\end{array}
\right) =\allowbreak \left( 
\begin{array}{ccc}
1 & 0 & 0 \\ 
0 & 1 & 0 \\ 
0 & 0 & 1%
\end{array}
\right) $

Finalement $\left( \vec{u},\vec{v},\vec{w}\right) $ est une base de $\mathbb{%
R}^{3}$

\item La matrice de passage de la base $\mathcal{B}$ \`{a} la base $\mathcal{%
C}$ est form\'{e}e par les coordonn\'{e}es (en colonne) des vecteurs de $%
\mathcal{C}$ dans la base $\mathcal{B}.$ C'est donc la matrice $P$ \'{e}%
crite ci-dessus.

\item $P^{-1}=\left( 
\begin{array}{rrr}
1/3 & 1/3 & 1/3 \\ 
1/2 & -1/2 & 0 \\ 
1/6 & 1/6 & -1/3%
\end{array}
\right) $

\item On pourrait utiliser la formule de changement de base pour obtenir les
coordonn\'{e}es des images de $\vec{u}$, $\vec{v}$, $\vec{w}$ dans la base $%
\mathcal{C}.$

Mais comme ces vecteurs ne nous ont pas \'{e}t\'{e}s donn\'{e}s au hasard,
on peut tenter directement \`{a} partir de leurs coordonn\'{e}es dans la
base canonique et de la matrice de $f_{a,b}$ dans cette m\^{e}me base.

$\vec{u}$ a pour coordonn\'{e}es $\left( 1,1,1\right) $ donc $f_{a,b}\left( 
\vec{u}\right) $ a pour coordonn\'{e}es dans $\mathcal{B}$ : \newline
$\left( 
\begin{array}{rrr}
b & a & b \\ 
a & b & b \\ 
b & b & a%
\end{array}%
\right) \left( 
\begin{array}{c}
1 \\ 
1 \\ 
1%
\end{array}%
\right) =\allowbreak \left( 
\begin{array}{c}
2b+a \\ 
2b+a \\ 
2b+a%
\end{array}%
\right) =\left( 2b+a\right) \left( 
\begin{array}{c}
1 \\ 
1 \\ 
1%
\end{array}%
\right) $

Donc $f_{a,b}\left( \vec{u}\right) =\left( 2b+a\right) \vec{u}$ et de m\^{e}%
me

$f_{a,b}\left( \vec{v}\right) $ a pour coordonn\'{e}es : $\left( 
\begin{array}{rrr}
b & a & b \\ 
a & b & b \\ 
b & b & a%
\end{array}%
\right) \left( 
\begin{array}{r}
1 \\ 
-1 \\ 
0%
\end{array}%
\right) =\left( 
\begin{array}{c}
b-a \\ 
a-b \\ 
0%
\end{array}%
\right) =\left( b-a\right) \left( 
\begin{array}{r}
1 \\ 
-1 \\ 
0%
\end{array}%
\right) $

donc $f_{a,b}\left( \vec{v}\right) =$ $\left( b-a\right) \vec{v}$

Et enfin $\left( 
\begin{array}{rrr}
b & a & b \\ 
a & b & b \\ 
b & b & a%
\end{array}%
\right) \left( 
\begin{array}{r}
1 \\ 
1 \\ 
-2%
\end{array}%
\right) =\left( 
\begin{array}{c}
a-b \\ 
a-b \\ 
2b-2a%
\end{array}%
\right) =\left( a-b\right) \left( 
\begin{array}{r}
1 \\ 
1 \\ 
-2%
\end{array}%
\right) $ donc $f_{a,b}\left( \vec{w}\right) =$ $\left( a-b\right) \vec{w}$

\item La matrice de $f_{a,b}$ dans la base $\mathcal{C}$. est donc : $%
D_{a,b}=\left( 
\begin{array}{ccc}
2b+a & 0 & 0 \\ 
0 & b-a & 0 \\ 
0 & 0 & a-b%
\end{array}%
\right) $

\item $P$ est la matrice de passage de $\mathcal{B}$ dans $\mathcal{C}.$ Son
inverse $P^{-1}$ est donc la matrice de passage de $\mathcal{C}$ dans $%
\mathcal{B}.$ Donc la matrice de $f_{a,b}$ dans la base $\mathcal{C}$
s'obtient \`{a} partir de sa matrice $M_{a,b}$ dans la base $\mathcal{B}$
par : 
\begin{equation*}
P^{-1}M_{a,b}P=D_{a,b}.
\end{equation*}

\item $D_{a,b}$ \'{e}tant une matrice diagonale, elle est inversible si et
seulement si les coefficients de la diagonale sont tous non nuls.

Donc si et seulement si $a\neq -2b$ et $a\neq b$

\item On a alors comme $D_{a,b}$ est diagonale, son inverse en inversant les
coefficients de la diagonale :

\begin{equation*}
D_{a,b}^{-1}=\left( 
\begin{array}{ccc}
1/\left( 2b+a\right) & 0 & 0 \\ 
0 & 1/\left( b-a\right) & 0 \\ 
0 & 0 & 1/\left( a-b\right)%
\end{array}%
\right)
\end{equation*}

\item Les deux matrices $M_{a,b}$ et $D_{a,b}\acute{e}\tan t$ semblables,
elles sont simultan\'{e}ment inversible. Donc $M_{a,b}$ est inversible si et
seulement si $a\neq -2b$ et $a\neq b.$
\end{enumerate}


\end{enumerate}

\end{correction}