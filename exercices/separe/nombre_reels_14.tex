% Titre : nombre
% Filiere : BCPST
% Difficulte : 
% Type : TD 
% Categories :nombre
% Subcategories : 
% Keywords : nombre




\begin{exercice}
Chercher la borne inférieure, supérieure et dire si ce sont des maximum, minimum pour les ensembles suivants : 

\begin{enumerate}
\begin{minipage}[t]{0.45\textwidth}
\item $E_1= \{ x- \floor{x}\, |\, x\in \R\}$
\item $E_2= \{ -x^2+3x -1\, |\, x\in [-1,1]\}$

\end{minipage}
\begin{minipage}[t]{0.45\textwidth}
\item $E_3= \{\exp(-x^2) |\, x\in \R \}$
\item $E_4= \{\frac{1}{n!+1}|\, n\in \N^* \}$
\end{minipage}
\end{enumerate}

\end{exercice}


\%\%\%\%\%\%\%\%\%\%\%\%\%\%\%\%\%\%\%\%
\%\%\%\%\%\%\%\%\%\%\%\%\%\%\%\%\%\%\%\%
\%\%\%\%\%\%\%\%\%\%\%\%\%\%\%\%\%\%\%\%




\begin{correction}
Voir en ligne pour les détails. 

On trouve 
\begin{enumerate}
	\item $\sup(E_1) =1$ ce n'est pas un maximum. $\inf(E_1)= 0$ c'est un mininum.
	\item $\sup(E_2)= 1$ c'est un maximum. $\inf(E_2)=-5$ c'est un minimum.
	\item $\sup(E_3) =1$ c'est un maximum. $\inf(E_3) = 0$ ce n'est pas un minimum. 
	\item $\sup(E_4)=1/2$ c'est un maximum. $\inf(E_4)=0$ ce n'est pas un minimum. 
\end{enumerate}
\end{correction}