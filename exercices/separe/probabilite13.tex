% Titre : Chaine de Markov - Hamster
% Filiere : BCPST
% Difficulte :
% Type : DS, DM
% Categories : probabilite
% Subcategories : 
% Keywords : probabilite






\begin{exercice}
Roudoudou le hamster vit une vie paisible de hamster. Il a deux activités : manger et  dormir... 
On va voir Roudoudou à 00h00 ($n=0$). Il est en train de dormir. 
\begin{itemize}
\item Quand Roudoudou dort à l'heure $n$, il y a 7 chances sur 10 qu'il dorme à l'heure suivante et 3 chances sur 10 qu'il mange à l'heure suivante. 
\item Quand Roudoudou mange à l'heure $n$, il y a 2 chances sur 10 qu'il dorme à l'heure suivante et 8 chances sur 10 qu'il mange à l'heure suivante. 
\end{itemize}


On note $D_n$ l'événement 'Roudoudou dort à l'heure $n$' et $M_n$ 'Roudoudou mange à l'heure $n$'. On note $d_n =P(D_n)$ et $m_n=P(M_n)$ les probabilités respectives. 


\begin{enumerate}
\item Justifier que $d_n+m_n=1$. 
\item Montrer rigoureusement que $$d_{n+1} =  0,7d_n+0,2m_n$$
\item Exprimer de manière similaire $m_{n+1} $ en fonction de $d_n$ et $m_n$. 

\item Soit $A$ la matrice $$A=\frac{1}{10}\left(\begin{array}{ccc}
7 & 2\\
3 & 8
\end{array}
\right).$$
Résoudre en fonction de $\lambda \in \R$ l'équation $AX = \lambda X$ d'inconnue $\ddp X =\left(\begin{array}{c}
x \\
y 
\end{array}
\right)$. 
\item Soit $P = 
\left(\begin{array}{cc}
1 & 2\\	
-1 & 3
\end{array}
\right)$ Montrer que $P$ est inversible et calculer $P^{-1}$. 
\item Montrer que $P^{-1} A P =\frac{1}{5} \left(\begin{array}{cc}
 \frac{1}{2}& 0\\
0 &  1 
\end{array}
\right)$
\item Calculer $D^n$ où $D=\left(\begin{array}{cc}
 \frac{1}{2}& 0\\
0 &  1 
\end{array}
\right)$

\item En déduire que pour tout  $n\in \N$, $\ddp A^n=\left(\begin{array}{ccc}
3\left( 1/2\right)^n +2 & -2\left( 1/2\right)^n +2\\
-3\left( 1/2\right)^n +3& 2\left( 1/2\right)^n +3
\end{array}
\right)$.
\item En déduire la valeur de $d_n$ en fonction de $n$. 
\end{enumerate}
\end{exercice}



\begin{correction}
\begin{enumerate}
\item $D_n$ et $M_n$ forment un système complet d'événements donc $
d_n+m_n=1$. 
\item On cherche à calculer $d_{n+1} =P(D_{n+1})$ 
On applique la formule des probabilités totales avec le SCE $(M_N,D_N)$
\begin{align*}
d_{n+1} &= P(D_{n+1}\, |\, M_n) P(M_n) +P(D_{n+1}\, |\, D_n) P(D_n)\\
			&= P(D_{n+1}\, |\, M_n) m_n +P(D_{n+1}\, |\, D_n) d_n
\end{align*}
L'énoncé donne : $ P(D_{n+1}\, |\, M_n) = \frac{2}{10}$ et  $ P(D_{n+1}\, |\, D_n) = \frac{7}{10}$
et donc 
$$d_{n+1} = 0,7 d_n  +0,2 m_n$$

\item On cherche à calculer $m_{n+1} =P(M_{n+1})$ 
On applique la formule des probabilités totales avec le SCE $(M_N,D_N)$
\begin{align*}
m_{n+1} &= P(M_{n+1}\, |\, M_n) P(M_n) +P(M_{n+1}\, |\, D_n) P(D_n)\\
			&= P(M_{n+1}\, |\, M_n) m_n +P(M_{n+1}\, |\, D_n) d_n
\end{align*}
L'énoncé donne : $ P(M_{n+1}\, |\, M_n) = \frac{8}{10}$ et  $ P(M_{n+1}\, |\, D_n) = \frac{3}{10}$
et donc 
$$m_{n+1} = 0,3 d_n  +0,8 m_n$$

\item 
On obtient le système d'équations
$$\left\{  
\begin{array}{cc}
7x +2y  &=10\lambda x\\
3x +8y  &=10\lambda y
\end{array}\right.$$



$
\equivaut
\left\{  
\begin{array}{cc}
(7-10\lambda) x +2y  &=0\\
3x +(8-10\lambda)y  &=0
\end{array}\right.
\equivaut 
\left\{  
\begin{array}{cc}
3x +(8-10\lambda)y  &=0\\
(7-10\lambda) x +2y  &=0
\end{array}\right.$

$L_2 \leftarrow3*L_2- (7-10\lambda)L_1$

$
\equivaut 
\left\{  
\begin{array}{cc}
3x +(8-10\lambda)y  &=0\\
(-100\lambda^2 +150\lambda -50 )y  &=0
\end{array}\right.
\equivaut
\left\{  
\begin{array}{cc}
(7-10\lambda) x +2y  &=0\\
(2\lambda^2 -3\lambda +1) y  &=0
\end{array}\right.
\equivaut
\left\{  
\begin{array}{cc}
(7-10\lambda) x +2y  &=0\\
(2\lambda-1)(\lambda-1) y  &=0
\end{array}\right.
$

Le système est de Cramer pour $(2\lambda-1)(\lambda-1)\neq 0$ et l'unique solution est alors $(0,0)$. 

Pour $\lambda=1$ on obtient 
$\equivaut
\left\{  
\begin{array}{cc}
-3 x +2y  &=0\\
0 &=0
\end{array}\right.
$
et les solutions sont de la forme : 
$$\{ (2a,3a ) \, |\, a\in \R\} $$

Pour $\lambda=\frac{1}{2}$ on obtient 
$\equivaut
\left\{  
\begin{array}{cc}
2 x +2y  &=0\\
0 &=0
\end{array}\right.
$
et les solutions sont de la forme : 
$$\{ (a,-a ) \, |\, a\in \R\} $$

\item Le determinant de $P$ vaut $det(P) = 3+2 = 5 \neq 0$ donc $P$ est inversible. 
Son inverse vaut 
$$P^{-1} = \frac{1}{5} \left( 
\begin{array}{cc}
3 & -2 \\
1 & 1
\end{array}
\right)$$

\item Ce n'est que du calcul. 

\item $$D^n =  \left( 
\begin{array}{cc}
\frac{1}{2^n}& 0 \\
0 & 1
\end{array}
\right)$$
A prouver par récurrence ou  dire que c'est du cours pour des matrices diagonales. 
\item 
On prouve tout d'abord par récurrence que pour tout n :
$Q(n) : " A^n = P D^n P^{-1} "$.
Initialisation. La proposition est vraie pour $n=0$ les deux cotés valent l'identité. 

On suppose $Q(n) $ vraie pour un $n \in \N$ fixé. On a 
$A^{n} =   P D^n P^{-1}$ et donc
\begin{align*}
A^{n+1} &=A P D^n P^{-1}\\
&=PDP^{-1} P D^n P^{-1} \\
&= PD \Id D^n P^{-1}  \\
&=PD  D^n P^{-1}\\
&=PD^{n+1} P^{-1}
\end{align*}

Ensuite c'est du calcul. 
\item 
Et d'après les questions 2 et 3 on  a
$$A \binom{d_n}{m_n}= \binom{d_{n+1}}{m_{n+1}}$$
et par récurrence 
$$A^n \binom{d_0}{m_0}= \binom{d_n}{m_n}$$
D'après l'énoncé $d_0= 1$ c'est l'événement certain. 
et donc $$ \binom{d_n}{m_n}= 
\frac{1}{5} \binom{3\left( 1/2\right)^n +2 }{-3\left( 1/2\right)^n +3}$$
En particulier 

$$d_n=\frac{1}{5} (3\left( 1/2\right)^n +2) $$

\end{enumerate}
\end{correction}