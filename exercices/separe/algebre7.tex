% Titre : Tchebychev (Pb)
% Filiere : BCPST
% Difficulte :
% Type : DS, DM
% Categories : algebre
% Subcategories : 
% Keywords : algebre





\begin{exercice}
On considère la suite de polynômes $(T_n)_{n\in \N}$ définie par 
$$ T_0=1 \quadet T_1=X \quadet \forall n\in \N,\, T_{n+2}=2X T_{n+1}-T_n$$
\begin{enumerate}
\item \begin{enumerate}
\item Calculer $T_2$, $T_3$ et $T_4$.
\item Calculer le degré et le coefficient de $T_n$ pour tout $n\in\N$. 
\item Calculer le coefficient constant de $T_n$. 
\end{enumerate}
\item \begin{enumerate}
\item Soit $\theta \in \R$. Montrer que pour tout $n\in \N$ on  a  $T_n(\cos(\theta)) =\cos(n\theta)$.
\item En déduire que $\forall x\in [-1,1], $ on a $T_n(x) =\cos(n \arccos(x))$. 
\end{enumerate}
\item \begin{enumerate}
\item En utilisant la question 2a), déterminer les racines de $T_n$ sur $[-1,1]$. 
\item Combien de racines distinctes a-t-on ainsi obtenues ? Que peut on en déduire ? 
\item Donner la factorisaiton de $T_n$ pour tout $n\in \N^*$. 
\end{enumerate}
\end{enumerate}
\end{exercice}
\begin{correction}
\begin{enumerate}
\item \begin{enumerate}
\item $T_2 = 2X^2-1$, $T_3 =4X^3-3X$, $T_4 = 8X^4 -8X^2+1 $
\item Montrons par récurrence que $\deg(T_n) =n$. 
Comme la suite est une suite récurrente d'ordre 2, on va poser comme proposition de récurrence $$P(n) : ' \deg(T_n) =n \text{ ET } \deg(T_{n+1}) =n+1'$$
C'est vrai pour $n=0,1,2$ et $3$.  On suppose qu'il existe un entier $n_0\in \N$ tel que $P(n_0)$ soit vrai et montrons $P(n_0+1)$. 
On cherche donc à vérifier $\deg(T_{n_0+1}) =n_0+1 \text{ ET } \deg(T_{n_0+2}) =n_0+2'$. La première égalité est vraie par hypothèse de récurrence. 
La seconde vient de la relation $T_{n_0+2} =2 X T_{n_0+1} -T_{n_0}$
En effet, par hypothèse de récurrence $T_{n_0+1}$ est de degré $n_0+1$ donc $2 X T_{n_0+1}$ est de degrés $n_0 +2$. Comme $\deg(T_{n_0}) =n_0<n_0+2$, on a 
$$\deg(T_{n_0+2} )=\max(\deg(2 X T_{n_0+1}),\deg(T_{n_0})) = n_0+2$$

Ainsi par récurrence pour tout $n\in \N$, $\deg(T_n)= n$.
\item La récurrence précédente montre que le coefficient dominant, notons le $c_n$ vérifie $c_{n+2} = 2 c_{n+1}$. Ainsi $c_n = 2^nc_0 =2^n$. 

\end{enumerate}
\item 
\begin{enumerate}
\item Montrons le résultat par récurrence. On pose 
$$Q(n) : " \forall \theta\in \R, T_n(\cos(\theta)) =\cos(n\theta)  \text{ ET } T_{n+1}(\cos(\theta)) =\cos((n+1)\theta)"$$

$Q(0)$ est vraie par définition de $T_0 $ et $T_1$ 

Supposons qu'il existe $n\in \N$ tel que $Q(n)$ soit vrai et montrons $Q(n+1)$. Il suffit de montrer que $\forall \theta \in \R$
$$T_{n+2}(\cos(\theta) )=\cos((n+2)\theta)$$

On a par définition de $T_{n+2}$ 
$$T_{n+2} (\cos(\theta))  = 2\cos(\theta) T_{n+1}(\cos(\theta)) -T_n(\cos(\theta))$$
Par hypothèse de récurrence on a 
$T_{n+1}(\cos(\theta)) =\cos((n+1)\theta)$ et 
$T_{n}(\cos(\theta))=\cos(n\theta) $ donc  
$$T_{n+2} (\cos(\theta)) =2 \cos(\theta) \cos((n+1) \theta) - \cos(n \theta)$$
Les formules trigonométriques donnent : 
\begin{align*}
2 \cos(\theta) \cos((n+1) \theta)   &=\cos(\theta+(n+1) \theta) +\cos(\theta-(n+1) \theta)\\
&=\cos((n+2) \theta) + \cos(-n\theta)\\
&=\cos((n+2) \theta) + \cos(n\theta)
\end{align*}
Donc 
$$T_{n+2} (\cos(\theta))  = \cos((n+2) \theta) + \cos(n\theta)-\cos(n\theta) = \cos((n+2)\theta)
$$


Par récurrence, pour tout $\theta \in \R$ et tout $n\in \N$: 
$$T_n(\cos(\theta ) ) =\cos(n\theta)$$

On peut répondre maintenant facilement à la question 2c) (avec la faute de frappe coefficient constant au lieu de coefficient dominant). Le coefficient constant vaut $T_n(0) = T_n(\cos(\pi/2))  =\cos(n \pi/2) $ 

Donc $T_n(0) = 0$ pour $n=2k$, $k\in \Z$. Pour $n= 2k+1$, $k\in \Z$, on a alors $T_{2k+1}(0) = \cos(( 2k+1 ) \pi/2 ) =\frac{(-1)^k}{2}$

\item Soit $x\in [-1,1]$ on  note $x =\cos(\theta)$, avec $\theta \in [0,\pi]$ on a  alors 
$\theta =\arccos( x) $. D'après la question précédente on a donc pour tout $x\in [-1,1]$: 
$$T_n(x) = \cos( n \arccos(x))$$



\end{enumerate}
\item 
\begin{enumerate}
\item Pour tout $\theta $ tel que $n\theta  \equiv \frac{\pi}{2}[\pi]$,  on a $\cos(n\theta) =0$ 

Ainsi pour tout $\theta $ tel que $\theta \equiv \frac{\pi}{2n}[\frac{\pi}{n}]$, 
$$T_n(\cos(\theta) ) =0$$

On obtient ainsi $n$ racines entre $[-1,1]$ données par 
$$\{  \cos\left(\frac{\pi+2k\pi }{2n}\right)\, |  \,  k\in [0,n-1] \}$$

\item On a obtenu $n$ racines. Comme $T_n$ est de degrés $n$, on a obtenu toutes les racines, ainsi $T_n$ se factorise de la manière suivante\footnote{ sans oublier le coéfficient dominant, merci Marie.}  : 
\item 
$$T_ n (X)  =2^n\prod_{k=0}^{n-1} \left(X - \cos\left(\frac{\pi+2k\pi }{2n}\right) \right) $$

\end{enumerate}

\end{enumerate}
\end{correction}