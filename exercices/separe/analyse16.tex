% Titre : Simplification Produit
% Filiere : BCPST
% Difficulte :
% Type : DS, DM
% Categories : analyse
% Subcategories : 
% Keywords : analyse




\begin{exercice}
Simplifier  
$$\prod_{k=1}^n \left(1-\frac{1}{k^2}\right) \quad \text{ et } \quad \prod_{k=2}^n \left(1-\frac{1}{k^2}\right).$$
En déduire la valeur de $\ddp \lim_{n\tv \infty} \prod_{k=2}^n \left(1-\frac{1}{k^2}\right)$
\end{exercice}




\begin{correction}
$$\prod_{k=1}^n \left(1-\frac{1}{k^2}\right) = \left(1-\frac{1}{1^2}\right) \times \prod_{k=2}^n \left(1-\frac{1}{k^2}\right) =0 \times \prod_{k=2}^n \left(1-\frac{1}{k^2}\right) =0$$
et
$$\prod_{k=2}^n \left(1-\frac{1}{k^2}\right) =\prod_{k=2}^n\left( \frac{k-1}{k}\frac{k+1}{k}\right) = \prod_{k=2}^n\left( \frac{k-1}{k}\right) \prod_{k=2}^n\left(  \frac{k+1}{k}\right) $$ 
On reconnait deux produits téléscopiques, donc 
$$\prod_{k=2}^n \left(1-\frac{1}{k^2}\right)  = \frac{2-1}{n} \times \frac{n+1}{2} =\frac{n+1}{2n}$$

$\ddp \lim_{n\tv \infty} \frac{n+1}{n}=1$, donc 
$$\lim_{n\tv \infty} \prod_{k=2}^n \left(1-\frac{1}{k^2}\right)  =\frac{1}{2}$$

\end{correction}