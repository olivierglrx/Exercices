
\begin{correction} \;
La d\'erivation d'une somme finie est une m\'ethode tr\`{e}s classique qui permet d'obtenir plein de nouvelles sommes. Il s'agit juste d'utiliser le fait que $(f+g)^{\prime}=f^{\prime}+g^{\prime}$ et ainsi la d\'eriv\'ee d'une somme est \'egale \`{a} la somme des d\'eriv\'ees.
\begin{enumerate}
\item D'apr\`{e}s le bin\^{o}me de Newton, on sait que: \fbox{$f(x)=(1+x)^n$.}
\item La fonction $f$ est ainsi d\'erivable sur $\R$ comme compos\'ee de fonctions d\'erivables. La fonction $f$ est d\'efinie par deux expressions diff\'erentes que l'on peut d\'eriver:
\begin{itemize}
\item[$\bullet$] D'un c\^{o}t\'e, la fonction $f$ vaut: $f(x)=(1+x)^n$. Ainsi, en d\'erivant, on obtient que: \begin{center}
\fbox{$\forall x\in\R,\ f^{\prime}(x)=n(1+x)^{n-1}$.}
\end{center}
\item[$\bullet$] De l'autre c\^{o}t\'e, la fonction $f$ vaut $f(x)=\ddp \sum\limits_{k=0}^{n} \binom{n}{k}x^k=1+nx+\dots+nx^{n-1}+x^n=1+\ddp \sum\limits_{k=1}^{n} \binom{n}{k}x^k$. La d\'eriv\'ee d'une somme \'etant \'egale \`{a} la somme des d\'eriv\'ees, on obtient que: \begin{center}
\fbox{$\forall x\in\R,\ f^{\prime}(x)=0+\ddp \sum\limits_{k=1}^{n} k\binom{n}{k}x^{k-1}$}
\end{center} 
\vsec
car le premier terme pour $k=0$ est constant donc sa d\'eriv\'ee est nulle. 

\vsec
%ATTENTION, la somme commence bien \`{a} $k=1$ car le terme pour $k=0$ dans $f(x)$ est le terme constant $1$ qui est nul lorsqu'on d\'erive.
\end{itemize}
On obtient donc que: \fbox{$\forall x\in\R,\ g(x)=\ddp \sum\limits_{k=1}^{n} k\binom{n}{k}x^{k-1}=n(1+x)^{n-1}$.}
\item Il s'agit de remarquer que $S=g(1)=f^{\prime}(1)$ et ainsi, on obtient que: \fbox{$S=n2^{n-1}$.} On retrouve bien le m\^{e}me r\'esultat.
\end{enumerate}
%\end{enumerate}
\end{correction}