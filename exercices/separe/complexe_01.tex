% Titre : complexe
% Filiere : BCPST
% Difficulte : 
% Type : TD 
% Categories :complexe
% Subcategories : 
% Keywords : complexe




\begin{exercice}  \;
Mettre les complexes suivants sous forme alg\'ebrique simple:
\begin{enumerate}
\begin{minipage}[t]{0.3\textwidth}
\item $z=\ddp\frac{1-3i}{1+3i}$ 
\item $z=(i-\sqrt{2})^3$
\item $z=\ddp\frac{1+4i}{1-5i}$ 
\item  $z=\left(\ddp\frac{\sqrt{3}-i}{1+i\sqrt{3}}  \right)^9$
\end{minipage}
\begin{minipage}[t]{0.3\textwidth}
\item $z=\ddp\frac{(1+i)^2}{(1-i)^2}$
\item $z=\ddp\frac{1}{\frac{1}{i+1}-1}$
\item $z=(1+i)^{2019}$
\item $z=\ddp\frac{2+5i}{1-i}+\ddp\frac{2-5i}{1+i}$
\end{minipage}
\begin{minipage}[t]{0.3\textwidth}
\item $z=(5-2i)^3$ 
\item $z=\ddp\frac{1}{(4-i)(3+2i)}$ 
\item $z=\ddp\frac{(3+i)(2-3i)}{-2i+5}$
\item $z=(\sqrt{3}-2i)^4$
\end{minipage}
\end{enumerate}
\end{exercice}


\%\%\%\%\%\%\%\%\%\%\%\%\%\%\%\%\%\%\%\%
\%\%\%\%\%\%\%\%\%\%\%\%\%\%\%\%\%\%\%\%
\%\%\%\%\%\%\%\%\%\%\%\%\%\%\%\%\%\%\%\%




\begin{correction}   \;
Dans cet exercice, je ne d\'etaille pas forc\'ement tous les calculs, je ne donne que la m\'ethode g\'en\'erale ou des indications.
\begin{enumerate}
\item \textbf{Mettre sous forme alg\'ebrique $\mathbf{z=\ddp\frac{1-3i}{1+3i}}$:} 
$\fbox{$z=-\ddp\frac{4}{5}-\ddp\frac{3i}{5}.$}$ On a un quotient de nombres complexes dont on vaut la forme alg\'ebrique: on multiplie par le conjugu\'e du d\'enominateur.  
\item \textbf{Mettre sous forme alg\'ebrique $\mathbf{z=(i-\sqrt{2})^3}$:} $\fbox{$z=\sqrt{2}+5i.$}$ On utilise ici une identit\'e remarquable.
\item \textbf{Mettre sous forme alg\'ebrique $\mathbf{z= \ddp\frac{1+4i}{1-5i}}$:}  $\fbox{$z=-\ddp\frac{19}{26}+i\frac{9}{26}.$ }$ On a un quotient de nombres complexes dont on vaut la forme alg\'ebrique: on multiplie par le conjugu\'e du d\'enominateur.  
\item \textbf{Mettre sous forme alg\'ebrique $\mathbf{z=\left(\ddp\frac{\sqrt{3}-i}{1+i\sqrt{3}}  \right)^9}$:}  $\fbox{$z=(-i)^9=-i.$}$ Ici plusieurs m\'ethodes marchent bien: Soit on commence par mettre sous forme exponentielle le nombre complexe $\ddp\frac{\sqrt{3}-i}{1+i\sqrt{3}}$ en mettant sous forme exponentielle le num\'erateur d'un c\^{o}t\'e et le d\'enominateur de l'autre c\^{o}t\'e puis on passe \`{a} la puissance 9. Soit on commence par mettre sous forme alg\'ebrique le nombre complexe $\ddp\frac{\sqrt{3}-i}{1+i\sqrt{3}}$ en multipliant par le conjugu\'e du d\'enominateur et on passe \`{a} la puissance 9.
\item \textbf{Mettre sous forme alg\'ebrique $\mathbf{z= \ddp\frac{(1+i)^2}{(1-i)^2}}$:}  $\fbox{$z=-1.$}$ L\`{a} encore il y a plusieurs m\'ethodes qui marchent bien. Une possibilit\'e est de mettre sous forme exponentielle $1+i$ d'un c\^{o}t\'e et $1-i$ de l'autre c\^{o}t\'e puis de les passer au carr\'e et enfin de faire le quotient.  
\item \textbf{Mettre sous forme alg\'ebrique $\mathbf{ z=\ddp\frac{1}{\frac{1}{i+1}-1}}$:}  $\fbox{$z=-1+i.$}$ On peut par exemple commencer par tout mettre sous le m\^{e}me d\'enominateur en bas et on obtient $z=\ddp\frac{1}{\frac{-i}{1+i}}=\ddp\frac{1+i}{-i}=i(1+i)$.
\item \textbf{Mettre sous forme alg\'ebrique $\mathbf{z=(1+i)^{2019}}$:}  $\fbox{$z=- 2^{1009} + 2^{1009} i.$  }$ Ici il faut commencer par mettre sous forme exponentielle $1+i$ et on obtient que $1+i=\sqrt{2}e^{i\frac{\pi}{4}}$. Ensuite on passe \`{a} la puissance et on obtient que: $z=\left( \sqrt{2}e^{i\frac{\pi}{4}}  \right)^{2019}=2^{1009} \sqrt{2} e^{i\frac{2019\pi}{4}}$. Il faut alors compter le nombre de tours complets que l'on a fait dans $\ddp\frac{2019\pi}{4}$. Une fa\c{c}on de voir les choses est d'\'ecrire : $\ddp\frac{2019}{8}\times 2\pi$ et de faire la division euclidienne de $2019$ par 8. On obtient: $2019=252\times 8 + 3$ et ainsi on a: $\ddp\frac{2019}{8}\times 2\pi=252\times 2\pi + \frac{3\pi}{4}$. Ainsi on a: $z=2^{1009} \sqrt{2} \times e^{i \left(252\times 2\pi + \frac{3\pi}{4}\right)}= 2^{1009} \sqrt{2} \left(-\frac{\sqrt{2}}{2} + \frac{\sqrt{2}}{2} i\right) = - 2^{1009} + 2^{1009} i$.
\item  \textbf{Mettre sous forme alg\'ebrique $\mathbf{z=\ddp\frac{2+5i}{1-i}+\ddp\frac{2-5i}{1+i}}$:}  $\fbox{$z=-3$. }$ On peut par exemple mettre sous forme alg\'ebrique chaque terme de la somme de fa\c{c}on s\'epar\'ee en multipliant par le conjugu\'e puis on les somme.
\item \textbf{Mettre sous forme alg\'ebrique $\mathbf{z=(5-2i)^3}$:}  $\fbox{$z=65-142 i.$}$ On utilise une identit\'e remarquable.
\item \textbf{Mettre sous forme alg\'ebrique $\mathbf{z= \ddp\frac{1}{(4-i)(3+2i)}}$:}  $\fbox{$z=\ddp\frac{14}{221}-i\ddp\frac{5}{221}$.}$ On peut multiplier par le conjugu\'e du d\'enominateur \`{a} savoir $(4+i)(3-2i)$.
\item \textbf{Mettre sous forme alg\'ebrique $\mathbf{z=\ddp\frac{(3+i)(2-3i)}{-2i+5}}$:}  $\fbox{$z=\ddp\frac{69}{29}-i\ddp\frac{17}{29}$.}$  On multiplie par le conjugu\'e du d\'enominateur.
\item \textbf{Mettre sous forme alg\'ebrique $\mathbf{z=(\sqrt{3}-2i)^4}$:}  $\fbox{$z=-47+8\sqrt{3}i$.}$ On d\'eveloppe avec le bin\^{o}me de Newton.
\end{enumerate}
\end{correction}