% Titre : Géométrie, droite et inclusion
% Filiere : BCPST
% Difficulte :
% Type : DS, DM
% Categories : autres
% Subcategories : 
% Keywords : autres




\begin{exercice}
Soit $D_1 = \{ (x,y,z) \in \R^3 \, |\, 2x-y=-2 \text{ et } 3y -2z=0\}$
$D_2 = \{ (x,y,z) \in \R^3 \, |\, x+y-z=-1 \text{ et } 3x -z=-3\}$

\begin{enumerate}
\item Soit $P_1=(x_1,y_1,z_1)\in D_1$. Exprimer $x_1$ et $y_1$ en fonction de $z_1$. 
\item Etablir que $D_1\subset D_2$
\item Soit $P_2=(x_2,y_2,z_2)\in D_2$. Exprimer $x_2$ et $y_2$ en fonction de $z_2$. 
\item Etablir que $D_2\subset D_1$. 
\end{enumerate}

\end{exercice}

\begin{correction}
Soit $P_1 =(x_1,y_1,z_1)\in D_1$. On a 
$x_1-y_1 =-2$ et $3y_1-2z_1= 0$ donc 
\begin{center}
\fbox{$y_1 = \frac{2}{3}z_1$ }
\end{center}
et  $2x_1 =-2 +y_1 = -2 + \frac{2}{3}z_1$
donc 
\begin{center}
\fbox{$x_1 = -1+\frac{1}{3}z_1$ }
\end{center}
Ainsi 
$$x_1+y_1-z-1= -1 + \frac{1}{3}z_1 + \frac{2}{3}z_1 -z_1=-1$$
et 
$$3x_1-z_1 = -3 +z_1-z_1=-3$$

Donc $P_1 \in D_2$. Le résultat étant vrai pour tout $P_1 \in D_1$, on a donc 

\begin{center}
\fbox{$D_1 \subset D_2$ }

\end{center}


Soit $P_2 =(x_2,y_2,z_2)\in D_2$. On a 
$x_2+y_2 -z_2=-1$ et $3x_2-z_2= -3$ donc 
\begin{center}
\fbox{$x_2 = -1 +\frac{1}{3}z_2$ }

\end{center}

et  $y_2 =-1+z_2-x_2 = -\frac{1}{3}z_2+z_2=\frac{2}{3}z_2$
donc 
\begin{center}
\fbox{$y_2=\frac{2}{3}z_2$ }

\end{center}


Ainsi 
$$2x_2-y_2 = -2 +\frac{2}{3}z_2-\frac{2}{3}z_2 =-2$$
et 
$$3y_2-z_2 = 3 \frac{2}{3}z_2 -z_2=0$$

Donc $P_2 \in D_1$. Le résultat étant vrai pour tout $P_2 \in D_2$, on a donc 

\begin{center}
\fbox{$D_2 \subset D_1$ }

\end{center}
\end{correction}