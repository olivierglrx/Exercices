% Titre : suites
% Filiere : BCPST
% Difficulte : 
% Type : TD 
% Categories :suites
% Subcategories : 
% Keywords : suites



\begin{exercice}
\'Etudier la suite $\suiteu$ d\'efinie par 
$\left\lbrace\begin{array}{l}
u_0=1\vsec\\
u_{n+1}=u_n^2+u_n.
\end{array}\right.$
\end{exercice}


\%\%\%\%\%\%\%\%\%\%\%\%\%\%\%\%\%\%\%\%
\%\%\%\%\%\%\%\%\%\%\%\%\%\%\%\%\%\%\%\%
\%\%\%\%\%\%\%\%\%\%\%\%\%\%\%\%\%\%\%\%



\begin{correction}
\textbf{\'Etudions la suite $\mathbf{\suiteu}$ d\'efinie par $\mathbf{\left\lbrace\begin{array}{l}
u_0=1\vsec\\
u_{n+1}=u_n^2+u_n.
\end{array}\right.}$}\\
\noindent C'est une suite de type $u_{n+1}=f(u_n)$, on donne les id\'ees de l'\'etude. Ainsi la r\'edaction dans une copie doit \^{e}tre beaucoup plus d\'etaill\'ee qu'ici.
\begin{enumerate}
 \item \textbf{\'Etude des variations de la fonction $\mathbf{f}$ associ\'ee: $\mathbf{x\mapsto x^2+x}$:}\\
\noindent La fonction $f$ est d\'efinie, continue et d\'erivable sur $\R$ et 
$$\forall x\in\R,\ f^{\prime}(x)=2x+1.$$
On obtient ainsi le tableau de variation suivant:
\begin{center}
\begin{tikzpicture}
 \tkzTabInit{ $x$          /1,%
	%$\sin{(3x)}$             /1,
       %$\cos{(5x)}$     /1,%
       $f'(x)$      /1,%
       $f$       /2}%
     { $-\infty$, $-\ddp\demi$ ,$+\infty$ }%
  %\tkzTabLine {0,$+$,t,$+$,t,$+$,0,$-$,t}%
  %\tkzTabLine {t,$+$,0,$-$,0,$+$,t,$+$,0}%
  \tkzTabLine {,$-$,0,$+$,}%
  \tkzTabVar{
     % {-/ $1$       /,%
       +/ $+\infty$        /,
        -/$-\ddp\frac{1}{4}$           /,%
       +/$+\infty$           /,
      % +/           /,%
       %-/ $-1$          /,
       %R/           /,%
       %-/ $-\infty$ /}
                      }
 \tkzTabVal[draw]{2}{3}{0.3}{$0$}{$0$}
 \tkzTabVal[draw]{2}{3}{0.6}{$1$}{$2$}
\end{tikzpicture}
\end{center}
\item \textbf{\'Etude du signe de la fonction $\mathbf{g: x\mapsto f(x)-x=x^2}$:}\\
\noindent \fbox{Cette fonction est toujours positive sur $\R$ et elle ne s'annule qu'en 0.}
\item \textbf{Calcul des limites \'eventuelles:}\\
\noindent La fonction $f$ est continue sur $\R$, ainsi, si la suite $\suiteu$ converge, elle ne peut converger que vers $l$ v\'erifiant 
$$l=f(l)\Leftrightarrow g(l)=0\Leftrightarrow l=0.$$
Ainsi, \fbox{0 est la seule limite \'eventuelle de la suite.}
\item \textbf{Montrons que la suite est bien d\'efinie et elle appartient \`{a} $\mathbf{I}$ intervalle stable par $\mathbf{f}$}:\\
\noindent On remarque que $[ 1,+\infty[$ est un intervalle stable par $f$ et que $u_0=1\in\lbrack 1,+\infty\lbrack$. Un raisonnement par r\'ecurrence permet alors de v\'erifier que \fbox{la suite est bien d\'efinie et que $\forall n\in\N,\quad u_n\geq 1.$}
\noindent 
\item \textbf{\'Etude de la monotonie de la suite:}\\
Comme $g$ est positive sur $\R$, \fbox{la suite est croissante.}
\noindent 
\item \textbf{\'Etude de la convergence de la suite:}\\
\begin{itemize}
\item[$\star$] La suite $\suiteu$ est croissante donc d'apr\`{e}s le th\'eor\`{e}me sur les suites monotones, elle converge ou elle diverge vers $+\infty$.
\item[$\star$] On suppose par l'absurde que la suite $\suiteu$ converge vers un r\'eel $l$. On a alors:
\begin{itemize}
\item[$\circ$] La suite $\suiteu$ converge vers $l$.
\item[$\circ$] Pour tout $n\in\N$: $u_n\geq 1$.
\end{itemize}
D'apr\`{e}s le th\'eor\`{e}me de passage \`{a} la limite, on obtient donc que: $l\geq 1$. Absurde car la seule limite \'eventuelle de la suite $\suiteu$ est 0. Ainsi \fbox{la suite $\suiteu$ diverge vers $+\infty$.}
\end{itemize}
\end{enumerate}



\end{correction}