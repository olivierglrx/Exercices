% Titre : Pivot de Gauss
% Filiere : BCPST
% Difficulte :
% Type : DS, DM
% Categories : info
% Subcategories : 
% Keywords : info






\begin{exercice} [Extrait du Concours Agro-Veto 2019]
Des éléments de syntaxe Python, et en particulier l'usage du module numpy, sont donnés en annexe . Dans tout ce qui suit, les variables $n, p, A, M, i, j$ et $c$ vérifient les conditions suivantes qui ne seront pas rappelées à chaque question :
\begin{itemize}


\item $n$ et $p$ sont des entiers naturels tels que $p \geq n \geq 2$;
\item $A$ est une matrice carrée à $n$ lignes inversible;
\item $M$ est une matrice à $n$ lignes et $p$ colonnes telle que la sous-matrice carrée constituée des $n$ premières colonnes de $M$ est inversible;
\item  $i$ et $j$ sont des entiers tels que $0 \leq i \leq n-1$ et $0 \leq j \leq n-1$;
\item  $c$ est un réel non nul.
\end{itemize}
On note $L_{i} \leftarrow L_{i}+c L_{j}$ l'opération qui ajoute à la ligne $i$ d'une matrice la ligne $j$ multipliée par c.
\begin{enumerate}


\item Soit la fonction initialisation:
\begin{lstlisting}{python}
def initialisation(A):
    n = np.shape(A)[0]
    mat = np.zeros((n,2*n))
    for i in range(0, n):
        for j in range(0, n):
            mat[i,j] = A[i,j]
    return(mat)
\end{lstlisting}
Pour chacune des affirmations suivantes, indiquer si elle est vraie ou fausse, en justifiant. L'appel initialisation (A) renvoie:
\begin{enumerate}


\item une matrice rectangulaire à $n$ lignes et $2 n$ colonnes remplie de zéros;
\item une matrice de même taille que $A$;
\item une erreur au niveau d'un range;
\item une matrice rectangulaire telle que les $n$ premières colonnes correspondent aux $n$ colonnes de $A$, et les autres colonnes sont nulles.\end{enumerate}
\item  Les trois fonctions multip, ajout et permut suivantes ne renvoient rien : elles modifient les matrices auxquelles elles s'appliquent.
\begin{enumerate}


\item Que réalise la fonction multip?
\begin{lstlisting}{python}
def multip(M, i, c):
    p = np.shape(M)[1]
    for k in range(0, p):
        M[i,k] = c*M[i,k]
\end{lstlisting}


\item Compléter la fonction ajout, afin qu'elle effectue l'opération $L_{i} \leftarrow L_{i}+c L_{j}$.
\begin{lstlisting}{python}
def ajout(M, i, j, c):
    p = np.shape(M)[1]
    for k in range(0, p):
        _____ ligne(s) a completer _____ 
\end{lstlisting}

\item Écrire une fonction permut prenant pour argument $M, i$ et $j$, et qui modifie $M$ en échangeant les valeurs des lignes $i$ et $j$.
\end{enumerate}
Dans la suite du sujet, l'expression "opération élémentaire sur les lignes" fera référence à l'utilisation de permut, multip ou ajout.
\item  Soit la colonne numéro $j$ dans la matrice $M$. On cherche le numéro $r$ d'une ligne où est situé le plus grand coefficient (en valeur absolue) de cette colonne parmi les lignes $j$ à $n-1$. Autrement dit, $r$ vérifie :
$$
|A[r, j]|=\max \{|A[i, j]| \text { pour } i \text { tel que } j \leq i \leq n-1\} .
$$
Écrire une fonction \texttt{rang\_pivot} prenant pour argument $M$ et $j$, et qui renvoie cette valeur de $r$.

 Lorsqu'il y a plusieurs réponses possibles pour $r$, dire (avec justification) si l'algorithme renvoie le plus petit $r$, le plus grand $r$ ou un autre choix.
 
  (L'utilisation d'une commande max déjà programmée dans Python est bien sûr proscrite.)
  
\item Soit la fonction mystere:
\begin{lstlisting}{python}
def mystere(M):
    n = np.shape(M)[0]
    for j in range(0, n):
        r = rang_pivot(M, j)
        permut(M, r, j)
        for k in range(j+1, n):
            ajout(M, k, j, -M[k,j]/M[j,j])
        print(M) 
\end{lstlisting} 

\begin{enumerate}
\item On considère dans cette question l'algorithme mystere appliqué à la matrice \[M_{1}=\left(\begin{array}{ccc}3 & 2 & 2 \\ -6 & 0 & 12 \\ 1 & 1 & -3\end{array}\right)\] Indiquer combien de fois la ligne print (M) est exécutée ainsi que les différentes valeurs qu'elle affiche
\item De façon générale, que réalise cet algorithme?
\end{enumerate} 
\item On considère la fonction reduire:
\begin{lstlisting}{python}
def reduire(M):
    n = np.shape(M)[0]
    mystere(M)
    for i in range(0, n):
        multip(M, i, 1/M[i,i])
    #Les lignes suivantes sont \'a compl\'eter :
        __________________________        
\end{lstlisting}
Compléter la fonction afin que la portion de code manquante effectue les opérations élémentaires suivantes sur les lignes :

pour $j$ prenant les valeurs $n-1, n-2, \ldots, 1$, faire :


\hspace{2cm}	pour $k$ prenant les valeurs $j-1, j-2, \ldots, 0$, faire :

$$
\mathrm{L}_{\mathrm{k}} \leftarrow \mathrm{L}_{\mathrm{k}}-\mathrm{M}[\mathrm{k}, \mathrm{j}] \mathrm{L}_{\mathrm{j}}
$$



Indiquer ce que réalise cette fonction.
\end{enumerate}

\begin{center}
\textbf{\huge{Annexe}}
\end{center}
On considère que le module numpy, permettant de manipuler des tableaux à deux dimensions, est importé via import numpy as np. Pour une matrice $M$ à $n$ lignes et $p$ colonnes, les indices vont de 0 à $n-1$ pour les lignes et de 0 à $p-1$ pour les colonnes.

\begin{center}
\begin{tabular}{ll}
\hline \multicolumn{1}{c}{ Python } & \multicolumn{1}{c}{ Interprétation } \\
\hline $\mathrm{abs}(\mathrm{x})$ & Valeur absolue du nombre $x$ \\
\hline $\mathrm{M}[\mathrm{i}, \mathrm{j}]$ & Coefficient d'indice $(i, j)$ de la matrice $M$ \\
\hline $\mathrm{np} \cdot$ zeros $((\mathrm{n}, \mathrm{p}))$ & Matrice à $n$ lignes et $p$ colonnes remplie de zéros \\
\hline $\mathrm{T}=\mathrm{np} . \mathrm{shape}(\mathrm{M})$ & Dimensions de la matrice $M$ \\
$\mathrm{~T}[0]$ ou np.shape $(\mathrm{M})[0]$ & Nombre de lignes \\
$\mathrm{T}[1]$ ou np.shape $(\mathrm{M})[1]$ & Nombre de colonnes \\
\hline $\mathrm{M}[\mathrm{a}: \mathrm{b}, \mathrm{c}: \mathrm{d}]$ & Matrice extraite de $M$ constituée des lignes $a$ à $b-1$ et des \\
& colonnes $c$ à $d-1:$ \\
& si $a$ (resp. $c)$ n'est pas précisé, l'extraction commence à la \\
& première ligne (resp. colonne) \\
& si $b$ (resp. $d)$ n'est pas précisé, l'extraction finit à la dernière \\
& ligne (resp. colonne) incluse \\
\hline
\end{tabular}
\end{center}
\end{exercice}


\begin{correction}

\end{correction}