
\begin{exercice} \;
On d\'efinit la suite $\suiteu$ par 
$\left\lbrace\begin{array}{l}
u_0\in\R\vsec\\
u_{n+1}=\ddp\frac{1}{3}u_n^2-u_n+3
\end{array}\right.$
\begin{enumerate}
\item \'Etudier la fonction $f$ associ\'ee.
\item \'Etudier le signe de $g: x\mapsto f(x)-x$.
\item Calculer les limites \'eventuelles de la suite $\suiteu$.
\item Que peut-on dire de la suite $\suiteu$ lorsque $u_0=3$ ou $u_0=0$ ?
\item On suppose que $u_0\in\rbrack 0,3\lbrack$.
\begin{enumerate}
\item Montrer que la suite est bien d\'efinie et que pour tout $n\in\N$: $u_n\in\rbrack 0,3\lbrack$.
\item \'Etudier la monotonie de la suite $\suiteu$.
\item \'Etudier le comportement \`{a} l'infini de la suite $\suiteu$.
\end{enumerate}
\item On suppose que $u_0>3$.
\begin{enumerate}
\item Montrer que la suite est bien d\'efinie et que pour tout $n\in\N$: $u_n>3$.
\item \'Etudier la monotonie de la suite $\suiteu$.
\item \'Etudier le comportement \`{a} l'infini de la suite $\suiteu$.
\end{enumerate} 
\item On suppose que $u_0<0$.
\begin{enumerate}
\item Montrer que $u_1>3$.
\item En d\'eduire le comportement \`{a} l'infini de la suite $\suiteu$.
\end{enumerate} 
\end{enumerate}
\end{exercice}