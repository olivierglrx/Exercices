% Titre : Système linéaire (produit - changement de variables)
% Filiere : BCPST
% Difficulte :
% Type : DS, DM
% Categories : algebre
% Subcategories : 
% Keywords : algebre




\begin{exercice}
Résoudre le système suivant où $x,y,z$ sont des réels positifs: 
$$\left\{ \begin{array}{ccc}
x^2y^2z^6 & =& 1\\
x^4y^5z^{13}& =& 2 \\
x^2yz^7 & =& 3
\end{array}\right. $$
\end{exercice}

\begin{correction}
Comme tous les éléments sont positifs on peut prendre le logartihme. 
On note $X= \ln(x), Y=\ln(y)$ et $Z=\ln(z)$ on obtient : 
$$\left\{ \begin{array}{ccc}
2X +2Y+6Z& =& 0\\
4X+5Y+13Z& =& \ln(2) \\
2X+Y+7Z & =& \ln(3)
\end{array}\right. $$
On résout ensuite le système en $(X,Y,Z)$. Tout d'abord on échelonne le système : 
$$\left\{ \begin{array}{rcl}
2X +2Y+6Z& =& 0\\
4X+5Y+13Z& =& \ln(2) \\
2X+Y+7Z & =& \ln(3)
\end{array}\right. \equivaut 
\left\{ \begin{array}{rcl}
2X +2Y+6Z& =& 0\\
0+Y+Z& =& \ln(2) \\
-Y+Z & =& \ln(3)
\end{array}\right. \equivaut 
\left\{ \begin{array}{rcl}
2X +2Y+6Z& =& 0\\
Y+Z& =& \ln(2) \\
2 Z & =& \ln(3) +\ln(2)
\end{array}\right.$$
Une fois que le système est échelonné, on résout en remontant les lignes. 
On obtient : 
$$\left\{ \begin{array}{rcl}
2X +2Y+6Z& =& 0\\
Y+Z& =& \ln(2) \\
Z & =& \ln(\sqrt{6})
\end{array}\right. \equivaut 
\left\{ \begin{array}{rcl}
2X +2Y+6Z& =& 0\\
Y& =& \ln(\frac{2}{\sqrt{6}}) \\
Z & =& \ln(\sqrt{6})
\end{array}\right. \equivaut 
\left\{ \begin{array}{rcl}
2X & =& -\ln(\frac{4}{6})  -\ln(6^3)\\
Y& =& \ln(\frac{2}{\sqrt{6}}) \\
Z & =& \ln(\sqrt{6})
\end{array}\right.
$$


Soit encore  $X= \ln( \sqrt{\frac{1}{6^2 4}}) $ , $Y = \ln(\frac{2}{\sqrt{6}}) $ et $Z= \ln(\sqrt{6})$. 
D'où 

\conclusion{$x= \frac{1}{12}, \quad  y = \frac{2}{\sqrt{6}}  \quadet z= \sqrt{6}$}
\end{correction}