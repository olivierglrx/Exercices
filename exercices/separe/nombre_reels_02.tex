% Titre : nombre
% Filiere : BCPST
% Difficulte : 
% Type : TD 
% Categories :nombre
% Subcategories : 
% Keywords : nombre




\begin{exercice}
Résoudre l'équation pour $x\in \R$ de paramètre $a$  : 
$$\frac{1}{x-a} \geq x$$
\end{exercice}


\%\%\%\%\%\%\%\%\%\%\%\%\%\%\%\%\%\%\%\%
\%\%\%\%\%\%\%\%\%\%\%\%\%\%\%\%\%\%\%\%
\%\%\%\%\%\%\%\%\%\%\%\%\%\%\%\%\%\%\%\%




\begin{correction}
L'ensemble de définition est $D_a= \R\setminus \{ a\}$. 
On a pour tout $x\in D_a$ : 
\begin{eqnarray*}
(I(a))  &\Longleftrightarrow &\frac{1}{x-a}  -x  \geq 0\\
	&\Longleftrightarrow &\frac{1-x(x-a)}{x-a}  \geq 0\\
	&\Longleftrightarrow &\frac{- x^2 + ax+1}{x-a}  \geq 0\\
	&\Longleftrightarrow &\frac{x^2-ax-1}{x-a}  \leq 0 \\
\end{eqnarray*}
Le discriminant de $x^2-ax-1$ est $\Delta(a) = a^2+4>0$ pour tout $a\in \R$. 
Les racines sont 
$$r_+ (a) = \frac{a+\sqrt{a^2+4}}{2} \quad \text{ et} \quad r_- (a) = \frac{a-\sqrt{a^2+4}}{2} $$
On va résoudre
\begin{equation}\tag{$I_+$}
r_+(a) \geq a 
\end{equation} 
 et 
\begin{equation}\tag{$I_-$}
r_-(a) \geq a
\end{equation} 
Résolvons $(I_+)$
\begin{eqnarray}
r_+(a) \geq a &\Longleftrightarrow & \sqrt{a^2+4} \geq a
\end{eqnarray}
Si $a\geq 0$, $ r_+(a) \geq a \Longleftrightarrow  a^2+4 \geq a^2$ toujours vrai. Donc $a\geq 0 $ solution. \\
Si $a\leq 0$, $a$ est solution car $ \sqrt{a^2+4} \geq 0\geq a$
Les solutions de $(I_+)$ sont $S_+=\R$


Les solutions de $(I_-)$ sont $S_-=\emptyset$.


Les solutions de $I(a)$ sont donc données par (tableau de signes)
\conclusion{$]-\infty, r_-(a)]\cup ]a, r_+(a)]$)}

\end{correction}