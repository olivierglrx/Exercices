% Titre : suites
% Filiere : BCPST
% Difficulte : 
% Type : TD 
% Categories :suites
% Subcategories : 
% Keywords : suites




\begin{exercice} \;
Soient $(a_n)_{n\in\N}$ et $(b_n)_{n\in\N}$ deux suites telles que $a_0=0$, $b_0=1$ et pour tout $n\in\N$
$$a_{n+1}=-2a_n+b_n\qquad \hbox{et}\qquad b_{n+1}=3a_n.$$
\begin{enumerate}
\item D\'emontrer que la suite $(a_n+b_n)_{n\in\N}$ est constante.
\item Pour tout $n\in\N$, exprimer $a_n$ en fonction de $n$.
\item Pour tout $n\in\N$, d\'eterminer $b_n$ en fonction de $n$.
\end{enumerate}
\end{exercice}


\%\%\%\%\%\%\%\%\%\%\%\%\%\%\%\%\%\%\%\%
\%\%\%\%\%\%\%\%\%\%\%\%\%\%\%\%\%\%\%\%
\%\%\%\%\%\%\%\%\%\%\%\%\%\%\%\%\%\%\%\%





\begin{correction} \;
\textbf{Soient $\mathbf{(a_n)_{n\in\N}}$ et $\mathbf{(b_n)_{n\in\N}}$ deux suites telles que $\mathbf{a_0=0}$, $\mathbf{b_0=1}$ et pour tout $\mathbf{n\in\N}$}
$$\mathbf{a_{n+1}=-2a_n+b_n\qquad \hbox{et}\qquad b_{n+1}=3a_n.}$$
\begin{enumerate}
\item \textbf{D\'emontrer que la suite $\mathbf{(a_n+b_n)_{n\in\N}}$ est constante:}\\
\noindent Soit $n\in\N$, on a:
$$a_{n+1}+b_{n+1}=-2a_n+b_n+3a_n=a_n+b_n.$$
Ainsi \fbox{la suite $(a_n+b_n)_{n\in\N}$ est constante} et donc pour tout $n\in\N$: $a_n+b_n=a_0+b_0=1$. Donc \fbox{$\forall n\in\N,\ a_n+b_n=1$.}
\item \textbf{Pour tout $\mathbf{n\in\N}$, exprimer $\mathbf{a_n}$ en fonction de $\mathbf{n}$:}\\
\noindent Soit $n\in\N$. On a, en utilisant le fait que pour tout $n\in\N,\ a_n+b_n=1$, que pour tout $n\in\N$: $b_n=1-a_n$. Ainsi on obtient que pour tout $n\in\N$:
$$a_{n+1}=-2a_n+b_n \Leftrightarrow a_{n+1}=1-3a_n.$$
On reconna\^{i}t une suite arithm\'etico-g\'eom\'etrique .
\begin{itemize}
\item[$\bullet$] Calcul de la limite \'eventuelle: on r\'esout: $l=1-3l\Leftrightarrow l=\ddp\frac{1}{4}$.
\item[$\bullet$] \'Etude d'une suite auxiliaire: pour tout $n\in\N$, on pose $v_n=a_n-\ddp\frac{1}{4}$. Montrons que $\suitev$ est une suite g\'eom\'etrique de raison $-3$. Soit $n\in\N$, on a:
$$v_{n+1}=a_{n+1}-\ddp\frac{1}{4}=1-3a_n-\ddp\frac{1}{4}=-3\left(a_n-\ddp\frac{1}{4} \right)=-3v_n.$$
Ainsi la suite $\suitev$ est bien une suite g\'eom\'etrique de raison $\ddp\frac{1}{4}$ et de premier terme $v_0=a_0-\ddp\frac{1}{4}=-\ddp\frac{1}{4}$.
On en d\'eduit l'expression explicite de la suite $\suitev$: pour tout $n\in\N$, on a: $v_n=-\ddp\frac{1}{4}(-3)^n$.
\item[$\bullet$] Expression explicite de $a_n$ pour tout $n\in\N$:\\
\noindent Pour tout $n\in\N$, on a: $a_n=v_n+\ddp\frac{1}{4}=-\ddp\frac{1}{4}(-3)^n+\ddp\frac{1}{4}$.On a donc: \fbox{$\forall n\in\N,\ a_n=\ddp\frac{1}{4} \left( 1-(-3)^n  \right)$.}
\end{itemize}
\item \textbf{Pour tout $\mathbf{n\in\N}$, d\'eterminer $\mathbf{b_n}$ en fonction de $\mathbf{n}$:}\\
\noindent Comme pour tout $n\in\N$, on a: $b_{n+1}=3a_n$, on a: $b_n=3a_{n-1}$. Puis en utilisant le r\'esultat de la question pr\'ec\'edente, on obtient que \fbox{$\forall n\in\N,\ b_n=\ddp\frac{3}{4} \left( 1-(-3)^{n-1}  \right)$.}
\end{enumerate}
\end{correction}

