

\begin{correction} 
\begin{enumerate}
\item \textbf{R\'esolution dans $\mathbf{\R}$ de $\mathbf{\sqrt{x+1}=x-1}$:}\\
\begin{itemize}
\item[$\star$] Domaine de d\'efinition: $\mathcal{D}=\lbrack -1,+\infty\lbrack$
\item[$\star$] Attention, pour pouvoir \'elever au carr\'e, il faut que les termes des deux c\^ot\'es soient du m\^eme signe ! Il faut toujours faire des cas :
\begin{itemize}
\item[$\bullet$] Cas 1: si $x<1$: on ne peut pas \'elever au carr\'e.\\
\noindent Une racine carr\'ee \'etant toujours positive, on a $\mathcal{S}_1=\emptyset$.
\item[$\bullet$] Cas 2: si $x\geq 1$: on peut passer au carr\'ee dans l'\'egalit\'e tout en conservant l'\'equivalence, les deux membres \'etant positifs. On obtient comme r\'esultat $x=0$ ou $x=3$. Or, on est sous l'hypoth\`ese $x\geq 1$ donc $\mathcal{S}_2=\lbrace 3 \rbrace$.
\end{itemize}
Synth\`ese : on a $\mathcal{S} = \mathcal{S}_1 \cup \mathcal{S}_2$, soit : \fbox{$\mathcal{S}=\lbrace 3\rbrace$}.
\end{itemize} 
%---
\item \textbf{R\'esolution dans $\mathbf{\R}$ de $\mathbf{\sqrt{x+4}+\sqrt{x+2}\leq 1}$:}\\
\noindent \begin{itemize}
\item[$\star$] Domaine de d\'efinition: $\mathcal{D}=\lbrack -2,+\infty\lbrack$.
\item[$\star$] Les deux termes \'etant positifs, on peut passer au carr\'e dans l'in\'egalit\'e tout en conservant l'\'equivalence et on obtient 
$$\sqrt{x+4}+\sqrt{x+2}\leq 1 \Leftrightarrow \sqrt{(x+4)(x+2)}\leq -\ddp\frac{5}{2}-x.$$
Il faut ensuite faire deux cas :
\begin{itemize}
\item[$\bullet$] Cas 1: si $x>-\ddp\frac{5}{2}$: on ne peut pas \'elever au carr\'e.\\
Comme une racine est toujours sup\'erieure ou \'egale \`a 0, on obtient $\mathcal{S}_1=\emptyset$.
\item[$\bullet$] Cas 2 : si $x\leq -\ddp\frac{5}{2}$ : impossible car $\mathcal{D}=\lbrack -2,+\infty\lbrack$ donc $\mathcal{S}_2=\emptyset$.
\end{itemize} 
Synth\`ese : on a  \fbox{$\mathcal{S}=\emptyset$}.
\end{itemize} 
%---
\item \textbf{R\'esolution dans $\mathbf{\R}$ de $\mathbf{\sqrt{x^2-3}>5x-9}$:}\\
\noindent \begin{itemize}
\item[$\star$] Domaine de d\'efinition: $\mathcal{D}= \; \rbrack -\infty, -\sqrt{3}\rbrack\cup\lbrack \sqrt{3},+\infty\lbrack$.
\item[$\star$]  On fait deux cas pour \'elever au carr\'e : 
 \begin{itemize}
 \item[$\bullet$]  Cas 1: Si $x\leq \ddp\frac{9}{5}$ : on ne peut pas \'elever au carr\'e.\\
\noindent Une racine carr\'ee \'etant toujours positive ou nulle, on obtient $\mathcal{S}_1=\rbrack -\infty, -\sqrt{3}\rbrack\cup\left\lbrack \sqrt{3},\ddp\frac{9}{5}\right\rbrack$.
 \item[$\bullet$]  Cas 2: Si $x > \ddp\frac{9}{5}$.\\
\noindent Les deux termes de l'in\'equation sont alors positifs et on peut donc passer au carr\'e tout en conservant l'\'equivalence. On obtient 
$$\sqrt{x^2-3}>5x-9 \Leftrightarrow 4x^2-15x+14<0.$$
Les racines sont alors $\ddp\frac{7}{4}$ et $2$. L'ensemble solution est alors pour ce cas, en n'oubliant pas de regarder \`a la fois le domaine de d\'efinition et l'hypoth\`ese $x > \ddp\frac{9}{5}$, $\mathcal{S}_2=\left\rbrack \ddp\frac{9}{5}, 2\right\lbrack$.
\end{itemize} 
Synth\`ese : on a  \fbox{$\mathcal{S}=\rbrack -\infty, -\sqrt{3}\rbrack\cup\lbrack \sqrt{3},2\lbrack$}.
\end{itemize}  
%---
\item \textbf{R\'esolution dans $\mathbf{\R}$ de $\mathbf{e^x-1\geq \sqrt{e^{x+1}-e^x-e+1}}$:}\\
\noindent 
\begin{itemize}
\item[$\star$] Domaine de d\'efinition: l'in\'equation est bien d\'efinie si et seulement si 
$$e^{x+1}-e^x-e+1 \geq 0\Leftrightarrow (e-1)e^x-e+1\geq 0\Leftrightarrow e^x\geq \ddp\frac{e-1}{e-1}$$ 
car $e-1>0$. Ainsi on obtient que: $e^{x+1}-e^x-e+1 \geq 0\Leftrightarrow e^x\geq 1\Leftrightarrow x\geq 0$ en composant par la fonction $\ln{}$ qui est bien strictement croissante sur $\R^{+\star}$. Ainsi $\mathcal{D}=\R^+$.
\item[$\star$] On peut remarquer que sur $\mathcal{D}$, on a toujours $e^x-1\geq 0$. Ainsi les deux termes de l'in\'equation sont toujours positifs et on peut passer au carr\'e tout en conservant l'\'equivalence. On obtient que:
$$e^x-1\geq \sqrt{e^{x+1}-e^x-e+1} \Leftrightarrow e^{2x}-(1+e)e^x+e\geq 0.$$
On pose alors $X=e^x$ et on doit r\'esoudre $X^2-(1+e)X+e\geq 0$. Le discriminant vaut $\Delta=(e-1)^2$ et les racines sont 1 et $e$. Ainsi on obtient:
$$e^x-1\geq \sqrt{e^{x+1}-e^x-e+1} \Leftrightarrow e^x\leq 1\ \hbox{ou}\ e^x\geq e\Leftrightarrow x\leq 0\ \hbox{ou}\ x\geq 1$$
en composant par la fonction $\ln{}$ qui est strictement croissante sur $\R^{+\star}$. Comme $\mathcal{D}=\R^+$, on obtient que:
$$e^x-1\geq \sqrt{e^{x+1}-e^x-e+1} \Leftrightarrow x=0\ \hbox{ou}\ x\geq 1.$$
\item[$\star$] Conclusion: \fbox{$\mathcal{S}=\lbrack 1,+\infty\lbrack \; \cup \; \lbrace0\rbrace$}.
\end{itemize}

%--
\item \textbf{R\'esolution dans $\mathbf{\R}$ de $\mathbf{\sqrt{(x+3)(x-1)}\geq 2x-1}$:}\\
\noindent 
\begin{itemize}
\item[$\star$] Domaine de d\'efinition: l'in\'equation est bien d\'efinie si et seulement si $(x+3)(x-1) \geq 0$. Il s'agit d'un polyn\^{o}me de degr\'e 2 dont les racines sont $-3$ et $1$. Ainsi $\mathcal{D}= \; \rbrack -\infty, -3\rbrack\cup\lbrack 1,+\infty\lbrack$.
\item[$\star$] On \'etudie deux cas:
\begin{itemize}
\item[$\bullet$] Cas 1: si $2x-1<0\Leftrightarrow x<\ddp\demi$: on ne peut pas \'elever au carr\'e.\\
\noindent On se place donc sur $\rbrack -\infty, -3\rbrack$. Comme une racine carr\'ee est un nombre positif ou nul, elle est bien toujours sup\'erieure ou \'egale \`{a} un nombre strictement n\'egatif. Ainsi l'in\'egalit\'e est toujours v\'erifi\'ee sur cet ensemble et on obtient que $\mathcal{S}_1=\rbrack -\infty, -3\rbrack$.
\item[$\bullet$] Cas 2: si $2x-1\geq 0\Leftrightarrow x\geq \ddp\demi$.\\
\noindent On se place donc sur $\lbrack 1,+\infty \lbrack$. Les deux termes sont alors positifs et on peut donc passer au carr\'e tout en conservant l'\'equivalence. On obtient que:
$$\sqrt{(x+3)(x-1)}\geq 2x-1 \Leftrightarrow x^2+2x-3\geq 4x^2-4x+1\Leftrightarrow 3x^2-6x+4\leq 0.$$
Le discriminant vaut $\Delta=-12$ et ainsi pour tout $x$, on a: $3x^2-6x+4 >0$. Donc $\mathcal{S}_2=\emptyset$.
\end{itemize}
Synth\`ese : on a \fbox{$\mathcal{S}=\rbrack -\infty, -3\rbrack$}.
\end{itemize}
%---
%\item \textbf{R\'esolution dans $\mathbf{\R}$ de $\mathbf{x\sqrt{x}-4x-\sqrt{x}+4>0}$:}\\
%\noindent 
%\begin{itemize}
%\item[$\star$] Domaine de d\'efinition: L'in\'equation est bien d\'efinie sur $\mathcal{D}=\R^+$.
%\item[$\star$] On pose $X=\sqrt{x}$ et l'in\'equation revient \`{a}: $X^3-4X^2-X+4>0$. 1 est racine \'evidente et on peut donc factoriser par $X-1$. Par identification des coefficients d'un polyn\^{o}me, on obtient que: 
%$X^3-4X^2-X+4=(X-1)(X^2-3X-4)$. Ainsi: $X^3-4X^2-X+4>0 \Leftrightarrow (X-1)(X^2-3X-4)>0$. Le discriminant de $X^2-3X-4$ vaut $\Delta=25$ et les racines sont -1 et 4. Un tableau de signe donne dont: 
%$X^3-4X^2-X+4>0 \Leftrightarrow -1<X<1\ \hbox{ou}\ X>4$. Comme $X=\sqrt{x}$ et qu'une racine carr\'ee est un nombre toujours positif ou nul, on obtient que: 
%$x\sqrt{x}-4x-\sqrt{x}+4>0 \Leftrightarrow \sqrt{x}<1\ \hbox{ou}\ \sqrt{x}>4$. Dans les deux in\'egalit\'es, les deux termes sont bien positifs, on peut donc bien passer au carr\'ee tout en conservant l'\'equivalence et on obtient ainsi:
%$$  x\sqrt{x}-4x-\sqrt{x}+4>0 \Leftrightarrow x<1\ \hbox{ou}\ x>16.$$
%\item[$\star$] Conclusion: \fbox{$\mathcal{S}=\lbrack 0,1\lbrack\cup\rbrack 16,+\infty\lbrack$} en prenant en compte le domaine de r\'esolution.
%\end{itemize}
%---
\item \textbf{R\'esolution dans $\mathbf{\R}$ de $\mathbf{\sqrt{x+4}-\sqrt{x+2}=1}$:}\\
\noindent \begin{itemize}
\item[$\star$] Domaine de d\'efinition $\mathcal{D}=\lbrack -2,+\infty\lbrack$ 
\item[$\star$] On utilise ici la forme conjugu\'ee, c'est-\`a-dire on multiplie le num\'erateur et le d\'enominateur par la quantit\'e strictement positive $\sqrt{x+4}+\sqrt{x+2}$. On obtient alors l'\'equation \'equivalente \`a r\'esoudre
$$\ddp\frac{2}{\sqrt{x+4}+\sqrt{x+2}}=1\Leftrightarrow 2=\sqrt{x+4}+\sqrt{x+2}.$$
On est ainsi ramen\'e \`a pratiquement la m\^eme \'equation que tout \`a l'heure que je vous laisse r\'esoudre. On obtient \fbox{$\mathcal{S}=\left\lbrace -\ddp\frac{7}{4}\right\rbrace$.}
\end{itemize}
%\item \textbf{R\'esolution dans $\mathbf{\R}$ de $\mathbf{x+1>\sqrt{x^2+2x}}$:}\\
%\noindent \begin{itemize}
%\item[$\star$] Domaine de d\'efinition: $\mathcal{D}=\rbrack -\infty, -2\rbrack\cup\lbrack 0,+\infty\lbrack$ 
%\item[$\star$] Cas 1: Si $x\leq -1$:\\
%\noindent Une racine carr\'ee \'etant toujours positive ou nulle, on obtient $\mathcal{S}_1=\emptyset$.
%\item[$\star$] Si $x>-1$:\\
%\noindent Les deux termes de l'in\'egalit\'e sont alors positifs et on peut donc passer au carr\'e tout en conservant l'\'equivalence. On obtient l'in\'egalit\'e \'equivalente suivante $1>0$ qui est toujours v\'erifi\'ee. Ainsi, l'ensemble des r\'eels qui sont dans le domaine de d\'efinition et qui v\'erifie 'hypoth\`ese $x>-1$ sont solutions et donc $\mathcal{S}_2=\R^+$.
%\item[$\star$] Conclusion: \fbox{$\mathcal{S}=\R^+$.}
%\end{itemize}  
\item \textbf{R\'esolution dans $\mathbf{\R}$ de $\mathbf{1\leq \left(\ddp\frac{x-3}{x-1}   \right)^2\leq 9}$:}\\
\noindent \begin{itemize}
\item[$\star$] Domaine de d\'efinition $\mathcal{D}=\R\setminus\lbrace 1 \rbrace$.
\item[$\star$] La racine carr\'ee est une fonction strictement croissante sur $\R^+$ et tous les termes de l'in\'equation sont bien positifs, on peut donc composer par la racine carr\'ee. On obtient (ATTENTION $\sqrt{a^2}=|a|$),
$$1\leq \left| \ddp\frac{x-3}{x-1} \right|\leq 3.$$ 
On fait alors deux cas pour enlever les valeurs absolues :
\noindent \begin{itemize}
\item[$\bullet$] Cas 1: $\ddp\frac{x-3}{x-1} \geq 0\Leftrightarrow x\in\rbrack -\infty, 1\lbrack\cup\lbrack 3,+\infty \lbrack$.\\
\noindent On doit alors r\'esoudre l'in\'equation $1\leq \ddp\frac{x-3}{x-1}\leq 3$. Les r\'eels $x$ doivent donc v\'erifier $1\leq \ddp\frac{x-3}{x-1}$ et $\ddp\frac{x-3}{x-1}\leq 3$. La r\'esolution de la premi\`ere in\'equation donne
$$\ddp\frac{x-3}{x-1}\geq 1 \; \Leftrightarrow \;  \ddp\frac{-2}{x-1}\geq 0 \; \Leftrightarrow \; x-1\leq 0.$$
Le premier ensemble solution est ainsi $\rbrack -\infty,1\rbrack$. La deuxi\`eme in\'equation donne
$$\ddp\frac{x-3}{x-1}\leq 3 \; \Leftrightarrow \; \ddp\frac{x-3}{x-1}- 3 \leq 0 \; \Leftrightarrow \; \ddp\frac{-2x}{x-1}\leq 0.$$
Un tableau de signe donne alors que le deuxi\`eme ensemble solution est alors $\rbrack -\infty, 0\rbrack\cup \lbrack 3,+\infty\lbrack$.\\
\noindent On obtient ainsi, en faisant l'intersection de ces deux ensembles et en v\'erifiant qu'on est bien aussi dans $\rbrack -\infty, 1\lbrack\cup\lbrack 3,+\infty \lbrack$, que $\mathcal{S}_1=\rbrack -\infty, 0\rbrack$.

\item[$\bullet$] Cas 2: $\ddp\frac{x-3}{x-1} \leq 0\Leftrightarrow x\in\rbrack  1, 3 \rbrack$.\\
\noindent On doit alors r\'esoudre l'in\'equation $1\leq -\ddp\frac{x-3}{x-1}\leq 3$. Les r\'eels $x$ doivent donc v\'erifier $1\leq \ddp\frac{3-x}{x-1}$ et $\ddp\frac{3-x}{x-1}\leq 3$. La r\'esolution de la premi\`ere in\'equation donne
$$\begin{array}{lll}
\ddp\frac{3-x}{x-1}\geq 1 &\Leftrightarrow & \ddp\frac{-2x+4}{x-1}\geq 0.
\end{array}$$
Un tableau de signe permet de trouver le premier ensemble de d\'efinition.
Le premier ensemble solution est ainsi $\rbrack 1,2\rbrack$. La deuxi\`eme in\'equation donne
$$\ddp\frac{3-x}{x-1}\leq 3 \; \Leftrightarrow \; \ddp\frac{3-x}{x-1}- 3\leq 0 \; \Leftrightarrow \; \ddp\frac{-2x+3}{x-1}\leq 0.$$
Un tableau de signe donne alors que le deuxi\`eme ensemble solution est alors $\left \lbrack \ddp\frac{3}{2},3\right\rbrack$.\\
\noindent On obtient ainsi, en faisant l'intersection de ces deux ensembles et en v\'erifiant qu'on est bien aussi dans $\rbrack  1, 3 \rbrack$, que $\mathcal{S}_2=\left\lbrack \ddp\frac{3}{2},2\right\rbrack$.
\end{itemize}
Synth\`ese : l'ensemble des solutions correspond alors \`a la r\'eunion des deux sous-ensembles $\mathcal{S}_1$ et $\mathcal{S}_2$. Ainsi, on obtient : \fbox{$ \mathcal{S}= \rbrack -\infty, 0\rbrack\cup \left\lbrack \ddp\frac{3}{2},2\right\rbrack$}.
\end{itemize} 
%---
\item \textbf{R\'esolution dans $\mathbf{\R}$ de $\mathbf{\sqrt{9^x-1}>3^x-2}$:}\\
\noindent 
\noindent \begin{itemize}
\item[$\star$] Domaine de d\'efinition: L'in\'equation est bien d\'efinie si: $9^x-1\geq0\Leftrightarrow e^{x\ln{9}}\geq 1\Leftrightarrow x\ln{9}\geq 0$ en composant par la fonction $\ln{}$ qui est bien strictement croissante sur $\R^{+\star}$. Ainsi comme $\ln{9}>0$, on obtient que: $\mathcal{D}=\R^+$.
\item[$\star$]  On doit ensuite \'etudier deux cas:
\begin{itemize}
\item[$\bullet$] Cas 1: si $3^x-2< 0\Leftrightarrow x<\ddp\frac{\ln{2}}{\ln{3}} $ : on ne peut pas \'elever au carr\'e.\\
\noindent On se place donc sur $\left\lbrack 0,\ddp\frac{\ln{2}}{\ln{3}} \right\lbrack$. Sur cet ensemble l'in\'equation est toujours v\'erifi\'ee car une racine carr\'ee est un nombre positif ou nul et elle est donc bien toujours sup\'erieure ou \'egale \`{a} un  nombre strictement n\'egatif. Donc $\mathcal{S}_1=\left\lbrack 0,\ddp\frac{\ln{2}}{\ln{3}} \right\lbrack$.
\item[$\bullet$] Cas 2: si $3^x-2\geq  0\Leftrightarrow x\geq \ddp\frac{\ln{2}}{\ln{3}} $.\\
\noindent On se place donc sur $\left\lbrack \ddp\frac{\ln{2}}{\ln{3}},+\infty \right\lbrack$. Sur cet ensemble, les deux termes sont positifs et on peut donc passer au carr\'e tout en conservant l'\'equivalence. Ainsi on obtient que:
$$\sqrt{9^x-1}>3^x-2 \Leftrightarrow 9^x-1>3^{2x}-4\times 3^x+4\Leftrightarrow 4\times 3^x-5>0\Leftrightarrow x>\ddp\frac{\ln{\left( \frac{5}{4} \right)}}{\ln{3}}$$
en composant par la fonction $\ln{}$ qui est strictement croissante sur $\R^{+\star}$ et car $\ln{3}>0$. Or on est sur 
$\left\lbrack \ddp\frac{\ln{2}}{\ln{3}},+\infty \right\lbrack$ et comme $\ddp\frac{5}{4}<2$ et que la fonction $\ln{}$ est strictement croissante, on obtient que $\ddp\frac{  \ln{\left( \frac{5}{4} \right)}  }{\ln{3}}<\ddp\frac{\ln{2}}{\ln{3}}$, on obtient que: $\mathcal{S}_2=\left\lbrack \ddp\frac{\ln{2}}{\ln{3}},+\infty \right\lbrack$.
\end{itemize}
Synth\`ese : \fbox{$\mathcal{S}= \R^+ $}.
\end{itemize}
\end{enumerate}
\end{correction}