
\begin{correction} \;
Pour toutes ces suites, on conjecture le r\'esultat en it\'erant la relation de r\'ecurrence puis on le d\'emontre rigoureusement par r\'ecurrence. Je ne fais pas ici la r\'ecurrence mais elle doit \^etre pr\'esente dans toute copie. Je ne donne ici que le r\'esultat, \`a savoir $u_n$ en fonction de $n$.
\begin{enumerate}
% \item $\forall n\in\N,\quad u_n=u_0^{2^n}$. 
\item $\forall n\in\N,\quad u_n=\ddp\frac{3^{n-1}}{2^{n-1}}nu_1=\ddp\frac{3^{n-1} }{2^{n-1}}n.$
\item M\'ethode 1 : on conjecture que $\forall n\in\N,\,u_n=2\times 2^3\times 2^{3^2}\times\dots\times 2^{3^{n-1}} u_0^{3^n}=2^{\sum\limits_{k=0}^{n} 3^k } = 2^{\frac{3^{n+1}-1}{2}}$ et on fait une r\'ecurrence.\\
M\'ethode 2 : on pose $u_n = 2^{v_n}$, et on essaye de calculer la suite $\suitev$. On a $u_0=2=2^1$, donc $v_0=1$. De plus, on a :
$$u_{n+1} =2(u_n)^3 \; \Leftrightarrow \;  2^{v_{n+1}} =  2\times (2^{v_n})^3 \; \Leftrightarrow \; 2^{v_{n+1}} =  2^{3v_n+1} \; \Leftrightarrow \; v_{n+1} = 3v_{n}+1.$$
On en d\'eduit que $\suitev$ est une suite arithm\'etico-g\'eom\'etrique. La m\'ethode habituelle donne ensuite $v_n=\ddp\frac{3}{2}\times\frac{3^{n}}{2}-\frac{1}{2}$, soit $u_n= 2^{\frac{3^{n+1}-1}{2}}$.
\end{enumerate}
\end{correction}