% Titre : Etude de $ f(x) = \ln\left(\sqrt{\frac{1}{2}+ \sin(x)}\right)$
% Filiere : BCPST
% Difficulte :
% Type : DS, DM
% Categories : analyse
% Subcategories : 
% Keywords : analyse



\begin{exercice}
\begin{enumerate}
\item Résoudre $\sin(x) \geq \frac{-1}{2} $ sur $[0,2\pi],$ puis sur $\R$
\item Donner l'ensemble de définition et de dérivabilité de $f$ définie par 
$$ f(x) = \ln\left(\sqrt{\frac{1}{2}+ \sin(x)}\right)$$
\item Rappeler la formule de dérivée d'une composée $(f\circ g) '$. 
\item Calculer la dérivée de $f$ sur son ensemble de dérivabililité. 
\item Calculer l'équation de la tangente à la courbe représentative de $f$ en $\frac{\pi}{6}$. 
\item On rappelle que la fonction $a\% b$ en Python renvoie le reste de la division de $a$ par $b$, c'ets à dire l'unique réel $r$ entre $[0,b[$ tel qu'il existe $k\in \Z$ vérifiant $a= kb+r$. Cette fonction  peut prendre des paramètres $a,b$ réels, pas nécessairement entier. 
\begin{enumerate}
\item Ecrire une fonction Python \texttt{reste} qui prend en paramètre un réel $x$ et qui retourne son reste modulo $2\pi$. 
\item Ecrire une fonction python  \texttt{definition} qui prend en paramètre un réel $x$ et renvoi $1$ si $x\in D_f$ et $0$ sinon. 
\item Ecrire une fonction python \texttt{f} qui prend en parmètre un réel $x$, qui renvoie un message d'erreur si $x\notin D_f$ et retourne la valeur de $f(x)$ sinon.  
\end{enumerate}
\end{enumerate}

\end{exercice}
\begin{correction}
\begin{enumerate}
\item Sur $[0,2\pi]$ les solutions sont $S_0=[0,\frac{7\pi}{6}]\cup [\frac{11\pi}{6},2\pi]$. Sur $\R$, les solutions sont $S=\bigcup_{k\in \Z} [2k\pi,\frac{7\pi}{6}+2k\pi]\cup [\frac{11\pi}{6}+2k\pi,2\pi+2k\pi]$
\item Pour que $f(x)$ soit défini il faut que : 

\begin{itemize}
\item  $\frac{1}{2}+\sin(x) \geq 0$ (racine défini sur $R_+$.)
\item  $\sqrt{(\frac{1}{2}+\sin(x) )} >0$  (ln définie sur $\R_+^*$. )
\end{itemize}
Ces equations donnent $$D_f = \bigcup_{k\in \Z} [2k\pi,\frac{7\pi}{6}+2k\pi[\cup ]\frac{11\pi}{6}+2k\pi,2\pi+2k\pi]=\bigcup_{k\in \Z} ]\frac{-\pi}{6}+2k\pi,\frac{7\pi}{6}+2k\pi[ $$

La fonction est dérivable sur le même ensemble $D_f=D_{f'}$

\item $(f\circ g)' = g' \times f'\circ g$. 
\item Pour tout $x\in D_{f}$ on a :
$f(x) = \frac{1}{2}\ln(\frac{1}{2}+\sin(x))$ et donc
$$f'(x) = \frac{1}{2} \cos(x) \frac{1}{\frac{1}{2}+\sin(x)}= \frac{\cos(x)}{1+2\sin(x)}$$
\item LA fonction est dérivable en $\frac{\pi}{6}$, donc la courbe admet une tangente en ce point.  L'équation de la tangente est $y-f(\frac{\pi}{6}) =f'(\frac{\pi}{6}) (x-\frac{\pi}{6}) $ Ce qui donne : 
$$ y =\frac{\sqrt{3}}{4}(x-\frac{\pi}{6})$$ 


\item
\begin{lstlisting}
from math import *
def reste(x):
  r=x%2*pi
  return(r)
  
def definition1(x):
  r=reste(x)
  if  0<= r<7*pi/6 or  11*pi/6<r:
    return(1)
  else:
    return(0)

def definition2(x): #autre solution
  if  sin(x)<1/2:
    return(1)
  else:
    return(0)
    
def f(x):
  if definition1(x)==1:
    return(log(sqrt( 1/2 +sin(x)))
  else:
    return('x n est pas dans le domaine de defiition')
   
\end{lstlisting}
\end{enumerate}
\end{correction}