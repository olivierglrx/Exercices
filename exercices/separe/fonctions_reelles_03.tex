% Titre : fonctions
% Filiere : BCPST
% Difficulte : 
% Type : TD 
% Categories :fonctions
% Subcategories : 
% Keywords : fonctions




\begin{exercice}  \; Calculer $f\circ g$ et $g\circ f$ apr\`es avoir indiqu\'e pour quels r\'eels cela a un sens :
\begin{enumerate}
\item $f:\ x\mapsto 2x^2-x+1\quad \hbox{et}\quad g:\ x\mapsto 2\sqrt{x-3}$.
\item $f:\ x\mapsto \ddp\frac{2x^2-8}{x}\quad\hbox{et}\quad g:\ x\mapsto x+\ddp\frac{1}{x}.$
\end{enumerate}
\end{exercice}


\%\%\%\%\%\%\%\%\%\%\%\%\%\%\%\%\%\%\%\%
\%\%\%\%\%\%\%\%\%\%\%\%\%\%\%\%\%\%\%\%
\%\%\%\%\%\%\%\%\%\%\%\%\%\%\%\%\%\%\%\%



\begin{correction}  \;
\begin{enumerate}
\item
\begin{itemize}
\item[$\bullet$] \'Etude de $f\circ g$:
\begin{itemize}
\item[$\star$] Domaine de d\'efinition: La fonction $f\circ g$ est bien d\'efinie si et seulement si $x\in\mathcal{D}_g$ et $g(x)\in\mathcal{D}_f$. Comme $\mathcal{D}_f=\R$, la fonction $f\circ g$ est bien d\'efinie si et seulement si $x\in\mathcal{D}_g$, \`{a} savoir si et seulement si $x-3\geq 0\Leftrightarrow x\geq 3$. Ainsi on obtient: \fbox{$\mathcal{D}_{f\circ  g}=\lbrack 3,+\infty\lbrack$}. 
\item[$\star$] Expression: Pour tout $x\geq 3$, on a: $f\circ g(x)=f\lbrack g(x)\rbrack= 2(g(x))^2-(g(x))+1=8(x-3)-2\sqrt{x-3}+1=8x-23-2\sqrt{x-3}$.
\end{itemize}
\item[$\bullet$] \'Etude de $g\circ f$:
\begin{itemize}
\item[$\star$] Domaine de d\'efinition: La fonction $g\circ f$ est bien d\'efinie si et seulement si $x\in\mathcal{D}_f$ et $f(x)\in\mathcal{D}_g$. Comme $\mathcal{D}_f=\R$, la fonction $g\circ f$ est bien d\'efinie si et seulement si $f(x)\in\mathcal{D}_g$, \`{a} savoir si et seulement si $f(x)-3\geq 0\Leftrightarrow 2x^2-x-2\geq 0$. Le discriminant vaut $\Delta=17$ et les deux racines sont $\ddp\frac{1-\sqrt{17}}{4}$ et $\ddp\frac{1+\sqrt{17}}{4}$. Ainsi on obtient: \fbox{$\mathcal{D}_{g\circ  f}=\left\rbrack -\infty,\ddp\frac{1-\sqrt{17}}{4}\right\rbrack\cup\left\lbrack \ddp\frac{1+\sqrt{17}}{4},+\infty\right\lbrack$}.
\item[$\star$] Expression: Pour tout $x\in\mathcal{D}_{g\circ f}$, on a: $g\circ f(x)=g\lbrack f(x)\rbrack=\sqrt{2x^2-x-2}$.
\end{itemize}
\end{itemize}
\item
\begin{itemize}
\item[$\bullet$] \'Etude de $f\circ g$:
\begin{itemize}
\item[$\star$] Domaine de d\'efinition: La fonction $f\circ g$ est bien d\'efinie si et seulement si $x\in\mathcal{D}_g$ et $g(x)\in\mathcal{D}_f$. Comme $\mathcal{D}_f=\R^{\star}=\mathcal{D}_g$, la fonction $f\circ g$ est bien d\'efinie si et seulement si $x\not= 0$ et $g(x)\not= 0$. On a: $g(x)\not= 0\Leftrightarrow \ddp\frac{x^2+1}{x}\not= 0\Leftrightarrow x^2+1\not= 0$: toujours vrai. Ainsi \fbox{$\mathcal{D}_{f\circ  g}=\R^{\star}$}.
\item[$\star$] Expression: Pour tout $x\not= 0$, on a:
 $f\circ g(x)=f\lbrack g(x)\rbrack=\ddp\frac{2(g(x))^2-8}{g(x)}=\ddp\frac{2\left( \frac{x^2+1}{x}\right)^2-8}{\frac{x^2+1}{x}}=\ddp\frac{2(x^2+1)^2-8x^2}{x^2}\times \ddp\frac{x}{x^2+1}=\ddp\frac{2x^4-4x^2+2}{x(x^2+1)}$.
\end{itemize}
\item[$\bullet$] \'Etude de $g\circ f$: \begin{itemize}
\item[$\star$] Domaine de d\'efinition: La fonction $g\circ f$ est bien d\'efinie si et seulement si $x\in\mathcal{D}_f$ et $f(x)\in\mathcal{D}_g$. Comme $\mathcal{D}_f=\R^{\star}=\mathcal{D}_g$, la fonction $f\circ g$ est bien d\'efinie si et seulement si $x\not= 0$ et $f(x)\not= 0$. On a: $f(x)\not= 0\Leftrightarrow 2x^2-8\not= 0\Leftrightarrow x=2\ \hbox{ou}\ x=-2$. Ainsi \fbox{$\mathcal{D}_{g\circ  f}=\R\setminus\lbrace  -2,0,2\rbrace$}.
\item[$\star$] Expression: Soit $x\in\R\setminus\lbrace  -2,0,2\rbrace$, on a: 
$g\circ f(x)=g\lbrack f(x)\rbrack=f(x)+\ddp\frac{1}{f(x)}=\ddp\frac{2x^2-8}{x}+\ddp\frac{x}{2x^2-8}=\ddp\frac{ 4x^4-31x^2+64 }{x(2x^2-8)}$.
\end{itemize}
\end{itemize}
\end{enumerate}
\end{correction}
%--------------------------------------------------
%------------------------------------------------
