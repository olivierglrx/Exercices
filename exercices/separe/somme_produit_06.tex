
\begin{exercice}  \;  \textbf{Sommes et d\'erivation:} 
Soit $n\in\N^{\star}$ et $S=\ddp \sum\limits_{k=1}^n k\ddp \binom{n}{k}$. %Calculons $S$ en utilisant une m\'ethode par d\'erivation terme \`a terme.
%\begin{enumerate}
% \item M\'ethode 1: Avec la formule des chefs.\\
%\noindent Calculer $S$ directement en utilisant une propri\'et\'e des coefficients bin\^omiaux. \\
%\noindent De la m\^eme fa\c{c}on, calculer alors $T=\ddp \sum\limits_{k=1}^n k(k-1)\ddp \binom{n}{k}$ puis $\ddp \sum\limits_{k=1}^n k^2\ddp \binom{n}{k}$ (on pourra \'ecrire que $k^2=k(k-1)+k$).
%\item M\'ethode 2: En d\'erivant.
\begin{enumerate}
\item
On pose, pour tout $x$ dans $\bR$, $f(x)=\ddp \sum\limits_{k=0}^n \ddp \binom{n}{k}x^k$. Calculer $f(x)$.
\item 
En d\'eduire, pour tout $x$ dans $\bR$, la valeur de $g(x)=\ddp \sum\limits_{k=1}^n k\ddp \binom{n}{k}x^{k-1}$, puis en d\'eduire $S$.
\end{enumerate}
%\end{enumerate}
\end{exercice}