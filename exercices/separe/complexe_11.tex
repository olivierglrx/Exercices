% Titre : complexe
% Filiere : BCPST
% Difficulte : 
% Type : TD 
% Categories :complexe
% Subcategories : 
% Keywords : complexe




\begin{exercice}  \;
R\'esoudre dans $\bC$ les \'equations suivantes et mettre les solutions sous forme exponentielle.
\begin{enumerate}
\begin{minipage}[t]{0.45\textwidth}
\item $z^2=i$
\item $z^3=i$
\item $z^4+4=0$ 
\end{minipage}
\begin{minipage}[t]{0.45\textwidth}
\item $z^2=3-4i$
\item $z^4=j$ (on rappelle que $j=e^{\frac{2i\pi}{3}}$).
\end{minipage}
\end{enumerate}
\end{exercice}


\%\%\%\%\%\%\%\%\%\%\%\%\%\%\%\%\%\%\%\%
\%\%\%\%\%\%\%\%\%\%\%\%\%\%\%\%\%\%\%\%
\%\%\%\%\%\%\%\%\%\%\%\%\%\%\%\%\%\%\%\%




\begin{correction}   \;
\begin{enumerate}
 \item \textbf{R\'esolution de $\mathbf{z^2=i}$ :}
Comme 0 n'est pas solution, on cherche les solutions $z$ sous la forme exponentielle $z=re^{i\theta}$ avec $r>0$ et $\theta\in\R$.
$$ 
z^2=i \Leftrightarrow  r^2e^{2i\theta}=e^{i\frac{\pi}{2}}
\Leftrightarrow  \left\lbrace\begin{array}{l}
r^2=1\vsec\\
\exists k\in\Z,\ 2\theta=\ddp\frac{\pi}{2}+2k\pi
\end{array}\right. 
\Leftrightarrow  \left\lbrace\begin{array}{l}
r=1\vsec\\
\exists k\in\Z,\ \theta=\ddp\frac{\pi}{4}+k\pi.
\end{array}\right. 
$$
Ainsi,
$$\fbox{$\mathcal{S}=\left\lbrace  e^{i\frac{\pi}{4}},e^{i\frac{5\pi}{4}} \right\rbrace=\left\lbrace 
\ddp\frac{1+i}{\sqrt{2}},\ddp\frac{-1-i}{\sqrt{2}}  \right\rbrace   .$}$$
%----------------------------------------------------
 \item \textbf{R\'esolution de $\mathbf{z^3=i}$ :}
 Comme 0 n'est pas solution, on cherche les solutions $z$ sous la forme exponentielle $z=re^{i\theta}$ avec $r>0$ et $\theta\in\R$.
$$ 
z^3=i \Leftrightarrow  r^3e^{3i\theta}=e^{i\frac{\pi}{2}}
\Leftrightarrow  \left\lbrace\begin{array}{l}
r^3=1\vsec\\
\exists k\in\Z,\ 3\theta=\ddp\frac{\pi}{2}+2k\pi
\end{array}\right. 
\Leftrightarrow  \left\lbrace\begin{array}{l}
r=1\vsec\\
\exists k\in\Z,\ \theta=\ddp\frac{\pi}{6}+\ddp\frac{2k\pi}{3}.
\end{array}\right. 
$$
Ainsi,
$$\fbox{$\mathcal{S}=\left\lbrace  e^{i\frac{\pi}{6}},e^{i\frac{5\pi}{6}},e^{i\frac{3\pi}{2}} \right\rbrace=\left\lbrace 
\ddp\frac{\sqrt{3}+i}{2},\ddp\frac{-\sqrt{3}+i}{2},-i  \right\rbrace   .$}$$
%------------------------------------------------------
\item  \textbf{R\'esolution de $\mathbf{z^4=-4}$ :}
0 n'est pas solution, on cherche donc les solutions sous la forme $z=re^{i\theta}$ avec $r>0$ et $\theta\in\R$.
On obtient alors
$$
z^4=-4 \Leftrightarrow  r^4e^{4i\theta}=4e^{i\pi}
\Leftrightarrow  \left\lbrace\begin{array}{lll}
r^4=4\vsec\\
\exists k\in\Z,\ 4\theta=\pi+2k\pi
\end{array} \right.
\Leftrightarrow  \left\lbrace\begin{array}{lll}
r=\sqrt{2}\vsec\\
\exists k\in\Z,\ \theta=\ddp\frac{\pi}{4}+\ddp\frac{k\pi}{2}.
\end{array} \right.$$
Ainsi, les solutions sont $\fbox{$ \mathcal{S}=\left\lbrace \sqrt{2}e^{i\frac{\pi}{4}},\sqrt{2}e^{i\frac{3\pi}{4}},\sqrt{2}e^{-i\frac{3\pi}{4}} ,\sqrt{2}e^{-i\frac{\pi}{4}}\right\rbrace .$}$
%-----------------------------------------------------
\item \textbf{R\'esolution de $\mathbf{z^2=3-4i}$ :}
\begin{itemize}
 \item[$\bullet$]
On commence par essayer d'appliquer la m\'ethode du cours et on cherche donc \`a mettre $3-4i$ sous forme exponentielle. On ne trouve pas de forme exponentielle simple. Pour les racines secondes d'un nombre complexe, il existe aussi une autre m\'ethode qui utilise la forme alg\'ebrique. 
\item[$\bullet$]  M\'ethode avec la forme alg\'ebrique pour les racines SECONDES d'un nombre complexe.\\
On cherche donc $z$ sous la forme $z=x+iy$. On obtient donc
$$z^2=3-4i\Leftrightarrow (x+iy)^2=3-4i  \Leftrightarrow x^2-y^2+2xy i = 3-4i \Leftrightarrow \left\lbrace\begin{array}{l}
x^2-y^2=3\vsec\\
2xy=-4.
\end{array}\right.$$
On obtient donc, car $xy=-2$, donc $x\not=0$ :
$$\begin{array}{lllll}
z^2=3-4i&\Leftrightarrow& \left\lbrace\begin{array}{l}
x^2-y^2=3\vsec\\
y=-\ddp\frac{2}{x}
\end{array}\right.
\Leftrightarrow 
\left\lbrace\begin{array}{ll}
\ddp x^2 - \frac{4}{x^2}=3&\vsec\\
y=-\ddp\frac{2}{x}
\end{array}\right.
&\Leftrightarrow & 
\left\lbrace\begin{array}{lll}
x^4-3x^2-4 = 0\vsec\\
y=-\ddp\frac{2}{x}
\end{array}\right.
\end{array}$$
Dans la premi\`ere \'equation on pose $X=x^2$. On obtient alors : $X^2-3X-4=0$, dont les solutions sont $X_1=-1$ et $X_2=4$. On revient \`a $x$ : on a $x^2=-1$ qui est impossible, ou $x^2=4 \Leftrightarrow x=2$ ou $x=-2$. On en d\'eduit dont gr\^ace \`a la deuxi\`eme \'equation :
$$z^2=3-4i \Leftrightarrow \left\{\begin{array}{rcr} x&=&2\\b&=&-1\end{array}\right. \textmd{ ou } \left\{\begin{array}{rcr} x&=&-2\\b&=&1\end{array}\right. $$
Ainsi, les solutions sont $\fbox{$ \mathcal{S}=\left\lbrace  2-i, -2+i\right\rbrace .$}$
\end{itemize}
%---------------------------------------------
\item \textbf{R\'esolution de $\mathbf{z^4=j}$ :}
0 n'est pas solution, on cherche donc les solutions $z$ sous la forme $z=re^{i\theta}$ avec $r>0$ et $\theta\in\R$. On obtient
$$
z^4=j \Leftrightarrow  r^4e^{4i\theta}=j
\Leftrightarrow  \left\lbrace\begin{array}{l}
r^4=1\vsec\\
\exists k\in\Z,\ 4\theta=\ddp\frac{2\pi}{3}+2k\pi
\end{array}\right.
\Leftrightarrow  \left\lbrace\begin{array}{l}
r=1\vsec\\
\exists k\in\Z,\ \theta=\ddp\frac{\pi}{6}+\ddp\frac{k\pi}{2}.
\end{array}\right.
$$
Ainsi, $\fbox{$ \mathcal{S}=\left\lbrace  e^{i\frac{\pi}{6}},e^{i\frac{2\pi}{3}},e^{i\frac{-5\pi}{6}},e^{i\frac{-\pi}{3}}  \right\rbrace .$}$
%----------------------------------------------------
\end{enumerate}
\end{correction}