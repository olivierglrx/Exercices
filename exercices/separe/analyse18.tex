% Titre : $u_{n+1}=\sin(u_n)$ (Pb)
% Filiere : BCPST
% Difficulte :
% Type : DS, DM
% Categories : analyse
% Subcategories : 
% Keywords : analyse




\begin{exercice}
Soit $\suite{u}$ la suite définie par 
$$\left\{ 
\begin{array}{ccl}
u_0&=&1\\
u_{n+1} &=& \sin(u_n)
\end{array}
\right.$$

\begin{enumerate}
\item Montrer que pour tout $n\in \N$, $0<u_n<\frac{\pi}{2}$.
\item On note $f(x) = \sin(x)-x$. Montrer que pour tout $x\in \R_+^*$, $f(x)<0.$
\item En déduire le sens de variation de $\suite{u}$.
%\item Montrer que $\suite{u}$ est bornée.
\item En déduire que $\suite{u}$ converge vers $\ell \in \R$
\item  Montrer que $f(x)=0 \equivaut x=0$.
\item Déterminer la valeur de $\ell$. 
\end{enumerate}

Info 
\begin{enumerate}
\item Ecrire une fonction qui prend en paramètre $n\in \N$ et qui retourne la valeur de $u_n$. (Pour ceux qui n'ont pas encore vu les fonctions, vous pouvez écrire un script qui demande à l'utilisateur la valeur de $n$ souhaité et qui retourne la valeur de $u_n$ sans les fonctions, mais bon c'est pas si différent... ) 
\item 
Ecrire une fonction qui prend en paramètre $e\in \R^+$ et qui retourne la valeur du premier terme $n_0\in \N$ telle que $|u_{n_0}-\ell| \leq e$ et la valeur de $u_{n_0}$. (même remarque) 
\end{enumerate}

\end{exercice}

\begin{correction}
\begin{enumerate}
\item On fait une récurrence. 
Pour tout $n\in \N$ on note  $P(n)$ la propriété définie par:  $" 0<u_n<\frac{\pi}{2}"$
Par définition $u_0= 1$, et on a bien $0<1<\frac{\pi}{2}$ (car $\pi>3$) 
Donc la propriété $P$ est vraie au rang $0$.  

 On suppose qu'il existe $n_0\in \N$ tel que $P_{n_0}$ soit vraie et on  va montrer que ceci implique $P_{n_0+1}$ 

En effet, pour tout $x\in ]0,\frac{\pi}{2}[,$ $\sin(x) \in ]0,1[\subset ]0,\pi/2[$ \footnote{en d'autres termes, $]0,\pi/2[ $ est stable par la fonction sinus}. Donc 
si $P_{n_0}$ est vraie, c'est à dire  $u_{n_0} \in  ]0,\frac{\pi}{2}[ $, on a  alors $u_{n_0+1}=\sin(u_{n_0}) \in ]0,1[$.
De nouveau comme $1< \frac{\pi}{2}$ ceci implique $P_{n_0+1}$. 

Par récurrence, la propriété $P(n)$ est vraie pour tout $n\in \N$. 


\item  La fonction $f$ est dérivable sur $\R$ et $f'(x) =\cos(x)-1\leq 0$. 
Donc $f$ est décroissante et $f(0)=0$. Donc pour tout $x\in \R_+^*,$ $f(x)< 0$.
\item $u_{n+1}-u_n =\sin(u_n)-u_n=f(u_n)$
Comme pour tout $n\in \N$, $u_n>0$ d'après la question 1, on a donc 
$f(u_n) <0$ d'après la question 2. Ainsi pour tout $n\in \N$
$$u_{n+1} \leq u_n$$ ce qui assure que la suite $\suite{u}$ est décroissante. 

\item La suite $\suite{u}$ est minorée (par 0) d'après la question $1$ et décroissante d'après la question précédente. Par théorème de la limite monotone, la suite converge vers $\ell \geq 0$ 

\item L'étude de $f$ a montré que $f(x)<0$ sur $\R^*_+$  et $f(x)>0$ sur $\R^*_-$. Ainsi $f(x)=0 \implique x=0$. Réciproquement, si $x=0$ , $f(0) =\sin(0)-0=0$. L'équivalence est bien montrée. 

\item Comme $\suite{u}$ converge vers $\ell\in \R$ on a aussi 
$\lim u_{n+1} =\ell$. De plus, comme la fonction sinus est continue sur $\R$ on a $\lim \sin(u_n) = \sin(\lim u_n) $. Ainsi la limite $\ell$ satisfait  $\ell =\sin(\ell)$. Ce qui d'après la question précédente implique $\ell=0$. 

Finalement $$\lim u_n= 0$$

\end{enumerate}

INFO


\begin{lstlisting}[language =Python]
from math import sin
def u(n):
  x=1	#valeur de u0
  for i in range(n):
     x=sin(x) 	#relation de recurrence que l'on applique n fois avec range(n)
  return(x)

from math import abs
def limite(e):
   L=0 #valeur de la limite
   n=0  #on met en place un compteur
   val=u(n)  #valeur de u0 
 
   while abs(val-L)>e: #tant que la valeur de |u(n)-L| est plus grande que e
      n+=1 #on incremente la valeur du compteur de 1 
      val =u(n) #on actualise la valeur de u(n)
      
   return(n, u(n))
\end{lstlisting}



\end{correction}