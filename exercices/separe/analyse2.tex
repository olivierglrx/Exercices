% Titre : Calcul ensemble de définition
% Filiere : BCPST
% Difficulte :
% Type : DS, DM
% Categories : analyse
% Subcategories : 
% Keywords : analyse



\begin{exercice}
Donner l'ensemble de définition de $f(x) = \sqrt{ (x^2-4)\ln\left(\frac{1}{x}\right)}$
\end{exercice}


\begin{correction}


Le logarithme est défini sur $\R^*_+$, on obtient donc comme condition 
$$\frac{1}{x}>0$$
Cette  condition équivaut à $x>0$.

La racine est définie sur $\R^+$ donc on obtient comme deuxième condition 
$$(x^2-4)\ln(\frac{1}{x}) \geq 0.$$
Ceci équivaut à $(x-2)(x+2) \ln(x)\leq 0$ et un tableau de signe donne comme solution $[1,2]$.
Ce dernier ensemble est donc l'ensemble de définition de $f$ :

\begin{center}
\fbox{$D_f =[1,2]$}
\end{center}
\end{correction}