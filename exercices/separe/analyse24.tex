% Titre : Etude de $f(x) = x+\cos\left(\frac{1}{x}\right).$
% Filiere : BCPST
% Difficulte :
% Type : DS, DM
% Categories : analyse
% Subcategories : 
% Keywords : analyse





\begin{exercice}
Soit $f$ la fonction définie pour tout $x$ par $f(x) = x+\cos\left(\frac{1}{x}\right).$
\begin{enumerate}
\item Donner le domaine de définiiton et de dérivabilité de $f$. 
\item Pour tout $n\in\N^* $ donner l'équation de la tangente 
$(T_n)$ à $\cC_f$ au point d'abscisse $n$. 
\item Calculer les coordonnées de  l'intersection entre $(T_n)$ et l'axe des abscisses. On note $x_n$ la coordonnée non nulle. 
\item Calculer la limite de $\suite{x}$. 
\end{enumerate}
\end{exercice}

\begin{correction}
\begin{enumerate}
\item $f$ est définie et dérivable sur $\R^*$, sa dérivée vaut :
$$f'(x) = 1 + \frac{1}{x^2}\sin(\frac{1}{x})$$
\item L'équation de la tangente en $n\in \N$ est 
$ y- f(n) = f'(n) (x-n)$
soit 
$$y -n -\cos(\frac{1}{n}) = ( 1 + \frac{1}{n^2}\sin(\frac{1}{n}))(x-n)$$
\item L'axe des abscisses a pour équation $y=0$. Il n'est donc pas parallèle à $(T_n)$. Notons $A_n$ le point d'interesection entre ces deux droites. 
Les coordonnées de $A_n$ sont donc $(x_n,0)$ (car appartiennent à l'axe des abscisses) et vérifient : 
$$  0-n -\cos(\frac{1}{n}) = ( 1 + \frac{1}{n^2}\sin(\frac{1}{n}))(x_n-n)$$
car $A_n$ appartient à $(T_n)$. Remarquons que pour $n\in \N^*$, $\frac{1}{n^2}\sin(\frac{1}{n})>0 $ et donc $( 1 + \frac{1}{n^2}\sin(\frac{1}{n})) \neq 0$ et en particulier $$x_n =  \frac{-n -\cos(\frac{1}{n})}{ 1 + \frac{1}{n^2}\sin(\frac{1}{n}) }+n$$

\item En mettant au même dénominateur on obtient : 
$$x_n = \frac{-n -\cos(\frac{1}{n}) +n +\frac{1}{n}\sin(\frac{1}{n})}{ 1 + \frac{1}{n^2}\sin(\frac{1}{n}) }=\frac{ -\cos(\frac{1}{n}) +\frac{1}{n}\sin(\frac{1}{n})}{ 1 + \frac{1}{n^2}\sin(\frac{1}{n}) } $$

Comme $ \lim_{n\tv \infty} \frac{1}{n^2} = 0$ et que la fonction $\sin$  est bornée, on obtient 
$$ \lim_{n\tv \infty} \frac{1}{n}\sin(\frac{1}{n})=0\quadet \lim_{n\tv \infty}  \frac{1}{n^2}\sin(\frac{1}{n})= 0$$
Comme $\cos(0)=1$ et que la fonction $\cos$ est continue en $0$ on obtient : 
$$  \lim_{n\tv \infty} \cos(\frac{1}{n}) = 1$$

Finalement $$ \lim_{n\tv \infty} x_n =-1$$


\end{enumerate}
\end{correction}