
\begin{exercice} Moyenne arithm\'etico-g\'eom\'etrique.\\
\noindent On consid\`ere deux suites $(a_n)_{n\in\N}$ et $(b_n)_{n\in\N}$ d\'efinies par 
$$\left\lbrace\begin{array}{l}
0<a_0<b_0\vsec\\
\forall n\in\N,\ a_{n+1}=\ddp\sqrt{a_nb_n}\vsec\\
\forall n\in\N,\ b_{n+1}=\ddp\frac{a_n+b_n}{2}.
\end{array}\right.$$
$a_{n+1}$ est la moyenne g\'eom\'etrique de $a_n$ et $b_n$; $b_{n+1}$ est la moyenne arithm\'etique de $a_n$ et de $b_n$.
\begin{enumerate}
 \item 
Montrer que les suites $(a_n)_{n\in\N}$ et $(b_n)_{n\in\N}$ sont bien d\'efinies et que pour tout $n\in\N$, $0<a_n\leq b_n$.
\item 
Montrer que la suite $(a_n)_{n\in\N}$ est croissante et que la suite $(b_n)_{n\in\N}$.est d\'ecroissante.
\item 
Montrer que
$$\forall n\in\N,\ b_{n+1}-a_{n+1}\leq \ddp\frac{b_n-a_n}{2}.$$
\item 
En d\'eduire que les suites $(a_n)_{n\in\N}$ et $(b_n)_{n\in\N}$ convergent vers la m\^eme limite.
\end{enumerate}
La limite des suites $(a_n)_{n\in\N}$ et $(b_n)_{n\in\N}$ ne d\'epend que de $a_0$ et $b_0$. On ne peut pas l'exprimer \`a l'aide des fonctions usuelles. On l'appelle moyenne arithm\'etico-g\'eom\'etrique de $a_0$ et $b_0$.
\end{exercice}
