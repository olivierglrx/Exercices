% Titre : somme
% Filiere : BCPST
% Difficulte : 
% Type : TD 
% Categories :somme
% Subcategories : 
% Keywords : somme



\begin{exercice}
Soit $n\in\N^{\star}$. Calculer $\ddp \sum\limits_{k=1}^n \ln{\left( 1+\ddp\frac{1}{k}  \right)}$.
\end{exercice}



\%\%\%\%\%\%\%\%\%\%\%\%\%\%\%\%\%\%\%\%
\%\%\%\%\%\%\%\%\%\%\%\%\%\%\%\%\%\%\%\%
\%\%\%\%\%\%\%\%\%\%\%\%\%\%\%\%\%\%\%\%




\begin{correction}  \; 
\begin{align*}
\sum_{k=1}^n \ln\left(1+\frac{1}{k} \right)&= 	\sum_{k=1}^n \ln\left(\frac{k+1}{k} \right)\\
								&= 	\sum_{k=1}^n \ln(k+1) -\ln(k)
\end{align*}
On reconnait alors une somme telescopique et on a donc :
$$\sum_{k=1}^n \ln\left(1+\frac{1}{k} \right) = \ln(n+1)-\ln(1) = \ln(n+1)$$

\end{correction}