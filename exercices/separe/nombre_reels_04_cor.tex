
\begin{correction}  \; \textbf{Valeurs de $\mathbf{m}$ pour que $\mathbf{mx^2+(m-2)x+2m-2=0}$ ait deux racines r\'eelles distinctes:}\\
\noindent Il faut que $m$ soit non nul (pour avoir une \'equation d'ordre 2), et que $\Delta>0$, \`{a} savoir: $-7m^2+4m+4>0$. \\
On \'etudie le discrimant de ce trin\^ome en $m$ : on a $\delta = 2\times 64$. Les racines du trin\^ome en $m$ sont donc :
$$m_1=\ddp\frac{2-4\sqrt{2}}{7} \quad \textmd{ et } \quad m_2=\ddp\frac{2+4\sqrt{2}}{7}.$$ 
Ainsi, l'\'equation a deux racines r\'eelles distinctes pour : \fbox{$m\in \left\rbrack  \ddp\frac{2-4\sqrt{2}}{7},\ddp\frac{2+4\sqrt{2}}{7} \right\lbrack \, \backslash\, \{0\}$}.
\end{correction}