% Titre : Calcul de la dérivée de $\arcsin$ [Agro 2015]
% Filiere : BCPST
% Difficulte :
% Type : DS, DM
% Categories : analyse
% Subcategories : 
% Keywords : analyse





\begin{exercice}
\begin{enumerate}
\item Que vaut $\arcsin(1/2)$ et $\arcsin(-\sqrt{2}/2)$ ?
\item Tracer le graphe de la fonction $\arcsin$ dans le plan usuel muni d'un repère orthonormé $(O, \vec{i}, \vec{j})$.
\item Soit $x\in [-1, 1]$, calculer  $\sin(\arcsin(x))$ ?
\item Soit $x\in [-1, 1]$, montrer que  $\cos(\arcsin(x)) = \sqrt{1-x^2}$.
%\item Soient $f,g $ deux fonctions dérivables, on rappelle que $(f\circ g )' =g'\times f'\circ g$. On admet que $\arcsin$ est dérivable sur $]-1, 1[$. Calculer sa dérivée. \\

 Pour tout $n\in \N$, on pose $f_n  : x\mapsto \cos(2n \arcsin(x))$ 
\item Calculer $f_0$, $f_1$ et $f_2$. 
\item \begin{enumerate}
\item Soient $a$ et $b$ deux réels, exprimer $\cos(a+b) +\cos(a-b)$ uniquement en fonction de $\cos(a)$ et $\cos(b)$. 
\item En déduire que pour tout entier $n$ on  a :
$$\forall x \in [-1, 1], \, \quad f_{n+2}(x) + f_n(x) =2(1-2x^2) f_{n+1}(x).$$
\end{enumerate}
\end{enumerate}


\end{exercice}

\begin{correction}
\begin{enumerate}
\item $\arcsin(\frac{1}{2})= \frac{\pi}{6}$ et $\arcsin(\frac{-\sqrt{2}}{2})= \frac{-\pi}{4}$

\item 

\item Pour $x\in [-1, 1]$ l'équation $\sin(\theta)=x$ admet une unique solution dans $[\frac{-\pi}{2}, \frac{\pi}{2}]$ notée 
$\theta=\arcsin(x)$. On a donc 
$$\sin(\arcsin(x)) = x.$$

\item On a $\cos^2(\arcsin(x)) =1-\sin^2(\arcsin(x)=1-x^2$
Or $\cos$ est positif pour tout $x\in [\frac{-\pi}{2}, \frac{\pi}{2}]$ donc 
$$\cos(\arcsin(x)) =\sqrt{1-x^2}$$

\item $$f_0(x) = \cos(0 \arcsin(x)) = 1$$
\begin{align*}
f_1(x) &= \cos(2 \arcsin(x)\\
		&= \cos^2(\arcsin(x)) -\sin^2(\arcsin(x))\\
		&=1-x^2-x^2\\
		&=1-2x^2
\end{align*}


\begin{align*}
f_2(x) &= \cos(4 \arcsin(x))\\
		&= \cos^2(2\arcsin(x)) -\sin^2(2\arcsin(x))\\
		&=(1-2x^2)^2-(2\sin(\arcsin(x)\cos(\arcsin(x))^2\\
		&=1-4x^2+4x^4 - 4(x(\sqrt{1-x^2})^2\\
		&=1-4x^2+4x^4 - 4(x^2(1-x^2))\\
		&=1-8x^2+8x^4
\end{align*}

\item
\begin{enumerate}
\item $\cos(a+b)+\cos(a-b)= 2\cos(a)\cos(b)$
\item 
\begin{align*}
f_{n+2} (x) + f_n(x) =&  \cos(2(n+2) \arcsin(x))  + \cos(2n \arcsin(x)) 
\end{align*}

\begin{align*}
f_{n+2} (x)  +f_n(x)&= \cos(2(n+2) \arcsin(x))  +  \cos(2n \arcsin(x))\\
							&=2 \cos\left(\frac{2(n+2)+2n}{2} \arcsin(x)\right) \cos\left(\frac{2(n+2)-2n}{2} \arcsin(x)\right)\\
							&=2\cos(2(n+1) \arcsin(x) \cos(2\arcsin(x))\\
							&=2 f_{n+1}(x) f_1(x)\\
							&=2(1-2x^2) f_{n+1}(x).
\end{align*}

\end{enumerate}
\end{enumerate}
\end{correction}