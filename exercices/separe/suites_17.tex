% Titre : suites
% Filiere : BCPST
% Difficulte : 
% Type : TD 
% Categories :suites
% Subcategories : 
% Keywords : suites




\begin{exercice} \; \'Etudier la monotonie des suites d\'efinies par
\begin{enumerate}
\begin{minipage}[t]{0.4\textwidth}
 \item
$\forall n\in\N,\ u_n=\left(\ddp \sum\limits_{k=0}^n \ddp\frac{1}{2^k}\right)-n$ 
\item 
$\forall n\in\N,\ u_n=\ddp\frac{n!}{2^{n+1}}$ 
\item 
$\forall n\in\N^{\star},\ u_n=\ddp\frac{\ln{(n)}}{n}$ 
\end{minipage}
\begin{minipage}[t]{0.4\textwidth}
\item 
$\forall n\in\N,\ u_n=\ddp \sum\limits_{k=0}^{2n}\ddp\frac{(-1)^k}{k+1}$ 
\item 
$\forall n\in\N,\ u_n=n+2(-1)^n$ 
\item $\forall n\in\N,\ u_n=\ddp \sum\limits_{k=2}^{n}\ddp\frac{1}{k\ln{(k)}}$ 
%\item $\forall n\in\N,\ u_n=\ddp\frac{5^n}{n!}$ 
\end{minipage}
\end{enumerate}
\end{exercice}


\%\%\%\%\%\%\%\%\%\%\%\%\%\%\%\%\%\%\%\%
\%\%\%\%\%\%\%\%\%\%\%\%\%\%\%\%\%\%\%\%
\%\%\%\%\%\%\%\%\%\%\%\%\%\%\%\%\%\%\%\%




\begin{correction} \;
\'Etude de la monotonie des suites suivantes.
\begin{enumerate}
 \item La suite $\suiteu$ est d\'efinie par une somme, on \'etudie donc le signe de $u_{n+1}-u_n$.
$$u_{n+1}-u_n=\sum\limits_{k=0}^{n+1}\ddp\frac{1}{2^k}-(n+1)-\sum\limits_{k=0}^{n}\ddp\frac{1}{2^k}+n=\ddp\frac{1}{2^{n+1}}-1.$$
Or, pour tout $n\in\N$, on a: $\ddp\frac{1}{2^{n+1}}<1$. Ainsi la suite $\suiteu$ est d\'ecroissante.
\item La suite $\suiteu$ est plut\^ot de type produit. Comme tous ses termes sont strictement positifs, on compare $\ddp\frac{u_{n+1}}{u_n}$ \`a 1. On obtient
$$\ddp\frac{u_{n+1}}{u_n}=\ddp\frac{(n+1)!}{2^{n+2}} \times \ddp\frac{2^{n+1}}{n!}=\ddp\frac{n+1}{2}.$$
Un calcul rapide donne
$$\ddp\frac{n+1}{2}<1\Leftrightarrow n+1<2\Leftrightarrow n<1.$$
Ainsi, la suite $(u_n)_{n\in\N^{\star}}$ est croissante ou encorrectione la suite $\suiteu$ est croissante \`{a} partir du rang 1.
\item 
La suite $(u_n)_{n\in\N^{\star}}$ est une suite d\'efinie explicitement et $u_n=f(n)$ avec
$$f:\ x\mapsto f(x)=\ddp\frac{\ln{x}}{x}.$$
L'\'etude de la monotonie de la fonction $f$ sur $\lbrack 1,+\infty\lbrack$ permet d'en d\'eduire directement la monotonie de la suite.\\
\noindent La fonction $f$ est d\'erivable sur $\R^{+\star}$ comme quotient dont le d\'enominateur ne s'annule pas de fonctions d\'erivables. On obtient
$$\forall x\in\R^{+\star},\ f^{\prime}(x)=\ddp\frac{1-\ln{x}}{x^2}.$$
\'Etudions le signe de $1-\ln{x}$ ($x^2\geq 0$ donc le signe de la d\'eriv\'ee est bien le signe de $1-\ln{x}$):\\
\noindent $1-\ln{x}>0  \Leftrightarrow  \ln{x}<1
\Leftrightarrow x<e $ car la fonction exponentielle est strictement croissante.
Ainsi, la fonction $f$ est strictement d\'ecroissante sur $\lbrack e,+\infty\lbrack$. Ainsi, \`a partir du rang 3, la suite $(u_n)_{n\geq 3}$ est d\'ecroissante.
\item La suite $\suiteu$ est d\'efinie par une somme, on \'etudie donc le signe de $u_{n+1}-u_n$.
$$\begin{array}{lll}
u_{n+1}-u_n&=&\sum\limits_{k=0}^{2(n+1)}\ddp\frac{(-1)^k}{k+1}-\sum\limits_{k=0}^{2n}\ddp\frac{(-1)^k}{k+1}\vsec\\
&=& \sum\limits_{k=0}^{2n+2}\ddp\frac{(-1)^k}{k+1}-\sum\limits_{k=0}^{2n}\ddp\frac{(-1)^k}{k+1}\vsec\\
&=& \ddp\frac{1}{2n+3}-\ddp\frac{1}{2n+2}\vsec\\
&=& \ddp\frac{-1}{(2n+3)(2n+2)}.
\end{array}$$
Ainsi, la suite $\suiteu$ est d\'ecroissante.
\item 
On remarque que
$$u_{n+1}-u_n=n+1+2(-1)^{n+1}-n-2(-1)^n=1+2(-1)^{n+1}+2(-1)^{n+1}=1+4(-1)^{n+1}.$$
Ainsi, si $n=2p$ pair, on obtient: $u_{2p+1}-u_{2p}=5>0$ et si $n=2p+1$ impair, on obtient: $u_{2p+2}-u_{2p+1}=-3<0$. Ainsi la suite $\suiteu$ n'est pas monotone.
\item La suite $\suiteu$ est d\'efinie par une somme, on \'etudie donc le signe de $u_{n+1}-u_n$.
$$u_{n+1}-u_n = \sum\limits_{k=2}^{n+1}\ddp\frac{1}{k \ln k}-\sum\limits_{k=2}^{n}\ddp\frac{1}{k \ln k} = \ddp\frac{1}{(n+1) \ln (n+1)} > 0.$$
Ainsi, la suite $\suiteu$ est croissante.
\end{enumerate}
\end{correction}