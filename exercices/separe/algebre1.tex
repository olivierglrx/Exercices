% Titre : Système linéaire
% Filiere : BCPST
% Difficulte :
% Type : DS, DM
% Categories : algebre
% Subcategories : 
% Keywords : algebre





\begin{exercice}
Soit $\lambda \in \R$. On considère le système suivant 
$$(S_\lambda)\quad  \left\{ \begin{array}{ccccc}
2x &+y& & =& \lambda x\\
 &y & & =& \lambda y \\
 -x&-y&+z&=&\lambda z
\end{array}\right. $$

\begin{enumerate}
\item Mettre le système sous forme échelonné. 
\item En donner le rang en fonction de $\lambda$. 
\item Déterminer $\Sigma$ l'ensemble des réels $\lambda$ pour lequel ce système \underline{n'est pas} de Cramer. 
\item Pour $\lambda \in \Sigma$, résoudre $S_\lambda$
\item Quelle est la solution si $\lambda \notin \Sigma$ ? 
\end{enumerate}
\end{exercice}
\begin{correction}
\begin{enumerate}
\item En échangeant les lignes et les colonnes on peut voir que le système est déjà échelonné ! 

$$(S_\lambda)\equivaut  \left\{ \begin{array}{ccccc}
(2-\lambda)x &+y& & =&0 \\
 &(1-\lambda)y & & =& 0 \\
 -x&-y&+(1-\lambda)z&=&0
\end{array}\right. $$

$L_3\leftarrow L_1, L_2 \leftarrow _3, L_1\leftarrow L_2$
$$
(S_\lambda)\equivaut  \left\{ \begin{array}{ccccc}
 -x&-y&+(1-\lambda)z&=&0\\
(2-\lambda)x &+y& & =&0 \\
 &(1-\lambda)y & & =& 0 
\end{array}\right.$$
$ C_3\leftarrow C_1, C_2 \leftarrow C_3, C_1\leftarrow C_2$
$$
\equivaut \left\{ \begin{array}{ccccc}
 (1-\lambda)z&-x&-y&=&0\\
 &(2-\lambda)x&+y & =&0 \\
 &  & (1-\lambda)y& =& 0 
\end{array}\right.$$


\item Si $(2-\lambda)\neq 0$ et $(1-\lambda)\neq 0$ c'est-à-dire si $\lambda \notin\{ 1,2\}$ 
\conclusion{ Le système est triangulaire de rang $3$. }

Si  $(2-\lambda)= 0$,  c'est-à-dire si $\lambda=2$ on a:
$$S_2\equivaut  \left\{ \begin{array}{ccccc}
 -z&-x&-y&=&0\\
 & &+y & =&0 \\
 &  & -y& =& 0 
\end{array}\right.$$
$$S_2\equivaut  \left\{ \begin{array}{ccccc}
 -z&-x&-y&=&0\\
 & &+y & =&0 
\end{array}\right.$$
\conclusion{Le système est de rang 2. }

Si  $(1-\lambda)= 0$,  c'est-à-dire si $\lambda=1$ on a:
$$S_1\equivaut  \left\{ \begin{array}{ccccc}
 &-x&-y&=&0\\
 & x&+y & =&0 \\
 &  &0 & =& 0 
\end{array}\right.$$
$$S_1\equivaut  \left\{ \begin{array}{ccccc}
 &-x&-y&=&0
\end{array}\right.$$
\conclusion{Le système est de rang 1. }


\item Le système n'est pas de Cramer, si $\lambda\in \{1,2\}$.

Si $\lambda=1$ les solutions sont données par 
\conclusion{$S_1=\{ (-y, y, z)\, | (y,z)\in \R^2\}$}

Si $\lambda=2$ les solutions sont données par 
\conclusion{$S_2=\{ (-z, 0, z)\, | z\in \R^2\}$}

\item Si $\lambda\notin \Sigma$, le système est de Cramer, il admet une unique solution. Or il est homogène donc, $(0,0,0)$ est solution, c'est donc la seule :
\conclusion{ $S=\{ (0,0,0)\}$}
\end{enumerate}
\end{correction}