\begin{correction}  \;
\begin{enumerate}
\item $f(x)=\sqrt{x^2}$: 
\begin{itemize}
\item[$\bullet$] Domaine de d\'efinition: La fonction $f$ est bien d\'efinie si $x^2\geq 0$: toujours vrai. Ainsi $\mathcal{D}_f=\R$. 
\item[$\bullet$] \'Etude de la parit\'e: $\mathcal{D}_f$ est centr\'e en 0, et  $\forall x\in\mathcal{D}_f$, on a: $f(-x)=\sqrt{(-x)^2}=\sqrt{x^2}=f(x)$.
Donc la fonction $f$ est paire.
\end{itemize}
%---
\item $f(x)=x^2+x^4+x^6+x^8$:
\begin{itemize}
\item[$\bullet$] Domaine de d\'efinition: La fonction $f$ est bien d\'efinie sur $\R$ donc $\mathcal{D}_f=\R$.
\item[$\bullet$] \'Etude de la parit\'e:  $\mathcal{D}_f$ est centr\'e en 0 et $\forall x\in\mathcal{D}_f$: $f(-x)=(-x)^2+(-x)^4+(-x)^6+(-x)^8=x^2+x^4+x^6+x^8=f(x)$.
Donc la fonction $f$ est paire.
\end{itemize}
%---
\item $f(x)=x+x^3+x^5+2x^7$:
\begin{itemize}
\item[$\bullet$] Domaine de d\'efinition: la fonction $f$ est bien d\'efinie pour tout $x\in\R$ donc $\mathcal{D}_f=\R$.
\item[$\bullet$] \'Etude de la parit\'e: $\mathcal{D}_f$ est centr\'e en 0, et $\forall x\in\R$: $f(-x)=-x+(-x)^3+(-x)^5+2(-x)^7=-x-x^3-x^5-2x^7=-(x+x^3+x^5+2x^7)=-f(x)$
Donc la fonction $f$ est impaire.
\end{itemize}
%---
\item $f(x)=\ddp\sqrt{\ddp\frac{1-|x|}{2-|x|}}$:
\begin{itemize}
\item[$\bullet$] Domaine de d\'efinition: La fonction $f$ est bien d\'efinie si et seulement si $\ddp\frac{1-|x|}{2-|x|}\geq 0$ et $2-|x|\not= 0$. Comme il y a une valeur absolue, on fait des cas:
\begin{itemize}
\item[$\star$] Si $x\geq 0$: on doit r\'esoudre: $\ddp\frac{1-x}{2-x}\geq 0$. Un tableau de signe en prenant en compte le fait que $x\geq 0$ donne: $x\in\lbrack 0,1\rbrack\cup\rbrack 2,+\infty\lbrack$.
\item[$\star$] Si $x<0$: on doit r\'esoudre: $\ddp\frac{1+x}{2+x}\geq 0$. Un tableau de signe en prenant en compte le fait que $x< 0$ donne: $x\in\rbrack -\infty,-2\lbrack\cup\lbrack -1,0\rbrack$.
\end{itemize}
Ainsi, on obtient $\mathcal{D}_f=\rbrack -\infty,-2\lbrack\cup\lbrack -1,1\rbrack\cup\rbrack 2;+\infty\lbrack$.
\item[$\bullet$] \'Etude de la parit\'e: $\mathcal{D}_f$ est centr\'e en 0, et $\forall x\in\mathcal{D}_f$, on a: $f(-x)=\ddp\sqrt{\ddp\frac{1-|-x|}{2-|-x|}}=\ddp\sqrt{\ddp\frac{1-|x|}{2-|x|}}=f(x)$ car $|-x|=|-1|\times |x|=|x|$.
Donc la fonction $f$ est paire.
\end{itemize}
%---
\item $f(x)=\ddp\frac{x^3+3x}{x^2+|x|}$:
\begin{itemize}
\item[$\bullet$] Domaine de d\'efinition: La fonction $f$ est bien d\'efinie si et seulement si $x^2+|x|\not= 0$. Or cette expression est toujours positive, comme somme de termes positif, et s'annule uniquement si les deux termes s'annule, c'est-\`a-dire si et seulement si $x=0$.
Ainsi, on obtient $\mathcal{D}_f=\R\setminus\lbrace 0\rbrace$.
\item[$\bullet$] \'Etude de la parit\'e: $\mathcal{D}_f$ est centr\'e en 0, et $\forall x\in\mathcal{D}_f$: $f(-x)=\ddp\frac{(-x)^3-3x}{(-x)^2+|-x|}=\ddp\frac{-x^3-3x}{x^2+|x|}=-f(x)$.
Donc la fonction $f$ est impaire.
\end{itemize}
\item $f(x)=|x+1|-|x-1|$:
\begin{itemize}
\item[$\bullet$] Domaine de d\'efinition: La fonction $f$ est bien d\'efinie sur $\R$ et ainsi $\mathcal{D}_f=\R$.
\item[$\bullet$] \'Etude de la parit\'e: $\mathcal{D}_f$ est centr\'e en 0, et  $\forall x\in\mathcal{D}_f$: $f(-x)=|-x+1|-|-x-1|=|-(x-1)|-|-(x+1)|=|-1||x-1|-|-1||x+1|=|x-1|-|x+1|=-\left( |x+1|-|x-1|\right)=-f(x)$.
Donc la fonction $f$ est impaire.
\end{itemize}
\item $f(x)=\sin{x}+\cos{x}$ 
\begin{itemize}
\item[$\bullet$] Domaine de d\'efinition: La fonction $f$ est bien d\'efinie sur $\R$ et ainsi $\mathcal{D}_f=\R$.
\item[$\bullet$] \'Etude de la parit\'e: pas de parit\'e: la fonction $f$ n'est ni paire, ni impaire.
\item[$\bullet$] \'Etude de la p\'eriodicit\'e: $\forall x\in\mathcal{D}_f$, $x+2\pi\in\mathcal{D}_f$ et $f(x+2\pi)=\sin{(x+2\pi)}+\cos{(x+2\pi)}=\sin{x}+\cos{x}=f(x)$ en utilisant la $2\pi$ p\'eriodicit\'e des fonctions sinus et cosinus.
Ainsi la fonction $f$ est $2\pi$ p\'eriodique.
\end{itemize}
%---
\item $f(x)=\cos{x}+\cos{(2x)}$ 
\begin{itemize}
\item[$\bullet$] Domaine de d\'efinition: La fonction $f$ est bien d\'efinie sur $\R$ et ainsi $\mathcal{D}_f=\R$.
\item[$\bullet$] \'Etude de la parit\'e: $\mathcal{D}_f=\R$ est centr\'e en 0, et $\forall x\in\mathcal{D}_f=\R$: $f(-x)=\cos{(-x)}+\cos{(-2x)}=\cos{x}+\cos{2x}=f(x)$ en utilisant le fait que la fonction cosinus est paire.
Donc la fonction $f$ est paire.
\item[$\bullet$] \'Etude de la p\'eriodicit\'e: $\forall x\in\mathcal{D}_f=\R$, $x+2\pi\in\mathcal{D}_f=\R$ et $f(x+2\pi)=\cos{(x+2\pi)}+\cos{(2(x+2\pi))}=\cos{x}+\cos{(2x+4\pi)}=\cos{(x)}+\cos{(2x)}=f(x)$ en utilisant la $2\pi$ p\'eriodicit\'e de la fonction cosinus.
Ainsi la fonction $f$ est $2\pi$ p\'eriodique.
\end{itemize}
\end{enumerate}
\end{correction}
%--------------------------------------------------
%------------------------------------------------
