% Titre : Simplification  de $\sqrt[3]{2+\sqrt{5}}$ 
% Filiere : BCPST
% Difficulte :
% Type : DS, DM
% Categories : autres
% Subcategories : 
% Keywords : autres



\begin{exercice}
On considère les nombres réels $\alpha =\sqrt[3]{2+\sqrt{5}}$ et $\beta =\sqrt[3]{2-\sqrt{5}}$. On rappelle que pour tout réel $y$ on note $\sqrt[3]{y}$ l'unique solution de l'équation $x^3=y$ d'inconnue $x$.

Le but de l'exercice est de donner des expressions simplifiées de $\alpha$ et $\beta$. 

\begin{enumerate}
\item Ecrire un script Python qui permet d'afficher une valeur approchée de $\alpha$.
\item 
\begin{enumerate}
\item Calculer $\alpha \beta$ et $\alpha^3+\beta^3$.
\item Vérifier que $\forall (a,b)\in \R^2$, $(a+b) ^3 = a^3 +3a^2b+3ab^2 +b^3$.
\item En déduire que  $(\alpha+\beta)^3= 4-3(\alpha+\beta)$ 
\end{enumerate}
\item On pose $u=\alpha +\beta$ et on considère la fonction polynomiale $P : x\mapsto x^3+3x-4$. 
\begin{enumerate}
\item A l'aide de  la question précédente montrer que $u$ est une racine de $P$ c'est-à-dire que $P(u)=0$. 
\item Trouver une autre racine \og évidente \fg\, de $P$.
\item Trouver trois nombres réels $a$, $b$, et $c$ tels que $\forall x\in \R, P(x) = (x-1)(ax^2+bx+c)$
\item Résoudre l'équation $P(x)=0$ pour $x\in \R$.
\item En déduire la valeur de $u$. 
\end{enumerate}
\item On considère la fonction polynomiale $Q : x\mapsto Q(x) = (x-\alpha)(x-\beta)$
\begin{enumerate}
\item A l'aide des questions précédentes, développer et simplifier $Q(x)$ pour tout nombre réel $x$. 
\item En déduire des expressions plus simples de $\alpha $ et $\beta$. 
\end{enumerate}
\end{enumerate}
\end{exercice}

\begin{correction}
\begin{enumerate}
\item \begin{lstlisting}
print((2+5**(1/2))**(1/3))
\end{lstlisting}
\item \begin{enumerate}
\item \begin{align*}
\alpha \beta &= \sqrt[3]{2+\sqrt{5}}\sqrt[3]{2-\sqrt{5}}\\
					&=\sqrt[3]{(2+\sqrt{5})(2-\sqrt{5}}\\
					&=\sqrt[3]{4-5}\\
					&=\sqrt[3]{-1}\\					
					&=-1
\end{align*}

\begin{align*}
\alpha^3+\beta^3&= 2+\sqrt{5}+2-\sqrt{5}\\
							&=4
\end{align*}

\conclusion{$\alpha\beta =-1$ et $\alpha^3+\beta^3=4$}

\item 
\begin{align*}
(a+b)^3&=(a+b)^2(a+b)\\
			&=(a^2+2ab+b^2)(a+b)\\
			&=a^3+2a^2b+ab^2+ba^2+2ab^2+b^3\\
			&=a^3+3a^2b+3ab^2+b^3
\end{align*}

\conclusion{ $\forall (a,b)\in \R^2$, $(a+b) ^3 = a^3 +3a^2b+3ab^2 +b^3$.}
\item 
\begin{align*}
(\alpha+\beta)^3& = \alpha^3 +3\alpha^2\beta +3\alpha \beta^2 +\beta^3\\
&=\alpha^3  +\beta^3 +3\alpha\beta(\alpha + \beta)\\
&=4-3\alpha\beta
\end{align*}
\end{enumerate}
\item 
\begin{align*}
P(u) &= P(\alpha+\beta) \\
	&= (\alpha+\beta)^3+3\alpha\beta -4\\
	&=0 \quad \text{ d'après la question précédente}
\end{align*}

\conclusion{ $u$ est racine de $P$}

\item $1$ est aussi racine de $P$, en effet : $P(1) =1+3-4=0$

\conclusion{ $1$ est racine de $P$}

\item Développons 
$(x-1) (ax^2+bx+c)$ on obtient 
$$(x-1) (ax^2+bx+c) =ax^3 +(b-a)x^2 +(c-b)x -c$$
En identifiant avec $P$, on a : 
$a=1, b-a=0, c-b=3, -c=-4$
c'est à dire 
\conclusion{ $a=1$, $b=1$ et $c=4$}  


\item D'après la question précédente $P(x) =(x-1) (x^2+x+4)$
Le discriminant de $x^2+x+4$ est $\Delta =1-4*4 =-15<0$ 
$x^2+x+4>0$ pour tout $x\in \R$. Ainsi $P(x)=0$ admet pour unique solution 
\conclusion{ $\cS=\{1\}$} 
 
 \item $1$ est racine de $P$, c'est la seule. Comme $u$ est aussi racine, 
 \conclusion{$u=1$ }

\item 
\begin{enumerate}
\item Développons $Q$:
\begin{align*}
Q(x) &= (x-\alpha) (x-\beta)\\
		&=x^2 -(\alpha+\beta)x +\alpha\beta\\
		&=x^2 -x -1
\end{align*}

\conclusion{ Pour tout $x\in \R, \, Q(x) =x^2-x-1$}

\item L'expression $Q(x) = (x-\alpha) (x-\beta)$ montre que les racines de $Q$ sont   $\alpha$ et $\beta$. 

D'autre part, on connait une autre expression des racines de $Q$ à l'aide du discriminant $\Delta =1+4=5$, les racines de $Q$ sont 
$$r_1 = \frac{1+\sqrt{5}}{2}\quadet r_1 = \frac{1-\sqrt{5}}{2}$$

Remarquons que $r_1<r_2$ et on  a $\alpha<\beta$ donc 
\conclusion{ $\alpha =  \frac{1+\sqrt{5}}{2} $ et $\beta =  \frac{1-\sqrt{5}}{2} $ }


\end{enumerate}
\end{enumerate}
\end{correction}