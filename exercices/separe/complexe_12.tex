% Titre : complexe
% Filiere : BCPST
% Difficulte : 
% Type : TD 
% Categories :complexe
% Subcategories : 
% Keywords : complexe




\begin{exercice}  \;
Soit $n\in\bN^{\star}$. R\'esoudre dans $\bC$ les \'equations suivantes et mettre les solutions sous forme exponentielle.
\begin{enumerate}
\item $z^n=(z-1)^n$, $n\in\bN^{\star}$
\item $(z+1)^n=(z-1)^n$ 
\end{enumerate}
\end{exercice}


\%\%\%\%\%\%\%\%\%\%\%\%\%\%\%\%\%\%\%\%
\%\%\%\%\%\%\%\%\%\%\%\%\%\%\%\%\%\%\%\%
\%\%\%\%\%\%\%\%\%\%\%\%\%\%\%\%\%\%\%\%




\begin{correction}   \;
\begin{enumerate}
%----------------------------------------------------
\item  \textbf{R\'esolution de $\mathbf{z^n=(z-1)^n}$ :}
On peut tout de suite remarquer que $z=1$ n'est pas solution. On obtient alors pour tout $z\not= 1$,
$$\begin{array}{llll}
z^n=(z-1)^n&\Leftrightarrow & \left( \ddp\frac{z}{z-1} \right)^n=1\vsec\\
& \Leftrightarrow & \exists k\in\lbrace 0,\dots, n-1\rbrace,\ \ddp\frac{z}{z-1}=e^{\frac{2ik\pi}{n}}& \textmd{ (racines $n$-i\`emes de l'unit\'e)}\vsec\\
& \Leftrightarrow & \exists k\in\lbrace 0,\dots, n-1\rbrace,\ z=e^{\frac{2ik\pi}{n}}(z-1).
\end{array}$$
Si $k=0$, $z=z-1$ n'a pas de solution. Ainsi, on peut prendre $k\in\lbrace 1,\dots,n-1\rbrace$. On obtient alors
$$\begin{array}{lll}
z^n=(z-1)^n&\Leftrightarrow & \exists k\in\lbrace 1,\dots, n-1\rbrace,\ z(1-e^{\frac{2ik\pi}{n}})=-e^{\frac{2ik\pi}{n}}\vsec\\
&\Leftrightarrow & \exists k\in\lbrace 1,\dots, n-1\rbrace,\ z=\ddp\frac{-e^{\frac{2ik\pi}{n}}}{e^{\frac{ik\pi}{n}}\times \left(-2i\sin{\left( \ddp\frac{k\pi}{n}\right)}\right)   }\vsec\\
&\Leftrightarrow & \exists k\in\lbrace 1,\dots, n-1\rbrace,\ z=e^{\frac{ik\pi}{n}}\ddp\frac{-i}{2\sin{\left( \ddp\frac{k\pi}{n}\right)}} = \frac{1}{2\sin{\left( \ddp\frac{k\pi}{n}\right)}} e^{i\frac{k\pi}{n}+ \frac{3pi}{2}}.
\end{array}$$
Donc on a : \fbox{$ \mathcal{S}=\left\lbrace \ddp  \frac{1}{2\sin{\left( \ddp\frac{k\pi}{n}\right)}} e^{i\frac{k\pi}{n}+ \frac{3pi}{2}}, k \in \intent{ 1, n-1 } \right\rbrace$}
%----------------------------------------------------
\item Exercice tr\`{e}s classique: L'id\'ee ici est de se ramener \`{a} la r\'esolution d'une \'equation type racine n-i\`{e}me de l'unit\'e.
\begin{itemize}
\item[$\bullet$] Comme 1 n'est pas solution de l'\'equation, on peut supposer que $z\not= 1$. Ainsi, on peut bien diviser par $(z-1)^n$ qui est bien non nul. Ainsi, on a
$$(z+1)^n=(z-1)^n\Leftrightarrow \left( \ddp\frac{z+1}{z-1}\right)^n=1\Leftrightarrow Z^n=1$$
en posant $Z=\ddp\frac{z+1}{z-1}$.
\item[$\bullet$] R\'esolution des racines n-i\`{e}me de l'unit\'e: \`{a} savoir faire: voir cours:\\
\noindent On obtient donc que les solutions sont les $Z$ de la forme
$$Z_k=e^{\frac{2ik\pi}{n}},\quad k\in\intent{ 0,n-1}.$$
\item[$\bullet$] On repasse alors \`{a} $z$ et on cherche donc les $z$ tels que: $\ddp\frac{z+1}{z-1}=e^{\frac{2ik\pi}{n}}$ avec $k\in\intent{ 0,n-1}$ fix\'e. On obtient alors
$$\ddp\frac{z+1}{z-1}=e^{\frac{2ik\pi}{n}}\Leftrightarrow z+1=e^{\frac{2ik\pi}{n}} (z-1)\Leftrightarrow z\left(1- e^{\frac{2ik\pi}{n}} \right)=-e^{\frac{2ik\pi}{n}}-1\Leftrightarrow z\left(e^{\frac{2ik\pi}{n}} -1\right)=e^{\frac{2ik\pi}{n}}+1.$$
Ici, il faut faire attention car on ne peut JAMAIS diviser par un nombre sans v\'erifier qu'il est bien NON nul. Or on a:
$$e^{\frac{2ik\pi}{n}} -1=0\Leftrightarrow e^{\frac{2ik\pi}{n}} =1\Leftrightarrow \ddp\frac{2k\pi}{n}=2k^{\prime}\pi\Leftrightarrow k=nk^{\prime}$$
avec $k^{\prime}\in\Z$. Or $k\in\intent{ 0,n-1}$ donc le seul $k$ qui v\'erifie cela est $k=0$. 
\begin{itemize}
\item[$\star$] Pour $k=0$, on obtient: $0=2$ donc il n'y a pas de solution pour $k=0$.
\item[$\star$] Pour $k\not= 0$, \`{a} savoir pour $k\in\intent{ 1,n-1}$, on sait que $1- e^{\frac{2ik\pi}{n}}\not= 0$ et on peut donc bien diviser. On obtient
$$z=\ddp\frac{e^{\frac{2ik\pi}{n}}+1}{e^{\frac{2ik\pi}{n}} -1}=-i\cot{\left( \ddp\frac{k\pi}{n} \right)}$$
en utilisant la m\'ethode de l'angle moiti\'e.
\end{itemize}
\item[$\bullet$] Conclusion: $\fbox{$ \mathcal{S}=\left\lbrace  z\in\bC,\exists k\in\intent{ 1,n-1},\ z=-i\cot{\left( \ddp\frac{k\pi}{n} \right)}  \right\rbrace $}$
\end{itemize}
%----------------------------------------------------
\end{enumerate}
\end{correction}