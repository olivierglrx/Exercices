\begin{correction}  \;
On consid\`{e}re deux fonctions $f$ et $g$ toutes les deux d\'efinies sur $\R$.
\begin{enumerate}
\item On suppose que $f$ et $g$ sont deux fonctions impaires. Montrons que $f\circ g$ est impaire.
\begin{itemize}
\item[$\bullet$] $\R$ est bien centr\'e en 0.
\item[$\bullet$] Soit $x\in\R$: $f\circ g(-x)=f\lbrack g(-x)\rbrack=f\lbrack -g(x)\rbrack$ car la fonction $g$ est impaire. Puis comme la fonction $f$ est elle aussi impaire, on obtient: $f\lbrack -g(x)\rbrack=-f\lbrack g(x)\rbrack=-f\circ g(x)$. Ainsi: $f\circ g(-x)=-f\circ g(x)$.
\end{itemize} 
Donc $f\circ g$ est impaire et on a bien montr\'e que la compo\'ee de deux fonctions impaires est impaire.
\item On suppose par exemple que $f$ est paire et que $g$ est impaire. Montrons que $f\circ g$ est paire.
\begin{itemize}
\item[$\bullet$] $\R$ est bien centr\'e en 0.
\item[$\bullet$] Soit $x\in\R$: $f\circ g(-x)=f\lbrack g(-x)\rbrack=f\lbrack -g(x)\rbrack$ car la fonction $g$ est impaire. Puis comme la fonction $f$ est paire, on obtient: $f\lbrack -g(x)\rbrack=f\lbrack g(x)\rbrack=f\circ g(x)$. Ainsi: $f\circ g(-x)=f\circ g(x)$.
\end{itemize} 
Donc $f\circ g$ est paire et on a bien montr\'e que la compo\'ee d'une fonction paire et d'une fonction impaire est paire.
\item On suppose que $f$ et $g$ sont deux fonctions impaires. Montrons que $f+g$ est impaire:
\begin{itemize}
\item[$\bullet$] $\R$ est bien centr\'e en 0.
\item[$\bullet$] Soit $x\in\R$: $(f+ g)(-x)=f(-x)+ g(-x)=-f(x)-g(x)$ car les fonctions $f$ et $g$ sont impaires. Puis on obtient: 
$(f+g)(-x)=-(f(x)+g(x))=-(f+g)(x)$.
\end{itemize} 
Donc $f + g$ est impaire et on a bien montr\'e que la somme de deux fonctions impaires est impaire.
\item On suppose que $f$ et $g$ sont deux fonctions impaires. Montrons que $f\times g$ est paire:
\begin{itemize}
\item[$\bullet$] $\R$ est bien centr\'e en 0.
\item[$\bullet$] Soit $x\in\R$: $(f\times g)(-x)=f(-x)\times g(-x)=-f(x)\times (-g(x))$ car les fonctions $f$ et $g$ sont impaires. Puis on obtient: 
$(f\times g)(-x)=f(x)\times g(x)=(f\times g)(x)$.
\end{itemize} 
Donc $f \times g$ est paire et on a bien montr\'e que le produit de deux fonctions impaires est paire.
\end{enumerate}
\end{correction}
%--------------------------------------------------
