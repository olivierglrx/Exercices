% Titre : Calcul de $e^{i\pi/7}$
% Filiere : BCPST
% Difficulte :
% Type : DS, DM
% Categories : autres
% Subcategories : 
% Keywords : autres




\begin{exercice}
Soit $\omega =e^{\frac{2i\pi}{7}}$. On considère $A=\omega+\omega^2 +\omega^4$ et $B =\omega^3+\omega^5 +\omega^6$

\begin{enumerate}
\item Calculer $\frac{1}{\omega}$ en fonction de $\overline{\omega}$
\item Montrer que pour tout $k\in \intent{0,7}$ on a 
$$\omega^k =\overline{\omega}^{7-k}.$$
\item En déduire que $\overline{A}=B$.
\item Montrer que la partie imaginaire de $A$ est strictement positive. (On pourra montrer que $\sin\left( \frac{2\pi}{7}\right)-\sin\left( \frac{\pi}{7}\right)>0$.)
\item  Rappelons la valeur de la  somme d'une suite géométrique : $\forall q\neq 1, \, \forall n\in \N : $
$$\sum_{k=0}^n q^k =\frac{1-q^{n+1}}{1-q}.$$
Montrer alors que $\ddp \sum_{k=0}^6 \omega^k =0$. En déduire que $A+B=-1$.
\item Montrer que $AB=2$. 

\item En déduire la valeur exacte de $A$.


\end{enumerate}
\end{exercice}
\begin{correction}
\begin{enumerate}
\item $$\frac{1}{\omega} = e^{\frac{-2i\pi}{7}} =\overline{\omega}$$
\item On a $\omega^7 = e^{7\frac{2i\pi}{7}}=e^{2i\pi}=1 $ donc pour tout $k\in \intent{0,7}$ on a 
$$\omega^{7-k}\omega^{k}=1$$
D'où 
$$\omega^k=\frac{1}{\omega^{7-k}}=\overline{\omega}^{7-k}$$
\item On  a d'après la question précédente : 
$$\overline{\omega} =\omega^{6}$$
$$\overline{\omega^2} =\omega^{5}$$
$$\overline{\omega^4} =\omega^{3}$$
Ainsi on a : 
\begin{align*}
\overline{A}&=\overline{\omega+\omega^2+\omega^4} \\
					&=\overline{\omega}+\overline{\omega^2}+\overline{\omega^4} \\
					&=\omega^6+\omega^5+\omega^3\\
					&= B. 
\end{align*}


\item $$\Im(A) =\sin(\frac{2\pi}{7})+\sin(\frac{4\pi}{7})+\sin(\frac{8\pi}{7})=\sin(\frac{2\pi}{7}) +\sin(\frac{4\pi}{7}) -\sin(\frac{\pi}{7})$$

Comme $\sin$ est croissante sur $[0, \frac{\pi}{2}[$ 
$$\sin(\frac{\pi}{7}) \leq \sin(\frac{2\pi}{7})$$
Donc 
$$\Im(A) \geq \sin(\frac{4\pi}{7})>0$$


\item On a 
$$\sum_{k=0}^6 \omega^k = \frac{1-\omega^7}{1-\omega} = 0$$

Or $$A+B= \sum_{k=1}^6 \omega^k =  \sum_{k=0}^6 \omega^k-1=-1$$



\item  $AB = \omega^{4}+\omega^{6}+\omega^{7}+\omega^{5}+\omega^{7}+\omega^{8}+\omega^{7}+\omega^{9}+\omega^{10}$ 
D'où 
$$AB= 2\omega^7 + \omega^4(1+\omega^{1}+\omega^{2}+\omega^{3}+\omega^{4}+\omega^{5}+\omega^{6})=2\omega^7=2$$

\item $A$ et $B$ sont donc les racines du polynome du second degré $X^2+X+2$. Son discriminant vaut $\Delta  =1-8 = -7$ donc 
$$A\in \{\frac{-1 \pm i\sqrt{7}}{2}\}$$

D'après la question 4, $\Im(A)>0$ donc 

$$A= \frac{-1+ i\sqrt{7}}{2}$$

\end{enumerate}

\end{correction}