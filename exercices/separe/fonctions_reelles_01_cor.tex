\begin{correction}  \;
\begin{enumerate}
 \item $f(x)=\sqrt{x^3}$: la fonction $f$ est bien d\'efinie si et seulement si $x^3\geq 0\Leftrightarrow x\geq 0$ car la fonction racine cubique est strictement croissante sur $\R$. Ainsi \fbox{$\mathcal{D}_f=\R^+$}.
 %--
\item $f(x)= \ddp\frac{1}{x-\frac{1}{x}}$. La fonction $f$ est bien d\'efinie si et seulement si: $x\not= 0$ et $x-\ddp\frac{1}{x}\not= 0\Leftrightarrow \ddp\frac{x^2-1}{x}\not= 0\Leftrightarrow x\not=\pm 1$. Ainsi, on obtient: \fbox{$\mathcal{D}_f=\R\setminus\lbrace -1,0,1\rbrace$}.
%---
\item $f(x)=\sqrt{x-3} +\ddp\frac{\sqrt{5+x}}{x} $. La fonction $f$ est bien d\'efinie si et seulement si $x-3\geq 0$, $5+x\geq 0$ et $x\not= 0$. Ainsi, on obtient: \fbox{$\mathcal{D}_f=\lbrack 3,+\infty\lbrack$}.
%--
\item $f(x)=\ln{\left(\ddp\frac{e^x+1}{e^x-1}\right)}$. La fonction $f$ est bien d\'efinie si $\ddp\frac{e^x+1}{e^x-1}>0$ et $e^x-1\not= 0$. Comme le num\'erateur est strictement positif comme somme de deux termes strictement positifs, on a: $\ddp\frac{e^x+1}{e^x-1}>0\Leftrightarrow e^x-1>0\Leftrightarrow x>0$. Donc \fbox{$\mathcal{D}_f=\R^{+\star}$}.
%--
\item $f(x) = \ln{\left(\ddp\frac{2-x}{x+4}\right)}$. La fonction $f$ est bien d\'efinie si $\ddp\frac{2-x}{x+4}>0$ et $x+4\not= 0$ (faire un tableau de signe). Donc \fbox{$\mathcal{D}_f=\rbrack -4,2\lbrack$}.
\end{enumerate}
\end{correction}