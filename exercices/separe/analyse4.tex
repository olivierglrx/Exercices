% Titre : Etude de $I_n=\int_1^e (\ln(x))^n dx$ 
% Filiere : BCPST
% Difficulte :
% Type : DS, DM
% Categories : analyse
% Subcategories : 
% Keywords : analyse




\begin{exercice}
On considère pour tout $n\in \N$ l'intégrale 
$$I_n = \int_1^e (\ln(x))^n dx$$

\begin{enumerate}
\item \begin{enumerate}
\item Démontrer que pour tout $x\in ]1,e[ $ et pour tout entier naturel $n\in \N$ on  a $ (\ln(x))^n  - (\ln(x))^{n+1} >0$.
\item En déduire que la suite $\suite{I}$ est décroissante.
\end{enumerate}
\item \begin{enumerate}
\item Calculer $I_1$ à l'aide d'une intégration par parties. 
\item Démontrer, toujours à l'aide d'une intégration par parties que, pour tout $n\in \N$, $I_{n+1} = e- (n+1)I_n$
\end{enumerate}
\item \begin{enumerate}
\item Démontrer que pour tout $n\in \N$, $I_n\geq0$.
\item Démontrer que pour tout $n\in \N$, $(n+1) I_n\leq e$.
\item En déduire la limite de $\suite{I}$. 
\item Déterminer la valeur de $nI_n +(I_n+I_{n+1})$  et en déduire la limite de $nI_n$. 
\end{enumerate}
\end{enumerate}
\end{exercice}

\begin{correction}
\begin{enumerate}
\item Pour tout $x\in ]1,e[ $ , $0<\ln(x) <1$, donc $\ln(x)^n \ln(x)< \ln(x)^n $. On obtient bien 
\conclusion{$\ln(x)^n -\ln(x)^{n+1}>0$}
\item En intégrant, par positivité de l'intégrale on a 
$$\int_1^e \ln(x)^n -\ln(x)^{n+1}dx >0$$
\conclusion{Donc $I_n>I_{n+1}$ et la suite est bien décroissante. }

\item vu en cours. 
$$\int_1^e \ln(x) dx= [x\ln(x)]_1^e - \int_1^e x \frac{1}{x}dx$$\\
Donc \conclusion{$\int_1^e \ln(x) dx = e-(e-1) =1$}

\item On pose $u'(x)= 1$ et $v(x) = (\ln(x))^{n+1}$. On  a
$u(x)=x$ et $v'(x) = (n+1) \frac{1}{x}  (\ln(x))^{n}$. Et finalement 
\begin{align*}
I_{n+1} &=  \int_1^e 1 (\ln(x))^{n+1} dx\\
			 &= [x (\ln(x))^{n+1}]^e_1 -\int_1^e  x  (n+1) \frac{1}{x}  (\ln(x))^{n} dx  \\
			  &= e -(n+1)I_n 
\end{align*}

\item Comme $\ln(x)\geq 0$ pour tout $x\in [1,e]$, $\ln(x)^n\geq 0$.
\conclusion{
 Par positivité de l'intérgale, $I_n$ est positive. }
\item D'après la question 2b, $(n+1)I_n = e -I_{n+1}$ et d'après la question précédente pour tout $n\in\N$, $I_n \geq 0$ donc 
$e-I_{n+1} \leq e$. \conclusion{On  a bien $(n+1)I_n\leq e$. }
\item Les question précédentes montre que 
$$0\leq I_n \leq \frac{e}{n+1}$$
Comme $\lim_{n\tv+\infty} \frac{e}{n+1}=0$, le théorème des gendarmes assure que 
\conclusion{La suite $\suite{I}$ converge et sa limite vaut $0$. }

\item D'après la question 2b, $I_{n+1} = e- (n+1)I_n$ donc 
$$(n+1)I_n+I_{n+1} =e$$
et finalement 
$nI_n + (I_n +I_{n+1}) =e$
Comme $\lim I_n = \lim I_{n+1} =0$ on obtient 
\conclusion{$\ddp \lim_{n\tv+\infty}  nI_n = e.$}

\end{enumerate}
\end{correction}