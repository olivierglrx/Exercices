
\begin{correction}   \;
\begin{itemize}
\item[$\bullet$] \textbf{Calculer le module de $\mathbf{z_1=t^2+2ti-1}$:} On a 
$$|z_1|^2=(t^2-1)^2+4t^2=t^4+2t^2+1=(1+t^2)^2.$$
Ainsi, $\fbox{$|z_1|=\sqrt{(1+t^2)^2}=|1+t^2|=1+t^2$}$ car la somme de deux nombres positifs est positive.
\item[$\bullet$] \textbf{Calculer le module de $\mathbf{z_2=1-\cos{(t)}+i\sin{(t)} }$:} On a:
$$|z_2|^2=(1-\cos{t})^2+\sin^2{t}=2(1-\cos{t})=4\sin^2{\left(\ddp\frac{t}{2}\right)}.$$
Ainsi, $|z_2|=2|\sin{\left(\frac{t}{2}\right)}|$. Il faut alors discuter selon le signe du sinus qui n'est pas toujours positif.
\begin{itemize}
\item[$\star$] Si $\sin{\left(\ddp\frac{t}{2}\right)}\geq 0$, on a $\fbox{$|z_2|=2\sin{\left( \ddp\frac{t}{2} \right)}.$}$ \'Etude de $\sin{\left(\ddp\frac{t}{2}\right)}\geq 0$:
$$\sin{\left(\ddp\frac{t}{2}\right)}\geq 0 \Leftrightarrow  \exists k\in\Z,\ 0+2k\pi\leq \ddp\frac{t}{2}\leq \pi+2k\pi\Leftrightarrow \exists k\in\Z,\ 4k\pi\leq t \leq 2\pi+4k\pi.$$
\item[$\star$] Si $\sin{\left(\ddp\frac{t}{2}\right)}\leq 0$, on a $\fbox{$|z_2|=-2\sin{\left( \ddp\frac{t}{2} \right)}.$}$ \'Etude de $\sin{\left(\ddp\frac{t}{2}\right)}\leq 0$:
$$
\sin{\left(\ddp\frac{t}{2}\right)}\leq 0 \Leftrightarrow  \exists k\in\Z,\ \pi+2k\pi\leq \ddp\frac{t}{2}\leq 2\pi+2k\pi\Leftrightarrow  \exists k\in\Z,\ 2\pi+4k\pi\leq t \leq 4\pi+4k\pi.$$
\end{itemize}
%----
\item[$\bullet$] \textbf{Mettre $\mathbf{z_2}$ sous forme exponentielle:} On distingue donc deux cas selon le signe de $\sin{\left(\ddp\frac{t}{2}\right)}$.
\begin{itemize}
\item[$\star$] Cas 1: Lorsque $t$ v\'erifie: $\exists k\in\Z,\ 4k\pi\ < t  < 2\pi+4k\pi$ (0 ne se met pas sous forme exponentielle, il faut donc \'etudier uniquement les nombres complexes non nuls ce qui explique les in\'egalit\'es strictes):\\
\noindent On a alors $|z_2|=2\sin{\left( \ddp\frac{t}{2} \right)}$ et donc 
$$\begin{array}{lllll}
z_2&=2\sin{\left( \ddp\frac{t}{2} \right)} \left\lbrack   \ddp\frac{1-\cos{(t)}}{2\sin{\left( \ddp\frac{t}{2} \right)}}+i\ddp\frac{\sin{(t)}}{2\sin{\left( \ddp\frac{t}{2} \right)}}  \right\rbrack\\ &=2\sin{\left( \ddp\frac{t}{2} \right)} \left\lbrack 
\ddp\frac{  2\sin^2{\left( \ddp\frac{t}{2} \right)}   }{2\sin{\left( \ddp\frac{t}{2} \right)}}+i\ddp\frac{  2\sin{\left( \ddp\frac{t}{2} \right)}\cos{\left( \ddp\frac{t}{2} \right)} }{2\sin{\left( \ddp\frac{t}{2} \right)}}
   \right\rbrack\\
   &= 2\sin{\left( \ddp\frac{t}{2} \right)} \left\lbrack \sin{\left( \ddp\frac{t}{2} \right)} +i\cos{\left( \ddp\frac{t}{2} \right)}    \right\rbrack\\
   & =2\sin{\left( \ddp\frac{t}{2} \right)} \left\lbrack i\left( \cos{\left( \ddp\frac{t}{2} \right)} -i \sin{\left( \ddp\frac{t}{2} \right)}   \right)\right\rbrack\vsec\\
   &= 2\sin{\left( \ddp\frac{t}{2} \right)} \left\lbrack i\left( \cos{\left( -\ddp\frac{t}{2} \right)} +i \sin{\left( -\ddp\frac{t}{2} \right)}  \right) \right\rbrack \\
    = &2\sin{\left( \ddp\frac{t}{2} \right)} \left\lbrack e^{i\frac{\pi}{2}} \times e^{-i\frac{t}{2}} \right\rbrack\vsec\\
   &= 2\sin{\left( \ddp\frac{t}{2} \right)}  e^{i\frac{\pi-t}{2}} &&
   \end{array}$$
Dans ce cas, on a donc obtenu que $\fbox{$z_2=2\sin{\left( \ddp\frac{t}{2} \right)}  e^{i\frac{\pi-t}{2}}.$}$   
\item[$\star$] Cas 2: Lorsque $t$ v\'erifie: $ \exists k\in\Z,\ 2\pi+4k\pi < t  < 4\pi+4k\pi$:\\
\noindent On a alors $|z_2|=-2\sin{\left( \ddp\frac{t}{2} \right)}$ et donc en refaisant le m\^{e}me type de raisonnement que ci-dessus:
$$
z_2=-2\sin{\left( \ddp\frac{t}{2} \right)}\left\lbrack  - e^{i\frac{\pi-t}{2}} \right\rbrack=-2\sin{\left( \ddp\frac{t}{2} \right)}\left\lbrack  e^{i\pi}e^{i\frac{\pi-t}{2}} \right\rbrack=-2\sin{\left( \ddp\frac{t}{2} \right)} e^{i\frac{3\pi-t}{2}}.$$
Dans ce cas, on a donc obtenu que $\fbox{$z_2=-2\sin{\left( \ddp\frac{t}{2} \right)}  e^{i\frac{3\pi-t}{2}}.$}$  
\end{itemize}
\item[$\bullet$] \textbf{Autre m\'ethode :} On peut \'egalement utiliser la m\'ethode de l'angle moiti\'e. On a en effet :
$$z_2 = 1-\cos t+i \sin t = 1-e^{-it} = e^{-i\frac{t}{2}} \left(e^{i\frac{t}{2}} - e^{-i\frac{t}{2}}\right) = 2i \sin\left(\frac{t}{2}\right) e^{-i\frac{t}{2}} = 2 \sin\left(\frac{t}{2}\right) e^{i\frac{\pi-t}{2}}.$$
On reprend ensuite les m\^emes cas, et on obtient les m\^emes r\'esultats que pr\'ec\'edemment.
\end{itemize}
\end{correction}