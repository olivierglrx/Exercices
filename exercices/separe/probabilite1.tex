% Titre : Urne Indépendance d'évenements (facile)
% Filiere : BCPST
% Difficulte :
% Type : DS, DM
% Categories : probabilite
% Subcategories : 
% Keywords : probabilite




\begin{exercice}
 Une urne contient 12 boules numérotées de 1 à 12. On en tire une hasard, et on considère les événements
\begin{center}
A="tirage d'un nombre pair''\\
B="tirage d'un multiple de 3''
\end{center}
Les événements A et B sont-ils indépendants?
 Reprendre la question avec une urne contenant 13 boules.
\end{exercice}



\begin{correction}
\begin{enumerate}
  \item On a:

$$
\begin{gathered}
A=\{2,4,6,8,10,12\} \\
B=\{3,6,9,12\} \\
A \cap B=\{6,12\} .
\end{gathered}
$$
On a donc $P(A)=1 / 2, P(B)=1 / 3$ et $P(A \cap B)=1 / 6=P(A) P(B) .$ Les événements $A$ et $B$ sont indépendants.\\
\item  Les événements $A, B$ et $A \cap B$ s'écrivent encore exactement de la même façon. Mais cette fois, on a : $P(A)=6 / 13, P(B)=4 / 13$ et $P(A \cap B)=2 / 13 \neq 24 / 169$. Les événements $A$ et $B$ ne sont pas indépendants. C'est conforme à l'intuition. Il n'y a plus la même répartition de boules paires et de boules impaires, et dans les multiples de 3 compris entre 1 et 13 , la répartition des nombres pairs et impairs est restée inchangée.
\end{enumerate}
\end{correction}